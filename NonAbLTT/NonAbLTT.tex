%! TeX root: ../main.tex
\documentclass[../main.tex]{subfiles}

\begin{document}
\section{Non-Abelian Lubin-Tate Theory: An Overview} % (fold)
\label{sec:Non-Abelian Lubin-Tate Theory: An Overview}
In the preceeding chapter we used formal $\cO_F$-modules to understand the 
maximial abelian extension of a local field $F$. The hope of non-Abelian Lubin-Tate
theory is to gain insight about the Abelian extensions of $F$ by considering
certain moduli spaces of formal $\cO_F$-modules.
More precisely, attached to a formal $\cO_F$-module $H_0$ over $\Fqbar$ (determined
up to isomorphism by its height $n$), we attach a system of rigid spaces 
$\{M_K\}_{K \subset \GL_n(\cO_F)}$, the so called Lubin-Tate Tower. For $l \neq p$, 
the system of $l$-adic compactly supported cohomology groups $\{H_c^{i}(M_K,
\bar \Q_l)\}_K$ admits commuting actions by $\GL_n(F)$, $W_F$ and $D^\times$,
where the latter denotes the units of the central divison algebra $D =
\End_{\FMOver {\cO_K}{\Fqbar}}(H_0) \otimes \Q$. This yields a correspondence of 
representations of the respective groups, and Harris and Taylor showed in
\cite{HTShimura} that the cohomology of middle degree induces (a version of)
the Local Langlands Correspondence for $\GL_n$. Our goal is an explicit
description of (a part of) this correspondence, and we obtain such a
description by understanding (a part of) the Lubin-Tate tower explicitely.

\subsection{Lubin--Tate Deformation Spaces} % (fold)
\label{sub:Lubin-Tate Deformation Spaces}
The aim of this section is to constuct, associated to a formal $\cO_K$-module
$H_0 \in \FMOver Ak$ and an integer $m$, a certain moduli problem $\cM_{H_0, m}$,
parametrizing "deformations of $H_0$ with Drinfeld level $\varpi^m$-structure".
By results of Drinfeld \cite{drinfel1974elliptic}, these moduli problems turn
out to be representable by formal schemes. 

\subsubsection{The Tower of Deformation Spaces} % (fold)
\label{ssub:The Tower of Deformation Spaces}
We mostly follow \cite[Chapter 2]{Strauch2008DefSp} in the following
exposition. 
Let $H_0$ be a formal $\cO_F$-module law over $k$ and let $R$ be an element of 
$\cC$.

\begin{defi}[Deformation]
  A deformation of $H_0$ to $R$ is a pair $(H, \iota)$ where
  $H$ is a formal $\cO_K$-module over $R$ and $\iota$ is a quasi-isogeny
  \begin{equation*}
    \iota: H_0 \dashto H \otimes_R \Fqbar.
  \end{equation*}
  Two deformations $(H, \iota)$ and $(H', \iota')$ are isomorphic if there is
  an isomorphism $\tau: H \to H'$ with $\iota' \circ \tau = \iota$. 
\end{defi}
The Lubin-Tate space without level structure is the moduli space of such deformations.
More precisely, we define it as the functor
\begin{equation*}
  \cM_0: \cC \to \Set, \quad R \mapsto \{\text{deformations } (H, \iota) \text{
  of $H_0$}\}/\simeq.
\end{equation*}

Note that we have a stratification
\begin{equation*}
  \cM_0 = \coprod_{j \in \Z} \cM_0^{(j)},
\end{equation*}
where $\cM_0^{(j)}$ parametrizes deformations $(H, \iota)$ with $\height(\iota) = j$.

We next introduce variants of this moduli problem with a certain level structure.

\begin{defi}[Drinfeld level $\varpi^m$-structure]
  Let $H \in \FMOver {\cO_F}{R}$ be a $\varpi$-divisible formal $\cO_F$-module
  of height $n>0$ and let $m$ be a non-negative 
  integer. A Drinfeld level $\varpi^m$-structure on $H$ is a morphism of
  $\cO_F$-module objects 
  \begin{equation*}
    \phi: \underline{(\varpi^{-m}\cO_F/\cO_F)}_{\spf R}^n \to H
  \end{equation*}
  in the category of formal schemes over $\spf R$,
  such that after choosing a coordinate $H \cong \spf R\llbr T \rrbr$, the 
  power series $[\varpi]_H(T) \in R\llbr T \rrbr$ satisfies the divisibility constraint
  \begin{equation*}
    \prod_{x \in (\varpi^{-1}\cO_F/\cO_F)^n}(T - \phi(x)) \ \bigg\vert \ [\varpi]_H(T).
  \end{equation*}
\end{defi}

\begin{defi}[Lubin--Tate Deformation Space with Level Structure]\label{def:LTDefSpaceWithLevel}
  Let $\cM_m: \cC \to \Set$ be the functor assigning to $R \in \cC$ the set 
  \begin{equation*}
    \cM_m(R) \coloneqq \{(H, \iota, \phi) \mid (H, \iota) \in \cM_0(R) \text{ and }\phi
    \text{ a Drinfeld level $\varpi^m$-structure on $H$}\}/\simeq.
  \end{equation*}
\end{defi}

By results of Drinfeld, the functor $\cM_m$ is representable.

\begin{thm}[Representability of the Lubin--Tate Deformation Space with Level Structure]\label{thm:RepresentabilityOfDefSpaceWithLevel}
  The functor $\cM_m^{(0)}$ is representable by a regular local ring $A_m \in \cC$ of 
  dimension $m-1$.
\begin{proof}
  This is \cite[Proposition 4.3]{drinfel1974elliptic}.
\end{proof}
\end{thm}

Note that for non-negative integers $m' \leq m$, we have natural morphisms $\cM_m
\to \cM_{m'}$ by restricting the level structure. We obtain a tower
$\{\cM_m\}_{m \geq 0}$ of Lubin--Tate deformation spaces.
% subsubsection The Tower of Deformation Spaces (end)

\subsubsection{Group Actions on the Tower of Lubin--Tate Deformation Spaces} % (fold)
\label{ssub:Group Actions on the Tower of Lubin--Tate Deformation Spaces}
We describe actions of $\GL_n(F) \times D^\times$ on the tower
$\{\cM_m\}_{m \geq 0}$. More precisely, given an element $d \in D^\times$ and an element
$g \in \GL_n(F)$, we construct, for sufficiently large $m\geq 0$, morphisms
\begin{equation*}
  d_m \colon \cM_m^{(j)} \to \cM_m^{(j')} \quad \text{and} \quad
  g_{m,m''} \colon \cM_m^{(j)} \to \cM_{m''}^{(j'')},
\end{equation*}
where $j'= j+\val_\varpi(\Nrd(d))$, $j'' = j - \val_\varpi(\det g)$ and $m''
= m-d$ is an integer differeing from $m$ by an integer depending on $g$. 

The action of $D^\times$ is easy to describe. Given $R \in \cC$ and 
$d \in D^\times$, we put
\begin{equation*}
  (H, \iota, \phi).d = (H, \iota \circ d, \phi).
\end{equation*}

The group $\GL_n(F)$ acts in a less simple matter. The idea is to have
$\GL_n(F)$ act on the level structure of triples $(H, \iota, \phi) \in \cM_m(R)$.
Akin to the action of $D^\times$, we would like to define the action as
$(H, \iota, \phi).g = (H, \iota, \phi \circ g)$, but this only makes sense
if $g \in \GL_n(\cO_F)$. To nonetheless define an 'action' of $\GL_n(F)$, we
allow ourselves to also change the underlying formal group.
We need the following notion of quotients of formal $\cO_F$-modules.

\begin{defi}[Quotient of Formal Module by Finite Subgroup]\label{def:QuotientModule}
  Let $H$ be a formal $\cO_F$-module law and let $G \subset R^\cici$ be a finite
  sub $\cO_F$-module, where we equip $R^\cici$ with the $\cO_F$-module structure 
  coming from $H$. Then, we define the quotient $H/G$ as the following formal
  $\cO_F$-module law. 
  \red{How is this supposed to work please?? It should be something like
  \begin{equation*}
    (H/G)(X,Y) = g(H(g^{-1}(X), g^{-1}(Y)), \quad \text{where}\quad 
    g(T) = \prod_{a \in F} (T - a).
  \end{equation*}
  But this doesn't make sense. 
  }
\end{defi}

\begin{lem}\label{lem:PropertyOfQuotient}
  The quotient as above is a formal $A$-module. If $\# P = q^n$, the induced
  morphism $H \to H/G$ of formal module laws over $R$ is of height $n$.
\begin{proof}
\end{proof}
\end{lem}

\begin{lem}\label{lem:DrinfeldLevelOnQuotients}
  Let $H$ be a $\varpi$-divisible formal $A$-module over $R\in\cC$ and let 
  $\phi$ be a Drinfeld $\varpi^m$-level structure on $H$. Suppose that 
  $P \subset (\varpi^{-m} \cO_F/\cO_F)^n$ is a submodule and that 
  there is an injection 
  \begin{equation*}
    \left(\varpi^{-m'} \cO_F/\cO_F \right)^n \to \frac{(\varpi^{-m}
    \cO_F/\cO_F)^n}{P}.
  \end{equation*}
  Then, the morphism
  \begin{equation*}
    \phi': \left(\varpi^{-m'} \cO_F/\cO_F \right)^n \to H/\phi(P)
  \end{equation*}
  is a Drinfeld $\varpi^m$-level structure.
\begin{proof}
  This is \cite[Proposition 4.4]{drinfel1974elliptic}.
\end{proof}
\end{lem}

First assume that $g \in \GL_n(F)$ is such that $g^{-1} \in \Mat_{n\times
n}(\cO_F)$ and $g \in \varpi^{-d} \Mat_{n \times n}(\cO_F)$ for some non-negative
integer $d$. In this case, we construct for all integers $m \geq d$ a natural
transformation
\begin{equation*}
  g_{m, m-d}: \cM_m \to \cM_{m-d}.
\end{equation*}
Note that $g \cO_F^n \subset \varpi^{-d} \cO_F^n$, and that multiplication with
$g$ yields an injection
\begin{equation*}
  (\varpi^{m-d}\cO_F/\cO_F)^n \xto g (\varpi^{-m}\cO_F^n/g\cO_F^n) = 
  \frac{(\varpi^{-m} \cO_F^n /\cO_F^n)}{(g \cO_F^n/\cO_F^n)}.
\end{equation*}
Now, given a tuple $(H, \iota, \phi) \in \cM_m(R)$, we put $$(H, \iota, \phi).g = (H', \iota', \phi'),$$ where 
$$H'= H/\phi(g\cO_F^n/\cO_F^n)$$ 
is a the quotient of $H$ as in Definition \ref{def:QuotientModule}, 
$$\iota'\colon H_0 \to H \otimes \bar \FF_q \to H' \otimes \bar \FF_q$$ 
is the corresponding quasi-isogeny of height 
$(\height(\iota) - \val_\varpi( \det g))$, and 
$$\phi'\colon (\varpi^{m-d}\cO_F/\cO_F)^n \to H'$$ 
is the Drinfeld $\varpi^{m-d}$-level structure obtained by Lemma
\ref{lem:DrinfeldLevelOnQuotients}. 
For varying choices of $d$, this gives a system of maps compatible with the 
transition functions $\cM_m \to \cM_{m'}$. Indeed, given integers $m \geq d'
\geq d$ with $d$ as above, the triangle 
\begin{equation*}
\begin{tikzcd}[ampersand replacement=\&]
	{\cM_{m}} \& {\cM_{m-d}} \\
	\& {\cM_{m-d'}}
	\arrow["{g_{m,m-d}}", from=1-1, to=1-2]
	\arrow["{g_{m,m-d'}}"', from=1-1, to=2-2]
	\arrow[from=1-2, to=2-2]
\end{tikzcd}
\end{equation*}
commutes.

If $g \in \GL_n(F)$ is an arbitrary element, we may choose an integer $r$ such
that $(\varpi^{-r} g)^{-1} \in \Mat_{n \times n}(\cO_F)$. We now pick 
$d > 0$ in a way that $\varpi^{-r}g \in \varpi^{-d}\Mat_{n \times n}(\cO_F)$. 
Now, for $m \geq d$, we obtain natural transformations
\begin{equation*}
  g_{m,m-d}\colon \cM_{m} \to \cM_{m-d}, \quad (H, \iota, \phi).g = (H, \iota \circ \varpi^{-r}, \phi).(\varpi^{-r}g).
\end{equation*}
By the same reason as above, this yields, for varying choices of $d$, a
compatible system of natural transformations.
This construction is independent of the choice of $r$, as $(H, \iota,
\phi).(\varpi\cdot\id) \simeq (H, \iota \circ [\varpi^{-1}], \phi)$.
Furthermore, given $g, g' \in \GL_n(F)$ with suitable choices of integers 
$d, d'$, we obtain a commutative triangle
\begin{equation*}
\begin{tikzcd}[ampersand replacement=\&]
	{\cM_{m}} \& {\cM_{m-d}} \\
	\& {\cM_{m-d-d'}}
	\arrow["{g_{m,m-d}}", from=1-1, to=1-2]
	\arrow["{(gg')_{m,m-d-d'}}"', from=1-1, to=2-2]
	\arrow["{g'_{m-d,m-d-d'}}", from=1-2, to=2-2]
\end{tikzcd}
\end{equation*}
This finishes the construction of the $\GL_n(F)$-action.

\red{Further observations: 
For $m' \leq m$, we have $A_{m'} = A_m^{K_{m'}}$.
We may, for $K \subset \GL_n(\cO_F)$ compact open and contained in $K_{m}$, define
$A_K = A_{m}^K$. This gives refined tower $\{\cM_K\}_K$. 
}
% subsubsection Group Actions on the Tower of Lubin--Tate Deformation Spaces (end)

\subsubsection{The Weil Descent Datum on the Deformation Space} % (fold)
\label{ssub:The Weil Descent Datum}
We now fix an embedding $\br E \inj \C_p$ and consider for integers $m \geq 0$ the
base change $\cM_{m, \cO_{\C_p}} = \cM \cotimes_{\cO_\br E} \cO_{\C_p}$. 
We recall the notion of Weil descent data and make use of this notion to describe
an action of the Weil group $\Weil_E$ on $\cM_{m, \cO_{\C_p}}$.

Let $\Phi \in \Gal(\br E / E)$ be the automorphism corresponding to the
$q$-th power Frobenius automorphism of the residue field of $E$. Given
an $\cO_\br E$-algebra $\cO_\br E \xto i R$, we write $\Phi^*R$ for the 
$\cO_\br E$-algebra with structure morphism $\cO_\br E \xto \Phi \cO_\br E \xto i R$.
The identity on $R$ yields a morphism $R \to \Phi^* R$, which preserves only the 
$\cO_E$-algebra structure. Note that $\Phi^* R$ admits the following equivalent
descriptions as $\cO_\br E$-algebra:
\begin{equation}\label{eq:PhiRDescription}
\begin{tikzcd}[ampersand replacement=\&]
	{\Phi^*R} \& {R \cotimes_{\cO_{\br E, \Phi^{-1}}}\cO_{\br E} \ } \& {\ R \cotimes_{\cO_{\br E}, \id} \Phi^*\cO_{\br E}} \\
	\& {\cO_{\br E}}
	\arrow["{r \mapsto r \otimes 1}", draw=none, from=1-1, to=1-2]
	\arrow["\sim"{description}, from=1-1, to=1-2]
	\arrow["{(r,x) \mapsto r \otimes\Phi(x)}", draw=none, from=1-2, to=1-3]
	\arrow["\sim"{description}, from=1-2, to=1-3]
	\arrow["{i \circ \Phi}", from=2-2, to=1-1]
	\arrow["{1 \otimes \id}"', from=2-2, to=1-2]
	\arrow["{1 \otimes \Phi}"', from=2-2, to=1-3]
\end{tikzcd}
\end{equation}

More generally, given any functor $\cG \colon \Adm{\cO_\br E} \to \Set$, we denote by
$\Phi^* \cG$ the fiber product $\cG \times_{\spf(\cO_\br E)} \spf(\Phi^*
\cO_\br E)$, interpreted as a functor $\Adm{\cO_\br E} \to \Set$ under the
equality of categories $\Adm {\Phi^*\cO_\br E} = \Adm {\cO_\br E}$. In
this situation, we have the notion of Weil descent data
(cf. \cite[Definition 3.45]{rapoport1996period}).

\begin{defi}[Weil Descent Datum]\label{def:WeilDescentDatum}
  A Weil Descent Datum for $\cG$ is an isomorphism
  \begin{equation*}
    \alpha \colon \cG \xto\sim \Phi^* \cG
  \end{equation*}
  of functors $\Adm{\cO_\br E} \to \Set$.
\end{defi}

To make this definition a bit more tangible, we give the following example.
\red{Is this example too boring?}
\begin{xpl}
  \begin{enumerate}
    \item Suppose that $p \neq 2$ and let $\ell \neq p$ be a prime number. Write 
  $\zeta_\ell \in \cO_{\br \Q_p}$ for an $\ell$-th root of unity. Let
  $\cP_{\zeta_\ell}$ be the functor parametrizing
  square roots of $\zeta_\ell$. One readily sees that $\cP_{\zeta_\ell}$ is
  representable by \red{In what category?}
  $\spf(R)$, where $R$ denotes the the $\cO_{\br \Q_p}$-algebra
  $R = \frac{\cO_{\br \Q_p}[X]}{(X^2 - \zeta_\ell)}$. 

  Now, $\Phi^*\cP_{\zeta_\ell}$ is readily seen to be the functor
  parametrizing square roots of $\Phi^{-1}(\zeta_\ell) = \zeta_\ell^{p^{-1}}$,
  where $p^{-1}$ denotes the inverse residue class of $p$ mod $\ell$. 
  Hence, a Weil descent datum for $\cP_{\zeta_\ell}$ is equvialent to a 
  $\cO_{\br \Q_p}$-linear isomorphism of rings
  \begin{equation*}
    \alpha \colon \Phi^* R = \frac{\cO_{\br \Q_p}[X]}{(X^2 - \zeta_\ell^{p^{-1}})} \to
    \frac{\cO_{\br \Q_p}[X]}{(X^2 - \zeta_\ell)} = R, 
  \end{equation*}
  One easily finds that any such isomorphism must send $X$ to $aX$, where $a$
  is a square root of $\zeta_\ell^{p^{-1}-1}$, and conversely any such square
  root yields a Weil descent datum.
    \item More generally, if $\cG = R$ is representable by the formal spectrum of a 
      ring formally of finite type over $\cO_{\br E}$, a Weil descent datum 
      for $\cG$ is an isomorphism of topological $R$-algebras
      \begin{equation*}
        \Phi^* R \cong \frac{ \cO_{\br E} [ X_1, \dots, X_n ]}{(f_1^{\Phi^{-1}},
        \dots, f_n^{\Phi^{-1}})}
        \to 
        \frac{ \cO_{\br E}[ X_1, \dots, X_n ]}{(f_1, \dots,
        f_n)} = R,
      \end{equation*}
      where $f_i^{\Phi^{-1}} \in \cO_\br E\llbr X_1, \dots, X_n\rrbr$ denotes the 
      power series obtained by applying $\Phi^{-1}$ to the coefficients of $f_i$.
      This follows quickly from the descriptions of $\Phi^*R$ in
      \eqref{eq:PhiRDescription}. A similar statement is true for admissible algebras
      formally of finite type over $\cO_\br E$.
  \end{enumerate}
\end{xpl}

An element $w \in \Weil_E$ with $w|_{\br E} = \Phi^m$ for $m \in \Z$, induces
a morphism
\begin{equation*}
  \cG \times_{\spf{\cO_{\br E}}} \spf(\cO_{\C_p}) \xto{(1,w)}
  \Phi^{m, *}\cG \times_{\spf{\cO_{\br E}}} \spf(\cO_{\C_p}).
\end{equation*}
If $\cG$ admits a Weil descent datum, we obtain a commuatative square 
of isomorphisms of functors $\FSchOver {\cO_E}^{\opp} \to \Set$
\begin{equation}\label{eq:WeilDescentActionCommSquare}
\begin{tikzcd}[ampersand replacement=\&]
	{\cG \times_{\cO_\br E} \cO_{\C_p}} \& {\Phi^{-m, *}\cG \times_{\cO_\br E} \cO_{\C_p}} \\
	{\Phi^{m, *}\cG \times_{\cO_\br E} \cO_{\C_p}} \& {\cG \times_{\cO_\br E} \cO_{\C_p}}.
	\arrow["{(\Phi^{-m, *} \alpha^{-m}, 1)}", from=1-1, to=1-2]
	\arrow["{(1, \spf(w))}"', from=1-1, to=2-1]
	\arrow["{(1,\spf(w))}", from=1-2, to=2-2]
	\arrow["{(\alpha^{-m}, 1)}", from=2-1, to=2-2]
\end{tikzcd}
\end{equation}
Here, $\alpha^{m}$ denotes the isomorphism 
$$\alpha^m: \cG \xto \alpha \Phi^{*} \cG \xto {\Phi^* \alpha} \Phi^{2, *}\cG
\xto{\Phi^{2,*}\alpha} \dots \xto{\Phi^{m-1,*}\alpha} \Phi^{m,*}\cG$$ 
obtained by iterating the isomorphism $\alpha$, and 
$\alpha^{-m}$ denotes the inverse of $\alpha^{m}$. 
We write $\delta_w$ for the automorphism of $\cG \times_{\cO_\br E} \cO_{\C_p}$
described in the square \eqref{eq:WeilDescentActionCommSquare}.
For $w_1, w_2 \in \Weil_E$, we have $\delta_{w_1 \circ w_2} = \delta_{w_1}
\circ \delta_{w_2}$,
so the assignment $w \mapsto \delta_{w^{-1}}$ defines a right-action of 
$\Weil_E$ on $\cG \times_{\spf(\cO_{\br E})} \cO_{\C_p}$. 

\begin{xpl}
  Again, suppose that the functor 
  $\cG$ is representable by the formal spectrum of some $\cO_{\br E}$ algebra $R$,
  and assume for simplicity that $R$ admits a finite presentation
  \begin{equation*}
    R = \frac{\cO_\br E \llbr X_1, \dots, X_n\rrbr}{(f_1, \dots, f_n)}.
  \end{equation*}
  Furthermore, assume that $\cG$ admits a Weil descent datum $\alpha$. 
  Then by the example above, the inverse of $\Phi^{-m, *} \alpha^{-m}$
  yields (upon choosing an embedding $\br E \to \C_p$) an isomorphism of
  $\cO_{\C_p}$-algebras
  \begin{equation*}
    \rho^m: \frac{\cO_{\C_p} \llbr X_1, \dots, X_n\rrbr}{(f_1^{\Phi^m},
    \dots, f_n^{\Phi^m})} \to 
    \frac{\cO_{\C_p} \llbr X_1, \dots, X_n\rrbr}{(f_1, \dots, f_n)},
  \end{equation*}
  uniquely determined by the images $\rho(X_i)$. 
  Let $w \in \Weil_E$ be an element satisfying $w|_{\br E} = \Phi^{m}$.
  Then $\delta_w$ corresponds on the level of global sections to an isomorphism
  \begin{equation*}
    R \cotimes_{\cO_\br E} \cO_{\C_p} = \frac{\cO_{\C_p} \llbr X_1, \dots,
    X_n\rrbr}{(f_1,
      \dots, f_n)} \xto w  \frac{\cO_{\C_p} \llbr X_1, \dots,
        X_n\rrbr}{(f_1^{\Phi^m}, \dots, f_n^{\Phi^m})} \xto {\rho^m}
        \frac{\cO_{\C_p} \llbr X_1, \dots, X_n\rrbr}{(f_1, \dots, f_n)} = R
        \cotimes_{\cO_\br E} {\cO_{\C_p}}.
  \end{equation*}
  That is, $\delta_w(a) = w(a)$ for $a \in \cO_{\C_p}$ and $\delta_w(X_i) =
  \rho^m(X_i)$ for $i= 1, \dots, n$. The commutativity of the square
  \eqref{eq:WeilDescentActionCommSquare}
  reflects the fact that $w(\rho^m(X_i)) = \Phi^{m}(\rho^m(X_i))$. Also note that 
  $\delta_w$ does not respect the $\cO_{\C_p}$- or 
  $\cO_{\br E}$-structure, it is only an isomorphism of $\cO_E$-algebras.
\end{xpl}

We define a Weil descent datum $\alpha$ on the functor $\cM_m$ as follows. 
Given $R\in \cC$ and $(H, \iota, \phi) \in \cM_m(R)$, we put
\begin{equation*}
  \alpha(R) \left( H, \iota, \phi \right) = (H', \iota',\phi'),
\end{equation*}
where
\begin{itemize}
  \item the formal group $H'$ is equal to $H$, however interpreted as a formal
    module over $\Phi^*R$. That is, $H' = H \otimes_{R} \Phi^* R = \Phi^* H$.
  \item the quasi-isogeny $\iota'$ is defined as
    \begin{equation*}
      H_0 \overset{\Frob_q^{-1}}{\dashrightarrow} H_0^{(-q)} = H_0
      \otimes_{\bar \FF_q, \Frob^{-1}}
      \bar \FF_q \dashrightarrow H \otimes_{R, \Phi^{-1}} (R/\fm_R) = (\Phi^*
      H) \otimes_R \bar \FF_q.
    \end{equation*}
    Here $\Frob_q^{-1}$ denotes the inverse quasi-isogeny of the relative
    Frobenius morphism $\Frob_q: H^{(-q)} \to H$. In particular, 
    $\height(\iota') = \height(\iota)-1$.
  \item the level structure $\phi'$ is
    \begin{equation*}
      \underline{\left({\varpi^{-m}\cO_E}/{\cO_E} \right)}^m_{\spf R}
      \cong \Phi^* \underline{\left({\varpi^{-m}\cO_E}/{\cO_E} \right)}^m_{\spf R}
      \xto{\Phi^* \phi} \Phi^* H.
    \end{equation*}
\end{itemize}

As explained above, this yields a right action of $\Weil_E$ on $\cM_{m, \cO_{\C_p}}$.
Given $w \in \Weil_E$ such that $w|_\br E = \Phi^m$ for $m \in \Z$, we find
that $\delta_w$ restricts to an isomorphism $\cM_{m, \cO_{\C_p}}^{(j)} \to \cM_{m,
\cO_{\C_p}}^{(j+m)}$. 

\begin{xpl}
  Suppose that $E'/\br E$ is a finite extension. Then $R = \cO_{E'} \in \cC$, 
  and we may describe the action of $\Weil_E$ on 
  $\cM_{m, \cO_{\C_p}}(R)$ as follows. Given (an equivalence class of)
  a triple $(\cH, \iota, \phi) \in \cM_m(R)$ such that 
  $\cH = \FGG(H)$, and an element $w \in \Weil_E$ such that 
  $w|_{\br E} = \Phi^m$ for $m \in \Z$, we find 
  $$\delta_w(H, \iota, \phi) = (\FGG(H^w), \iota^w \circ \Frob_q^m, \phi^w),$$ 
  where $H^w$ is the formal group law over $E'$ obtained by applying 
  $w|_{E'}$ to $H$ coefficient-wise. Likewise, 
  $$\iota^w: H_0^{(q^m)} \dashrightarrow H^w \otimes \bar \FF_q
  \quad \text{and} \quad
  \phi^w:  \underline{\left({\varpi^{-m}\cO_E}/{\cO_E} \right)}^m_{\spf R}
      \to H^w$$
  denote the corresponding quasi-isogeny and level structure.
\end{xpl}
% subsection The Lubin-Tate Tower (end)


% (end)
\subsection{The Local Langlands Correspondence for the General Linear Group} % (fold)
\label{sub:The Local Langlands Correspondence for the General Linear Group}
The aim of this section is to review some of the results in \cite{HTShimura}. 
Recall from the previous section the tower of deformation spaces 
$\{\cM_K\}_{K \subset \GL_n(\cO_F)}$. Let $M_K \in \RigOver {\br F}$ denote the
(rigid) generic fiber of $\cM_K \in \FSchOver {\cO_{\br F}}$\todo{reference
(Berthelot? Reference )}. Fix an algebraic closure $\C_p$ of $F$ and a prime number $\ell
\neq p$. This section is concerned with the representation-theoretic aspects of
the vector space
\begin{equation*}
  \HLT \coloneqq \lim_K H^{n-1}_c (M_{K} \otimes_{\breve F} \C_p, \bar \Q_l).
\end{equation*}

By functoriality of the generic fiber functor, the action of $D^\times \times
\GL_n(F)$ on the tower $\{\cM_K\}_{K \subset \GL_n(\cO_F)}$ from the right
yields an action of $D^\times \times \GL_n(F)$ on $\HLT$ from the left. 

For $K \subset \GL_n(\cO_F)$, write $M_{K, \varpi^\Z}$ for the quotient of 
$M_K$  by the action of the subgroup
$\varpi^\Z \subset D^\times$. Writing $\cM_m = \coprod_{\delta \in \Z} \cM_m^{(\delta)}$
induces $M_K = \coprod_{\delta \in\Z} M_K^{(\delta)}$, and 
the action of $\varpi$ induces for any $\delta \in \Z$ an isomorphism 
$M_K^{(\delta)} \cong M_K^{(\delta +n)}$. Hence, $M_{K, \varpi^\Z}$ is 
isomorphic to $\coprod_{0 \leq \delta \leq n-1} M_{K}^{(\delta)}$. 

Let $l \neq p$ be a prime number and fix an isomorphism $\bar \Q_l \cong \C$.

\begin{defi}[Cohomology of the Lubin--Tate tower]
  We write $\HLT = \lim_{K} H^{n-1}_c (M_{K, \varpi^\Z} \otimes_{\breve F} C, \bar \Q_l)$.
\end{defi}

\begin{thm}[Non-Abelian Lubin--Tate theory]
  Let $\pi$ be an irreducible supercuspidal representation of $\GL_n(F)$ whose central
  character is trivial on $\varpi^\Z$. We write $\rec_F(\pi)$ for the irreducible 
  smooth representation of $W_F$ corresponding to $\pi$ undet the local Langlands
  correspondence, and $\JL(\pi)$ for the irreducible smooth representation of
  $D^\times$ corresponding to $\pi$ under the local Jacquet-Langlands correspondence.
  Then we have 
  \begin{equation*}
    \HLT[\pi^\vee] = \pi^\vee \boxtimes \JL(\pi) \boxtimes \rec_F(\pi)(\tfrac{1-n}2)
  \end{equation*}
  as representations of $\GL_n(F) \times D^\times \times W_F$. 
\begin{proof}
\end{proof}
\end{thm}


We set 
\begin{equation} \label{eq:DefHLTHprLT}
  \begin{gathered}
  \HLT \coloneqq \lim_K H^{n-1}_c(M_{K, \varpi^\Z} \otimes_{\breve F} C, \bar Q_l)
  \\ \text{and} \\
  \HLT' \coloneqq \lim_K H^{n-1}_c(M_{K}^{(0)} \otimes_{\breve F} C, \bar Q_l).
  \end{gathered}
\end{equation}
Also, we set 
\begin{equation}\label{eq:DefGandG1}
\begin{gathered} 
  G \coloneqq \GL_n(F) \times D^\times/\varpi^\Z \times W_F \\  \text{ and } \\
  G^1 \coloneqq \{(g,d,\sigma) \in \GL_n(F) \times D^\times \times W_F \mid 
  \det(g)^{-1} \Nrd(d) \Art_F^{-1}(\sigma) = 1\}.
\end{gathered}
\end{equation}
\begin{lem}\label{lem:G1subG}
  The natural map $G^1 \to G$ is injective and realizes $G^1$ as a co-compact closed
  normal subgroup of $G$.
\begin{proof}
  The morphism $G^1 \to G$ is clearly injective, Further, the image of the natural
  homomorphism is isomorphic to the kernel of the map 
  $\nu: G \to F^\times/\varpi^{n\Z}$, given by $\nu(g, \bar d, \sigma) = 
  \bar{\det(g)^{-1}\Nrd(d)\Art_F^{-1}(\sigma)}$. The claim follows.
\end{proof}
\end{lem}

We have actions $G \lacts \HLT$ and $G^1 \lacts \HLT'$. \red{Again, this uses
Weil-Descent Data; make this precise.}
\begin{thm}[Non-Abelian Lubin-Tate Theory]\label{thm:NonAbLTT}
  Let $\pi$ be a irreducible supercuspidal representation of $\GL_n$ whose cetral 
  character is trivial on $\varpi^\Z$. Then, as representations of 
  $\GL_n(F) \times D^\times \times \Weil_F$, the $\pi^\vee$-supercuspidal part
  of $\HLT$ has the form
  \begin{equation} \label{eq:DefRecJL}
    {\HLT}_{,\pi^\vee} = \pi^\vee \boxtimes \JL(\pi) \boxtimes
                  \rec_F(\pi)(\tfrac {1-n}2),
  \end{equation}
  $\JL(\pi)$ is a representation of $D^\times$ and $\rec_F(\pi)$ is a representation of 
  $W_F$. The assignments $\pi \mapsto \JL(\pi)$ and 
  $\pi \mapsto \rec_F(\pi)$ satisfy the conditions imposed on the 
  Jacquet--Langlands and local Langlands correspondeces for $\GL_n$.
\end{thm}

\begin{lem}\label{lem:GActsSmoothlyOnHLT}
  These actions are smooth.
  \begin{proof}
    \red{TODO}
  \end{proof}
\end{lem}

\begin{lem}\label{lem:InductionStatementOnHLT}
  The $G$-representation $\cInd_{G^1}^G (H'_\LT)$ is isomorphic to $H_\LT$. 
\begin{proof}
    \red{TODO}
\end{proof}
\end{lem}

% subsection The Local Langlands Correspondence for the General Linear Group (end)

% section Non-Abelian Lubin-Tate Theory: An Overview (end)

\end{document}
