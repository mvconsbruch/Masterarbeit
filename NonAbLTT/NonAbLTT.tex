%! TeX root: ../main.tex
\documentclass[../main.tex]{subfiles}

\begin{document}
\section{Non-Abelian Lubin-Tate Theory: An Overview} % (fold)
\label{sec:Non-Abelian Lubin-Tate Theory: An Overview}
In the preceeding chapter we used formal $\cO_E$-modules to understand the 
maximial abelian extension of a local field $E$. The hope of non-Abelian Lubin-Tate
theory is to gain insight about the structure of non-Abelian extensions of $E$
by considering certain moduli spaces of formal $\cO_E$-modules.
More precisely, attached to a formal $\cO_E$-module $H_0$ over $\Fqbar$
(determined
up to isomorphism by its height $n$), we attach a system of rigid spaces 
$\{M_K\}_{K \subset \GL_n(\cO_E)}$, the so called Lubin-Tate Tower. For $l \neq p$, 
the system of $l$-adic compactly supported cohomology groups $\{\hHc^{i}(M_K,
\bar \Q_l)\}_K$ admits commuting actions by $\GL_n(E)$, $\Weil_E$ and $D^\times$,
where the latter denotes the units of the central divison algebra $D =
\End_{\FMOver {\cO_E}{\Fqbar}}(H_0) \otimes \Q$. This yields a correspondence of 
representations of the respective groups, and Harris and Taylor showed in
\cite{HTShimura} that the cohomology of middle degree induces (a version of)
the Local Langlands Correspondence for $\GL_n(E)$. Our goal is an explicit
description of (a part of) this correspondence, and we obtain such a
description by understanding (a part of) the Lubin--Tate tower explicitely.

\subsection{Lubin--Tate Deformation Spaces} % (fold)
\label{sub:Lubin-Tate Deformation Spaces}
Let $\XX \in \FMLOver {\cO_E}\Fqbar$ be the reduction of the standard formal
$\cO_E$-module law $H$ of height $n$ (cf. \cref{sub:Hazewinkels FuncEq and the
Standard Formal Module}).
The aim of this section is to constuct a certain moduli problem $\cM_{H_0, m}$,
parametrizing deformations of $\XX$ with Drinfeld level $\varpi^m$-structure.
By results of Drinfeld \cite{drinfel1974elliptic}, these moduli problems turn
out to be representable by formal schemes.

\subsubsection{The Tower of Deformation Spaces} % (fold)
\label{ssub:The Tower of Deformation Spaces}
We mostly follow \cite[Chapter 2]{Strauch2008DefSp} in the following
exposition. 
 We work with the following variant of the deformation 
functor introduced in \cref{sub:Deformations of Formal Modules}.

We define a functor $\cM_0\colon \cC \to \Set$ on components $R \in \cC$ via
\begin{equation*}
  \cM_0(R) = \{(\cF, \iota) \mid \cF \in \FMOver {\cO_E}R \text { and }
  \iota\colon \FGG(\XX) \to \cF \otimes_R \Fqbar \text{ a quasi-isogeny}\}/\sim,
\end{equation*}
where we say that two pairs $(\cF, \iota)$ and $(\cF', \iota')$ are isomorphic
if and only if there is an isomorphism $\alpha \colon \cF \to \cF'$,
such that $\iota' = \alpha_0 \circ \iota$, where $\alpha_0$ denotes the reduction
of $\alpha$. Note that, as all formal $\cO_E$-module laws of the same height
are isomorphic, this functor only depends on the height $n$.

We have a stratification
\begin{equation*}
  \cM_0 = \coprod_{j \in \Z} \cM_0^{(j)},
\end{equation*}
where $\cM_0^{(j)}$ parametrizes deformations $(\cF, \iota)$ with $\height(\iota) = j$.
Any quasi-isogeny $\tau\colon \XX \dashrightarrow \XX$ yields an isomorphism, 
functorial in $R$,
\begin{equation*}
  \cM_0^{(j)}(R) \to \cM_0^{(j + \height(\tau))}(R), \quad
  (\cF, \iota) \mapsto (\cF, \iota \circ \tau). 
\end{equation*}
In particular, for any two integers $j, j' \in \Z$, the two spaces
$\cM_0^{(j)}$ and $\cM_0^{(j')}$ are isomorphic. Hence, by
\cref{thm:RepresentabilityOfDefSpaceWOLevel}, $\cM_0^{(j)}$ is representable by 
a local $\cO_\br E$-algebra $A^{(j)}_0 \in \cC$, non-canonically isomorphic 
to $\cO_\br E \llbr u_1, \dots, u_{n-1} \rrbr$.

We next introduce variants of this moduli problem with a certain level structure.

\begin{defi}[Drinfeld level $\varpi^m$-structure]
  Let $\cF \in \FMOver {\cO_E}{R}$ be a $\varpi$-divisible formal $\cO_E$-module
  of height $n>0$ and let $m$ be a non-negative 
  integer. A Drinfeld level $\varpi^m$-structure on $\cF$ is a morphism of
  $\cO_E$-modules 
  \begin{equation*}
    \phi\colon {(\varpi^{-m}\cO_E/\cO_E)} \to \cF(R)
  \end{equation*}
  such that after choosing a coordinate $\cF \cong \spf R\llbr T \rrbr$, the 
  power series $[\varpi]_H(T) \in R\llbr T \rrbr$ satisfies the divisibility
  constraint
  \begin{equation*}
    \prod_{x \in (\varpi^{-1}\cO_E/\cO_E)^n}(T - \phi(x)) \ \bigg\vert \ [\varpi]_H(T).
  \end{equation*}
\end{defi}

\begin{defi}[Lubin--Tate Deformation Space with Level Structure]\label{def:LTDefSpaceWithLevel}
  Let $\cM_m: \cC \to \Set$ be the functor assigning to $R \in \cC$ the set 
  \begin{equation*}
    \cM_m(R) \coloneqq \{(\cF, \iota, \phi) \mid (\cF, \iota) \in \cM_0(R) \text{ and }\phi
    \text{ a Drinfeld level $\varpi^m$-structure on $\cF$}\}/\simeq.
  \end{equation*}
\end{defi}

Just as in the case without level, any two functors $\cM_m^{(j)}$ and
$\cM_m^{(j')}$ (non-canonically) isomorphic.
Furthermore, for non-negative integers $m' \leq m$, we have
natural morphisms
$\cM_m \to \cM_{m'}$ by restricting the level structure. 
By results of Drinfeld, each functor $\cM_m^{(0)}$ is representable.

\begin{thm}[Representability of the Lubin--Tate Deformation Space with Level Structure]\label{thm:RepresentabilityOfDefSpaceWithLevel}
  The functor $\cM_m^{(0)}$ is representable by a regular local ring $A_m \in \cC$ of 
  dimension $m-1$.
\begin{proof}
  This is \cite[Proposition 4.3]{drinfel1974elliptic}.
\end{proof}
\end{thm}

Our main interest will
not lie in the individual spaces $\cM_m$, but rather in the resulting tower
$\{\cM_m\}_{m \geq 0}$, which we call the tower of Lubin--Tate deformation
spaces.

% subsubsection The Tower of Deformation Spaces (end)

\subsubsection{The Case of Height One} % (fold)
\label{ssub:The Case of Height One}
Let $\hat E^\ab$ denote the closure of the maximal abelian extension of $E$ inside
$\Cp$, and let $E_m^\LT \coloneq E_{\varpi, m}\br E \subset E^\ab$ be the composition
of $\br E$ with the totally ramified extension constructed in \cref{sec:Local
Class Field Theory}. As all Lubin--Tate formal $\cO_E$-modules of height one
are isomorphic (this follows either by \cref{thm:RepresentabilityOfDefSpaceWOLevel} 
or directly by \cite[Lemma 2]{LubinTateFormalMult}), this extension is independent
of $\varpi$. 

Suppose that $\XX \in \FMLOver {\cO_E}\Fqbar$ is the reduction of the standard
$\cO_E$-module law $H$ over $\Fqbar$ of height one. 
By \cref{thm:RepresentabilityOfDefSpaceWOLevel}, the corresponding deformation functor
$\cM_0^{(0)}$ is isomorphic to $\cO_\br E$, so for any $R \in \cC$, all deformations
$(\cF, \iota)$ of $\XX$ to $R$ differ by a (unique) isomorphism. We may therefore
choose a coordinate on
$\cF$ that gives rise to the unique Lubin--Tate module law $F$ with
$[\varpi]_F(T) = \varpi T + T^q$. Now, a morphism 
\begin{equation*}
  \phi \colon \varpi^{-m}\cO_E / \cO_E \to \fm_R = F(R)
\end{equation*}
constitutes a Drinfeld $\varpi^m$ level structure if and only if 
$\phi$ maps the generator $e = \varpi^{-m} \in \varpi^{-m} \cO_E/\cO_E$ 
to a root of the Eisenstein polyonomial 
\begin{equation*}
  \eta_1([\varpi^{m-1}](T)) = \frac{[\varpi^m](T)}{[\varpi^{m-1}](T)} = \eta_m(T)
\end{equation*}
(cf. \cref{sec:Local Class Field Theory}). 
Hence, we find that 
\begin{equation}\label{eq:HeightOneDefSpaceIsOEm}
  \cM_{m}^{(0)} = \frac{\cO_\br E[T]}{\eta_m(T)} \cong \cO_{E_m^\LT},
\end{equation}
where the choice of an isomorphism is equivalent to the choice of a level structure 
on $F(\cO_{E_m^\LT})$ by \cref{lem:PropertiesOfFiniteLTExt}. 

Now let $n \geq 1$ and let $\XX$ be the reduction of the standard formal 
$\cO_E$-module of height $n$. Let $\wedge \XX$ be the determinant of $\XX$,
that is, the formal $\cO_E$-module with $\Dio(\XX) = \wedge^n \Dio(\XX)$, where
$\Dio$ denotes the Dieudonn\'e-module, cf. \cref{sub:Explicit Dieudonne Theory}.
Write $\cM_m^{(0)}$ for the deformation space of $\XX$ with Drinfeld $\varpi^m$
level structure
and $\cM_{m, \wedge}^{(0)}$ for the deformation space of $\wedge \XX$. 

Let $R \in \cC$ be arbitrary and let $(\cF, \iota) \in \cM_0^{(0)}(R)$. 
Recall the construction of determinants of formal $\cO_E$-modules
from \cref{sub:Determinants}. We have the following result.

\begin{lem}\label{lem:DeterminantOfDrinfeldStructure}
  We have $(\wedge^n \cF, \wedge^n \iota) \in \cM_{0, \wedge}^{(0)}(R)$.
  Moreover,
  let $\phi\colon \varpi^{-m} \cO_E^n/\cO_E^n \to  \cF^n(R)$ be a Drinfeld level
  $\varpi^m$ structure on $\cF$. Then 
  \begin{equation*}
    \wedge^n \phi \colon \varpi^{-m} \cO_E/\cO_E \to \wedge^n\cF(R), \quad 
  \varpi^{-m} \mapsto \delta_m(\varpi^{-m} e_1, \dots, \varpi^{-m} e_n)
  \end{equation*}
  is a Drinfeld level $\varpi^m$ structure on $\wedge^n \cF$.
\begin{proof}
  This is \cite[Proposition 2.11]{weinstein2016semistable}.
\end{proof}
\end{lem}

In particular, we obtain a natural map
\begin{equation*}
  \cM_m^{(0)}(R) \to \cM^{(0)}_{m, \wedge}(R), \quad (\cF, \iota, \phi)
  \mapsto (\wedge^n \cF, \wedge^n \iota, \wedge^n \phi).
\end{equation*}
In particular, upon choosing a Drinfeld level $\varpi^m$ structure on 
$F(\cO_{E^\LT_m})$, this provides a morphism $\cM_m^{(0)} \to \cO_{E^{\LT}_m}$. 

% subsubsection The Case of Height One (end)

\subsubsection{Group Actions on the Tower of Lubin--Tate Deformation Spaces} % (fold)
\label{ssub:Group Actions on the Tower of Lubin--Tate Deformation Spaces}
We describe actions of $\GL_n(E) \times D^\times$ on the tower
$\{\cM_m\}_{m \geq 0}$. More precisely, given an element $d \in D^\times$ and an element
$g \in \GL_n(E)$, we construct, for sufficiently large $m\geq 0$, morphisms
\begin{equation*}
  d_m \colon \cM_m^{(j)} \to \cM_m^{(j')} \quad \text{and} \quad
  g_{m,m''} \colon \cM_m^{(j)} \to \cM_{m''}^{(j'')},
\end{equation*}
where $j'= j+\val_\varpi(\Nrd(d))$, $j'' = j - \val_\varpi(\det g)$ and $m''
= m-d$ is an integer differeing from $m$ by an integer depending on $g$. 

The action of $D^\times$ is easy to describe. Given $R \in \cC$ and 
$d \in D^\times$, we put
\begin{equation*}
  (\cF, \iota, \phi).d = (\cF, \iota \circ d, \phi).
\end{equation*}

The group $\GL_n(E)$ acts in a less simple matter. Akin to the action of
$D^\times$, we would like to define the action as
$(\cF, \iota, \phi).g = (\cF, \iota, \phi \circ g)$, but this only makes sense
if $g \in \GL_n(\cO_E)$. To extend this action to all of $\GL_n(E)$, we
allow ourselves to also change the underlying formal group.
We need the following constructions.

\begin{defi}[Quotients with respect to Level Structure]\label{def:QuotientModule}
  Let $R \in \cC$ and let $(\cF, \iota, \phi) \in \cM_m(R)$. Let
  $P \subset (\varpi^{-m}\cO_E/\cO_E)^n$ be a submodule.
  We define the quotient $(\cF/\phi(P))$ as follows. Let 
  $(\cF^\univ, \iota^\univ, \phi^{\univ}) \in \cM_m(A_m)$ be the universal 
  triple and let $\alpha\colon A_m \to R$ be the morphism giving rise to
  $(\cF, \iota, \phi)$. Then $\phi^\univ(P) \subset \cF^\univ(A_m)$ is a 
  finite subset. As $A_m$ satisfies all the conditions imposed in
  \cref{sub:Quotients of Formal Modules by Finite Submodules}, we may take the quotient
  $(\cF^\univ/\phi^\univ(P))$ as in \cref{thm:Quotients}. We define
  $(\cF/\phi(P))$ as $(\cF^\univ/\phi^\univ(P))\otimes_{A_m,\alpha} R$. 
\end{defi}

\begin{lem}\label{lem:PropertyOfQuotient}
  If $\# P = q^c$, the induced morphism $\cF \to (\cF/\phi(P))$ of formal
  module laws over $R$ is of height $q^c$.
\begin{proof}
  It suffices to check this for the universal triple, where it is part 2 of 
  \cref{thm:Quotients}.
\end{proof}
\end{lem}

\begin{lem}\label{lem:DrinfeldLevelOnQuotients}
  As above, let $R \in \cC$, let $(\cF, \iota, \phi) \in \cM_m(R)$ and let 
  $P \subseteq (\varpi^{-m}\cO_E/\cO_E)^n$ be a submodule.   
  \begin{enumerate}
    \item There is a unique natural morphism of $\cO_E$-modules
      \begin{equation*}
        \bar \phi \colon \frac{(\varpi^{-m} \cO_E/\cO_E)^n}{P} \to (\cF/\phi(P))(R)
      \end{equation*}
      compatible with the maps $(\varpi^{-m}\cO_E/\cO_E)^n \to
      \frac{(\varpi^{-m} \cO_E/\cO_E)^n}{P}$ and $\cF \to (\cF/\phi(P))$. 
    \item Suppose that  there is an injection 
  \begin{equation*}
    \left(\varpi^{-m'} \cO_E/\cO_E \right)^n \to \frac{(\varpi^{-m}
    \cO_E/\cO_E)^n}{P}.
  \end{equation*}
  Then, the induced morphism 
  \begin{equation*}
  \phi'\colon ({\varpi^{-m'} \cO_E}/{\cO_E})^n  \to 
  (\cF/\phi(P))(R)
  \end{equation*}
  is a Drinfeld $\varpi^m$-level structure.
  \end{enumerate}
\end{lem}
\begin{proof}
  This is \cite[Proposition 4.4]{drinfel1974elliptic}. Again it suffices assume 
  that $R = A_m$ and that $\cF$ is the universal formal module with level
  $\varpi^m$ structure. We may choose a coordinate $\cF \cong \FGG(F)$. The first
  statement follows as over $A_m$, we have $f_{\phi(P)}(\phi(a)) = 0$ for $a
  \in P$, where $f_{\phi(P)}$ is the power series arising in the construction of 
  $(F/\phi(P))$, cf. \cref{thm:Quotients}. The second claim follows as the 
  morphism $(\varpi^{-m'} \cO_E^n/\cO_E^n) \to \fm_{A_m}$, induced by 
  $\phi'$, is injective.
\end{proof}

This allows us to construct the $\GL_n(E)$-action. Assume that $g \in \GL_n(E)$ is such that $g^{-1} \in \Mat_{n\times
n}(\cO_E)$ and $g \in \varpi^{-d} \Mat_{n \times n}(\cO_E)$ for some non-negative
integer $d$. In this case, we construct for all integers $m \geq d$ a natural
transformation
\begin{equation*}
  g_{m, m-d}: \cM_m \to \cM_{m-d}.
\end{equation*}
Note that $g \cO_E^n \subset \varpi^{-d} \cO_E^n$, and that multiplication with
$g$ yields an injection
\begin{equation*}
  (\varpi^{m-d}\cO_E/\cO_E)^n \xto g (\varpi^{-m}\cO_E^n/g\cO_E^n) = 
  \frac{(\varpi^{-m} \cO_E^n /\cO_E^n)}{(g \cO_E^n/\cO_E^n)}.
\end{equation*}
Now, given a tuple $(\cF, \iota, \phi) \in \cM_m(R)$, we put $$(\cF, \iota, \phi).g = (\cF', \iota', \phi'),$$ where 
$$\cF'= \cF/\phi(g\cO_E^n/\cO_E^n)$$ 
is a the quotient of $\cF$ as in Definition \ref{def:QuotientModule}, 
$$\iota'\colon \cF_0 \to \cF \otimes \bar \FF_q \to \cF' \otimes \bar \FF_q$$ 
is the corresponding quasi-isogeny of height 
$(\height(\iota) - \val_\varpi( \det g))$, and 
$$\phi'\colon (\varpi^{m-d}\cO_E/\cO_E)^n \to \cF'$$ 
is the Drinfeld $\varpi^{m-d}$-level structure obtained by Lemma
\ref{lem:DrinfeldLevelOnQuotients}. 
For varying choices of $d$, this gives a system of maps compatible with the 
transition functions $\cM_m \to \cM_{m'}$. Indeed, given integers $m \geq d'
\geq d$ with $d$ as above, the triangle 
\begin{equation*}
\begin{tikzcd}[ampersand replacement=\&]
	{\cM_{m}} \& {\cM_{m-d}} \\
	\& {\cM_{m-d'}}
	\arrow["{g_{m,m-d}}", from=1-1, to=1-2]
	\arrow["{g_{m,m-d'}}"', from=1-1, to=2-2]
	\arrow[from=1-2, to=2-2]
\end{tikzcd}
\end{equation*}
commutes.

If $g \in \GL_n(E)$ is an arbitrary element, we may choose an integer $r$ such
that $(\varpi^{-r} g)^{-1} \in \Mat_{n \times n}(\cO_E)$. We now pick 
$d > 0$ in a way that $\varpi^{-r}g \in \varpi^{-d}\Mat_{n \times n}(\cO_E)$. 
Now, for $m \geq d$, we obtain natural transformations
\begin{equation*}
  g_{m,m-d}\colon \cM_{m} \to \cM_{m-d}, \quad (\cF, \iota, \phi).g = (\cF, \iota \circ \varpi^{-r}, \phi).(\varpi^{-r}g).
\end{equation*}
By the same reason as above, this yields, for varying choices of $d$, a
compatible system of natural transformations.
This construction is independent of the choice of $r$, as $(\cF, \iota,
\phi).(\varpi\cdot\id) \simeq (\cF, \iota \circ [\varpi^{-1}], \phi)$.
Furthermore, given $g, g' \in \GL_n(E)$ with suitable choices of integers 
$d, d'$, we obtain a commutative triangle
\begin{equation*}
\begin{tikzcd}[ampersand replacement=\&]
	{\cM_{m}} \& {\cM_{m-d}} \\
	\& {\cM_{m-d-d'}}
	\arrow["{g_{m,m-d}}", from=1-1, to=1-2]
	\arrow["{(gg')_{m,m-d-d'}}"', from=1-1, to=2-2]
	\arrow["{g'_{m-d,m-d-d'}}", from=1-2, to=2-2]
\end{tikzcd}
\end{equation*}
This finishes the construction of the $\GL_n(E)$-action.


We will need to consider a refinement of the tower $\{\cM_m\}_{m \in \N}$. We
define the 
$m$-th congruence subgroups
\begin{equation}\label{eq:defCongruenceSubgroup}
  K_m = \ker(\GL_n(\cO_E) \to \GL_n(\cO_E/(\varpi^m))).
\end{equation}
For any compact open subgroup $K \subset \GL_n(\cO_E)$, we may choose integers
$m > m'$ such that $K_m \subseteq K \subseteq K_{m'}$. We define 
\begin{equation*}
  A_K^{(j)} \coloneq (A^{(j)}_m)^K, \quad \cM_K^{(j)} \coloneq \Spf(A_K^{(j)}), \quad
  \cM_K = \coprod_{j \in \Z} \cM_K^{(j)}.
\end{equation*}
By \cite[Proposition 2.2.5]{Strauch2008DefSp}, these constructions are well-defined
and yield the refined tower of deformation spaces 
$\{\cM_K\}_{K \subset \GL_n(\cO_E)}$, where $K$ ranges over 
the compact open subgroups of $\GL_n(\cO_E)$. Any ring
$A_K^{(j)}$ is local, noetherian, integrally closed and finite over
$A_{m'}^{(j)}$.
Furthermore,
$A_{K}^{(j)}[\tfrac 1 \varpi]$ is \'etale over $A_{m}^{(j)}[\tfrac 1 \varpi]$, 
and it is Galois with Galois group $K/K_m$. Conversely, $A_{m}^{(j)}[\tfrac 1 \varpi]$
is \'etale over $A_K^{(j)}$, and if $K$ is normal in $K_{m'}$ then it is also
Galois with Galois group $K_{m'}/K$. 

% subsubsection Group Actions on the Tower of Lubin--Tate Deformation Spaces (end)

\subsubsection{The Weil Descent Datum on the Deformation Space} % (fold)
\label{ssub:The Weil Descent Datum}
We now fix an algebraic closure
$\bar E/E$ and a finite Galois extension $E'/\br E$, identified inside $\bar E$. 
For integers $m \geq 0$, we consider the base change $\cM_{m, \cO_{{E'}}} = \cM
\cotimes_{\cO_\br E} \cO_{{E'}}$. 
We recall the notion of Weil descent data and make use of this notion to describe
an action of the Weil group $\Weil_E$ on $\cM_{m, \cO_{{E'}}}$. In the 
limit over all finite extensions of $E$, this yields an action of $\Weil_E$ on the
formal scheme $\cM_{m, \cO_{{\Cp}}}$.

Let $\Phi \in \Gal(E^\nr / E)$ be the automorphism corresponding to the
$q$-th power Frobenius automorphism on the residue field $\Fqbar$ of $E^\nr$. Given
an $\cO_\br E$-algebra $\cO_\br E \xto i R$, we write $\Phi^*R$ for the 
$\cO_\br E$-algebra with structure morphism $\cO_\br E \xto \Phi \cO_\br E \xto i R$.
The identity on $R$ yields a morphism $R \to \Phi^* R$, which preserves only the 
$\cO_E$-algebra structure. Note that $\Phi^* R$ admits the following equivalent
descriptions as $\cO_\br E$-algebra:
\begin{equation}\label{eq:PhiRDescription}
\begin{tikzcd}[ampersand replacement=\&]
	{\Phi^*R} \& {R \cotimes_{\cO_{\br E, \Phi^{-1}}}\cO_{\br E} \ } \& {\ R \cotimes_{\cO_{\br E}, \id} \Phi^*\cO_{\br E}} \\
	\& {\cO_{\br E}}
	\arrow["{r \mapsto r \otimes 1}", draw=none, from=1-1, to=1-2]
	\arrow["\sim"{description}, from=1-1, to=1-2]
	\arrow["{(r,x) \mapsto r \otimes\Phi(x)}", draw=none, from=1-2, to=1-3]
	\arrow["\sim"{description}, from=1-2, to=1-3]
	\arrow["{i \circ \Phi}", from=2-2, to=1-1]
	\arrow["{1 \otimes \id}"', from=2-2, to=1-2]
	\arrow["{1 \otimes \Phi}"', from=2-2, to=1-3]
\end{tikzcd}
\end{equation}

More generally, given any functor $\cG \colon \FSchOver{\cO_\br E}^\opp \to
\Set$, we denote by
$\Phi^* \cG$ the fiber product $\cG \times_{\spf(\cO_\br E)} \spf(\Phi^*
\cO_\br E)$, which is again a functor $\FSchOver{\cO_\br E}^\opp \to \Set$.
In this situation, we have the notion of Weil descent data
(cf. \cite[Definition 3.45]{rapoport1996period}).

\begin{defi}[Weil Descent Datum]\label{def:WeilDescentDatum}
  A Weil Descent Datum for $\cG$ is an isomorphism
  \begin{equation*}
    \alpha \colon \cG \xto\sim \Phi^* \cG
  \end{equation*}
  of functors $\Adm{\cO_\br E} \to \Set$.
\end{defi}

To make this definition a bit more tangible, we give the following example.

\begin{xpl}
  \begin{enumerate}
    \item Suppose that $p \neq 2$ and let $\ell \neq p$ be a prime number. Write 
  $\zeta_\ell \in \cO_{\br \Q_p}$ for an $\ell$-th root of unity. Let
  $\cP_{\zeta_\ell}$ be the functor parametrizing
  square roots of $\zeta_\ell$. One readily sees that $\cP_{\zeta_\ell}$ is
  representable by \red{In what category?}
  $\spf(R)$, where $R$ denotes the the $\cO_{\br \Q_p}$-algebra
  $R = \frac{\cO_{\br \Q_p}[X]}{(X^2 - \zeta_\ell)}$. 

  Now, $\Phi^*\cP_{\zeta_\ell}$ is the functor
  parametrizing square roots of $\Phi^{-1}(\zeta_\ell) = \zeta_\ell^{p^{-1}}$,
  where $p^{-1}$ denotes the inverse residue class of $p$ mod $\ell$. 
  Hence, a Weil descent datum for $\cP_{\zeta_\ell}$ is equvialent to a 
  $\cO_{\br \Q_p}$-linear isomorphism of rings
  \begin{equation*}
    \alpha \colon \Phi^* R = \frac{\cO_{\br \Q_p}[X]}{(X^2 - \zeta_\ell^{p^{-1}})} \to
    \frac{\cO_{\br \Q_p}[X]}{(X^2 - \zeta_\ell)} = R, 
  \end{equation*}
  One easily finds that any such isomorphism must send $X$ to $aX$, where $a$
  is a square root of $\zeta_\ell^{p^{-1}-1}$, and conversely any such square
  root yields a Weil descent datum.
    \item More generally, if $\cG = R$ is representable by the formal spectrum of a 
      ring of the form
      \begin{equation*}
        R = \frac{ \cO_{\br E}\llbr X_1, \dots, X_n \rrbr}{(f_1, \dots,
        f_n)},
      \end{equation*} a Weil descent datum 
      for $\cG$ is an isomorphism of topological $R$-algebras
      \begin{equation*}
        \Phi^* R \cong \frac{ \cO_{\br E} \llbr X_1, \dots, X_n \rrbr}{(f_1^{\Phi^{-1}},
        \dots, f_n^{\Phi^{-1}})}
        \to 
        \frac{ \cO_{\br E}\llbr X_1, \dots, X_n \rrbr}{(f_1, \dots,
        f_n)} = R,
      \end{equation*}
      where $f_i^{\Phi^{-1}} \in \cO_\br E\llbr X_1, \dots, X_n\rrbr$ denotes the 
      power series obtained by applying $\Phi^{-1}$ to the coefficients of $f_i$.
      This follows quickly from the descriptions of $\Phi^*R$ in
      \eqref{eq:PhiRDescription}. A similar statement is true for admissible algebras
      formally of finite type over $\cO_\br E$.
  \end{enumerate}
\end{xpl}

An element $w \in \Weil_E$ with $w|_{\br E} = \Phi^m$ for $m \in \Z$ induces
a morphism over $\Spf(\cO_E)$
\begin{equation*}
  \cG \times_{\spf{\cO_{\br E}}} \spf(\cO_{{E'}}) \xto{(1,w|_{E'})}
  \Phi^{m, *}\cG \times_{\spf{\cO_{\br E}}} \spf(\cO_{{E'}}).
\end{equation*}
If $\cG$ admits a Weil descent datum, we obtain a commuatative square 
of isomorphisms of functors $\FSchOver {\cO_E}^{\opp} \to \Set$ 
\begin{equation}\label{eq:WeilDescentActionCommSquare}
\begin{tikzcd}[ampersand replacement=\&]
	{\cG \cotimes_{\cO_\br E} \cO_{{E'}}} \& {\Phi^{-m, *}\cG \cotimes_{\cO_\br E} \cO_{E'}} \\
	{\Phi^{m, *}\cG \cotimes_{\cO_\br E} \cO_{{E'}}} \& {\cG \cotimes_{\cO_\br E} \cO_{E'}}.
	\arrow["{(\Phi^{-m, *} \alpha^{-m}, 1)}", from=1-1, to=1-2]
	\arrow["{(1, \spf(w))}"', from=1-1, to=2-1]
	\arrow["{(1,\spf(w))}", from=1-2, to=2-2]
	\arrow["{(\alpha^{-m}, 1)}", from=2-1, to=2-2]
\end{tikzcd}
\end{equation}
Here, $\alpha^{m}$ denotes the isomorphism 
$$\alpha^m: \cG \xto \alpha \Phi^{*} \cG \xto {\Phi^* \alpha} \Phi^{2, *}\cG
\xto{\Phi^{2,*}\alpha} \dots \xto{\Phi^{m-1,*}\alpha} \Phi^{m,*}\cG$$ 
obtained by iterating the isomorphism $\alpha$, and 
$\alpha^{-m}$ denotes the inverse of $\alpha^{m}$. 
We write $\delta_w$ for the automorphism of $\cG \cotimes_{\cO_\br E} \cO_{{E'}}$
described in the square \eqref{eq:WeilDescentActionCommSquare}.
For $w_1, w_2 \in \Weil_E$, we have $\delta_{w_1 \circ w_2} = \delta_{w_2}
\circ \delta_{w_1}$,
so the assignment $w \mapsto \delta_{w}$ defines a right-action of 
$\Weil_E$ on $\cG \times_{\spf(\cO_{\br E})} \cO_{{E'}}$. 

\begin{xpl}
  Again, suppose that the functor 
  $\cG$ is representable by the formal spectrum of some $\cO_{\br E}$-algebra $R$,
  and assume for simplicity that $R$ admits a finite presentation
  \begin{equation*}
    R = \frac{\cO_\br E \llbr X_1, \dots, X_n\rrbr}{(f_1, \dots, f_n)}.
  \end{equation*}
  Furthermore, assume that $\cG$ admits a Weil descent datum $\alpha$. 
  Then by the example above, the inverse of $\Phi^{-m, *} \alpha^{-m}$
  yields an isomorphism of $\cO_{{E'}}$-algebras
  \begin{equation*}
    \rho^m: \frac{\cO_{{E'}} \llbr X_1, \dots, X_n\rrbr}{(f_1^{\Phi^m},
    \dots, f_n^{\Phi^m})} \to 
    \frac{\cO_{{E'}} \llbr X_1, \dots, X_n\rrbr}{(f_1, \dots, f_n)},
  \end{equation*}
  uniquely determined by the images $\rho(X_i)$. 
  Let $w \in \Weil_E$ be an element satisfying $w|_{\br E} = \Phi^{m}$.
  Then $\delta_w$ corresponds on the level of global sections to an isomorphism
  \begin{equation*}
    R \cotimes_{\cO_\br E} \cO_{{E'}} = 
    \frac{\cO_{{E'}} \llbr X_1, \dots,
    X_n\rrbr}{(f_1, \dots, f_n)} \xto w  \frac{\cO_{{E'}} \llbr X_1, \dots,
        X_n\rrbr}{(f_1^{\Phi^m}, \dots, f_n^{\Phi^m})} \xto {\rho^m}
        \frac{\cO_{{E'}} \llbr X_1, \dots, X_n\rrbr}{(f_1, \dots, f_n)} = R
        \cotimes_{\cO_\br E} {\cO_{{E'}}}.
  \end{equation*}
  That is, $\delta_w(a) = w|_{E'}(a)$ for $a \in \cO_{{E'}}$ and $\delta_w(X_i) =
  \rho^m(X_i)$ for $i= 1, \dots, n$.  
  \end{xpl}

One quickly verifies the following.

\begin{lem}\label{lem:WeilGroupActionIsFunctorialInE}
  The action defined this way is functorial in $E'$. That is, if $E''/E$ is a
  finite Galois extension containing $E'$, then the induced morphism 
  of functors 
  $\cG \times_{\spf \cO_{\br \cO_E}} \spf(\cO_{E''}) \to 
  \cG \times_{\spf \cO_\br E} \spf(\cO_{E'})$
  is $\Weil_E$-equivariant. 
\end{lem}
As a direct consequence, we obtain an action of $\Weil_E$ on
$$\lim_{E'/E \text{ finite}} \cG \times_{\spf \cO_\br E} \spf(\cO_{E'}) = 
\cG \times_{\spf \cO_\br E} \spf(\cO_{\Cp}).$$

We now define a Weil descent datum $\alpha$ on $\cM_m$. As $\cM_m$ is representable
in $\cC$, it suffices to construct the isomorphism
$\cM_m \to \Phi^* \cM_m$ on objects $R \in \cC$. 
Given $R\in \cC$ and $(\cF, \iota, \phi) \in \cM_m(R)$, we put
\begin{equation*}
  \alpha(R) \left( \cF, \iota, \phi \right) = (\cF', \iota',\phi'),
\end{equation*}
where
\begin{itemize}
  \item the formal $\cO_E$-module $\cF'$ is given by $\cF' = \cF \otimes_{R, \id}
    \Phi^* R = \Phi^* \cF$.
  \item the quasi-isogeny $\iota'$ is defined as
    \begin{equation*}
      \XX \overset{\Frob_q^{-1}}{\dashrightarrow} \XX^{(-q)} = \XX
      \otimes_{\bar \FF_q, \Frob^{-1}}
      \Fqbar \dashrightarrow \cF \otimes_{R, \Phi^{-1}} (R/\fm_R) = (\Phi^*
      \cF) \otimes_R \Fqbar.
    \end{equation*}
    Here $\Frob_q^{-1}$ denotes the inverse quasi-isogeny of the relative
    Frobenius morphism $\Frob_q: \cF^{(-q)} \to \cF$. In particular, 
    $\height(\iota') = \height(\iota)-1$.
  \item the level structure $\phi'$ is
    \begin{equation*}
      \left({\varpi^{-m}\cO_E}/{\cO_E} \right)^m      
       \xto{\phi} \cF(R) = \Phi^* \cF(R).
    \end{equation*}
\end{itemize}

As explained above, this yields a right action of $\Weil_E$ on $\cM_{m, \cO_{{E'}}}$.
Given $w \in \Weil_E$ such that $w|_\br E = \Phi^m$ for $m \in \Z$, we find
that $\delta_w$ restricts to an isomorphism $\cM_{m, \cO_{{E'}}}^{(j)} \to \cM_{m,
\cO_{{E'}}}^{(j+m)}$. 

\begin{xpl}
  The action of $\Weil_E$ on 
  $(\cM_{m}\cotimes_{\cO_\br E} \cO_{E'})({\cO_{E'}})$ admits a simple description.
  Suppose we are given (an equivalence class of)
  a triple $(\cF, \iota, \phi) \in \cM_m({\cO_{E'}})$ 
  and an element $w \in \Weil_E$ such that 
  $w|_{\br E} = \Phi^m$ for $m \in \Z$. We may choose a coordinate
  $\cF = \FGG(F)$, where $F$ is some formal $\cO_E$-module law over $E'$. Now
  $$\delta_w(\cF, \iota, \phi) = (\FGG(F^w), \iota^w, \phi^w),$$ 
  where $F^w$ is the formal group law over $E'$ obtained by applying 
  $w|_{E'}$ to $F$ coefficient-wise. The quasi-isogeny $\iota^w$ may be
  described in these coordinates as the composite
  $$ \XX \xto {\Frob_q^m} \XX^{(q^m)} \xto {\iota^{(q^m)}} (F \otimes
  \Fqbar)^{(q^m)} = F^w \otimes \Fqbar,$$
  while the level structure $\phi^w$ is obtained by post-composing $\phi$ with $w$:
  $$\phi^w = w|_{\fm_{E'}} \circ \phi \colon (\varpi^{-m}\cO_E/\cO_E)^n 
  \to \fm_{\cO_{E'}}.$$
\end{xpl}
% subsection The Lubin-Tate Tower (end)


% (end)

\subsection{The \'Etale Cohomology of the Lubin--Tate Tower} % (fold)
\label{sub:The Local Langlands Correspondence for the General Linear Group}
In the preceeding paragraphs, we explained that $\{\cM_{K}\}_{K \subset
\GL_n(\cO_E)}$ is a 
tower of locally noetherian formal schemes with finite and generically \'etale 
transition maps, equipped with an action by $\GL_n(E) \times D^\times$ and 
a Weil descent datum. 

Let us write $M_{K}$ for the rigid generic fiber of $\cM_{K}$,
which we may consider as an analytic adic space over $\spa(\br E,\cO_\br E)$. 
Ranging over $m$, this yields a tower of spaces $\{M_{K}\}_{K \subset
\GL_n(\cO_E)}$ whose transition maps are finite and \'etale. Furthermore, each
component admits a Weil descent datum and an action by $D^\times$, and $\GL_n(E)$
acts on the tower as a pro-object. 

This section is concerned with the representation-theoretic aspects of
the compactly supported \'etale cohomology groups
\begin{equation*}
  \colim_K \hHc^{n-1} (M_{K} \times_{\spa{\br E,\cO_\br E}} \Spa(\Cp, \cO_\Cp),
  \bar \Q_l).
\end{equation*}
The right-action of $G \coloneqq \GL_n(E) \times D^\times \times \Weil_E$ on the tower
$\{\cM_{K, \cO_\Cp}\}_{K \subset \GL_n(\cO_E)}$ induces the structure of a
$G$-representation on this vector space. 
The supercuspidal part of this representation realizes the local Langlands
correspondence. 

\begin{defi}[Supercuspidal Representation]\label{def:SuperCusp}
  An admissible (cf. \cref{def:SmoothAndAdmRep}) representation $(\pi,V) \in
  \Rep{\GL_n(E)}$ is called supercuspidal if all of its matrix coefficients are
  constant modulo the center $E^\times \subseteq \GL_n(E)$.
\end{defi}

Here, a matrix coefficient is a function of the form $m(g) = \langle \pi(g)
\phi, \phi^\vee\rangle$, where $(\phi, \phi^\vee)$ ranges over the elements of
$V \times V^\vee$ (where $\vee$ denotes the contragradient). For details, and
the important relation to the method of parabolic induction, we refer to
\cite[Section 8]{getz2023introduction}.

We briefly recall the local Langlands are Jacquet--Langlands correspondences.
The local Langlands correspondence gives a bijection
\begin{equation*}
  \rec_E\colon \left\{
\begin{gathered}\text{supercuspidal representations} \\
\text{of $\GL_n(E)$}\end{gathered}\middle\}\right / \cong \ \ 
\xto{1:1} \  \left\{\begin{gathered}\text{irreducible representations} \\
\text{of $\Weil_E$} \end{gathered}\middle\}\right/\cong.
\end{equation*}
Of course not any such bijection is interesting, the local Langlands correspondence
predicts the existence of a unique bijection $\rec_E$ compatible with various
structures on both sides. See \cite[Section 12.4]{getz2023introduction} for a
detailed account of the axioms $\rec_E$ should satisfy.
In particular, the case $n=1$ recovers local class field theory: 
$\rec_E(\pi) = \pi \circ \Art_E^{-1}$. 

The local Jacquet--Langlands correspondence gives a bijection 
\begin{equation*}
  \JL\colon \left\{
\begin{gathered}\text{irreducible discrete series} \\
\text{representations of $\GL_n(E)$}\end{gathered}\middle\}\right / \cong \ \ 
\xto{1:1} \  \left\{\begin{gathered}\text{irreducible representations} \\
\text{of $D^\times$}. \end{gathered}\middle\}\right/\cong,
\end{equation*}
characterized by a natural character identity. This result is due to
Deligne--Bernstein--Kazhdan--Vigneras cf. \cite{deligne1984representations} and 
Rogawski \cite{rogawski1983representations}. Here, neither the definition of discrete
series representations or the precise statement is of major importance for us,
we only need the fact that all supercuspidal representations of $\GL_n(E)$ are
discrete series representations. 

By results of Boyer (cf. \cite{boyer1999mauvaise})
in the equal characteristic case and Harris--Taylor (cf. \cite{HTShimura}) in the 
mixed characteristic case, both $\rec_F(\varpi)$ and $\JL(\varpi)$ occur
(up to duals and twists) in the $\pi$-isotypic component of 
$\colim_K \hHc^{n-1} (M_{K} \otimes_{\br E} \Cp, \bar \Q_l)$. We state a
precise result below.
For our purposes, it will be sufficient to consider supercuspidal representations
whose central character is trivial on $\varpi^\Z$. For those representations,
the correspondence already takes place in the cohomology of the 
quotients $M_{K, \varpi^\Z} \coloneqq M_K/\varpi^\Z$ by the action of the
subgroup $\varpi^\Z \subseteq D^\times$. Recall that the action of 
$\varpi$ restricts to an isomorphism $\cM_K^{(j)} \to \cM_K^{(j+n)}$, so we find
$M_{K, \varpi^\Z} \cong \coprod_{0 \leq j < n} M_{K}^{(j)}$. We now define
\begin{equation} \label{eq:defHLT}
  \HLT \coloneq \colim_{K} \hHc^{n-1} (M_{K, \varpi^\Z} \otimes_{\br E} C, \bar \Q_l),
\end{equation}
where $K$ ranges over the compact open subgroups of $\GL_n(\cO_E)$. 
For an irreducible representation $\pi$ of $\GL_n$, we denote by ${\HLT}_{,\pi}$ the 
$\pi$-isotypical component of $\HLT$.
The precise statement of Non-Abelian Lubin--Tate Theory is the following.
\begin{thm}[Non-Abelian Lubin--Tate Theory]\label{thm:NonAbLTT}
  Let $\pi$ be an irreducible supercuspidal representation of $\GL_n(E)$ whose central
  character is trivial on $\varpi^\Z$.   
  Then we have 
  \begin{equation*}
    {\HLT}_{,\pi^\vee} = \pi^\vee \boxtimes \JL(\pi) \boxtimes \rec_F(\pi)(\tfrac{1-n}2)
  \end{equation*}
  as representations of $\GL_n(E) \times D^\times \times W_F$. 
\end{thm}

Let $\wedge \XX$ denote the determinant of $\XX$ as defined in 
\cref{sub:Determinants} (it is a normalized formal $\cO_E$-module law over
$\Fqbar$ of height $1$) 
and denote by $\cM_{m, \wedge}^{(0)}$ the corresponding deformation functors of
$\wedge \XX$. 
By \cref{ssub:The Case of Height One}, we have a natural
determinant map $\cM_{m}^{(0)} \to \cM_{m ,\wedge}^{(0)}$, and we may choose compatible
isomorphisms $\cM_{m, \wedge}^{(0)} \cong \cO_{E^\LT_m}$. 
We define $\cM'_{m,\cO_\Cp} \coloneq \cM_{m}^{(0)} \cotimes_{\cO_{E^\LT_m}} \cO_\Cp$.
As the determinant map is multilinear and alternating, we obtain an action of 
$\SL_n(\cO_E)$ on each space $\cM'_{m,\cO_\Cp}$. This allows us to construct a tower
$\{\cM'_{K, \cO_\Cp}\}_{K \subset \SL_n(\cO_E)}$, where $K$ ranges over the 
open subgroups of $\SL_n(\cO_E)$, by the same method as sketched 
at the end of \cref{ssub:Group Actions on the Tower of Lubin--Tate Deformation
Spaces}. Furthermore, it is a consequence of \cref{lem:WeinsteinDeterminantAndNorm}
that the group 
\begin{equation}\label{eq:DefGandG1}
  G^1 \coloneqq \{(g,d,\sigma) \in \GL_n(E) \times D^\times \times W_F \mid 
  \det(g)^{-1} \Nrd(d) \Art_F^{-1}(\sigma) = 1\}
\end{equation}
acts on this tower.

We define $M'_{K, C}$ as the generic fiber of $\cM'_{K, \cO_\Cp}$, which is 
an analytic adic space over $\spa(\Cp,\allowbreak \cO_\Cp)$. We obtain a tower
of adic spaces
$\{M'_{K, \Cp}\}_{K \subset \SL_n(\cO_E)}$ with finite \'etale transition maps. 

Much of our study of the Lubin--Tate tower will evolve around
this tower. It will therefore be convenient to define 
\begin{equation*}
  \HLT' \coloneqq \colim_K \hHc^{n-1}(M_{K,C}', \Qlbar).
\end{equation*}
By construction, $\HLT'$ is a representation of the group $G^1$. 
The vector spaces $\HLT$ and $\HLT'$ are closely related. 
As $\varpi^\Z$ acts trivially on $\HLT$, it is really a representation of the group
\begin{equation*}
  \bar G \coloneqq \GL_n(E) \times (D^\times/\varpi^\Z) \times \Weil_E.
\end{equation*}
We note the following relation between $G^1$ and $\bar G$.
\begin{lem}\label{lem:G1subG}
  The natural map $G^1 \to \bar G$ is injective and realizes $G^1$ as a co-compact closed
  normal subgroup of $\bar G$.
\begin{proof}
  The morphism $G^1 \to \bar G$ is clearly injective. Further, the image of the natural
  homomorphism is isomorphic to the kernel of the map 
  \begin{equation*}
   \theta \colon \bar G \to F^\times/\varpi^{n\Z}, \quad \nu(g, \bar d, \sigma) \mapsto 
 \bar{\det(g)\Nrd(d)^{-1}\Art_F^{-1}(\sigma)}.
  \end{equation*}
  The claim follows.
\end{proof}
\end{lem}

We have the following crucial result, cf. \cite[Section 4]{mieda2016geometric}.
\begin{prop}\label{lem:InductionStatementOnHLT}
  The $\bar G$-representation $\cInd_{G^1}^{\bar G} (H'_\LT)$ is isomorphic to $H_\LT$. 
\end{prop}

Finally, we need the following result, which is also in Section 4 of
\cite{mieda2016geometric}.
\begin{lem}\label{lem:GActsSmoothlyOnHLT}
  The action of $\Weil_E$ on $\HLT$ is smooth.
\end{lem}


% subsection The Local Langlands Correspondence for the General Linear Group (end)

% section Non-Abelian Lubin-Tate Theory: An Overview (end)

\end{document}
