%! TeX root: ../main.tex
\documentclass[../main.tex]{subfiles}

\begin{document}
\section{Non-Abelian Lubin-Tate Theory: An Overview} % (fold)
\label{sec:Non-Abelian Lubin-Tate Theory: An Overview}
In the preceeding chapter we uesd formal $\cO_K$-modules to understand the 
maximial abelian extension of a local field $K$. The hope of non-Ablian Lubin-Tate
theory is that the $l$-adic cohomology of certain moduli spaces of 
deformations of formal modules, which comes with commuting actions by $\GL_n$ 
and $W_K$, encodes information about the non-abelian extensions of $K$. 

\subsection{The Lubin-Tate Tower} % (fold)
\label{sub:The Lubin-Tate Tower}
\subsubsection{Deformations of Formal Modules} % (fold)
\label{subsub:Deformations of Formal Modules}
We mostly follow \cite[Chapter 2]{Strauch2008DefSp} for notation. Let $\cC$
denote the category of local, Noetherian $\cO_\br K$-modules with residue field
$\bar \FF_q$. Let $H_0$ be a formal $\cO_K$-module over $\bar \FF_q$. 

% subsection Deformations of Formal Modules (end)

\subsubsection{Deformations of Formal Modules with Drinfeld Level Structure} 

\label{subsub:Deformations of Formal Modules with Drinfeld Level Structure}
The following examples might shed some light on this definition.
\begin{xpl}
  \begin{itemize}
    \item $\Ghat$
    \item Things over $\FF_q$.
  \end{itemize}
\end{xpl}

\begin{itemize}
  \item Drinfeld Level
  \item Moduli Problem + Representability
  \item The Lubin-Tate Tower
\end{itemize}
% subsection Deformations of Formal Modules with Drinfeld Level Structure (end)

\subsubsection{The Group actions on the Tower and its Cohomology} % (fold)
\label{subsub:The Group actions on the Tower}
\begin{itemize}
  \item Action By $D^\times$ and $\GL_n$
  \item Action by $W_K$ via Weil descent Datum.
\end{itemize}
% subsection The Group actions on the Tower (end)


% subsection The Lubin-Tate Tower (end)

\subsection{The Local Langlands Correspondence for the General Linear Group} % (fold)
\label{sub:The Local Langlands Correspondence for the General Linear Group}

% subsection The Local Langlands Correspondence for the General Linear Group (end)

\subsection{The Lubin-Tate Perfectoid Space} % (fold)
\label{sub:The Lubin-Tate Perfectoid Space}

% subsection The Lubin-Tate Perfectoid Space (end)

% section Non-Abelian Lubin-Tate Theory: An Overview (end)
\end{document}
