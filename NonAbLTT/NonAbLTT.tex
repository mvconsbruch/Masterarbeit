%! TeX root: ../main.tex
\documentclass[../main.tex]{subfiles}

\begin{document}
\section{Non-Abelian Lubin-Tate Theory: An Overview} % (fold)
\label{sec:Non-Abelian Lubin-Tate Theory: An Overview}
In the preceeding chapter we used formal $\cO_K$-modules to understand the 
maximial abelian extension of a local field $K$. The hope of non-Abelian Lubin-Tate
theory is to gain insight about the Abelian extensions of $K$ by considering
certain moduli spaces of formal $\cO_K$-modules.
More precisely, attached to a formal $\cO_K$-module $H_0$ over $\Fqbar$ (determined
up to isomorphism by its height $n$), we attach a system of rigid spaces 
$\{M_K\}_{K \subset \GL_n(\cO_K)}$, the so called Lubin-Tate Tower. For $l \neq p$, 
the system of $l$-adic compactly supported cohomology groups $\{H_c^{i}(M_K,
\bar \Q_l)\}_K$ admits commuting actions by $\GL_n(K)$, $W_K$ and $D^\times$,
where the latter denotes the units of the central divison algebra $D =
\End_{\FMOver {\cO_K}{\Fqbar}}(H_0) \otimes \Q$. This yields a correspondence of 
representations of the respective groups, and Harris and Taylor showed in
\cite{HTShimura} that the cohomology of middle degree induces (a version of)
the Local Langlands Correspondence. Our goal is an explicit description of 
this correspondence, and we hope to obtain such a description by 
understanding the Lubin-Tate tower explicitely.
As it turns out, the the limit $\lim_{K} M_K$ is representable by a perfectoid space
which is is easier to describe than its individual layers.

\subsection{The Lubin-Tate Tower} % (fold)
\label{sub:The Lubin-Tate Tower}
\subsubsection{Deformations of Formal Modules} % (fold)
\label{subsub:Deformations of Formal Modules}
We mostly follow \cite[Chapter 2]{Strauch2008DefSp} for notation. Let $\cC$
denote the category of local, Noetherian $\cO_\br K$-modules with distinguished
isomorphisms $R/\fm_R \to \bar \FF_q$. Let $H_0$ be a formal $\cO_K$-module law
over $\bar \FF_q$. 
\begin{defi}[Deformation]
  Let $R \in \cC$. A deformation of $H_0$ to $R$ is a pair $(H, \iota)$ where
  $H$ is a formal $\cO_K$-module over $R$ and $\iota$ is a quasi-isogeny
  \begin{equation*}
    H_0 \to H \otimes_R \Fqbar.
  \end{equation*}
  Two deformations $(H, \iota)$ and $(H', \iota')$ are isomorphic if there is
  an isomorphism $\tau: H \to H'$ with $\iota' \circ \tau = \iota$. 
\end{defi}
The Lubin-Tate space at level $\fp^0$ is the moduli space of such deformations.
A priori, it is the functor
\begin{equation*}
  \cM: \cC \to \Set, \quad R \mapsto \{\text{deformations } (H, \iota) \text{
  of $H_0$}\}/\cong.
\end{equation*}



\begin{itemize}
  \item Deformations
  \item Representability of $\cM_0$.
\end{itemize}
% subsection Deformations of Formal Modules (end)

\subsubsection{Deformations of Formal Modules with Drinfeld Level Structure} % (fold)
\label{subsub:Deformations of Formal Modules with Drinfeld Level Structure}
\begin{itemize}
  \item Drinfeld Level
  \item Moduli Problem + Representability
  \item The Lubin-Tate Tower
\end{itemize}
% subsection Deformations of Formal Modules with Drinfeld Level Structure (end)

\subsubsection{The Group actions on the Tower and its Cohomology} % (fold)
\label{subsub:The Group actions on the Tower}
\begin{itemize}
  \item Action By $D^\times$ and $\GL_n$
  \item Action by $W_K$ via Weil descent Datum.
\end{itemize}
% subsection The Group actions on the Tower (end)


% subsection The Lubin-Tate Tower (end)

\subsection{The Local Langlands Correspondence for the General Linear Group} % (fold)
\label{sub:The Local Langlands Correspondence for the General Linear Group}

% subsection The Local Langlands Correspondence for the General Linear Group (end)

\subsection{The Lubin-Tate Perfectoid Space} % (fold)
\label{sub:The Lubin-Tate Perfectoid Space}

% subsection The Lubin-Tate Perfectoid Space (end)

% section Non-Abelian Lubin-Tate Theory: An Overview (end)
\end{document}
