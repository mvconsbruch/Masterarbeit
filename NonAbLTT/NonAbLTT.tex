%! TeX root: ../main.tex
\documentclass[../main.tex]{subfiles}

\begin{document}
\section{Non-Abelian Lubin-Tate Theory: An Overview} % (fold)
\label{sec:Non-Abelian Lubin-Tate Theory: An Overview}
In the preceeding chapter we used formal $\cO_F$-modules to understand the 
maximial abelian extension of a local field $F$. The hope of non-Abelian Lubin-Tate
theory is to gain insight about the Abelian extensions of $F$ by considering
certain moduli spaces of formal $\cO_F$-modules.
More precisely, attached to a formal $\cO_F$-module $H_0$ over $\Fqbar$ (determined
up to isomorphism by its height $n$), we attach a system of rigid spaces 
$\{M_K\}_{K \subset \GL_n(\cO_F)}$, the so called Lubin-Tate Tower. For $l \neq p$, 
the system of $l$-adic compactly supported cohomology groups $\{H_c^{i}(M_K,
\bar \Q_l)\}_K$ admits commuting actions by $\GL_n(F)$, $W_F$ and $D^\times$,
where the latter denotes the units of the central divison algebra $D =
\End_{\FMOver {\cO_K}{\Fqbar}}(H_0) \otimes \Q$. This yields a correspondence of 
representations of the respective groups, and Harris and Taylor showed in
\cite{HTShimura} that the cohomology of middle degree induces (a version of)
the Local Langlands Correspondence for $\GL_n$. Our goal is an explicit
description of this correspondence, and we obtain such descriptions by
understanding the Lubin-Tate tower explicitely.
As it turns out, the the limit $\lim_{K\subset \GL_n{\cO_F}} M_K$ is
representable by a perfectoid space
which is is easier to describe than its individual layers.

\subsection{The Lubin-Tate Tower} % (fold)
\label{sub:The Lubin-Tate Tower}
\subsubsection{Deformations of Formal Modules} % (fold)
\label{subsub:Deformations of Formal Modules}
We mostly follow \cite[Chapter 2]{Strauch2008DefSp} for notation. Let $\cC$
denote the category of local, Noetherian $\cO_\br K$-modules with distinguished
isomorphisms $R/\fm_R \to \bar \FF_q$. Let $H_0$ be a formal $\cO_K$-module 
over $\bar \FF_q$. 
\begin{defi}[Deformation]
  Let $R \in \cC$. A deformation of $H_0$ to $R$ is a pair $(H, \iota)$ where
  $H$ is a formal $\cO_K$-module over $R$ and $\iota$ is a quasi-isogeny
  \begin{equation*}
    \iota: H_0 \dashto H \otimes_R \Fqbar.
  \end{equation*}
  Two deformations $(H, \iota)$ and $(H', \iota')$ are isomorphic if there is
  an isomorphism $\tau: H \to H'$ with $\iota' \circ \tau = \iota$. 
\end{defi}
The Lubin-Tate space without level structure is the moduli space of such deformations.
More precisely, we define it as the functor
\begin{equation*}
  \cM_0: \cC \to \Set, \quad R \mapsto \{\text{deformations } (H, \iota) \text{
  of $H_0$}\}/\text{iso}.
\end{equation*}

\begin{thm}[Representability of $\cM_0$]
  The functor $\cM_0$ is (non-canonically) representable, by the noetherian
  local ring
  \begin{equation*}
    A_0 \cong \cO_\br K\llbr u_1, \dots, u_{n-1} \rrbr.
  \end{equation*}
\end{thm}
In particular, there is a universal deformation
$(F^\univ, \iota^\univ)$, with $F^\univ \in \FMOver {\cO_\br K} {A_0}$.
% subsection Deformations of Formal Modules (end)

\subsubsection{Deformations of Formal Modules with Drinfeld Level Structure} 

\label{subsub:Deformations of Formal Modules with Drinfeld Level Structure}
\begin{defi}[Drinfeld level $\fp^m$-structure]
  Let $R \in \cC$ and $H \in \FMOver {\cO_K}{R}$. A Drinfeld level 
  $\fp^m$-structure on $H$ is a morphism of $R$-group schemes 
  \begin{equation*}
    (\fp^{-m}/\cO_K)^{\oplus n} \to H(R)[\varpi^m]
  \end{equation*}
  such that after choosing a coordinate $H \cong \spf R\llbr T \rrbr$, the 
  power series $[\varpi]_H(T) \in R\llbr T \rrbr$ satisfies the divisibility constraint
  \begin{equation*}
    \prod_{x \in (\fp^{-1}/\cO_K)}(T - \phi(x)) \mid  [\varpi]_H(T).
  \end{equation*}
\end{defi}
The following examples might shed some light on this definition.
\begin{xpl}
  \begin{itemize}
    \item $\Ghat$
    \item Things over $\FF_q$.
  \end{itemize}
\end{xpl}

\begin{itemize}
  \item Drinfeld Level
  \item Moduli Problem + Representability
  \item The Lubin-Tate Tower
\end{itemize}
% subsection Deformations of Formal Modules with Drinfeld Level Structure (end)

\subsubsection{The Group actions on the Tower and its Cohomology} % (fold)
\label{subsub:The Group actions on the Tower}
\begin{itemize}
  \item Action By $D^\times$ and $\GL_n$
  \item Action by $W_K$ via Weil descent Datum.
\end{itemize}
% subsection The Group actions on the Tower (end)


\begin{defi}[Lubin-Tate rigid space with $K$-level structure]\label{def:LTRigidSpaceFinLev}
  \todo{}
\end{defi}

For $K \subset \GL_n(\cO_F)$, write $M_{K, \varpi^\Z}$ for the quotient of 
$M_K$ (cf. Definition \ref{def:LTRigidSpaceFinLev}) by the action of the subgroup
$\varpi^\Z \subset D^\times$. Writing $\cM_m = \coprod_{\delta \in \Z} \cM_m^{(\delta)}$
induces $M_K = \coprod_{\delta \in\Z} M_K^{(\delta)}$, and 
the action of $\varpi$ induces for any $\delta \in \Z$ an isomorphism 
$M_K^{(\delta)} \cong M_K^{(\delta +n)}$. Hence, $M_{K, \varpi^\Z}$ is 
isomorphic to $\coprod_{0 \leq \delta \leq n-1} M_{K}^{(\delta)}$. 

Let $l \neq p$ be a prime number and fix an isomorphism $\bar \Q_l \cong \C$.

\begin{defi}[Cohomology of the Lubin--Tate tower]
  We write $\HLT = \lim_{K} H^{n-1}_c (M_{K, \varpi^\Z} \otimes_{\breve F} C, \bar \Q_l)$.
\end{defi}

\begin{thm}[Non-Abelian Lubin--Tate theory]
  Let $\pi$ be an irreducible supercuspidal representation of $\GL_n(F)$ whose central
  character is trivial on $\varpi^\Z$. We write $\rec_F(\pi)$ for the irreducible 
  smooth representation of $W_F$ corresponding to $\pi$ undet the local Langlands
  correspondence, and $\JL(\pi)$ for the irreducible smooth representation of
  $D^\times$ corresponding to $\pi$ under the local Jacquet-Langlands correspondence.
  Then we have 
  \begin{equation*}
    \HLT[\pi^\vee] = \pi^\vee \boxtimes \JL(\pi) \boxtimes \rec_F(\pi)(\tfrac{1-n}2)
  \end{equation*}
  as representations of $\GL_n(F) \times D^\times \times W_F$. 
\begin{proof}
\end{proof}
\end{thm}

% subsection The Lubin-Tate Tower (end)

\subsection{The Local Langlands Correspondence for the General Linear Group} % (fold)
\label{sub:The Local Langlands Correspondence for the General Linear Group}

% subsection The Local Langlands Correspondence for the General Linear Group (end)

\subsection{The Lubin-Tate Perfectoid Space} % (fold)
\label{sub:The Lubin-Tate Perfectoid Space}

% subsection The Lubin-Tate Perfectoid Space (end)

% section Non-Abelian Lubin-Tate Theory: An Overview (end)
\end{document}
