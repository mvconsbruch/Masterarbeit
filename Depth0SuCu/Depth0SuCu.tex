%! TeX root: ../main.tex
\documentclass[../main.tex]{subfiles}

\begin{document}
\section{Explicit Non-Abelian Lubin--Tate Theory for Depth Zero Supercuspidal
Representations} % (fold)
\label{sec:Explicit Non-Abelian Lubin-Tate Theory for Depth Zero Supercuspidal Representations}

\subsection{The Special Affinoid} % (fold)
\label{sub:The Special Affinoid}
\begin{defi}[The special Affinoid]
  \todo{}
\end{defi}

% subsection The Special Affinoid (end)

\subsection{The Explicit Correspondence} % (fold)
\label{sub:The Explicit Correspondence}
Fix, for the remainder of the section, an isomorphism $\bar \Q_l \cong \C$ and
a regular character $\theta: \FF_{q^n}^\times \to \C^\times$. 
The datum of $\theta$ can be used to construct representations 
of $W_F$ and $D^\times$ and, making use of Deligne--Lusztig
theory, a representation of $\GL_n(F)$. We proceed as follows.
\begin{itemize}
  \item Let $\bar \tau_\theta$ be the character of $\Weil_{F_n}$ given by 
    the composition
    \begin{equation*}
      \Weil_{F_n} \to \Weil_{F_n}^\ab \xto{\Art_{F_n}^{-1}} F_n^\times \cong
      \Z \times \cO_{F_n}^\times \surj \FF_{q^n}^\times \xto \theta \C^\times
    \end{equation*}
  and put $\tau_\theta = \cInd_{\Weil_{F_n}}^{\Weil_F}(\bar \tau_\theta)$.
  \item Let $\bar \rho_\theta$ be the character on 
    $F^\times \cO_D^\times$ given by the composition
    \begin{equation*}
      F^\times\cO_D^\times \cong \varpi^\Z \times \cO_D^\times \surj 
      \FF_{q^n} \xto \theta \C^\times
    \end{equation*}
    and let $\rho_\theta = \cInd_{F^\times \cO_D^\times}^{D^\times}(\bar \rho_\theta)$.
  \item Let $\bar \pi_\theta$ be the representation of $F^\times \GL_n(\cO_F)$
    arising from post-composing $R_\theta$ (cf. Theorem
    \ref{thm:DLCorrespondenceForUs}) with the composition
    \begin{equation*}
      F^\times \GL_n(\cO_F) \cong \varpi^\Z \times \GL_n(\cO_F)
      \surj \GL_n(\cO_F) \surj \GL_n(\FF_q).
    \end{equation*}
    Let $\pi_\theta = \cInd_{F^\times \GL_n(\cO_F)}^{\GL_n(F)}(\bar \pi_\theta)$. 
\end{itemize}

\begin{lem}\label{lem:BarRepsAreSmooth}
  The representations $\bar \pi_\theta$, $\bar \rho_\theta$ and $\bar
  \tau_\theta$ are smooth, in particular
  $\pi_\theta$, $\rho_\theta$ and $\tau_\theta$ are smooth as well. Additionally, the
  representations $\pi_\theta$ and $\rho_\theta$ are irreducible, and
  $\pi_\theta$ is supercuspidal.
\begin{proof}
  By design, $\bar \pi_\theta$ is trivial on the compact open subgroup $1 +
  \varpi \Mat_{n \times n}(\cO_F)$ of $F^\times \GL_n(\cO_F)$. Similar
  statements hold for $\bar \rho_\theta$
  and $\bar \tau_\theta$. 
  \red{why is $\rho_\theta$ irreducible? Why is $\pi_\theta$ supercuspidal and
  irreducible?}
\end{proof}
\end{lem}

The aim of this section is to prove the following statement.
\begin{thm}[Explicit Non-Abelian Lubin--Tate Theory for Depth Zero Supercuspidal
  Representations]\label{thm:MainRes1}
  The representation $\JL(\pi_\theta)$ of $D^\times$ and the representation
  $\rec_F(\pi_\theta)$ of $\Weil_F$ take the form
  \begin{equation*}
    \JL(\pi_\theta) = \rho_{\theta}
    \quad \text{and} \quad \rec_F(\pi_\theta) = \Ind_{\Weil_{F_n}}^{\Weil_F} 
    (\tau_\theta \, \delta^{n-1}),
  \end{equation*}
  where $\delta: \Weil_{F_n} \to \{\pm 1\}$ is the unramified quadratic
  character. This is the character corresponding to $a \mapsto
  (-1)^{\val_{F_n}(a)}$ under the isomorphism $\Art_{F_n}: F_n^\times \to
  \Weil_{F_n}^\ab$. 
\end{thm}

We set 
\begin{equation} \label{eq:defJandJ1}
  J \coloneqq F^\times \GL_n(\cO_F) \times \cO_D^\times \times \Weil_{F_n}
  \text{ and } J^1 = J \cap G^1.
\end{equation}
Also, we define a morphism 
\begin{equation*}
  \Theta: J \to \GL_n(\FF_q) \times \FF_{q^n}^\times \times {\Frob_q^{n\Z}}, \quad
  (\varpi^m g,d,\sigma) \mapsto (\bar g, \bar {d^{-1}u_\sigma^{-1}}, \bar{\sigma}).
\end{equation*}

Recall that $H_\DL$ denotes the middle $l$-adic cohomology of 
$\DL_n$, cf. Section \ref{sub:The Deligne--Lusztig Correspondence for the
Explicit Example}.
\begin{lem}
  The morphism $\Theta$ makes $J$ act on ${\hH}_{\DL,\theta}$. This representation is 
  of the form 
  \begin{equation*}
    (g,d,\sigma) \mapsto \bar \pi_\theta(g) \otimes \bar \rho_{\theta^{-1}}(d)
    \otimes \left(\bar \tau_\theta \, \delta^{n-1}\right)^{-1}(\tfrac{1-n}2)(\sigma).
  \end{equation*}
  This representation is smooth.
\begin{proof}
  This is a direct calculation.
\end{proof}
\end{lem}

The input we get from Mieda's result is the following.
\begin{prop}\label{prop:J1EquivInjMor}
  There is an injective morphism of  $J^1$-representations
  \begin{equation*} 
    \Res_{J^1}^J( \hH_{\DL,\theta} ) \inj \Res_{J^1}^{G^1}( \hH'_{\LT}).
  \end{equation*}
\begin{proof}
  \red{This is \cite[Proposition 5.11]{mieda2016geometric}.}
\end{proof}
\end{prop}

\begin{lem} \label{lem:JEquivInjMor}
  The morphism in Proposition \ref{prop:J1EquivInjMor} naturally gives rise to
  an injective $J$-equivariant morphism
  \begin{equation*}
    \hH_{\DL,\theta} \inj \Res_J^G \hH_\LT.
  \end{equation*}
\begin{proof}
  We construct a sequence of $J$-equivariant injections
  \begin{equation*}
    \hH_{\DL,\theta} \inj \Ind_{J^1}^J ( \Res_{J^1}^J \hH_{\DL,\theta}) \inj
    \Ind_{J^1}^J (\Res_{J^1}^{G^1} \hH'_{\LT}) \\ \xto \sim \Res_J^{G^1J}(
    \Ind_{G^1}^{G^1J} \hH'_\LT) 
 \inj \Res_J^G \hH_\LT.
  \end{equation*}
  \textit{The first morphism.}
  This is the unit of the adjunction $\Res_{J^1}^J \ladj \Ind_{J^1}^J$ 
  applied at $\hH_{\DL,\theta}$, which is injective by Lemma
  \ref{lem:FrobRecUnitsAreInjective}.

  \textit{The second morphism.}
  This is $\Ind_{J^1}^J$ applied to the injective morphism in Proposition
  \ref{prop:J1EquivInjMor}. The resulting morphism is injective because
  $\Ind_{J^1}^J$ is exact, cf. Proposition
  \ref{prop:InducedRepresentationExact}. 

  \textit{The third morphism.} 
  The morphism is given by the inverse of the base-change morphism constructed in
  Lemma \ref{lem:BaseChangeForResInd}, which is applied with $H = G^1$, $N = J$. 
  Note that $G^1$ is normal in $G$, so the assumptions of the Lemma are
  satisfied. As $J$ is open in $G$, the map is an isomorphism. 
  
  \textit{The fourth morphism.}
  Since $G^1J$ is open in $G$, the unit of the adjunction $\cInd_{G^1J}^G \ladj
  \Res_{G^1J}^G$ yields a monomorphism of $G^1J$-representations 
  \begin{equation}\label{eq:MiedaEx1Morph1}
    \Ind_{G^1}^{G^1J} \hH_\LT' \to \Res_{G^1J}^G
    (\cInd_{G^1J}^G(\Ind_{G^1}^{G^1J} \hH'_\LT)).
  \end{equation}
  As $G^1J$ co-compact in $G$, we have $\cInd_{G^1J}^G =
  \Ind_{G^1J}^G$, so the right-hand
  side is isomorphic to $\Res_{G^1J}^G(\Ind_{G^1}^G \hH'_\LT) \cong
  \Res_{G^1J}^G(\hH_\LT)$ by Proposition \ref{prop:InductionOnTower} and 
  Lemma \ref{lem:InductionStatementOnHLT}. Hence, applying $\Res_J^{G^1J}$ to the
  morphism in \eqref{eq:MiedaEx1Morph1} yields the desired map.
\end{proof}
\end{lem}

The morphism constructed in Lemma \ref{lem:JEquivInjMor} yields, by Frobenius
reciprocity, a non-zero map of $G$-representations
\begin{equation} \label{eq:WantedMap}
  \Ind_{J}^G (\hH_{\DL,\theta}) \cong \pi_\theta \boxtimes \rho_{\theta^{-1}}
  \boxtimes (\tau_\theta \, \delta^{n-1})^{-1}(\tfrac{1-n}2) \to \hH_\LT.
\end{equation}
As $\pi_\theta$ is supercuspidal and its central character is trivial on 
$\varpi^\Z$, Theorem \ref{thm:NonAbLTT} yields a non-zero map
\begin{equation*}
  \rho_{\theta^{-1}} \boxtimes (\tau_\theta \, \delta^{n-1}) \to
  \JL(\pi_\theta)^\vee \boxtimes \rec_F(\pi_\theta)^\vee.
\end{equation*}
As $\rho_{\theta^{-1}}$ and $\JL(\pi_\theta)^\vee$ are irreducible, this implies
$\JL(\pi_\theta) = \rho_{\theta^{-1}}^\vee = \rho_\theta$. As 
$\rec_F(\pi_\theta)$ is irreducible and $\dim(\tau_\theta) = n = \dim
(\rec_F(\pi_\theta))$, this also implies $\tau_\theta \, \delta^{n-1} =
\rec_F(\pi_\theta)$. Admitting Proposition \ref{prop:J1EquivInjMor}, this
concludes the proof of Theorem \ref{thm:MainRes1}.

\subsubsection{Proof of Proposition \ref{prop:J1EquivInjMor}} % (fold)
\label{ssub:Proof of Proposition}

\begin{lem}\label{lem:GeneratorsForJ1}
  The group $J^1$ is generated by the following elements.
  \begin{itemize}
    \item $(g,1,1)$ for $g \in \GL_n(\cO_F)$ with $\deg g = 1$,
    \item $(1,d,1)$ for $d \in \cO_D^\times$ with $\Nrd d = 1$, 
    \item $(a,a,1)$ for $a \in F_n^\times$, 
    \item $(1, \Art^{-1}_{F_n}(\sigma)^{-1}, \sigma)$ for $\sigma \in W_{I_n}$, 
    \item and $(1, \varpi^{-1}, \sigma)$ for $\sigma \in W_{F_n}$ with
              $\Art^{-1}_{F_n}(\sigma) = \varpi$. 
  \end{itemize}
  The image of $J^1$ under the homomorphism $\Theta$ lies inside  
  \begin{equation*}
    \{(g,d,\sigma) \in \GL_n(\FF_q) \times \FF_{q^n}^\times \times \Frob_q^{n\Z} \mid 
    \det(g) \Norm_{\FF_{q^n}/\FF_q}(d) = 1\}.
  \end{equation*}
\begin{proof}
\end{proof}
\end{lem}

\begin{lem}\label{lem:J1ActionOnYb}
  Through the homomorphism $\Theta$, $J$ acts on $\DL_n$ over $Y_b$.
\begin{proof}
\end{proof}
\end{lem}

\begin{prop}\label{prop:J1ActionOnAffinoid}
  The action of $J^1$ on $M_{\infty, C}^{(0)}$ stabilizes $U$ and extends to
  $\cX$. The induced action on the special fiber $\cX_s$ is compatible with the
  action of $J^1$ on $Y_b$.
\begin{proof}
\end{proof}
\end{prop}
% subsubsection Proof of Proposition \ref{prop:J1EquivInjMor} (end)

% subsection (end)

% section The Explicit Local Langlands Correspondence for Depth Zero Supercuspidal Representations (end)
\end{document}
