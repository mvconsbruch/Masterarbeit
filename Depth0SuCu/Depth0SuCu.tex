%! TeX root: ../main.tex
\documentclass[../main.tex]{subfiles}

\begin{document}
\section{Explicit Non-Abelian Lubin--Tate Theory for Depth Zero Supercuspidal
Representations} % (fold)
\label{sec:Explicit Non-Abelian Lubin-Tate Theory for Depth Zero Supercuspidal Representations}

\subsection{Deligne--Lusztig Theory for Depth Zero Representations} % (fold)
\label{sub:Deligne-Lusztig Theory for Depth Zero Representations}
The aim of this section is to outline the construction of a correspondence
between characters of $\FF_{q^n}^\times$ (with values in $\C$) and 
cuspidal representations of $\GL_n(\FF_q)$. The correspondence we construct here
is an application of a more general theory due to Deligne--Lusztig. 
In \cite{delignelusztig1976}, they construct
for any connected reductive algebraic group $G$ over $\FF_q$ and any
Frobenius-stable 
maximal torus $T \subseteq G$ a correspondence associating to certain
characters $\theta$ of $T^F$ a virtual representation $R_{T,\theta}$ of 
$G^F$. These representations arise from the $l$-adic cohomology (with $l \neq p$)
of a certain space $\DL_{G,T,w}$ admitting commuting actions by $G^F$ and $T^F$. 
We give explicit descriptions of the occuring spaces and state the cohomological 
results.

\subsubsection{Deligne--Lusztig Varieties} % (fold)
\label{ssub:Deligne--Lusztig Varieties}
We begin by introducing (full) flags and their classifying objects, flag
varieties. Let $k$ be a field and let $V$ be a finite dimensional $k$-vector space
of dimension $n$.

\begin{defi}[Flag Variety]\label{def:FlagVariety}
  Let $X: \SchOver k \to \Set$ be the functor assigning to each $k$-scheme $S$
  the set 
  \begin{equation*}
    X(S) = \left\{\cF_1 \subset \cF_2 \subset \dots \subset \cF_{n-1} \subset
    \cO_S^n \ \big \vert  \begin{array}c
      \text{For all $i$, $\cF_i$ is locally a direct} \\
      \text{summand of $\cO_S \otimes_{\cO_k} \tilde V$, locally free of rank $i$}
  \end{array} \right\}.
  \end{equation*}
  Recall that a subsheaf $\cF_i \subset \cO_S^n$ is locally a direct summand
  if it is quasi-coherent, and for each $s \in S$ there is some neighbourhood 
  $U$ of $s$ such that $\cF_i|_U$ is a direct summand of $\cO_S^n|_U$. 
  The $S$-valued points of $S$ are called families of flags over $S$.
\end{defi}
Elements of $X(k)$ are called (full) Flags. They are given by an increasing
$n-1$-tuple of vector spaces 
$$\Flag = (F_1 \subsetneq F_2 \subsetneq \dots F_{n-1} \subsetneq V) \in X(k).$$
  %By \cite[Prposition 8.10]{gortz2020algebraic} and the subsequent discussion,
  %a subsheaf $\cF \subset \cO_S^n$ being locally a direct summand is equivalent
  %to the quotient sheaf $\cO_S^n/\cF$ being locally free.
  There are natural morphisms
  \begin{equation*}
    \nu_i: X \to \Grass_{n,i}, \quad (\cF_1 \subset \dots \subset \cF_{n-1}
    \subset \cO_S^n) \mapsto (\cO_S^n \surj \cF_i^\vee),
  \end{equation*}
  where $\Grass_{n,i}$ denotes the Grassmannian as defined in
  \cite[\href{https://stacks.math.columbia.edu/tag/089R}{Tag 089R}]{stacks-project}.  

\begin{prop}\label{prop:FlagVarisProjective}
  The induced morphism of functors
  \begin{equation*}
    X \to \Grass_{n,1} \times_{\spec k} \dots \times_{\spec k} \Grass_{n,n-1}
  \end{equation*}
  is representable by a closed embedding. In particular, as $\Grass_{n,d}$ is
  representable by a projective scheme for integers $1 \leq d \leq n-1$, 
  $X$ is representable by a projective scheme. 
\begin{proof}
  This can be checked directly on the standard affine cover of the Grassmannians,
  where the condition that $F_i$ is contained in $F_{i+1}$ is a polynomial
  condition.
\end{proof}
\end{prop}

There is a natural $\GL_n$-action on $X$, induced by the natural action of 
$\GL_n(S)$ on $\cO_S^n$. Given a flag $\Flag \in X(k)$, we
write $B_\Flag \subset \GL_n$ for the isotropy subgroup of $\Flag$ under this
action. In \cite{delignelusztig1976}, the authors work with schemes
arising as quotients $G/B$ where $B$ is some Borel subgroup. 
It is elementary that $B_\Flag$ is Borel, 
and the following proposition shows that $X$ is of this form.

\begin{lem}\label{lem:FlagVarietyAsQuotient}
  The morphism of schemes $\mu_{\Flag}: \GL_n \to X$, $g \mapsto g.\Flag$
  yields an isomorphism $\GL_n/B_{\Flag} \to X$. 
\begin{proof}
  As $X$ is separated, \cite[Proposition 7.13]{milne2017algebraic} yields an 
  isomorphism $\GL_n/B_{\Flag} \to O(\Flag)$, where $O(\Flag)$
  denotes the (unique) subscheme of $X$ with underlying topological space given by
  $\mu_{\Flag}(\abs G)$. As the action is faithful on geometric points, 
  we find $\abs{O(\Flag)} = \abs X$, and the claim follows.
\end{proof}
\end{lem}

\begin{prop}\label{prop:FlagVarIsSmooth}
  The scheme $X$ is smooth, and of pure dimension $\tfrac{n(n-1)}2$. 
\begin{proof}
  This follows by the fact that $\GL_n$ is smooth and the lemma above. 
  Alternatively, this can be checked directly on the moduli problem.
\end{proof}
\end{prop}

Write $T$ for the torus of diagonal matrices in $\GL_n$. 

\begin{defi}[Functor of Marked Flags]\label{def:MarkedFlagVar}
  Let $Y$ be the functor $\SchOver X \to \Set$ given by sending a morphism
  $S \to X$, corresponding to a family of flags $(\cF_i)_i \in X(S)$, to the set
  \begin{equation*}
    Y(S) = \{(e_i)_{1 \leq i \leq n-1} \mid e_i: \cO_S \to \cF_i/\cF_{i-1}
    \text{ is an isomorphism}\}.
  \end{equation*}
  Here, $\cF_0$ is the zero-sheaf.
\end{defi}
Just like $X$, the functor $Y$ comes with a natural action by $\GL_n$
and the natural morphism
\begin{equation}\label{eq:naturalmapYtoX}
  Y(S) \to X(S), \quad (\cF_i, e_i)_i \mapsto (\cF_i)_i
\end{equation}
is equivariant for this action. One readily checks that $Y$ is a sheaf in
$\SchOver X_{\mathrm{Zar}}$, the big Zariski site of schemes over $X$.

\begin{lem}\label{prop:RepresentabilityOfYMarkedFlags}
  The morphism described in \eqref{eq:naturalmapYtoX} turns $Y$ into a
  $T \times_k X$-torsor \begin{proof}
  For any $(\cF_i)_i \in X(S)$, the quotient sheaves $\cF_i/\cF_{i-1}$ are
  (Zariski-) locally free of rank $1$. Hence there is a Zariski-cover $\sqcup_j
  \spec R_j \to S$
  with $(\cF_i/\cF_{i-1})|_{\spec R_j} \cong \cO_{\spec R_j}$ for all $i$. The claim
  follows as on $\spec R_j$, any two isomorphisms $e_i$, $e_i'$ differ by a
  unit in $R_j$, so any tuple of isomorphisms differs by an element in $(R_j^\times)^n
  \cong T(R_j)$.
\end{proof}
\end{lem}

As an immediate corollary, we obtain the following.
\begin{prop}\label{prop:RepAndPropsOfY}
  The functor $Y$ is representable, the morphism $Y \to X$ is smooth and affine.
\begin{proof}
  Let $\Flag^\univ$ be the universal flag coming from the identity on $X$. Then
  we may pick a Zariski-cover $X' = \sqcup_j U_j \to X$ trivializing $Y$, so
  that (as functors)
  $Y \times_X X' \cong T \times_{\spec k} X \times_X X'$. Hence $Y \times_X U_j$ is
  representable for all $j$. By the Yoneda-lemma and the fact that $Y$ is a sheaf over 
  $X$ we obtain natural glueing data
  \begin{equation*}
    \phi_{j k}: (Y|_{U_j})|_{U_j \cap U_k} \xto \sim (Y|_{U_k})|_{U_j \cap U_k}.
  \end{equation*}
  These are readily seen to satifsy the cocycle condition. This turns $Y$ into a 
  scheme over $X$. As $Y$ is locally on $X$ isomorphic to $T \times_{\spec k}
  X$, it is relatively affine and smooth.
\end{proof}
\end{prop}

Just like $X$, the scheme $Y$ admits representations as quotient of algebraic groups.
Let $\Flag \in X(k)$ be a flag and $e_\bullet$ be a marking. Let 
$U_{\Flag, e_\bullet} \subset \GL_n$ be the isotropy subgroup of the corresponding
element of $Y(k)$ under the action of $\GL_n$.
\begin{lem}\label{lem:MarkedFlagVarietyAsQuotient}
  In this situation, $Y \cong \GL_n/U_{\Flag, e_\bullet}$.
\begin{proof}
  This can be shows using the same arguments as in the proof of
  Lemma \ref{lem:FlagVarietyAsQuotient}.
\end{proof}
\end{lem}

We next recall basic facts of the Bruhat decomposition for $\GL_n$. 
Consider the diagonal action of $\GL_n$ on $X \times X$.
\begin{prop}\label{prop:BruhatDecomp}
  There is a decomposition into disjoint locally closed subschemes
  \begin{equation*}
    X \times X = \bigcup_{\sigma \in \Sigma_n}^{\cdot} O_\sigma.
  \end{equation*}
  Here, $O_\sigma$ is equal to the orbit of the element 
  $(\Flag, \sigma.\Flag) \in (X \times X)(k)$. Each $O_\sigma$ is smooth of dimension
  $\dim(X) + l(\sigma)$, where $l(\sigma)$ denotes the coxeter length of $\sigma$.
\begin{proof}
  See \red{Reference. Also, should follows from usual bruhat decomp.}.
\end{proof}
\end{prop}

Given a basis $(v_1, \dots, v_n)$ of $V$, we write $\Flag(v_1, \dots, v_n) \in X(k)$ for
the flag
\begin{equation*}
  0 \subsetneq \langle v_1 \rangle \subsetneq \langle v_1, v_2 \rangle 
  \subsetneq \dots \subsetneq \langle v_1, \dots, v_n \rangle = V. 
\end{equation*}
It is equipped with a natural marking, given by the residue classes of the
vectors $v_i$. 


We next introduce the concept of relative position. 

\begin{defi}[Relative Position]\label{def:RelPos}
  A pair of flags $(\Flag,\Flag') \in X(k) \times X(k)$ is in relative position
  $\sigma \in \Sigma_n$,
  if there is a basis $(v_1, \dots, v_n)$ of $V$ such that for each $i = 
  1, \dots, n-1$, the sub-vector space $F_i \subset V$ is equal to the span of
  the vectors $v_1, \dots, v_i$ and $F'_i \subset V$ is equal to the span 
  of the vectors $v_{\sigma(1)}, \dots, v_{\sigma(i)}$.
\end{defi}
\begin{prop}\label{prop:RelativePosWellDef}
  For any pair of flags $(\Flag, \Flag')$, there is a unique $\sigma \in 
  \Sigma_n$ such that $(\Flag, \Flag')$ is in relative position $\sigma$. 
\begin{proof}
  This statement is equivalent to the Bruhat decomposition for $\GL(V)$. See 
  \cite[Chapter 21]{milne2017algebraic} for details.
\end{proof}
\end{prop}
There is a scheme-theoretic analogue of \eqref{eq:BruhatDecompInOrbits}. 


We now specialize to the case where $k = \bar \FF_q$ is an algebraic closure of 
the finite field with $q$ elements. Then $X = X_0 \times_{\spec \FF_q} \spec \bar \FF_q$,
where $X_0$ is the variety of full flags over $\FF_q$. Hence $X$ comes with
the (absolute) Frobenius map $\Frob: X \to X$ defined over $\FF_q$. 
Write $\gamma_{\Frob}: X \to X \times_{\spec \FF_q} X$ for the morphism sending
a family of flags $\Flag$ to the pair $(\Flag, \Frob (\Flag))$. 
For $\sigma \in \Sigma_n$, we define the space
\begin{equation*}
  X_\sigma \coloneqq O_\sigma \times_{X \times_{\spec \FF_q}X, \gamma_\Frob} X.
\end{equation*}
On $k$-rational points of $X$, $\Frob$ acts by sending a flag $\Flag(v_1,
\dots, \allowbreak v_n)$
to the flag $\Flag(\Frob(v_1), \dots, \allowbreak \Frob(v_n))$. Thereby, the
set of $k$-rational points $X_\sigma(k)$ is given by those flags $\Flag$ which
are in relative position $\sigma$ to the flag $\Frob(\Flag)$. The group
$\GL_n(\FF_q)$ naturally acts on $X_\sigma(k)$. More generally, the group sheaf
of Frobenius-invariants $\GL_n^\Frob$ acts on $X_\sigma$. 

We obtain 

% subsubsection Deligne--Lusztig Varieties (end)

\begin{defi}[Deligne--Lusztig Variety for $\GL_n(\FF_q)$]\label{def:DLVariety}
  \todo{}
\end{defi}

\begin{lem}
  We have $$\DL_n = \coprod_{b} Y_b,$$
  where $b$ runs over the set elements in $\FF_{q^n}^\times$ satisfying 
  $b^{q-1} = (-1)^{n-1}$. 
\end{lem}

\begin{equation} \label{eq:defDLCohom}
  \HDL \coloneqq \hH^{n-1}_c(\DL_n^\perf, \bar \Q_l) \quad \text{and} \quad
  \HYb = \hH^{n-1}_c(\Yb^\perf, \bar \Q_l).
\end{equation}

\begin{defi}[Regular Character on $\FF_{q^n}^\times$]
  We say that a character $\theta: \FF_{q^n}^\times \to \C^\times$ is 
  regular if it does not factor through the norm morphism
  $\FF_{q^n}^\times \to \FF_{q^m}^\times$ for any $m \leq n$.
\end{defi}

\begin{prop}[Deligne--Lusztig Correspondence] \label{prop:DLCorrespondence}
  Let $\theta: \FF_{q^n}^\times \to \C^\times$. be a regular character.
  As representations of $\GL_n(\FF_q) \times \FF_{q^n}^\times$, there is an
  isomorhphism 
  \begin{equation*}
    {\HDL}_{, \theta} \cong  R_\theta \boxtimes \theta,
  \end{equation*}
  where $R_\theta$ is an \red{irreducible?} representation of $\GL_n(\FF_q)$. 
\end{prop}

\begin{prop}
  We also have a natural action of $\Frob_q^{n\Z}$ on $\HDL$. Here, $\Frob_q^n$
  acts by the scalar $(-1)^{n-1} q^{\frac{n(n-1)}2}$. 
  \begin{proof}
    This is somewhere in \cite{digne1985fonctions}, according to \cite[Lemma
    5.10]{mieda2016geometric}.
  \end{proof}
\end{prop}

\begin{thm}[]
  For a regular character $\theta: \FF_{q^n}^\times \to \C^\times$, the 
  $\theta$-isotypic component of $\HDL$ is, as a representation of 
  $\GL_n(\FF_q) \times \FF_{q^n}^\times \times \Frob_q^{n\Z}$, given by 
  \begin{equation*}
    {\HDL}_{,\theta} = R_\theta \boxtimes \theta \boxtimes (\delta^{n-1})(\tfrac{1-n}2).
  \end{equation*}
  Here, $\delta: W_{F_n} \to \{\pm 1\}$ is the non-trivial quadratic unramified character.
  \red{QUESTION: How do we make sense of half tate-twists?}
\end{thm}

% subsection Deligne-Lusztig Theory for Depth Zero Representations (end)

\subsection{The Special Affinoid} % (fold)
\label{sub:The Special Affinoid}
\begin{defi}[The special Affinoid]
  \todo{}
\end{defi}

% subsection The Special Affinoid (end)

\subsection{The Explicit Correspondence} % (fold)
\label{sub:The Explicit Correspondence}
Fix, for the remainder of the section, a regular character $\theta: \FF_{q^n}^\times
\to \C^\times$. Here, regular means that $\theta$ does not factor through
the norm map $\Norm_{\FF_{q^n}/\FF_{q^m}}: \FF_{q^n}^\times \to
\FF_{q^m}^\times$ for any $m \leq n$. 
The datum of $\theta$ can be used to construct representations 
of $W_F$ and $D^\times$ and, making use of Deligne--Lusztig
theory, a representation of $\GL_n(F)$. We proceed as follows.
\begin{itemize}
  \item Let $\bar \tau_\theta$ be the character of $\Weil_{F_n}$ given by 
    the composition
    \begin{equation*}
      \Weil_{F_n} \to \Weil_{F_n}^\ab \xto{\Art_{F_n}^{-1}} F_n^\times \cong
      \Z \times \cO_{F_n}^\times \surj \FF_{q^n}^\times \xto \theta \C^\times
    \end{equation*}
  and put $\tau_\theta = \cInd_{\Weil_{F_n}}^{\Weil_F}(\bar \tau_\theta)$.
  \item Let $\bar \rho_\theta$ be the character on 
    $F^\times \cO_D^\times$ given by the composition
    \begin{equation*}
      F^\times\cO_D^\times \cong \varpi^\Z \times \cO_D^\times \surj 
      \FF_{q^n} \xto \theta \C^\times
    \end{equation*}
    and let $\rho_\theta = \cInd_{F^\times \cO_D^\times}^{D^\times}(\bar \rho_\theta)$.
  \item Let $\bar \pi_\theta$ be the representation of $F^\times \GL_n(\cO_F)$
    arising from post-composing $R_\theta$ (cf. Definition
    \ref{prop:DLCorrespondence}) with the composition
    \begin{equation*}
      F^\times \GL_n(\cO_F) \cong \varpi^\Z \times \GL_n(\cO_F)
      \surj \GL_n(\cO_F) \surj \GL_n(\FF_q).
    \end{equation*}
    Let $\pi_\theta = \cInd_{F^\times \GL_n(\cO_F)}^{\GL_n(F)}(\bar \pi_\theta)$. 
\end{itemize}

\begin{lem}\label{lem:BarRepsAreSmooth}
  The representations $\bar \pi_\theta$, $\bar \rho_\theta$ and $\bar
  \tau_\theta$ are smooth, in particular
  $\pi_\theta$, $\rho_\theta$ and $\tau_\theta$ are smooth as well. Additionally, the
  representations $\pi_\theta$ and $\rho_\theta$ are irreducible, and
  $\pi_\theta$ is supercuspidal.
\begin{proof}
  By design, $\bar \pi_\theta$ is trivial on the compact open subgroup $1 +
  \varpi \Mat_{n \times n}(\cO_F)$ of $F^\times \GL_n(\cO_F)$. Similar
  statements hold for $\bar \rho_\theta$
  and $\bar \tau_\theta$. 
  \red{why is $\rho_\theta$ irreducible? Why is $\pi_\theta$ supercuspidal and
  irreducible?}
\end{proof}
\end{lem}

The aim of this section is to prove the following statement.
\begin{thm}[Explicit Non-Abelian Lubin--Tate Theory for Depth Zero Supercuspidal
  Representations]\label{thm:MainRes1}
  The representation $\JL(\pi_\theta)$ of $D^\times$ and the representation
  $\rec_F(\pi_\theta)$ of $\Weil_F$ take the form
  \begin{equation*}
    \JL(\pi_\theta) = \rho_{\theta}
    \quad \text{and} \quad \rec_F(\pi_\theta) = \Ind_{\Weil_{F_n}}^{\Weil_F} 
    (\tau_\theta \, \delta^{n-1}),
  \end{equation*}
  where $\delta: \Weil_{F_n} \to \{\pm 1\}$ is the unramified quadratic
  character. This is the character corresponding to $a \mapsto
  (-1)^{\val_{F_n}(a)}$ under the isomorphism $\Art_{F_n}: F_n^\times \to
  \Weil_{F_n}^\ab$. 
\end{thm}

We set 
\begin{equation} \label{eq:defJandJ1}
  J \coloneqq F^\times \GL_n(\cO_F) \times \cO_D^\times \times \Weil_{F_n}
  \text{ and } J^1 = J \cap G^1.
\end{equation}
Also, we define a morphism 
\begin{equation*}
  \Theta: J \to \GL_n(\FF_q) \times \FF_{q^n}^\times \times {\Frob_q^{n\Z}}, \quad
  (\varpi^m g,d,\sigma) \mapsto (\bar g, \bar {d^{-1}u_\sigma^{-1}}, \bar{\sigma}).
\end{equation*}

Recall that $H_\DL$ denotes the middle $l$-adic cohomology of 
$\DL_n$, cf. Section \ref{sub:Deligne-Lusztig Theory for Depth Zero Representations}.
\begin{lem}
  The morphism $\Theta$ makes $J$ act on ${\hH}_{\DL,\theta}$. This representation is 
  of the form 
  \begin{equation*}
    (g,d,\sigma) \mapsto \bar \pi_\theta(g) \otimes \bar \rho_{\theta^{-1}}(d)
    \otimes \left(\bar \tau_\theta \, \delta^{n-1}\right)^{-1}(\tfrac{1-n}2)(\sigma).
  \end{equation*}
  This representation is smooth.
\begin{proof}
  This is a direct calculation.
\end{proof}
\end{lem}

The input we get from Mieda's result is the following.
\begin{prop}\label{prop:J1EquivInjMor}
  There is an injective morphism of  $J^1$-representations
  \begin{equation*} 
    \Res_{J^1}^J( \hH_{\DL,\theta} ) \inj \Res_{J^1}^{G^1}( \hH'_{\LT}).
  \end{equation*}
\begin{proof}
  \red{This is \cite[Proposition 5.11]{mieda2016geometric}.}
\end{proof}
\end{prop}

\begin{lem} \label{lem:JEquivInjMor}
  The morphism in Proposition \ref{prop:J1EquivInjMor} naturally gives rise to
  an injective $J$-equivariant morphism
  \begin{equation*}
    \hH_{\DL,\theta} \inj \Res_J^G \hH_\LT.
  \end{equation*}
\begin{proof}
  We construct a sequence of $J$-equivariant injections
  \begin{equation*}
    \hH_{\DL,\theta} \inj \Ind_{J^1}^J ( \Res_{J^1}^J \hH_{\DL,\theta}) \inj
    \Ind_{J^1}^J (\Res_{J^1}^{G^1} \hH'_{\LT}) \\ \xto \sim \Res_J^{G^1J}(
    \Ind_{G^1}^{G^1J} \hH'_\LT) 
 \inj \Res_J^G \hH_\LT.
  \end{equation*}
  \textit{The first morphism.}
  This is the unit of the adjunction $\Res_{J^1}^J \ladj \Ind_{J^1}^J$ 
  applied at $\hH_{\DL,\theta}$, which is injective by Lemma
  \ref{lem:FrobRecUnitsAreInjective}.

  \textit{The second morphism.}
  This is $\Ind_{J^1}^J$ applied to the injective morphism in Proposition
  \ref{prop:J1EquivInjMor}. The resulting morphism is injective because
  $\Ind_{J^1}^J$ is exact, cf. Proposition
  \ref{prop:InducedRepresentationExact}. 

  \textit{The third morphism.} 
  The morphism is given by the inverse of the base-change morphism constructed in
  Lemma \ref{lem:BaseChangeForResInd}, which is applied with $H = G^1$, $N = J$. 
  Note that $G^1$ is normal in $G$, so the assumptions of the Lemma are
  satisfied. As $J$ is open in $G$, the map is an isomorphism. 
  
  \textit{The fourth morphism.}
  Since $G^1J$ is open in $G$, the unit of the adjunction $\cInd_{G^1J}^G \ladj
  \Res_{G^1J}^G$ yields a monomorphism of $G^1J$-representations 
  \begin{equation}\label{eq:MiedaEx1Morph1}
    \Ind_{G^1}^{G^1J} \hH_\LT' \to \Res_{G^1J}^G
    (\cInd_{G^1J}^G(\Ind_{G^1}^{G^1J} \hH'_\LT)).
  \end{equation}
  As $G^1J$ co-compact in $G$, we have $\cInd_{G^1J}^G =
  \Ind_{G^1J}^G$, so the right-hand
  side is isomorphic to $\Res_{G^1J}^G(\Ind_{G^1}^G \hH'_\LT) \cong
  \Res_{G^1J}^G(\hH_\LT)$ by Proposition \ref{prop:InductionOnTower} and 
  Lemma \ref{lem:InductionStatementOnHLT}. Hence, applying $\Res_J^{G^1J}$ to the
  morphism in \eqref{eq:MiedaEx1Morph1} yields the desired map.
\end{proof}
\end{lem}

The morphism constructed in Lemma \ref{lem:JEquivInjMor} yields, by Frobenius
reciprocity, a non-zero map of $G$-representations
\begin{equation} \label{eq:WantedMap}
  \Ind_{J}^G (\hH_{\DL,\theta}) \cong \pi_\theta \boxtimes \rho_{\theta^{-1}}
  \boxtimes (\tau_\theta \, \delta^{n-1})^{-1}(\tfrac{1-n}2) \to \hH_\LT.
\end{equation}
As $\pi_\theta$ is supercuspidal and its central character is trivial on 
$\varpi^\Z$, Theorem \ref{thm:NonAbLTT} yields a non-zero map
\begin{equation*}
  \rho_{\theta^{-1}} \boxtimes (\tau_\theta \, \delta^{n-1}) \to
  \JL(\pi_\theta)^\vee \boxtimes \rec_F(\pi_\theta)^\vee.
\end{equation*}
As $\rho_{\theta^{-1}}$ and $\JL(\pi_\theta)^\vee$ are irreducible, this implies
$\JL(\pi_\theta) = \rho_{\theta^{-1}}^\vee = \rho_\theta$. As 
$\rec_F(\pi_\theta)$ is irreducible and $\dim(\tau_\theta) = n = \dim
(\rec_F(\pi_\theta))$, this also implies $\tau_\theta \, \delta^{n-1} =
\rec_F(\pi_\theta)$. Admitting Proposition \ref{prop:J1EquivInjMor}, this
concludes the proof of Theorem \ref{thm:MainRes1}.

\subsubsection{Proof of Proposition \ref{prop:J1EquivInjMor}} % (fold)
\label{ssub:Proof of Proposition}

\begin{lem}\label{lem:GeneratorsForJ1}
  The group $J^1$ is generated by the following elements.
  \begin{itemize}
    \item $(g,1,1)$ for $g \in \GL_n(\cO_F)$ with $\deg g = 1$,
    \item $(1,d,1)$ for $d \in \cO_D^\times$ with $\Nrd d = 1$, 
    \item $(a,a,1)$ for $a \in F_n^\times$, 
    \item $(1, \Art^{-1}_{F_n}(\sigma)^{-1}, \sigma)$ for $\sigma \in W_{I_n}$, 
    \item and $(1, \varpi^{-1}, \sigma)$ for $\sigma \in W_{F_n}$ with
              $\Art^{-1}_{F_n}(\sigma) = \varpi$. 
  \end{itemize}
  The image of $J^1$ under the homomorphism $\Theta$ lies inside  
  \begin{equation*}
    \{(g,d,\sigma) \in \GL_n(\FF_q) \times \FF_{q^n}^\times \times \Frob_q^{n\Z} \mid 
    \det(g) \Norm_{\FF_{q^n}/\FF_q}(d) = 1\}.
  \end{equation*}
\begin{proof}
\end{proof}
\end{lem}

\begin{lem}\label{lem:J1ActionOnYb}
  Through the homomorphism $\Theta$, $J$ acts on $\DL_n$ over $Y_b$.
\begin{proof}
\end{proof}
\end{lem}

\begin{prop}\label{prop:J1ActionOnAffinoid}
  The action of $J^1$ on $M_{\infty, C}^{(0)}$ stabilizes $U$ and extends to
  $\cX$. The induced action on the special fiber $\cX_s$ is compatible with the
  action of $J^1$ on $Y_b$.
\begin{proof}
\end{proof}
\end{prop}
% subsubsection Proof of Proposition \ref{prop:J1EquivInjMor} (end)

% subsection (end)

% section The Explicit Local Langlands Correspondence for Depth Zero Supercuspidal Representations (end)
\end{document}
