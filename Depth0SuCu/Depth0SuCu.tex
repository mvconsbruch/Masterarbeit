%! TeX root: ../main.tex
\documentclass[../main.tex]{subfiles}

\begin{document}
\section{Explicit Non-Abelian Lubin--Tate Theory for Depth Zero Supercuspidal
Representations} % (fold)
\label{sec:Explicit Non-Abelian Lubin-Tate Theory for Depth Zero Supercuspidal Representations}

In this final section, we explicitely calculate the representations
arising in Theorem \ref{thm:NonAbLTT} for depth zero supercuspidal
representations $\pi \in \Rep{\GL_n(F)}$, which essentially are representations
of $\GL_n(F)$
obtained by compact induction from Deligne--Lusztig theory. In
particular, we give explicit descriptions of $\rec_E(\pi)$ and $\JL(\pi)$. 
The main input is the construction of an affinoid in the Lubin--Tate perfectoid
space whose special fiber is isomorphic to the perfection of the 
Deligne--Lusztig variety constructed in Section \ref{sub:An Explicit Example}.
Mieda's results give a relation between the $\ell$-adic cohomology of $\DL_n$
and the representation $\HLT$, which allows to make some parts of $\HLT$
explicit.

\subsection{The Special Affinoid and its Formal Model} % (fold)
\label{sub:The Special Affinoid}
We define the subspace $U \subset M_{\infty,\cO_{\C_p}}^{(0)}$ as the rational
subset cut out by the inequalities $\abs{X_i} \leq \abs{\varpi}^{1/q^n-1}$ for
$i = 1, \dots, n$. We construct formal model $\cX \in \FSchOver {\cO_{\C_p}}$
of $U$ whose special fiber $\cX_s = 
\cX \times_{\spf(\cO_{\C_p})}\spec(\bar \FF_q)$ is isomorphic to the perfection of the
Deligne--Lusztig variety $\DL_n$ constructed in Section \ref{sub:An Explicit Example}.

The formal model $\cX = \spf C_n$ is defined as follows. Let $H$ be the
standard formal $\cO_E$-module over $\cO_\br E$ and let 
$$y = (y_1, y_2, \dots) \in T_\varpi H(\cO_\Cp)$$ be a primitive element, that is,
an element with $y_1 \neq 0$ and $[\varpi]_H(y_1) = 0$. Write $\xi = \lambda(y)$ 
for the corresponding element of $\Nilp^\flat(\cO_\Cp)$. 

\begin{lem}\label{lem:PropertiesOfXi}
  We have 
  \begin{equation*}
    \log_H(y) = \sum_{i = -\infty}^ \infty \frac{\xi^{q^in}}{\varpi} = 0
    \quad \text{and} \quad
    \abs \xi^{q^n-1} = \abs \varpi.
  \end{equation*}
\begin{proof}
  The first identity follows from Lemma \ref{lem:LogInTermsOfNil} and the fact that
  $\log_H(y_1)$ vanishes, cf. Lemma \ref{lem:KernelOfLog}. 
  For the second one, note that $[\varpi]_H(y_1) = 0$ and 
  $[\varpi]_H(y_{k+1}) = y_k$ implies that 
  $\abs{y_k} = \abs \varpi^{\frac{1}{q^{nk}(q^n-1)}}$. Hence, the claim follows by
  the equality $\xi = \lim_{k \to \infty} y_k^{q^{nk}}$, cf. Lemma
  \ref{lem:GroupStructureOnNilp}.
\end{proof}
\end{lem}

Let $E_n$ be the degree $n$ unramified extension of $E$, and let 
$(\alpha_1, \dots, \alpha_n)$ be a basis of $\cO_{E_n}$ over $\cO_E$. 
Let $t \coloneq \delta(\alpha_1 y, \dots, \alpha_n y) \in (T_\varpi {\wedge H})(\cO_\Cp)$
and write $\tau \in \Nilp^\flat(\cO_\Cp)$ for the corresponding system of $q$-th 
power roots. Also, the choice of $(\alpha_1, \dots, \alpha_n)$ lets us 
identify $E_n$ as a subfield of $\GL_n(E)$ and $D^\times$. 

\begin{lem}\label{lem:PropertiesOfTau}
  We have 
  \begin{equation*}
    \log_{\wedge H}(t) = \sum_{i = -\infty}^\infty (-1)^{i(n-1)} \frac{\tau^{q^i}}{\varpi^i} = 0
    \quad \text{and} \quad
    \abs \tau^{q-1} = \abs \varpi.
  \end{equation*}
  Furthermore, we have the congruence
  \begin{equation*}
    \tau \equiv \det(\alpha_i^{q^j}) \xi^{1 + q + \dots + q^{n-1}}
  \end{equation*}
  modulo the ideal generated by elements $z \in \cO_\Cp$ with with 
  $\abs{z} < \abs {\xi}^{1 + q + \dots + q^{n-1}}$. 
\begin{proof}
  We have $t \neq 0$ and $[\varpi]_{\wedge H}(t) = 0$, so the first two
  assertions follow just as above in the proof of Lemma
  \ref{lem:PropertiesOfXi}, appropiately adjusted. The congruence is a
  corollary of the approximation of $\Delta$ in Lemma \ref{lem:DeltaApprox}.
\end{proof}
\end{lem}

We now define the formal model $\cX$. We abbreviate with $(x_i^{q^{-m}})_{m \in
\N_0}$ the system of 
$q$-th power roots $(X_i^{q^{-m}}/\xi^{q^{-m}})_{m \in \N_0}$ of elements in
$\cO_\Cp \llbr X_1^{q^{-\infty}}, \dots, X_n^{q^{-\infty}}\rrbr$, and define
systems of $q$-th power roots $\Delta'$ and $\tau'$ as
\begin{equation*}
  \begin{array}c
  \Delta'(x_1, \dots, x_n)^{q^{-m}} \coloneq (\xi^{q^{-m}})^{-(1 + q + \dots +
  q^{n-1})} \Delta(\xi x_1, \dots, \xi x_n) \in \cO_\Cp \llbr x_1^{q^{-\infty}}, \dots
  , x_n^{q^{-\infty}}\rrbr  \\
  \text{and} \\
  \tau'^{q^{-m}} \coloneq (\xi^{q^{-m}})^{-(1 + q + \dots + q^{n-1})} \tau^{q^{-m}} \in
  \cO_\Cp^\times.
  \end{array}
\end{equation*}

\begin{lem}\label{lem:PropsOfDeltaPrAndTauPr}
  Let $\fm_\Cp$ denote the maximal ideal of $\cO_\Cp$. We have
  \begin{equation*}
    \Delta'(x_1, \dots, x_n)^{q^{-m}} \equiv (\det(x_i^{q^j})_{1 \leq i,j \leq
    n})^{q^{-m}} \mod \fm_\Cp.
  \end{equation*}
  In particular, as $\Delta$ has coefficients in $\cO_\br E$, we have 
  $\Delta'(x_1, \dots, x_n)^{q^{-m}} \in \cO_\Cp \langle x_1^{q^{-\infty}}, \dots
  , x_n^{q^{-\infty}}\rangle.$
  Also, $\tau'^{q^{-m}} \in \cO_\Cp^\times$. 
\begin{proof}
  The statement for $\Delta$ follows directly from the approximation in Lemma
  \ref{lem:DeltaApprox}. One easily checks $\abs \tau = 1$, implying the second
  statement.
\end{proof}
\end{lem}

We define 
\begin{equation*}
  \cX \coloneq \spf \left(
    \frac
    {\cO_\Cp \langle x_1^{q^{-\infty}}, \dots , x_n^{q^{-\infty}}
      \rangle}
    {(\Delta'(x_1, \dots, x_n)^{q^{-m}} - \tau'^{q^{-m}} \mid m \in \N_0)^-
      }\right)
\end{equation*}

\begin{prop}\label{prop:AffinoidIsFormalModel}
  The formal scheme $\cX$ is a formal model for $U$. 
\end{prop}
\begin{proof} 
  By definition, $U$ is isomorphic to the affinoid adic space
  \begin{equation*}
    \Spa(B_n[\tfrac 1\varpi], \tilde B_n) \quad \text{where} \quad 
    B_n \coloneqq \frac{\cO_\Cp \langle x_1^{q^{-\infty}}, \dots,
    x_n^{q^{-\infty}} \rangle}
    {\Delta^{q^{-m}}(\xi x_1, \dots, \xi x_n) - \tau^{q^{-m}} \mid m \in \N)^-}
  \end{equation*}
  and $\tilde B_n \subset B_n[\tfrac 1\varpi]$ denotes the normalization of 
  $B_n$ inside $B_n[\tfrac 1 \varpi]$. 
  We set 
  $$B'_n \coloneqq \frac{\cO_\Cp \langle x_1^{q^{-\infty}}, \dots,
  x_n^{q^{-\infty}} \rangle} {\Delta'(x_1, \dots, x_n)^{q^{-m}} -
  \tau'^{q^{-m}} \mid m \in \N)^-}.$$
  As $\xi$ is invertible in $B_n[\tfrac 1 \varpi]$, we have 
  $B'_n[\tfrac 1 \varpi] = B_n[\tfrac 1 \varpi]$. Furthermore, the images of 
  $B_n$ and $B'_n$ inside $B_n[\tfrac 1 \varpi]$ agree. The claim follows.
\end{proof}

%\begin{rmk} 
%  We give the following interpretation of this result. 
%  Disregarding finiteness conditions, $\cX$ naturally embedds into the
%  formal admissible blowing (cf. \red{cite Bosh} for this notion, although they
%  only define the formal admissible blowing up of formal schemes topologically
%  of finite type) up of $\cM_{\infty, \cO_\Cp}^{(0)}$ along the closed
%  subscheme corresponding to the ideal of $A_{\infty,\cO_\Cp}$ given by 
%  $\cI = (\xi^{q^{-m}}, X_1^{q^{-m}}, \dots, X_n^{q^{-m}} \mid m \in \N)^{-}$.
%  In light of Raynaulds view on the interplay between (reasonable) rigid spaces
%  over a non-Archimedian local field and (reasonable) formal schemes over 
%  its ring of integers, this realizes the inclusion
%  $U \inj M_{\infty, \Cp}^{(0)}$ as the inclusion of formal models
%  \begin{equation*}
%    \cX \inj \Bl_\cI (\cM_{\infty, \cO_\Cp}^{(0)}).
%  \end{equation*}
%\end{rmk}

\begin{prop}\label{prop:SpecialFiberOfAffinoidIsLusztig}
  Let $b^{1/q^m}$ be the residue class of $\tau'^{1/q^m}$ in $\Cp/\fm_{\Cp}$. 
  Then $b^{q-1} = (-1)^{n-1}$, and the special fiber of $\cX_s \coloneq \cX
  \times_{\spf \cO_\Cp} \spec (\Fqbar)$ is isomorphic to $Y_b^\perf$, the
  perfection of the component $Y_b \subset \DL_n$ defined in \eqref{eq:DefYb}.
\begin{proof}
  We first show that $b^{q^{-m}(q-1)} = (-1)^{n-1}$ for $m \in \Z$. 
  This follows from Lemma \ref{lem:PropertiesOfTau}. 
  Write $\bar \alpha_i \in \FF_{q^n}$ for the residue classes of the 
  elements $\alpha_i$ for $i = 1, \dots, n$. We have $\bar \alpha_i^{q^n}
  = \bar \alpha_i$, implying for $m \in \Z$ the equalities
  \begin{equation*}
    b^{q^{-(m-1)}} = \det(\bar \alpha_i^{q^j})_{1 \leq i,j \leq n}^{q^{-m}} =
    (-1)^{n-1} \det (\bar \alpha_i^{q^{j-1}})_{1 \leq i,j \leq n}^{q^{-m}} =
    (-1)^{n-1} b^{q^{-m}}.
  \end{equation*}
  This gives $b^{q^{-m}}(q-1) = (-1)^{n-1}$, as desired.
  By Lemma \ref{lem:PropsOfDeltaPrAndTauPr}, we find that $\cX_s$ is equal to
  \begin{equation*}
    \spec \left( \frac{\Fqbar [x_1^{q^{-\infty,}}, \dots,
    x_n^{q^{-\infty}}]}{(\det(x_i^{q^{j-1}})^{q^{-m}}- b^{q^{-m}}} \right),
  \end{equation*}
  which is precisely the perfection of $Y_b$.
\end{proof}
\end{prop}

We also have
\begin{prop}\label{prop:FlatnessOfFormalModel}
  The formal model $\cX$ is flat over $\spf(\cO_\Cp)$. 
\begin{proof}
  \red{todo; this is well-documented in \cite{mieda2016geometric}}.
\end{proof}
\end{prop}


% subsection The Special Affinoid (end)

\subsection{Comparison of the Group Actions} % (fold)
\label{sub:Proof of Proposition}
In Section \ref{sec:Non-Abelian Lubin-Tate Theory: An Overview}, we saw that
the Lubin--Tate perfectoid space 
$$M_{\infty, \C_p}^{(0)} = \cM_{\infty} \times_{\Spa(\hat E^\ab, \cO_{\hat
E^\ab})} \Spa(\Cp, \cO_\Cp)$$ 
admits a right action by the group
\begin{equation*}
  G^1 \subset G = \GL_n(E) \times D^\times \times W_E,
\end{equation*}
given by those elements $(g,d,\sigma)$ satisfying $\det(g) \Nrd(d)^{-1} \Art_E^{-1}(\sigma|_\br E) = 1$.
In this section, we construct a subgroup $J^1 \subset G^1$ stabilizing the
special affinoid $U$ constructed above, and we furthermore show that 
the action of $J^1$ on $U$ extends to an action of $J^1$ on the formal model
$\cX$. This induces a right action on the special fiber
$\cX_s$ of $\cX$, and as $\cX_s \cong Y_b^\perf \subset \DL_n^\perf$, this
yields an action of $J^1$ on the perfection of a part of the Deligne--Lusztig
variety constructed in Section \ref{sub:An Explicit Example}. Recall that the group 
\begin{equation*}
  \bar J^1 = (\GL_n(\FF_q) \times \FF_{q^n}^\times)^1 \times n\Z
\end{equation*}
acts on $Y_b$, and in particular on 
$Y_b^\perf$. We will show that the action of $J^1$ on $Y_b^\perf$
factors through a certain homomorphism $\Theta: J^1 \to \bar J^1$. These
results lay the representation-theoretic ground for the comparison between the
representations $\HLT$ and $\HDL$.

We set 
\begin{equation} \label{eq:defJandJ1}
  J \coloneqq F^\times \GL_n(\cO_F) \times \cO_D^\times \times \Weil_{F_n}
  \text{ and } J^1 = J \cap G^1.
\end{equation}
Also, we define a morphism $\Theta$ as follows. For $\sigma \in \Weil_{E_n}$ with
$\sigma|_\br E = \Phi^{n_\sigma}$ and $u_\sigma \coloneq \varpi^{-n_\sigma}
\Art^{-1}_{E_n}(\sigma|_{\hat E_n^\ab}) \in \cO_{E_n}^\times$, we set
\begin{equation*}
  \Theta \colon J \to \GL_n(\FF_q) \times \FF_{q^n}^\times \times {n\Z}, \quad
  (\varpi^m g,d,\sigma) \mapsto (\bar g, \bar {d^{-1}u_\sigma^{-1}}, n_\sigma).
\end{equation*}

It will be convenient to have a list of generators for the group $J^1$. 
\begin{lem}\label{lem:GeneratorsForJ1}
  The group $J^1$ is generated by elements of the form
  \begin{itemize}
    \item $(g,1,1)$ for $g \in \GL_n(\cO_E)$ with $\det g = 1$; for those
      elements we have $\Theta(g,1,1) = (\bar g, 1,0)$
    \item $(1,d,1)$ for $d \in \cO_D^\times$ with $\Nrd d = 1$; for those
      elements we have $\Theta(1,d,1) = (1 , \bar d,0)$
    \item $(a,a,1)$ for $a \in F_n^\times$; for those
      elements we have $\Theta(a,a,1) = (\bar a, \bar a,0)$
    \item $(1, \alpha^{-1}, \sigma)$ for $\sigma \in I_{E_n}$ 
      and $\alpha \in \cO_{E_n}$ with $\Art_{E_n}(\alpha) = \sigma|_{\hat
      E_n^\ab}$; for those
      elements we have $\Theta(1,\alpha,\sigma) = (1, 1,0)$.
    \item $(1, \varpi^{-1}, \sigma)$ for $\sigma \in W_{E_n}$ with
      $\Art^{-1}_{E_n}(\sigma|_{\hat E_n^\ab}) = \varpi$; for those
      elements we have $\Theta(1,\varpi^{-1},\sigma) = (1, 1,n)$. 
  \end{itemize}
  \end{lem}
In particular, the image of $J^1$ under the homomorphism $\Theta$ is 
$\bar J^1$, so the induced action of $J^1$ on $\DL_n$ stabilizes $Y_b$.

For $(g,d,\sigma) \in G^1$, recall the definition of $g^* \in \cO_\Cp\llbr
X_1^{q^{-\infty}}, \dots, X_n^{q^{-\infty}}\rrbr$ and 
$d^* \in \cO_\Cp \llbr X^{q^{-\infty}} \rrbr$ from Proposition
\ref{lem:ExplicitActionOnQthPowerRootSystems}. 
We wish to show the following Proposition.

\begin{prop}\label{prop:J1ActionOnAffinoid}
  The action of $J^1$ on $M_{\infty, C}^{(0)}$ stabilizes $U$ and extends to
  $\cX$. The induced action on the special fiber $\cX_s$ is compatible with the
  action of $J^1$ on $Y_b$.
\end{prop}

\begin{proof}
Note that by the description of the group action of $G^1$ on $A_{\infty, \cO_\Cp}$
in Proposition \ref{prop:ExplicitDescriptionOfActionOnAinfty}, the induced
actions on $C_n$ and $Y_b^\perf$ must take the form described in Figure
\ref{fig:TableOfGroupActionsOnModels}. 

\begin{figure}[H] 
\centering
\begin{center}
\begin{tabularx} {0.9\textwidth} { 
  |>{\centering\arraybackslash}X ||>{\centering\arraybackslash}X
  |>{\centering\arraybackslash}X |>{\centering\arraybackslash}X | }
 \hline
 Element & Action on $A_{\infty,\cO_\Cp}$ & Action on $C_n$ & Action on
 $Y_b^\perf$ \\ [0.5ex] 
 \hline\hline
 $(g,1,1)$ & $(X_1, \dots, X_n) \mapsto g^*(X_1, \dots, X_n)$ & $(x_1, \dots,
 x_n) \mapsto \xi^{-1}g^*(\xi x_1, \dots, \xi x_n)$ & $(x_1,\dots, x_n) \mapsto (x_1,
 \dots, x_n)\cdot\bar g$\\ 
 \hline
 $(1,d,1)$ & $X_i \mapsto d^{-1, *}(X_i)$ & $x_i \mapsto \xi^{-1} d^{-1, *}(\xi
 x_i)$ & $x_i
 \mapsto \bar d^{-1} x_i$ \\
 \hline
 $(a,a,1)$ & trivial & trivial & trivial \\
 \hline
 $(1, \alpha^{-1}, \sigma)$ & $a \mapsto \sigma(a)$& $a \mapsto \sigma(a)$, $x_i \mapsto  \tfrac{\xi}{\sigma(\xi)} x_i$ & trivial \\
 \hline
 $(1, \varpi, \sigma)$ & $a \mapsto \sigma(a)$& $a \mapsto \sigma(a)$, $x_i \mapsto
 \tfrac{\xi}{\sigma(\xi)} x_i$ & $a
 \mapsto a^{q^{-n}}$ \\ [1ex] 
 \hline
\end{tabularx}
\end{center}
\caption{Description of the group actions.}
\label{fig:TableOfGroupActionsOnModels}
\end{figure}

To show that the actions described above fulfill the desired properties, it suffices
to show the following claims.
\begin{enumerate}
  \item The power series 
    $$\xi^{-1}g^*(\xi x_1, \dots, \xi x_n) \in \Cp\llbr x_1, \dots x_n \rrbr$$
    lies inside $\cO_\Cp \langle x_1, \dots, x_n \rangle$ and reduces to $(x_1,
    \dots, x_n)\cdot \bar g$ modulo $\fm_\Cp$.
  \item The power series 
    \begin{equation*}
      \xi^{-1} d^{-1, *} (\xi x_i) \in \Cp \llbr x_i\rrbr 
    \end{equation*}
    lies inside $\cO_\Cp \langle x_i \rangle$ and reduces to $\bar d^{-1} x_i$
    modulo $\fm_\Cp$. 
  \item If $\sigma \in \Weil_{E_n}$ lies inside the inertia subgroup $I_{E_n}$,
    the element
    $\xi/\sigma(\xi)$ reduces to $1$ modulo $\fm_\Cp$.
  \item If $\sigma \in \Weil_{E_n}$ satisfies $\Art^{-1}_{E_n}(\sigma|_{\hat
    E_n^\ab}) = \varpi$, the element
    $\xi/\sigma(\xi)$ reduces to $1$ modulo $\fm_\Cp$. 
\end{enumerate}
The first claim follows directly from the description of $g^*$ in
Lemma \ref{lem:ExplicitActionOnQthPowerRootSystems}. Indeed, writing $g = (g_1,
\dots, g_n)$ for $g_i$ the $i$-th column vector of $g$, one quickly checks that modulo
$\fm_\Cp$, we have 
$$\xi^{-1} g_i^*(\xi x_1, \dots, \xi x_n) \equiv \sum_{j=1}^n \sumH
a_{ij}^{(0)} x_j = (x_1, \dots, x_n).\bar g_i \mod \fm_\Cp.$$ 
Similarly one obtains the second claim.

The third claim follows as $\sigma$ lies in the inertia subgroup,
hence $\xi \equiv \sigma(\xi) \mod \fm_\Cp$.

Finally, for the fourth statement, we use that by classical Lubin--Tate theory, $\sigma$
acts trivially on the system
$(y_k)_{k\in \N}$. As $\sigma$ is continuous, this implies $\sigma(\xi) = \xi$,
concluding the proof.
\end{proof}
% subsubsection Proof of Proposition \ref{prop:J1EquivInjMor} (end)

\subsection{The Explicit Correspondence} % (fold)
\label{sub:The Explicit Correspondence}
Fix, for the remainder of the section, an isomorphism $\bar \Q_l \cong \C$ and
a regular character $\theta: \FF_{q^n}^\times \to \C^\times$. 
The datum of $\theta$ can be used to construct representations 
of $W_F$ and $D^\times$ and, making use of Deligne--Lusztig
theory, a representation of $\GL_n(F)$. We proceed as follows.
\begin{itemize}
  \item Let $\bar \tau_\theta$ be the character of $\Weil_{F_n}$ given by 
    the composition
    \begin{equation*}
      \Weil_{F_n} \to \Weil_{E_n}^\ab \xto{\Art_{F_n}^{-1}} F_n^\times \cong
      \Z \times \cO_{F_n}^\times \surj \FF_{q^n}^\times \xto \theta \C^\times
    \end{equation*}
  and put $\tau_\theta = \cInd_{\Weil_{F_n}}^{\Weil_F}(\bar \tau_\theta)$.
  \item Let $\bar \rho_\theta$ be the character on 
    $F^\times \cO_D^\times$ given by the composition
    \begin{equation*}
      F^\times\cO_D^\times \cong \varpi^\Z \times \cO_D^\times \surj 
      \FF_{q^n} \xto \theta \C^\times
    \end{equation*}
    and let $\rho_\theta = \cInd_{F^\times \cO_D^\times}^{D^\times}(\bar \rho_\theta)$.
  \item Let $\bar \pi_\theta$ be the representation of $F^\times \GL_n(\cO_F)$
    arising from post-composing $R_\theta$ (cf. Theorem
    \ref{thm:DLCorrespondenceForUs}) with the composition
    \begin{equation*}
      F^\times \GL_n(\cO_F) \cong \varpi^\Z \times \GL_n(\cO_F)
      \surj \GL_n(\cO_F) \surj \GL_n(\FF_q).
    \end{equation*}
    Let $\pi_\theta = \cInd_{F^\times \GL_n(\cO_F)}^{\GL_n(F)}(\bar \pi_\theta)$. 
\end{itemize}

\begin{lem}\label{lem:BarRepsAreSmooth}
  The representations $\bar \pi_\theta$, $\bar \rho_\theta$ and $\bar
  \tau_\theta$ are smooth, in particular
  $\pi_\theta$, $\rho_\theta$ and $\tau_\theta$ are smooth as well. Additionally, the
  representations $\pi_\theta$ and $\rho_\theta$ are irreducible, and
  $\pi_\theta$ is supercuspidal.
\begin{proof}
  By design, $\bar \pi_\theta$ is trivial on the compact open subgroup $1 +
  \varpi \Mat_{n \times n}(\cO_F)$ of $F^\times \GL_n(\cO_F)$. Similar
  statements hold for $\bar \rho_\theta$
  and $\bar \tau_\theta$. 
  \red{why is $\rho_\theta$ irreducible? Why is $\pi_\theta$ supercuspidal and
  irreducible?}
\end{proof}
\end{lem}

The aim of this section is to prove the following statement.
\begin{thm}[Explicit Non-Abelian Lubin--Tate Theory for Depth Zero Supercuspidal
  Representations]\label{thm:MainRes1}
  The representation $\JL(\pi_\theta)$ of $D^\times$ and the representation
  $\rec_F(\pi_\theta)$ of $\Weil_F$ take the form
  \begin{equation*}
    \JL(\pi_\theta) = \rho_{\theta}
    \quad \text{and} \quad \rec_F(\pi_\theta) = \Ind_{\Weil_{F_n}}^{\Weil_F} 
    (\tau_\theta \, \delta^{n-1}),
  \end{equation*}
  where $\delta: \Weil_{F_n} \to \{\pm 1\}$ is the unramified quadratic
  character. This is the character corresponding to $a \mapsto
  (-1)^{\val_{F_n}(a)}$ under the isomorphism $\Art_{F_n}: F_n^\times \to
  \Weil_{E_n}^\ab$. 
\end{thm}


Recall that $H_\DL$ denotes the middle $l$-adic cohomology of 
$\DL_n$, cf. Section \ref{sub:The Deligne--Lusztig Correspondence for the
Explicit Example}.
\begin{lem}
  The morphism $\Theta$ makes $J$ act on ${\hH}_{\DL,\theta}$. This representation is 
  of the form 
  \begin{equation*}
    (g,d,\sigma) \mapsto \bar \pi_\theta(g) \otimes \bar \rho_{\theta^{-1}}(d)
    \otimes \left(\bar \tau_\theta \, \delta^{n-1}\right)^{-1}(\tfrac{1-n}2)(\sigma).
  \end{equation*}
  This representation is smooth.
\begin{proof}
  This is a direct calculation.
\end{proof}
\end{lem}

The input we get from Mieda's theory is the following.
\begin{prop}\label{prop:J1EquivInjMor}
  There is an injective morphism of  $J^1$-representations
  \begin{equation*} 
    \Res_{J^1}^J( \hH_{\DL,\theta} ) \inj \Res_{J^1}^{G^1}( \hH'_{\LT}).
  \end{equation*}
  \begin{proof}
    This is \cite[Proposition 5.11]{mieda2016geometric}. 
  \end{proof}
\end{prop}

\begin{lem} \label{lem:JEquivInjMor}
  The morphism in Proposition \ref{prop:J1EquivInjMor} naturally gives rise to
  an injective $J$-equivariant morphism
  \begin{equation*}
    \hH_{\DL,\theta} \inj \Res_J^G \hH_\LT.
  \end{equation*}
\begin{proof}
  We construct a sequence of $J$-equivariant injections
  \begin{equation*}
    \hH_{\DL,\theta} \inj \Ind_{J^1}^J ( \Res_{J^1}^J \hH_{\DL,\theta}) \inj
    \Ind_{J^1}^J (\Res_{J^1}^{G^1} \hH'_{\LT}) \\ \xto \sim \Res_J^{G^1J}(
    \Ind_{G^1}^{G^1J} \hH'_\LT) 
 \inj \Res_J^G \hH_\LT.
  \end{equation*}
  \textit{The first morphism.}
  This is the unit of the adjunction $\Res_{J^1}^J \ladj \Ind_{J^1}^J$ 
  applied at $\hH_{\DL,\theta}$, which is injective by Lemma
  \ref{lem:FrobRecUnitsAreInjective}.

  \textit{The second morphism.}
  This is $\Ind_{J^1}^J$ applied to the injective morphism in Proposition
  \ref{prop:J1EquivInjMor}. The resulting morphism is injective because
  $\Ind_{J^1}^J$ is exact, cf. Proposition
  \ref{prop:InducedRepresentationExact}. 

  \textit{The third morphism.} 
  The morphism is given by the inverse of the base-change morphism constructed in
  Lemma \ref{lem:BaseChangeForResInd}, which is applied with $H = G^1$, $N = J$. 
  Note that $G^1$ is normal in $G$, so the assumptions of the Lemma are
  satisfied. As $J$ is open in $G$, the map is an isomorphism. 
  
  \textit{The fourth morphism.}
  Since $G^1J$ is open in $G$, the unit of the adjunction $\cInd_{G^1J}^G \ladj
  \Res_{G^1J}^G$ yields a monomorphism of $G^1J$-representations 
  \begin{equation}\label{eq:MiedaEx1Morph1}
    \Ind_{G^1}^{G^1J} \hH_\LT' \to \Res_{G^1J}^G
    (\cInd_{G^1J}^G(\Ind_{G^1}^{G^1J} \hH'_\LT)).
  \end{equation}
  As $G^1J$ co-compact in $G$, we have $\cInd_{G^1J}^G =
  \Ind_{G^1J}^G$, so the right-hand
  side is isomorphic to $\Res_{G^1J}^G(\Ind_{G^1}^G \hH'_\LT) \cong
  \Res_{G^1J}^G(\hH_\LT)$ by Proposition \ref{prop:InductionOnTower} and 
  Lemma \ref{lem:InductionStatementOnHLT}. Hence, applying $\Res_J^{G^1J}$ to the
  morphism in \eqref{eq:MiedaEx1Morph1} yields the desired map.
\end{proof}
\end{lem}

The morphism constructed in Lemma \ref{lem:JEquivInjMor} yields, by Frobenius
reciprocity, a non-zero map of $G$-representations
\begin{equation} \label{eq:WantedMap}
  \Ind_{J}^G (\hH_{\DL,\theta}) \cong \pi_\theta \boxtimes \rho_{\theta^{-1}}
  \boxtimes (\tau_\theta \, \delta^{n-1})^{-1}(\tfrac{1-n}2) \to \hH_\LT.
\end{equation}
As $\pi_\theta$ is supercuspidal and its central character is trivial on 
$\varpi^\Z$, Theorem \ref{thm:NonAbLTT} yields a non-zero map
\begin{equation*}
  \rho_{\theta^{-1}} \boxtimes (\tau_\theta \, \delta^{n-1}) \to
  \JL(\pi_\theta)^\vee \boxtimes \rec_F(\pi_\theta)^\vee.
\end{equation*}
As $\rho_{\theta^{-1}}$ and $\JL(\pi_\theta)^\vee$ are irreducible, this implies
$\JL(\pi_\theta) = \rho_{\theta^{-1}}^\vee = \rho_\theta$. As 
$\rec_F(\pi_\theta)$ is irreducible and $\dim(\tau_\theta) = n = \dim
(\rec_F(\pi_\theta))$, this also implies $\tau_\theta \, \delta^{n-1} =
\rec_F(\pi_\theta)$, concluding the proof of Theorem \ref{thm:MainRes1}.

% subsection (end)

% section The Explicit Local Langlands Correspondence for Depth Zero Supercuspidal Representations (end)
\end{document}
