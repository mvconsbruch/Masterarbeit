%! TeX root: ../main.tex
\documentclass[../main.tex]{subfiles}

\begin{document}
\section{Explicit Non-Abelian Lubin--Tate Theory for Depth Zero Supercuspidal
Representations} % (fold)
\label{sec:Explicit Non-Abelian Lubin-Tate Theory for Depth Zero Supercuspidal Representations}
% INTRO (fold)
In this final section, we explicitely calculate the representations
arising in Theorem \ref{thm:NonAbLTT} for depth zero supercuspidal
representations $\pi \in \Rep{\GL_n(E)}$, which essentially are representations
of $\GL_n(E)$
obtained by compact induction from Deligne--Lusztig theory. In
particular, we give explicit descriptions of $\rec_E(\pi)$ and $\JL(\pi)$. 
The main input is the construction of an affinoid in the Lubin--Tate perfectoid
space whose special fiber is isomorphic to the perfection of the 
Deligne--Lusztig variety constructed in  \ref{sub:An Explicit Example}.
Mieda's results give a relation between the $\ell$-adic cohomology of $\DL_n$
and the representation $\HLT$, which allows to make some parts of $\HLT$
explicit.
% INTRO (end)

\subsection{The Special Affinoid and its Formal Model} % (fold)
\label{sub:The Special Affinoid}
We define the subspace $U \subset M'_{\infty,\Cp}$ as the rational
subset cut out by the inequalities $\abs{X_i(x)} \leq \abs{\varpi(x)}^{1/q^n-1}$ for
$i = 1, \dots, n$. We construct a formal model $\cX \in \FSchOver {\cO_{\Cp}}$
of $U$ whose special fiber $\cX_s = 
\cX \times_{\spf(\cO_{\Cp})}\spec(\bar \FF_q)$ is isomorphic to the perfection of the
Deligne--Lusztig variety $\DL_n$ constructed in  \ref{sub:An Explicit Example}.
Furthermore, the formal model $\cX$ is compatible with the group actions
defined on $\DL_n$ and $M_{\infty, \cO_\Cp}^{(0)}$. 

Let $H$ be the standard formal $\cO_E$-module over $\cO_\br E$ and write
$E_n$ for the degree $n$ unramified extension of $E$. 
The construction of the formal model $\cX$ requires a few choices. 
First, we fix a choice of basis $(\alpha_1, \dots, \alpha_n)$ of $E_n$ over $E$, 
which in particular allows us to view $E_n$ as a subfield of $\Mat_{n \times n}(E)$. 
By \cref{lem:MultByROUForStandardModule} and 
\cref{prop:ReductionOfHomsIsInjective}, $E_n$ is also naturally a subfield of
$D$.

Furthermore, we'll have to make the choice of a primitive element 
$$y = (y_1, y_2, \dots) \in T_\varpi H(\cO_\Cp),$$ that is,
an element with $y_1 \neq 0$ and $[\varpi]_H(y_1) = 0$. Write $\xi = \lambda(y)$ 
for the corresponding element of $\Nilp^\flat(\cO_\Cp)$. 

\begin{lem}\label{lem:PropertiesOfXi}
  We have 
  \begin{equation*}
    \log_H(y) = \sum_{i = -\infty}^ \infty \frac{\xi^{q^in}}{\varpi} = 0
    \quad \text{and} \quad
    \abs \xi^{q^n-1} = \abs \varpi.
  \end{equation*}
\begin{proof}
  The first identity follows from \cref{lem:LogInTermsOfNil} and the fact that
  $\log_H(y_1)$ vanishes, cf. \cref{lem:KernelOfLog}. For the second one, note
  that $[\varpi]_H(y_1) = 0$ and $[\varpi]_H(y_{k+1}) = y_k$ implies that
  $\abs{y_k} = \abs \varpi^{\frac{1}{q^{nk}(q^n-1)}}$. Hence, the claim follows
  by the equality $\xi = \lim_{k \to \infty} y_k^{q^{nk}}$, cf. 
  \cref{lem:GroupStructureOnNilp}.
\end{proof}
\end{lem}

Let $t \coloneq \delta(\alpha_1 y, \dots, \alpha_n y) \in (T_\varpi {\wedge H})(\cO_\Cp)$
and write $\tau \in \Nilp^\flat(\cO_\Cp)$ for the corresponding system of $q$-th 
power roots.

\begin{lem}\label{lem:PropertiesOfTau}
  We have 
  \begin{equation*}
    \log_{\wedge H}(t) = 
    \sum_{i = -\infty}^\infty (-1)^{i(n-1)} \frac{\tau^{q^i}}{\varpi^i} = 0 
    \quad \text{and} \quad 
    \abs \tau^{q-1} = \abs \varpi.
  \end{equation*}
  Furthermore, we have the congruence
  \begin{equation*}
    \tau \equiv \det(\alpha_i^{q^{j-1}})_{1 \leq i,j \leq n} \xi^{1 + q + \dots + q^{n-1}}
  \end{equation*}
  modulo the ideal generated by elements $z \in \cO_\Cp$ with with 
  $\abs{z} < \abs {\xi}^{1 + q + \dots + q^{n-1}}$. 
\begin{proof}
  We have $t_1 \neq 0$ and $[\varpi]_{\wedge H}(t) = 0$, so the first two
  assertions follow just as above in the proof of 
  \cref{lem:PropertiesOfXi}, appropiately adjusted. The congruence is a
  corollary of the approximation of $\Delta$ in \cref{lem:DeltaApprox}.
\end{proof}
\end{lem}

We now define the formal model $\cX$. We abbreviate with $(x_i^{q^{-m}})_{m \in
\N_0}$ the system of 
$q$-th power roots $(X_i^{q^{-m}}/\xi^{q^{-m}})_{m \in \N_0}$ of elements in
$\Cp \llbr X_1^{q^{-\infty}}, \dots, X_n^{q^{-\infty}}\rrbr$, and define
systems of $q$-th power roots $\Delta'$ and $\tau'$ as
\begin{equation*}
  \begin{array}c
  \Delta'(x_1, \dots, x_n)^{q^{-m}} \coloneq (\xi^{q^{-m}})^{-(1 + q + \dots +
  q^{n-1})} \Delta^{q^{-m}}(\xi x_1, \dots, \xi x_n) \in \Cp \llbr x_1^{q^{-\infty}}, \dots
  , x_n^{q^{-\infty}}\rrbr  \\
  \text{and} \\
  \tau'^{q^{-m}} \coloneq (\xi^{q^{-m}})^{-(1 + q + \dots + q^{n-1})} \tau^{q^{-m}} \in
  \Cp^\times.
  \end{array}
\end{equation*}

\begin{lem}\label{lem:PropsOfDeltaPrAndTauPr}
  The coefficients of $\Delta'$ are 
  integral, and we have
  \begin{equation*}
    \Delta'(x_1, \dots, x_n)^{q^{-m}} \equiv (\det(x_i^{q^j})_{1 \leq i,j \leq
    n})^{q^{-m}} \mod \fm_\Cp.
  \end{equation*}
  In particular, we have 
  $\Delta'(x_1, \dots, x_n)^{q^{-m}} \in \cO_\Cp \langle x_1^{q^{-\infty}}, \dots
  , x_n^{q^{-\infty}}\rangle.$
  Also, $\tau'^{q^{-m}} \in \cO_\Cp^\times$. 
\begin{proof}
  The statement about $\Delta'$ follows directly from the approximation in 
  \cref{lem:DeltaApprox}. One easily checks $\abs {\tau'} = 1$, implying the second
  statement.
\end{proof}
\end{lem}

We define 
\begin{equation}\label{eq:defQ}
  Q \coloneq \frac
    {\cO_\Cp \langle x_1^{q^{-\infty}}, \dots , x_n^{q^{-\infty}}
      \rangle}
      {(\Delta'(x_1, \dots, x_n)^{q^{-m}} - \tau'^{q^{-m}} \mid m \in \N_0)^-}
      \quad \text{and} \quad
  \cX \coloneq \spf \left( Q \right).
\end{equation}

\begin{prop}\label{prop:AffinoidIsFormalModel}
  The formal scheme $\cX$ is a formal model for $U$. 
\end{prop}
\begin{proof} 
  By definition, $U$ is isomorphic to the affinoid adic space $\Spa(B[\tfrac 1\varpi], \tilde B),$ where
  \begin{equation*}
    B \coloneqq \frac{\cO_\Cp \langle x_1^{q^{-\infty}}, \dots,
    x_n^{q^{-\infty}} \rangle}
    {(\Delta^{q^{-m}}(\xi x_1, \dots, \xi x_n) - \tau^{q^{-m}} \mid m \in \N)^-}
  \end{equation*}
  and $\tilde B \subset B[\tfrac 1\varpi]$ denotes the normalization of 
  $B$ inside $B[\tfrac 1 \varpi]$. 
  As $\xi$ is invertible in $B[\tfrac 1 \varpi]$, we have 
  $Q[\tfrac 1 \varpi] = B[\tfrac 1 \varpi]$. Furthermore, the images of 
  $B$ and $Q$ inside $B[\tfrac 1 \varpi]$ agree. The claim follows.
\end{proof}

%\begin{rmk} 
%  We give the following interpretation of this result. 
%  Disregarding finiteness conditions, $\cX$ naturally embedds into the
%  formal admissible blowing (cf. \red{cite Bosh} for this notion, although they
%  only define the formal admissible blowing up of formal schemes topologically
%  of finite type) up of $\cM_{\infty, \cO_\Cp}^{(0)}$ along the closed
%  subscheme corresponding to the ideal of $A_{\infty,\cO_\Cp}$ given by 
%  $\cI = (\xi^{q^{-m}}, X_1^{q^{-m}}, \dots, X_n^{q^{-m}} \mid m \in \N)^{-}$.
%  In light of Raynaulds view on the interplay between (reasonable) rigid spaces
%  over a non-Archimedian local field and (reasonable) formal schemes over 
%  its ring of integers, this realizes the inclusion
%  $U \inj M_{\infty, \Cp}^{(0)}$ as the inclusion of formal models
%  \begin{equation*}
%    \cX \inj \Bl_\cI (\cM_{\infty, \cO_\Cp}^{(0)}).
%  \end{equation*}
%\end{rmk}

\begin{prop}\label{prop:SpecialFiberOfAffinoidIsLusztig}
  Let $b^{1/q^m}$ be the residue class of $\tau'^{1/q^m}$ in $\cO_\Cp/\fm_{\Cp}$. 
  Then $b^{q-1} = (-1)^{n-1}$, and the special fiber of $\cX_s \coloneq \cX
  \times_{\spf(\cO_\Cp)} \spec (\Fqbar)$ is isomorphic to $Y_b^\perf$, the
  perfection of the component $Y_b \subset \DL_n$ defined in \eqref{eq:DefYb}.
\begin{proof}
  We first show that $b^{q^{-m}(q-1)} = (-1)^{n-1}$ for $m \in \Z$. 
  This follows from \cref{lem:PropertiesOfTau}. 
  Write $\bar \alpha_i \in \FF_{q^n}$ for the residue classes of the 
  elements $\alpha_i$ for $i = 1, \dots, n$. We have $\bar \alpha_i^{q^n}
  = \bar \alpha_i$, implying for $m \in \Z$ the equalities
  \begin{equation*}
    b^{q^{-(m-1)}} = \det(\bar \alpha_i^{q^j})_{1 \leq i,j \leq n}^{q^{-m}} =
    (-1)^{n-1} \det (\bar \alpha_i^{q^{j-1}})_{1 \leq i,j \leq n}^{q^{-m}} =
    (-1)^{n-1} b^{q^{-m}}.
  \end{equation*}
  This gives $b^{q^{-m}(q-1)} = (-1)^{n-1}$, as desired.
  By \cref{lem:PropsOfDeltaPrAndTauPr}, we find that $\cX_s$ is equal to
  \begin{equation*}
    \spec \left( \frac{\Fqbar [x_1^{q^{-\infty,}}, \dots,
    x_n^{q^{-\infty}}]}{(\det(x_i^{q^{j-1}})^{q^{-m}}- b^{q^{-m}})} \right),
  \end{equation*}
  which is precisely the perfection of $Y_b$.
\end{proof}
\end{prop}

Finally, we need to show that $\cX$ is flat over $\spf(\cO_\Cp)$. 

\begin{prop}\label{prop:FlatnessOfFormalModel}
  The formal model $\cX = \spf(Q)$ is flat over $\spf(\cO_\Cp)$. 
\begin{proof}
  We show that $Q$, as defined in \eqref{eq:defQ}, is a flat $\cO_\Cp$-agebra.
  For the moment, let us abbreviate $A = \cO_\Cp\langle x_1^{q^{-\infty}}, \dots, 
  x_n^{q^{-\infty}}\rangle$, 
  and let $I_m \subset A$ denote the ideal generated by the element 
  $(\Delta'^{q^{-m}} - \tau'^{q^{-m}}) \in A$. Furthermore, we write $I = I_1 +
  I_2 + \dots \subset A$ 
  and $\bar I$ for the closure of $I$ in $A$, so that $Q = A/\bar I$. 
  We first show that the (non-complete) quotient $A/I$ is $\cO_\Cp$-torsion free, 
  which suffices for flatness as $\cO_\Cp$ is a valuation ring,
  cf. \cite[\href{https://stacks.math.columbia.edu/tag/0539}{Tag
  0539}]{stacks-project}. From the factorization 
  \begin{equation}\label{eq:FlatnessFactorization}
    \Delta'^{q^{-m}} - \tau'^{q^{-m}} = (\Delta'^{q^{-(m+1)}} - \tau'^{q^{-(m+1)}})
    (\Delta'^{(q-1)q^{-(m+1)}} + \dots + \tau'^{(q-1)q^{-(m+1)}}),
  \end{equation}
  we deduce that $I_{m} \subset I_{m+1}$. Now $A/I = \colim_{m \in \N} A/I_m$, and
  it suffices to check that $A/I_m$ is flat for all $m \in \N$, which is equivalent
  to $\Delta'^{q^{-m}} - \tau'^{q^{-m}} \not \in \fm_\Cp A$ for all
  $m \in \N$. As the reduction
  $$(A/I) \otimes_{\cO_\Cp} \Fqbar = (A/\bar I) \otimes_{\cO_\Cp}
  \Fqbar \cong \frac{\Fqbar [x_1^{q^{-\infty,}}, \dots,
    x_n^{q^{-\infty}}]}{(\det(x_i^{q^{j-1}})^{q^{-m}}- b^{q^{-m}} | m \in \N)}$$
  is not isomorphic to $A \otimes_{\cO_\Cp} \Fqbar$, the reduction of $I$ modulo
  $\fm_\Cp$ cannot be trivial and there must be some 
  $m \in \N$ such that $(\Delta'^{q^{-m}} - \tau'^{q^{-m}}) \not \in \fm_\Cp A$. 
  By taking $q^{m-1}$-th powers, this readily implies $(\Delta' - \tau') \not \in
  \fm_\Cp A$, which, by the factorization in \eqref{eq:FlatnessFactorization},
  implies that $(\Delta'^{q^{-m}} - \tau'^{q^{-m}}) \not \in \fm_\Cp A$ for every $m$. 
  Hence, $A/I$ is flat over $\cO_\Cp$. 

  The ultra-metric inequality for the $\varpi$-adic metric on $A$ implies that 
  \begin{equation*}
    I \cap \fm_\Cp A = (0) \iff \bar I \cap \fm_\Cp A = (0),
  \end{equation*}
  and it follows that $Q = A/\bar I$ is flat over $\cO_\Cp$ as well. 
  This is what we wanted to show.
\end{proof}
\end{prop}

% subsection The Special Affinoid (end)

\subsection{Comparison of the Group Actions} % (fold)
\label{sub:Proof of Proposition}
In \cref{sec:Non-Abelian Lubin-Tate Theory: An Overview}, we saw that
the Lubin--Tate perfectoid space 
$$M'_{\infty, \Cp} = \cM_{\infty} \times_{\Spa(\hat E^\ab, \cO_{\hat
E^\ab})} \Spa(\Cp, \cO_\Cp)$$ 
admits a right action by the group
\begin{equation*}
  G^1 \subset G = \GL_n(E) \times D^\times \times \Weil_E,
\end{equation*}
given by those elements $(g,d,\sigma)$ satisfying $\det(g) \Nrd(d)^{-1} \Art_E^{-1}(\sigma|_\br E) = 1$.
In this section, we construct a subgroup $J^1 \subset G^1$ stabilizing the
special affinoid $U$ constructed above, and we furthermore show that 
the action of $J^1$ on $U$ extends to an action of $J^1$ on the formal model
$\cX$. This induces a right action on the special fiber
$\cX_s$ of $\cX$, and as $\cX_s \cong Y_b^\perf \subset \DL_n^\perf$, this
yields an action of $J^1$ on the perfection of a part of the Deligne--Lusztig
variety constructed in \cref{sub:An Explicit Example}. Recall that the group 
\begin{equation*}
  \bar J^1 = (\GL_n(\FF_q) \times \FF_{q^n}^\times)^1 \times n\Z
\end{equation*}
acts on $Y_b$, in particular it also acts on 
$Y_b^\perf$. We will show that the action of $J^1$ on $Y_b^\perf$
may be described in terms of the latter action, via a homomorphism
$\Theta\colon J^1 \to \bar J^1$. These
results lay the representation-theoretic ground for the comparison between the
representations $\HLT$ and $\HDL$.

We set 
\begin{equation} \label{eq:defJandJ1}
  J \coloneqq E^\times \GL_n(\cO_E) \times \cO_D^\times \times \Weil_{E_n}
  \text{ and } J^1 = J \cap G^1.
\end{equation}
Also, we define the morphism $\Theta$ as follows. For $\sigma \in \Weil_{E_n}$ with
$\sigma|_\br E = \Phi^{n_\sigma}$ and $u_\sigma \coloneq \varpi^{-n_\sigma}
\Art^{-1}_{E_n}(\sigma|_{\hat E_n^\ab}) \in \cO_{E_n}^\times$, we set
\begin{equation}\label{eq:ThetaMorDef}
  \Theta \colon J \to \GL_n(\FF_q) \times \FF_{q^n}^\times \times {n\Z}, \quad
  (\varpi^m g,d,\sigma) \mapsto (\bar g, \bar {d^{-1}u_\sigma^{-1}}, -n_\sigma).
\end{equation}

It will be convenient to have a list of generators for the group $J^1$. 
\begin{lem}\label{lem:GeneratorsForJ1}
  The group $J^1$ is generated by elements of the form
  \begin{itemize}
    \item $(g,1,1)$ for $g \in \GL_n(\cO_E)$ with $\det g = 1$; for those
      elements we have $\Theta(g,1,1) = (\bar g, 1,0)$
    \item $(1,d,1)$ for $d \in \cO_D^\times$ with $\Nrd d = 1$; for those
      elements we have $\Theta(1,d,1) = (1 , \bar d,0)$
    \item $(a,a,1)$ for $a \in E_n^\times$; for those
      elements we have $\Theta(a,a,1) = (\bar a, \bar a,0)$
    \item $(1, \alpha^{-1}, \sigma)$ for $\sigma \in I_{E_n}$ 
      and $\alpha \in \cO_{E_n}$ with $\Art_{E_n}^{-1}(\sigma|_{E_n^\ab}) = \alpha$;
      for those elements we have $\Theta(1,\alpha,\sigma) = (1, 1,0)$.
    \item $(1, \varpi^{-1}, \sigma)$ for $\sigma \in W_{E_n}$ with
      $\Art^{-1}_{E_n}(\sigma|_{E_n^\ab}) = \varpi$; for those
      elements we have $\Theta(1,\varpi^{-1},\sigma) = (1, 1, -n)$. 
  \end{itemize}
  \end{lem}
In particular, the image of $J^1$ under the homomorphism $\Theta$ is 
$\bar J^1$, so the induced action of $J^1$ on $\DL_n$ stabilizes $Y_b$.

For $(g,d,\sigma) \in G^1$, recall the definition of $\gamma_{g,i} \in \cO_\Cp\llbr
X_1^{q^{-\infty}}, \dots, X_n^{q^{-\infty}}\rrbr$ and 
$\delta_{d^{-1}} \in \cO_\Cp \llbr X^{q^{-\infty}} \rrbr$ from 
\cref{lem:ExplicitActionOnQthPowerRootSystems}. 
We wish to show the following proposition.

\begin{prop}\label{prop:J1ActionOnAffinoid}
  The action of $J^1$ on $M_{\infty, C}^{(0)}$ stabilizes $U$ and extends to
  $\cX$. The induced action on the special fiber $\cX_s$ is compatible with the
  action of $J^1$ on $Y_b$.
\end{prop}

\begin{proof}
For $g \in \GL_n(E)$, write $\gamma_{g,i} = g^*(X_i) 
\in \cO_\Cp \llbr X_1^{q^{-\infty}}, \dots, X_n^{q^{-\infty}} \rrbr$, where 
$g^*$ is the morphism defined in \cref{lem:ExplicitActionOnQthPowerRootSystems}.
 Also, denote by $g_i$ the $i$-th column vector of $g$.

Similarly, for $d \in D^\times$, write $\delta_d = d^*(X_i) \in \cO_\Cp\llbr
X_1^{q^{-\infty}}, \dots, X_n^{q^{-\infty}} \rrbr.$
By the description of the group action of $G^1$ on $A_{\infty, \cO_\Cp}$
in \cref{prop:ExplicitDescriptionOfActionOnAinfty}, the induced
actions on $Q$ and $Y_b^\perf$ must take the form described in Figure
\ref{fig:TableOfGroupActionsOnModels}. 

\begin{figure}[H] 
\centering
\begin{center}
\begin{tabularx} {0.95\textwidth} { 
  |>{\centering\arraybackslash}X ||>{\centering\arraybackslash}X
  |>{\centering\arraybackslash}X |>{\centering\arraybackslash}X | }
 \hline
 Element & Action on $A_{\infty,\cO_\Cp}$ & Action on $Q$ & Action on
 $Y_b^\perf$ \\ [0.5ex] 
 \hline\hline
 $(g,1,1)$ & $X_i \mapsto \gamma_{g,i}(X_1, \dots, X_n)$ & $x_i \mapsto
 \xi^{-1}\gamma_{g,i}(\xi x_1, \dots, \xi x_n)$ & $x_i \mapsto
 (x_1, \dots, x_n)\cdot\bar g_i$\\ 
 \hline
$(1,d,1)$ & $X_i \mapsto \delta_{d^{-1}}(X_i)$ & $x_i \mapsto \xi^{-1}
\delta_{d^{-1}}(\xi x_i)$ & $x_i \mapsto \bar d^{-1} x_i$ \\
 \hline
 $(a,a,1)$ & trivial & trivial & trivial \\
 \hline
 $(1, \alpha^{-1}, \sigma)$ & $a \mapsto \sigma(a)$& $a \mapsto \sigma(a)$, $x_i \mapsto  \tfrac{\xi}{\sigma(\xi)} x_i$ & trivial \\
 \hline
 $(1, \varpi, \sigma)$ & $a \mapsto \sigma(a)$& $a \mapsto \sigma(a)$, $x_i \mapsto
 \tfrac{\xi}{\sigma(\xi)} x_i$ & $a
 \mapsto \Frob_q^n(a)$ \\ [1ex] 
 \hline
\end{tabularx}
\end{center}
\caption{Description of the group actions.}
\label{fig:TableOfGroupActionsOnModels}
\end{figure}

To show that the actions described above fulfill the desired properties, it suffices
to verify the following claims.
\begin{enumerate}
  \item For $i = 1, \dots, n$, the power series 
    $$\xi^{-1}\gamma_{g,i}(\xi x_1, \dots, \xi x_n) \in \Cp\llbr x_1, \dots x_n \rrbr$$
    lies inside $\cO_\Cp \langle x_1, \dots, x_n \rangle$ and reduces to $(x_1,
    \dots, x_n)\cdot \bar g_i$ modulo $\fm_\Cp$.
  \item The power series 
    \begin{equation*}
      \xi^{-1} d^{-1, *} (\xi x_i) \in \Cp \llbr x_i\rrbr 
    \end{equation*}
    lies inside $\cO_\Cp \langle x_i \rangle$ and reduces to $\bar d^{-1} x_i$
    modulo $\fm_\Cp$. 
  \item If $\sigma \in \Weil_{E_n}$ lies inside the inertia subgroup $I_{E_n}$,
    the element
    $\xi/\sigma(\xi)$ reduces to $1$ modulo $\fm_\Cp$.
  \item If $\sigma \in \Weil_{E_n}$ satisfies $\Art^{-1}_{E_n}(\sigma|_{E_n^\ab}) 
    = \varpi$, the element $\xi/\sigma(\xi)$ reduces to $1$ modulo $\fm_\Cp$. 
\end{enumerate}
The first claim follows directly from the description of $g^*$ in
\cref{lem:ExplicitActionOnQthPowerRootSystems}. Indeed, one quickly checks
that modulo $\fm_\Cp$, we have 
$$\xi^{-1} \gamma_{g,i}^*(\xi x_1, \dots, \xi x_n) \equiv \sum_{j=1}^n \sumH
a_{ji}^{(0)} x_j = (x_1, \dots, x_n).\bar g_i \mod \fm_\Cp.$$ 
Similarly one obtains the second claim.

The third claim follows as $\sigma$ lies in the inertia subgroup,
hence $\xi \equiv \sigma(\xi) \mod \fm_\Cp$.

The fourth statement follows from classical Lubin--Tate theory and the construction
of the Artin map in \cref{sec:Local Class Field Theory}. 
Recall that $\xi = \lim_{k \to \infty} y_k^{q^{nk}}$, where $y =(y_1, y_2, \dots) 
\in T_\varpi H(\cO_\Cp)$ is a primitive element.
Thereby, each component of $y$ lies inside the (totally ramified) Lubin--Tate
extension $E_{n,\varpi}$ (this is an infinite extension, not to be confused with
the finite subextensions $E_{n, \varpi,m}$). By construction of the Artin map,
we have
$$\sigma|_{E_{n,\varpi}} = \Art_{E_{n}}(\varpi)|_{E_{n, \varpi}} =  \id_{E_{n,\varpi}},$$
hence $\sigma$ acts trivially on
the system $(y_k)_{k\in \N}$. As $\sigma$ is continuous, this implies
$\sigma(\xi) = \xi$, concluding the proof.
\end{proof}
% subsubsection Proof of Proposition \ref{prop:J1EquivInjMor} (end)

\subsection{The Explicit Correspondence} % (fold)
\label{sub:The Explicit Correspondence}
In this chapter we apply Mieda's \cref{thm:MiedaAppliedToLTT} to the rational subspace
$U$ and its formal model $\cX$. We show how this allows us to explicitly
calculate certain parts of the local Langlands correspondence for $\GL_n$ over
$E$. Fix, for the remainder of the section, an isomorphism $\Qlbar \cong \C$.

As usual, we let $E_n$ be the degree $n$ unramified extension of $E$. 
We need the following definitions.
\begin{defi}[Regular Tame Character]\label{def:TameCharacter}
  We say that a character $\theta\colon E_n^\times \to \C^\times$ is \emph{tame} if it is
  trivial on the subgroup $1 + \varpi \cO_{E_n} \subset E_n^\times$. 
  If $\theta$ is tame, then its restriction to $\cO_{E_n}^\times$ factors as
  $$\cO_{E_n}^\times \to \FF_{q^n}^\times \xto {\bar \theta} \C^\times.$$ 
  In this case, we say that $\theta$ is \emph{regular} if $\bar \theta$ is
  regular in the sense of \cref{def:RegChar}. 
\end{defi}

Fix a regular tame character $\theta\colon  E_n^\times \to \C^\times$. 
The datum of $\theta$ can be used to construct representations 
of $\Weil_E$ and $D^\times$ and, making use of Deligne--Lusztig
theory, a representation of $\GL_n(E)$. The construction is as follows.
\begin{itemize}
  \item Let $\bar \tau_\theta$ be the character of $\Weil_{E_n}$ given by 
    the composition
    \begin{equation*}
      \Weil_{E_n} \surj \Weil_{E_n}^\ab \xto{\Art_{E_n}^{-1}} E_n^\times \xto
      {\theta \delta^{n-1}} \C^\times
    \end{equation*}
    where $\delta\colon \Weil_{E_n} \to \{\pm 1\}$ the character corresponding to
    $a \mapsto (-1)^{\val_{E_n}(a)}$ under the Artin map $\Art_{E_n}\colon
    E_n^\times \to \Weil_{E_n}^\ab$,
    and define $\tau_\theta \coloneqq \cInd_{\Weil_{E_n}}^{\Weil_E}(\bar \tau_\theta)$.

  \item Let $\bar \rho_\theta$ be the character on 
    $E^\times \cO_D^\times$ given by the composition
    \begin{equation*}
      E^\times\cO_D^\times \cong \varpi^\Z \times \cO_D^\times \surj 
      \cO_D^\times \times \FF_{q^n} \xto {\theta|_{E^\times} \boxtimes \,
      \bar \theta} \C^\times
    \end{equation*}
    and define $\rho_\theta \coloneqq \cInd_{E^\times
    \cO_D^\times}^{D^\times}(\bar \rho_\theta)$.

  \item Let $\bar \pi_\theta$ be the representation of $E^\times \GL_n(\cO_E)$
    given by the composition
    \begin{equation*}
      E^\times \GL_n(\cO_E) \cong \varpi^\Z \times \GL_n(\cO_E)
      \surj \varpi^\Z \times \GL_n(\FF_q) \xto{\theta|_{E^\times} \boxtimes \, 
      R_{\bar\theta}} \GL(V_{\bar \theta}),
    \end{equation*}
    where $(R_{\bar \theta}, V_{\bar \theta})$ is the cuspidal representation 
    corresponding to $\bar \theta$ under the Deligne--Lusztig correspondence laid out
    in \cref{thm:DLCorrespondenceForUs}.
    Define $\pi_\theta \coloneqq \cInd_{E^\times \GL_n(\cO_E)}^{\GL_n(E)}(\bar
    \pi_\theta)$. 
\end{itemize}

\begin{lem}\label{lem:BarRepsAreSmooth}
  The representations   $\pi_\theta$, $\rho_\theta$ and $\tau_\theta$ are smooth. 
  Additionally, $\pi_\theta$ and $\rho_\theta$ are irreducible, and
  $\pi_\theta$ is supercuspidal.
  \begin{proof}
    By design, $\bar \pi_\theta$ is trivial on the compact open subgroup $1 +
    \varpi \Mat_{n \times n}(\cO_E)$ of $E^\times \GL_n(\cO_E)$. Similar
    statements hold for $\bar \rho_\theta$ and $\bar \tau_\theta$. Hence,
    $\pi_\theta$, $\rho_\theta$ and $\tau_\theta$ are smooth as well. 
    In the case that $n=2$, the fact that $\pi_\theta$ is supercuspidal and
    irreducible is laid out in detail in \cite[Section 11]{bushnell2006local}.
    The general case should add no complication. 
    Similarly, irreducibility of $\rho_\theta$ is shown, again in the case 
    $n=2$, in \cite[Section 54]{bushnell2006local}. 
  \end{proof}
\end{lem}

We call a representation $\pi$ \emph{depth zero supercuspidal} if it is equal
to $\pi_\theta$ for some regular tame character $\theta \colon E_n \to \C^\times$. 
The aim of this section is to prove the following statement.
\begin{thm}[Explicit Jacquet--Langlands and Local Langlands Correspondence for
  Depth Zero Supercuspidal Representations]\label{thm:MainRes1}
  Let $\theta\colon E_n \to \C^\times$ be a regular tame character.
  The representation $\JL(\pi_\theta)$ of $D^\times$ and the representation
  $\rec_E(\pi_\theta)$ of $\Weil_E$ corresponding to $\pi_\theta$ under the
  Jacquet--Langlands and local Langlands correspondence take the form
  \begin{equation*}
    \JL(\pi_\theta) = \rho_{\theta}
    \quad \text{and} \quad \rec_E(\pi_\theta) = \tau_\theta.
  \end{equation*}
\end{thm}

The rest of this section will be devoted to the proof of this theorem. 
Recall that $H_\DL$ denotes the middle degree $\ell$-adic cohomology of 
$\DL_n$, cf. \cref{sub:The Deligne--Lusztig Correspondence for the
Explicit Example}. Let us fix a regular tame character
$$\theta\colon E_n \to \C^\times.$$ For now, we assume that $\theta$ is
trivial on $\varpi^\Z$. 

\begin{lem}
  The morphism $\Theta$ from \eqref{eq:ThetaMorDef} makes $J$ act on
  ${\hH}_{\DL,\theta}$. If $\theta$ is trivial on $\varpi^\Z$, this
  representation is of the form 
  \begin{equation*}
    (g,d,\sigma) \mapsto \bar \pi_\theta(g) \otimes \bar \rho_{\theta^{-1}}(d)
    \otimes \bar \tau_\theta^{-1}(\tfrac{1-n}2)(\sigma).
  \end{equation*}
  This representation is smooth.
\begin{proof}
  This follows directly from \cref{thm:MainThmHDlStructure}, and the fact that 
  if $\theta$ is trivial on $\varpi^\Z$, the central characters of $\rho_\theta$
  and $\varpi_\theta$ are trivial on $\varpi^\Z$ as well. 
\end{proof}
\end{lem}

The input we get from Mieda's theory (\cref{sec:Mieda's Approach
to the Explicit Local Langlands Correspondence}) is the
following.
\begin{prop}\label{prop:J1EquivInjMor}
  If $\theta$ is trival on $\varpi^\Z$, then there is an injective morphism of
  $J^1$-representations
  \begin{equation*} 
    \Res_{J^1}^J( \hH_{\DL,\theta} ) \inj \Res_{J^1}^{G^1}( \hH'_{\LT}).
  \end{equation*}
  \begin{proof}
    By the results of the previous two sections, $\cX = \Spf(Q)$ is a formal
    model of the rational subset $U \subset M'_{\infty, \Cp}$
    (\cref{prop:AffinoidIsFormalModel}), 
    $Q$ is flat over $\cO_\Cp$ (\cref{prop:FlatnessOfFormalModel}),
    and there is group action of $J^1$ on $Y_b$, which is compatible with the group
    action on $\cX_s$ obtained from the group action of $J^1$ on $M_{\infty,
    \cO_\Cp}^{(0)}$ (\cref{sub:Proof of Proposition}).
    We are therefore in the situation of Mieda's Injectivity 
    \cref{thm:MiedaAppliedToLTT}, 
    which we want to apply with $V = \hH^{n-1}_c(Y_b, \Qlbar)_\theta$. 
    By part 3 of \cref{lem:ResultsAboutYb}, 
    the natural map $\hH_c^{n-1}(Y_b, \Qlbar)_\theta \to \hH^{n-1}(Y_b,
    \Qlbar)_\theta$ is an isomorphism. Hence the composition
    \begin{equation*}
      \hH^{n-1}_c(Y_b, \Qlbar)_\theta \inj \hH^{n-1}_c(Y_b, \Qlbar)
      \to \hH^{n-1}(Y_b, \Qlbar)
    \end{equation*}
    is injective. 
    \cref{thm:MiedaAppliedToLTT} now yields a $J^1$-equivariant injective morphism
    \begin{equation*}
      \hH_c^{n-1}(Y_b, \Qlbar)_\theta \inj \hH'_\LT.
    \end{equation*}
    By part 2 of \cref{lem:ResultsAboutYb}, the left-hand side
    $J^1$-equivariantly isomorphic to $\hH_c^{n-1}(\DL_n, \Qlbar)_\theta$. The
    claim follows.
  \end{proof}
\end{prop}

Recall that 
\begin{equation*}
  \bar G = \GL_n(E) \times (D^\times/\varpi^\Z) \times \Weil_E,
\end{equation*}
and that $G^1$ is naturally a co-compact and closed normal subgroup of $\bar G$
(\cref{lem:G1subG}). 

\begin{lem} \label{lem:JEquivInjMor}
  Assume that $\theta$ is trivial on $\varpi^\Z$. Then, the injection from
  \cref{prop:J1EquivInjMor} naturally gives rise to an injective
  $J$-equivariant morphism
  \begin{equation*}
    \hH_{\DL,\theta} \inj \Res_J^{\bar G} \hH_\LT.
  \end{equation*}
\begin{proof} % PROOF (fold)
  We construct a sequence of $J$-equivariant injections
  \begin{equation*}
    \hH_{\DL,\theta} \inj \Ind_{J^1}^J (\Res_{J^1}^{G^1} \hH'_{\LT}) \\ \xto
    \sim \Res_J^{G^1J}( \Ind_{G^1}^{G^1J} \hH'_\LT) \inj \Res_J^{\bar G}
    \hH_\LT.
  \end{equation*}
  \textit{The first morphism.}
  This morphism arises by adjunction from the morphism in 
  \cref{prop:J1EquivInjMor}, which is injective by 
  \cref{prop:InducedRepresentationExact} and \cref{lem:FrobRecUnitsAreInjective}.
  
  \textit{The second morphism.} 
  The morphism is given by the inverse of the base-change morphism constructed in
  \cref{lem:BaseChangeForResInd}, which is applied with $H = G^1$, $N = J$. 
  Note that $G^1$ is normal in ${\bar G}$, so the assumptions of the lemma are
  satisfied. As $J$ is open in ${\bar G}$, the map is an isomorphism. 
  
  \textit{The third morphism.}
  Since $G^1J$ is open in ${\bar G}$, the unit of the adjunction $\cInd_{G^1J}^{\bar G} \ladj
  \Res_{G^1J}^{\bar G}$ yields a monomorphism of $G^1J$-representations 
  \begin{equation}\label{eq:MiedaEx1Morph1}
    \Ind_{G^1}^{G^1J} \hH_\LT' \to \Res_{G^1J}^{\bar G}
    (\cInd_{G^1J}^{\bar G}(\Ind_{G^1}^{G^1J} \hH'_\LT)).
  \end{equation}
  As $G^1J$ co-compact in ${\bar G}$, we have $\cInd_{G^1J}^{\bar G} =
  \Ind_{G^1J}^{\bar G}$, so the right-hand
  side is isomorphic to $\Res_{G^1J}^{\bar G}(\Ind_{G^1}^{\bar G} \hH'_\LT) \cong
  \Res_{G^1J}^{\bar G}(\hH_\LT)$ by \cref{prop:InductionOnTower} and 
  \cref{lem:InductionStatementOnHLT}. Hence, applying $\Res_J^{G^1J}$ to the
  morphism in \eqref{eq:MiedaEx1Morph1} yields the desired map.
\end{proof} % PROOF (end)
\end{lem}

\begin{lem}\label{lem:FirstCaseOfMainRes1}
  If $\theta$ is trivial on $\varpi^\Z$, the statements of 
  \cref{thm:MainRes1} hold.   
\begin{proof}
  We show that $\hH_{\LT,\pi_\theta} = \pi_\theta \boxtimes \rho_\theta^\vee 
  \boxtimes \tau_\theta^{\vee}(\tfrac{1-n}2)$. 
  The morphism constructed in \cref{lem:JEquivInjMor} yields, by Frobenius
  reciprocity, a non-zero map of ${\bar G}$-representations
  \begin{equation} \label{eq:WantedMap}
    \Ind_{J}^{\bar G} (\hH_{\DL,\theta}) \cong \pi_\theta \boxtimes \rho_{\theta^{-1}}
    \boxtimes \tau_\theta^\vee (\tfrac{1-n}2) \to \hH_\LT,
  \end{equation}
  where we use that $\tau_{\theta^{-1}} = \tau_\theta^\vee$, which follows from 
  Frobenius reciprocity. The image of this morphism lies inside $\hH_{\LT,
  \pi_\theta}$, the $\pi_\theta$-isotypic component of $\HLT$. 
  As $\pi_\theta$ is supercuspidal and its central character is trivial on 
  $\varpi^\Z$, non-Abelian Lubin--Tate theory (\cref{thm:NonAbLTT}) gives 
  a map 
  \begin{equation*}
    \pi_\theta \boxtimes \rho_{\theta^{-1}} \boxtimes \tau_\theta^{\vee} \to
    \pi_\theta \boxtimes \JL(\pi_\theta)^\vee \boxtimes
    \rec_E(\pi_\theta)^\vee.
  \end{equation*}
  Applying $\Hom(\pi_\theta, -)$, we obtain a non-zero map 
  $\rho_{\theta^{-1}} \boxtimes \tau_\theta^{\vee} \to \JL(\pi_\theta)^\vee
  \boxtimes \rec_E(\pi_\theta)^\vee$.
  As $\rho_{\theta^{-1}}$ and $\JL(\pi_\theta)^\vee$ are irreducible, this 
  implies $\rho_\theta \cong \JL(\pi_\theta)$.
  As $\rec_E(\pi_\theta)$ is irreducible and $\dim(\tau_\theta) = n = \dim
  (\rec_E(\pi_\theta))$, this also implies $\tau_\theta = \rec_E(\pi_\theta)$,
  concluding the proof.
\end{proof}
\end{lem}

\begin{proof}[Proof of \cref{thm:MainRes1}]
  By the lemma above, all that is left to do is to check that the 
  assumption on $\theta$ to be trivial on $\varpi^\Z$ can be removed.   
  Let $\theta\colon E_n^\times \to \C^\times$ be an arbitrary regular
  tame character. Let $\xi\colon E^\times \to \C^\times$ be an unramified character 
  (i.e., a character trivial on $\cO_E^\times$) with $\xi(\varpi)^n =
  \theta(\varpi)$, and put $\theta_0 \coloneqq \theta \cdot (\xi \circ
  \Norm_{E_n/E})^{-1}$. Now $\theta_0$ is tame and we have $\bar \theta_0 =
  \bar \theta$, so $\theta_0$ is regular as well. By construction, we have
  $\theta_0(\varpi) = 1$, so $\theta_0$ is trivial on $\varpi^\Z$ and
  \cref{lem:FirstCaseOfMainRes1} implies that 
  $$\JL(\pi_{\theta_0}) = \rho_{\theta_0} \quad \rec_E(\pi_{\theta_0}) =
  \tau_{\theta_0}.$$
  On the other hand, we have
  \begin{equation*}
    \rho_{\theta} = \rho_{\theta_0} \otimes (\xi \circ \Nrd_{D/E}),
      \quad 
    \pi_{\theta} = \pi_{\theta_0} \otimes (\xi \circ \det)
      \quad \text{and} \quad
    \tau_{\theta} = \tau_{\theta_0} \otimes (\xi \circ \Art_E^{-1});
  \end{equation*}
  the first two statements follow quickly from definition and the third one
  is a consequence of \cref{thm:ArtinMapFunc}.
  
  As the Jacquet-Langlands and local Langlands correspondences are compatible
  with character twist, this gives
  \begin{equation*}
    \JL(\pi_\theta) = \rho_{\theta_0} \otimes (\xi \circ \Nrd_{D/E}) 
    \quad \text{and} \quad
    \rec_E(\pi_{\theta}) = \tau_{\theta_0} \otimes (\xi \circ \Art_E^{-1}).
  \end{equation*}
  The desired statement follows from the identites laid out above.
\end{proof}
% subsection (end)

% section The Explicit Local Langlands Correspondence for Depth Zero Supercuspidal Representations (end)
\end{document}
