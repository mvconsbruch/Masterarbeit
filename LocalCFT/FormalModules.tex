This section will serve as an introduction to formal groups and 
formal modules. Formal groups (or rather, formal group laws) were first
introduced by \textsc{Salomon Bochner} in 1946 as a natural means of studying Lie
Groups over fields of characteristic $0$, cf. \cite{Bochner1946FGrps}. 
The study of formal groups later became interesting for its own right, 
with pioneering works of Lazard \cite{Lazard1955FGrps}.

\subsection{Formal Modules} % (fold)
\label{sub:Formal Modules}
As promised in the introduction, we begin by defining {formal group
laws}.

\begin{defi}[Formal Group Law]
    Let $R$ be a ring. A (commututative, one-\allowbreak
    dimen\-sional) formal group law over $R$ is a power series $F(X,Y) \in
    R\llbr X, Y \rrbr$ such that $F(X,Y) \equiv X + Y$ modulo terms
    of degree $2$ and
    the following properties are satisfied:
    \begin{itemize}
        \item $F(F(X,Y),Z) = F(X,F(Y,Z))$,
        \item $F(X,Y) = F(Y,X)$,
        \item $F(X,0) = X$.
    \end{itemize}
\end{defi}
Given two formal group laws $F, G \in R\llbr X,Y\rrbr$, a morphism
$f: F\to G$ is a 
power series $f \in R\llbr T \rrbr$ such that $f(0) = 0$ and $f(F(X,Y)) =
G(f(X),f(Y))$.
Such a series is an isomorphism if there is an {inverse}, that
is, a power series $g \in R\llbr T \rrbr$ with $(f \circ g)(T) = T$.
This yields the category of formal group laws over $R$, which we notate by
$\FGLover R$.

The following statements about morphisms of formal group laws are 
useful and easily verified.
\begin{lem}\label{lem:FGLeasyfacts}
  Let $R$ be a ring and let $F,G \in R\llbr X,Y\rrbr$ be two formal
  group laws over $R$. 
  \begin{enumerate}
    \item Given two morphisms $f,g : F \to G$, the power series $G(f(T), g(T))
      \in R\llbr T \rrbr$ is a morphism of formal group laws 
      $F \to G$. In particular, $\Hom_\FGLover R(F,G)$ is an abelian
      group for any two formal group laws $F,G$.
    \item The abelian group $\End_\FGLover R(F)$ has a natural ring structure
      with multiplication given by concatenation.
    \item A morphism $f = c_1 T + c_2 T^2 + \dots \in R\llbr T \rrbr$ between
      $F$ and $G$ is an isomorphism if and only if $c_1 \in R^\times$.
  \end{enumerate}
\end{lem}

\begin{xpl} Let us introduce the following two formal group laws.
  \begin{itemize}
    \item \textit{The additive formal group law}. Write 
      $\GG_a$ for the formal group law with addition given by 
      $\GG_a(X,Y) = X + Y$. 
    \item We write $\GG_m$ for the formal group law associated with the 
      with $\GG_m(X,Y) = X + Y + XY$. 
  \end{itemize}
\end{xpl}

Next up is the definition of formal $A$-module laws. Naively, we'd like to say
that an $A$-module law is the same as that of a formal group law $F$
plus $A$-module structure, i.e. a morphism of rings $[\cdot]_F: A \to
\End_\FGLover R(F)$. But there is a subtlety going on here: Let 
\begin{equation*}
  \Lie: \FGLover R \to \Ab 
\end{equation*}
be the (constant) functor that sends $F \in \FGLover R$ to $(R,+)$, and morphisms
$f: G \to H$ given by a formal power series
$f = c_1 T + c_2 T^2 + \dots \in R\llbr T \rrbr$ to the endomorphism
of $R$ given by multiplication with $c_1$. The condition that 
$F(X,Y) \equiv X+Y$ modulo degree $2$ enforces that the induced map
$\End(F) \to \End(R)$ is a morphism of rings. Now, the $A$-module structure on $F$ 
yields an $A$-module structure on $R$, given by the concatenation\begin{equation*}
  A \xto{[\cdot]_F} \End(F) \xto{\Lie} \End(R), \quad a \mapsto \Lie([a]_F)
\end{equation*}

This is a morphism of rings, and we obtain an $A$-algebra structure on $R$. We'd
like the $A$-algebra structure on $R$ to be uniform. This
motivates the following definition.
\begin{defi}[Formal $A$-module law]
  Let $A$ be a ring and $R$ be an $A$-algebra with structure
  morphism $p: A \to R$. A (one-dimensional) $A$-module law over an $R$ is a
  pair $(F, ([a]_F)_{a \in A})$, where $F \in R\llbr X,Y \rrbr$ is 
  a formal group law and $[a]_F = p(a)X + c_2X^2 + \dots 
  \in R\llbr X \rrbr$ yield endomorphisms $F \to F$ such that the induced map
  \begin{equation*}
    A \to \End(F), \quad a \mapsto [a]_F 
  \end{equation*}
  is a morphism of rings.
\end{defi}
Similarly to above, we obtain a category of formal $A$-module laws over $R$,
which we denote by $\FMLOver A R$. Note that $\FGLover R \cong \FMLOver \Z R$.
Slightly abusing notation, we usually do not explicitely mention the $A$-structure
when referring to formal module laws, simply writing $F \in \FMLOver AR$, for
example. 

The following lemma explains a the functoriality of the assignment
$R \mapsto \FMLOver A R$.

\begin{lem}\label{lem:FMLFunc}
  The assignment $R \mapsto \FMLOver A R$ is functorial in the following sense.
  If $p: R \to R'$ is a morphism of $A$-algebras, we obtain a functor
  \begin{equation*}
    \FMLOver A R \to \FMLOver A {R'}, \quad F \mapsto p_*F,
  \end{equation*}
  where $p_*F$ is the formal $A$-module law obtained by applying $p$ to the
  coefficients of the formal power series representing addition
  and scalar multiplication of $F$. We sometimes write
  (with abuse of notation) $p_* F = F \otimes_R R'$. 
\end{lem}

Note that every formal module law $F \in \FMLOver A R$ yields a functor
\begin{equation}\label{eq:fmnilpfunc}
  \Alg R \to \Mod A, \quad S \mapsto \Nil(S),
\end{equation}
where $\Nil(S)$, the set of nilpotent elements of $S$, is equipped with
addition and scalars given by 
\begin{equation*}
  s_1 + s_2 = F(s_1, s_2) \in \Nil(S), \quad a s = [a]_F(s) \in \Nil(S).
\end{equation*}
This construction yields a functor (with slight abuse of notation)
\begin{equation}\label{eq:formfunc}
  \FMLOver A R \to \Fun(\Alg R, \Mod A),
\end{equation}
where $\Fun$ denotes the functor category.

Passing from discrete $R$-algebras to admissible $R$-algebras, this construction extends
naturally to a functor 
\begin{equation*}
  \Spf^F: \FMLOver AR \to \Fun (\Adm R, \Mod A), \quad F \mapsto \Spf R\llbr T \rrbr,
\end{equation*}
where we equip $\Spf R \llbr T \rrbr$ with the structure of an $A$-module object
using the endomorphisms coming from $F$. 
Following this line of thought leads naturally to the definition of
formal modules. 

\begin{defi}[Formal Group and Formal Module.]
  Let $X$ be an $A$-scheme, and let 
  let $\cF$ be an $A$-module object in $\FSchOver X$, the category of formal
  schemes over $X$. Suppose that there is a Zariski-covering
  $(\spec(R_i))_{i \in I}$ of $X$ with $\cF \times_{X} U_i \cong
  \spf(R_i\llbr T \rrbr).$ If for every $i\in I$ the induced $A$-module
  structure on $\spf(R_i\llbr T \rrbr)$ comes from a formal $A$-module law
  $F_i$ over $R_i$, we say that $\cF$ is a formal $A$-module. 
\end{defi}
\begin{defi}[Coordinate]
  Let $\cF$ be a formal $A$-module over $X$. The choice of a cover $\sqcup_{i
  \in I} \spec(R_i) \to X$ together with isomorphisms $\cF \times_X \spec(R_i)
  \cong \Spf(R_i\llbr T \rrbr)$ will be referred to as a coordinate of $\cF$. 
\end{defi}

Of course there is a functor 
\begin{equation*}
  \FMLOver AR \to \FMOver AR,
\end{equation*}
essentially forgetting the choice of module law. The observation of Lemma 
\ref{lem:FMLFunc} translates to formal modules, a morphism $p : R \to R'$ 
yields a functor 
\begin{equation*}
  p_*: \FMOver AR \to \FMOver A{R'}, \quad \cF \mapsto \cF \otimes_R R'.
\end{equation*}

\begin{xpl}
  The additive group law $\GG_a$ extends to a formal $A$-module over an affine base
  $\spec R$ by setting 
  \begin{equation*}
    [a]_{\GG_a}(T) = aT
  \end{equation*}
  for $a \in A$. More generally, we obtain a formal $A$-module over an arbitrary base
  scheme.

  The multiplicative formal group $\GG_m$ does not have an obvious 
  generalization to a formal $A$-module law for general $A$. In the case where $A$
  is the ring of integers of a local field, Lubin and Tate \cite{LubinTateFormalMult} 
  construct such generalizations. This construction, and the application to local
  class fild theory, will be discussed in section \ref{sub:Application: The
  Local Class Field Theory}.
\end{xpl}

\subsubsection{Formal DVR-Modules over Fields of Characteristic 0} % (fold)
\label{ssub:Formal DVR-Modules over Fields of Characteristic 0}
As above, let $A$ be a discrete valuation ring with uniformizer $\pi$ and finite 
residue field $k$. Let $K$ denote the field of fractions of $A$.
% subsubsection Formal DVR-Modules over Fields of Characteristic 0 (end)

\subsubsection{Formal DVR-Modules over Residue Fields}
Let $\FF_q$ denote the finite field with $q = p^n$ elements.
\begin{defi}[Frobenius]
  Given a formal group $F$ over $\FF_q$, let $\phi$ denote the Frobenius
  endomorphism. This is the endomorphism given by $f(T) = T^q$ after choosing
  a coordinate on $F$.
\end{defi}

Let $F$ and $G$ be formal group laws over $\FF_q$, and let $f: F \to G$
be a non-zero morphism between $F$ and $G$ given by a formal power series $f(T) = c_1 T
+ c_2 T + \dots$.

\begin{defi}[Height]
  In the above situation, the height of $f$, denoted $\height(f)$, is the
  greatest integer $h$ such that $f$ factors through $\phi^h: F \to F$. 
  In case that $f = 0$, we write $\height(f) = \infty$. 
\end{defi}

\subsubsection{Tate Modules and the Universal Cover} % (fold)
\label{sub:Tate Modules and the Universal Cover}
Let $A$ be a ring and $R$ be an $A$-algebra. Given $H \in \FMOver AR$ and $a \in A$,
we define the functor 
\begin{equation*}
  \Tilde H_a : \Adm R \to \Mod {A[\tfrac 1a]}, \quad
  S \mapsto \left\{(x_1, x_2, \dots) \in \prod_{\N} H(S) \mid [a]_H(x_{i+1}) =
  x_i \right\}.
\end{equation*}

\begin{defi}[The Universal Cover and Tate Module]
  In case where $A$ is a discrete valuation ring with uniformizer $\pi \in A$ and
  field of fractions $K$, we omit
  $\pi$ from notation and write $\tilde H = \tilde H_\pi$. This functor takes values in
  the category of $K$-vector spaces.
  Up to isomorphism, $\Tilde H$ is uniquely determined. We call this functor
  the universal cover of $H$. 

  The Tate-Module $T_\pi H$ is the subfunctor of $\Tilde H$ cut out out
  by the condition that $[\pi]_H(x_1) = 0$. Note that $T_\pi H$ does no longer 
  carry the structure of a $K$-vector space, it is an $A$-module. The Rational
  Tate Module $V_\pi H$ is the subfunctor of $\Tilde H$ cut out by the
  condition that $x_1$ has 
  $[\pi]_H$-torsion. Equivalently, we have 
  \begin{equation*}
    V_\pi H (S) = T_\pi H(S) \otimes_A K.
  \end{equation*}
\end{defi}



% subsection Tate Modules and the Universal Cover (end)


