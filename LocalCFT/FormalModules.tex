%! TeX root: ../main.tex
\documentclass[../main.tex]{subfiles}

\begin{document}
\subsection{Formal Modules} % (fold)
\label{sub:Formal Modules}

\subsubsection{Formal schemes} % (fold)
\label{ssub:Formal schemes}

\begin{defi}[Admissable and Adic Rings]\cite[Tag 07E8]{stacks-project}
    Let $A$ be a topological ring.
    \begin{enumerate}
        \item $A$ is called \emph{linearly topologized} if $0 \in R$ has a basis of 
            neighbourhoods, which are ideals.
        \item If $A$ is \emph{linearly topologized}, then an ideal $I \subset R$ is 
            called an \emph{ideal of definition}, if $I \subset R$ is open and
            every neighbourhood of $0$ contains $I^n$ for some $n \geq 0$.
        \item $A$ is called \emph{admissible} if $A$ is linearly topologized,
            contains an ideal of definition and $A$ is complete. That is, there
            is an isomorphism of topological rings $A \cong \underset{J}\lim
            A/J$, where $J$ is running through a fundamental systems of
            neighbourhoods of $0$ which are ideals and the right hand side is
            equipped with the limit topology. For an admissible ring $A$, 
            an admissible $A$-algebra is an admissible ring $R$ with a continuous
            homomorphism $A \to R$. These objects form a category, which we 
            denote by $\Adm_A$.
        \item $A$ is called \emph{adic} if $A$ is complete and it is $I$-adic
            for some ideal $I \subset A$. That is, if it is admissable with some
            ideal of definition $I$ such that $I^n$ is open for all $n \in \N$.
            For an adic ring $A$, an adic $A$-algebra is an adic ring $R$ with 
            a continuous homomorphism $A \to R$. These objects form a category,
            which we denote by $\Adic_A$. 
    \end{enumerate}
\end{defi}

We follow \cite{EGA} for the formalism of formal schemes (also, see
\cite[tag 0AHY]{stacks-project})
\begin{defi}[formal scheme]
    A formal scheme is a locally topologically ringed space $\fX$ locally
    isomorphic to $\Spf(B)$ for an admissible ring $B$. 
\end{defi}

\begin{defi}[Formally smooth morphism of formal schemes]
    Let $R$ be an admissible ring. A formal $R$-scheme $\fX \to \spf(R)$ is called
    formally smooth if for every discrete $R$-algebra $S$ and any
    nilpotent ideal $I$, the map $\fX(S) \to \fX(S/I)$ is surjective.
\end{defi}

\begin{defi}[Tangent space]
    Let $R$ be an admissible ring and let $\fX \to \spf(R)$ be a formally smooth
    formal scheme over $\spf(R)$. Suppose there is a section
    $s: \spf(R) \to \fX$ of the structure morphism with image contained in
    some affine open $\spf(B) \subset \fX$. Now $s$ corresponds
    to a continuous surjective homomorphism $B \to R$ with kernel $I$.
    Then we define the tangent space of $\fX$ at $s$ via 
    \begin{equation*}
        \cT_s \fX \coloneqq \Hom_{\Mod_R}(I/I^2, R).
    \end{equation*}
\end{defi}

\begin{defi}[Lie algebra]
    Let $R$ be an admissible ring and let $\cG$ be a formally smooth group
    object in the category of formal schemes over $\spf(R)$, with identity 
    $0_{\cG}: \spf(R) \to \cG$. Then the Lie algebra of $\cG$ is defined as
    \begin{equation*}
        \Lie(\cG) = \cT_{0_{\cG}} \cG,
    \end{equation*}
    i.e., it is the tangent space at the identity.
\end{defi}
% subsubsection Formal schemes (end)

\subsubsection{Formal Groups} % (fold)
\label{ssub:Formal Groups}

\begin{defi}[Formal Group Law]
    Let $R$ be a ring. A (commututative, one-\allowbreak
    dimen\-sional) formal group law over $R$ is a power series $F(X,Y) \in
    R\llbr X, Y \rrbr$ such that $F(X,Y) = X + Y + \textit{higher terms}$ and
    the following properties are satisfied:
    \begin{itemize}
        \item $F(F(X,Y),Z) = F(X,F(Y,Z))$,
        \item $F(X,Y) = F(Y,X)$,
        \item $F(X,0) = X$.
    \end{itemize}
    A formal group law can equivalently be described as a group structure
    on $\spf R\llbr T \rrbr$ in the category of formal schemes such that
    the identity is given by the morphism corresponding to the ring homomorphism
    $[0]: R\llbr T \rrbr \to R, T \mapsto 0$, such that the
    induced group structure on $\cT_{\spf([0])}\spf(R\llbr T \rrbr)$ coincides with
    the natural group structure (as $R$-module). We sometimes write 
    $X +_F Y$ instead of $F(X,Y)$.
\end{defi}

\begin{defi}[Morphisms of formal grup laws.]
    Let $F, G \in R\llbr X, Y \rrbr$ be formal group laws over
    a ring $R$. A \textit{morphism} of formal groups
    $F \to G$ is a power series $f(T) \in T R\llbr T \rrbr$ such that
    $f(F(X,Y)) = G(f(X), f(Y))$. This yields the
    \emph{category of formal groups over $R$}, which we denote $\FGL(R)$. 
    Equivalently, it is a morphism of formal groups
    \begin{equation*}
        (\spf(R\llbr T \rrbr), +_F) \to (\spf(R\llbr T \rrbr), +_G).
    \end{equation*}
\end{defi}

We now introduce formal $A$-module laws, which were first discussed in the context 
of local class field theory by Lubin and Tate in \cite{LubinTateFormalMult}.

\begin{defi}[Formal $A$-module law]
    Let $A$ be a ring and $R$ be an $A$-algebra with structure
    morphism $p: A \to R$. A (one-dimensional) $A$-module law over an $R$ is a
    pair $(F, ([a])_{a \in A})$, where $F \in R\llbr X,Y \rrbr$, is 
    a formal group law, and $[a]_F = p(a)X + \textit{higher terms}
    \in R\llbr X \rrbr$ are power
    series such that the following identities are satisfied:
    \begin{itemize}
        \item \textit{$a \mapsto [a]_F$ is a morphism of monoids:} $[ab]_F(X) = 
            [a]_F \circ [b]_F(X)$.
        \item \textit{distributivity:} $F([a]_F(X), [a]_F(Y)) = [a]_F(F(X,Y))$
    \end{itemize}
    Equvalently, a formal $A$-module law is an $A$-module structure on 
    $\spf(R\llbr T \rrbr)$, such that the addition yields a formal group law
    and the induced $A$-module structure on 
    $\Lie(\spf(R \llbr T \rrbr)) \cong R$ is identical to the natural structure 
    as an $A$-module.
\end{defi}

\begin{defi}[Category of $A$-module laws]
    Let $A$ be a ring and let $F = (F, ([a]_F)_{a\in A})$, $G = (G, ([a]_G)_{a
    \in A})$ be formal $A$-module laws over an $A$-algebra $R$. A
    \textit{morphism of formal module laws} between $F$ and $G$ is a power
    series $f(X) \in R\llbr X \rrbr$ such that
    $f(F(X,Y)) = G(f(X),f(Y))$. This gives rise to the category of formal
    $A$-module laws over $R$, denoted $\FGL_A(R)$. 
\end{defi}

\textbf{Remark.} The theory of $A$-module laws can easily be extended to higher 
dimensions: An $n$-dimensio\-nal formal group law over $R$ corresponds to a
tuple of power series $F_1, \dots, F_n \in \R\llbr X_1, \dots, X_n,\allowbreak
Y_1, \dots, Y_n \rrbr$, such that the evident relations are satisfied.
Similarly, the multiplication-by-$a$ maps become a tuple of power series in $n$
variables.

\begin{defi}[Base change]
    For a morphism of adic rings $p: R \to R'$ and a formal module $G$ over $R$,
    there is a formal module law $G \otimes_R R'$ over $R'$, which 
    is given by applying $p$ to the coefficients of the corresponding power
    series. This gives rise to a functor $- \otimes_R R': \FGL_A(R) \to \FGL_A(R')$, 
    which we call \emph{base change}. 
\end{defi}

\begin{lem}
    The assignment $\spf(R) \mapsto \FGL_A(R)$ together with the base change 
    functors yields a prestack on the category of affine formal schemes over $\spf(A)$.
\begin{proof}
    There is not much to check. \red{TODO}
\end{proof}
\end{lem}

\begin{defi}[Category of formal $A$-modules]
    Let $A$ be a ring.
    Let $\FG_A$ be the stackification of $\FGL_A$ on the big Zariski site in the 
    category of affine formal schemes over $\spec(A)$. For an $A$-algebra
    $R$, a \emph{formal $A$-module over $R$} is an object of $\FG_A(R)$.
    Concretely, a formal $A$-module $\cG \in \FG_A(R)$ is an
    affine formal group scheme such that there is a (finite) cover
    $\bigsqcup_i U_i \to \spec(R)$, where $U_i$ is the distinguished open associated 
    to some $f_i \in R$ and $\cG \times_{\spec(R)} U_i$ is isomorphic (as 
    formal group scheme) to $\spf(R[\tfrac1{f_i}] \llbr T \rrbr)$, equipped with
    a formal $A$-module law. A morphism of two formal groups is simply
    a morphism of $A$-module objects in the category of formal schemes over $A$.
\end{defi}
\textbf{Remark.} Here we defined \textit{one-dimensional} formal group laws. This 
definition can be extended to arbitrary dimensions in the evident way. The remark 
above explains how to obtain formal group laws of arbitrary dimension, and we obtain
formal groups of arbitrary dimension after stackification.

\vspace{6pt}
What follows is some study on the essential image of the (fully faithful)
functor $\FGL_A(R) \to
\FG_A(R)$.

\begin{defi}[]
    Let $R$ be a ring, and let $M$ be a $R$-module. Define
    $R\llbr M \rrbr$ as the completion of $\Sym^\bullet M$ with respect to 
    the $M$-adic topology. Then $R\llbr M \rrbr$ represents the functor 
    \begin{equation*}
        S \in \Adm_R \mapsto \Hom_{\Mod_R}(M, S^\cici).
    \end{equation*}
\end{defi}

\begin{lem}
    Let $A$ be an admissible ring and let $R$ be an admissible $A$-algebra. 
    Let $\cG \in \FG_A(R)$. Then the formal scheme $\cG$ is isomorphic to
    $\spf(R \llbr M \rrbr)$ for some finite projective $R$-module $M$.
\end{lem}
\begin{proof}
    \red{TODO!}
\end{proof}

\begin{lem}\label{lem:inversegrouphom}
    Let $F,G$ be one-dimensional formal $A$-module laws over $R \in \Alg_A$.
    Then any non-zero homomorphism $f\in \Hom_{\FGL_A(R)}(F,G)$ 
    given by a power series of the form $f(X) = c_1X + \textnormal{higher
    terms}$ with $c_1 \in R^\times$ is an isomorphism.
\begin{proof}
    Without loss of generality we can assume that $c_1 = 1$. Let us recursively
    contruct a left inverse $e: G \to F$ of $f$, that is, a power series
    $e \in R\llbracket X \rrbracket$ such that $e(f(X)) = X$. Set
    $e_1(X) = X$. Now let $n$ be arbitrary and suppose we have a power series 
    $e_n$ such that $e_n(f(X)) = X + cX^{n+1} + \textit{higher terms}$. Then we 
    set $e_{n+1}(X) = e_n(X) - cX^{n+1}$, and we verify that 
    $e_{n+1}(f(X)) = X + c'X^{n+2} + \textit{higher terms}$, as desired. Now set
    $e(X) \coloneqq \lim_{n \to \infty} e_{n}(X)$. By construction we have
    $e(f(X)) = X$. Using the same technique one can construct the (unique) right inverse 
    $e' \in R\llbr X \rrbr$ of $f$, i.e., a power series with $f(e'(X)) = X$. 
    One readily checks that $e = e'$, and that $e$ is a map of $A$-module laws.
\end{proof}
\end{lem}

% subsubsection Formal groups(end)

\subsubsection{Formal modules over rings of integers} % (fold)
\label{ssub:Formal modules over rings of integers}
For our applications, or more generally in the setting of local class field theory,
formal $A$-modules arise for $A$ the ring of integers of a
non-Archimedean local field $K$. So let us now fix such a field, denote by
$\cO_K$ its ring of integers, in which we fix a uniformizer $\varpi \in \cO_K$. 
Let $\breve K$ denote the completion of the maximal unramified extension of $K$.
We will furthermore be mostly interested in formal $\cO_K$-modules over noetherian local
$\cO_{\breve K}$-algebras with residue field $\bar \FF_q$, which are complete
and Hausdorff (although this is automatic for our notion of completeness) with
respect to the $\varpi$-adic topology. Let us denote the category of such rings
with $\cC$. For sake of clarity, the succeeding results and definitions on
formal groups will be stated in this setting. Most of the results readily
generalize to more general situations.


\begin{defi}[Isogenies]
    Let $\cF, \cG \in \FG_{\cO_K}(R)$ be two formal $\cO_K$-modules over $R$, and 
    let $f: \cF \to \cG$ be a morphism between $\cF$ and $\cG$. Write
    $\ker f = \cF \times_{\cG, 0_G} \spec(R)$ for the kernel of $f$. We call
    $f$ a \emph{isogeny} if $\ker f$ is finite and locally free over $\spec(R)$.
\end{defi}

\begin{defi}[The height function]
    Assume that $\pi = 0$ in $R$. 
    Let $\cG \in \FG_{\cO_K}(R)$, and pick some coordinate $\cG \cong
    \spf(k\llbr T \llbr)$ giving rise to some formal $A$-module structure $(G,
    [-]_G)$. The \emph{height} of $\cG$ is defined as the largest integer $n$
    such that $[\pi]_G(X) = a_1X^{q^n} + a_2 X^{q^{2n}} + \dots.$ If
    $[\varpi]_G = 0$, we say that $G$ has infinite height. 
    \vspace{6pt}
    More generally, if $R \in \Alg_A$ define the function
    \begin{equation*}
        \height_\cG: \spec(R) \to \Z_{\geq 0} \cup \{\infty\}, 
        x \mapsto \height(\cG \times_{\spec(R)} \spec(k(x)))
    \end{equation*}
\end{defi}

\begin{defi}[]
    Let $R \in \Adic_A$ and let $\cG$ be a formal $A$-module over $R$. Then 
    we call $G$ a $\pi$-divisible formal $A$-module if $[\pi]_{\cG}: \cG \to \cG$ 
    is a isogeny. 
\end{defi}


% subsubsection The height(end)


\subsubsection{The universal cover} % (fold)
We keep the conventions on $A$ and $R$ from the last section. 

\label{sub:The universal cover}
\begin{defi}[universal cover of $G$]
    Let $R \in \Adic_{\cO_{A}}$ and let $G$ be a $\varpi$-divisible
    formal $\cO_K$-module over $R$. We define a functor
    \begin{equation*}
        \tilde G: \Adic_R \to \Vec_K, \quad S \mapsto \lim_{x \mapsto
        [\varpi]_G(x)} G(S) = \{(x_0, x_1, \dots ) \mid [\varpi]_G(x_i) =
        x_{i-1}\}.
    \end{equation*}
    We call $\tilde G$ the \emph{universal cover} of $G$. 
\end{defi}

\begin{defi}[Tate module]
    For $A$ and $G$ as above, denote by
    $T_\varpi G$ The \textit{Tate module} associated to $G$. This is the functor
    sending $R \in \Adm_S$ to the
    $\cO_K$-module
    \begin{equation*}
        \{(x_1, x_2, \dots ) \in \tilde G(R) \mid [\varpi]_G(x_1) = 0\},
    \end{equation*}
    which is naturally a subfunctor of $\tilde G$. The rational Tate module $V_\varpi G$,
    given by the functor $R \mapsto (T_\varpi G)(R) \otimes_{\cO_K} K$ also embedds
    into $\tilde G$.
\end{defi}

\begin{defi}[Tilting]
    Let $R \in \Adic_A$. We write
    \begin{equation*}
        R^\flat = \lim_{x \mapsto x^q} R/(\pi) = \left\{(x_1, x_2 ,\dots) \in
        \prod_\N (R/(\pi)) \mid x_{i+1}^q = x_i\right\}
    \end{equation*}
    for the \emph{tilt} of $A$. We write 
    \begin{equation*}
        \Nilp^\flat(R) \coloneqq \{(x_1, x_2, \dots) \in R^\flat \mid 
        x_1 \in \Nilp(R/(\pi))\}.
    \end{equation*}
\end{defi}
\blue{$A^\flat$ is a perfect $A/(\pi)$-algebra.}

\begin{prop}\label{prop:Teichmuellerlift}
    Let $R \in \Adic_A$ with ideal of definition $I$ containing the image of $\pi$. Then
    the natural map
    \begin{equation*}
        \Pi: \lim_{x \mapsto x^q} R \to (R/I)^\flat 
    \end{equation*}
    is bijective. In particular, this equips the left hand side with the natural
    structure of an $A$-algebra. Similarly, we have
    \begin{equation*}
        \lim_{x \mapsto x^q} \Nilp(R) \xto \sim \Nilp^\flat(R).
    \end{equation*}
\begin{proof}
    We construct an inverse $\Psi$ of $\Pi$. Let $x = (x_1, x_2, \dots) \in (R/I)^\flat$ 
    and let $(\tilde x_1, \tilde x_2, \dots) \in \prod_\N R$ be a arbitrary
    lift. One can check that the sequence $(x_i^{q^i})_{i \in \N}$ is a
    Cauchy-sequence in $R$, and we obtain an element $x^\sharp = \lim_{i \to
    \infty} x_i^{q^i} \in R$. Similarly, we obtain elements $(x^\sharp)^{1/q} =
    \lim_{i \to \infty} x_i^{q^{i-1}}$, $(x^\sharp)^{1/q^2} = \lim_{i \to
    \infty} x_i^{q^{i-2}}$, \dots, yielding a lift $(x^\sharp,
    (x^\sharp)^{1/q}, (x^\sharp)^{1/q^2}, \dots)$. It turns out that the choice of the 
    elements $\tilde x_i$ does not change the element $x^\sharp$ and it's distinguished
    $q$-power roots, so the map 
    $$\Psi: (A/I)^\flat \to \lim_{x \mapsto x^q} A , \quad x \mapsto x^\sharp$$
    is well-defined. 
    It is immediate that $\Pi \circ \Phi = \id$, and using the uniqueness of the
    lifts it is readily confirmed that $\Phi \circ \Pi = \id$. 

    The second part of the proposition follows by identifying the image of 
    $\lim_{x\mapsto x^q} \Nilp(R) \subset \lim_{x \mapsto x^q} R$ under 
    $\Pi$. 
\end{proof}
\end{prop}


\begin{prop}\label{prop:UnivCovRepFq}
    Assume that $R \in \Adic_{A}$ is a $\bar \FF_q$-algebra (i.e., $\pi$
    vanishes in $A$) and let $G \in \FG_{\cO_K}(R)$ be a $\pi$-divisible
    formal group of height $n$ in the essential image of 
    $\FGL_A(R) \to \FG_A(R)$. The choice of a coordinate $G \cong \spf(R\llbr
    T \rrbr)$ yields an isomorphism $\tilde G \cong \spf(R\llbr
    T^{1/q^\infty}\rrbr)$. In particular, there is a isomorphism of functors
    \begin{equation*}
        \tilde G \cong \Nilp^\flat.
    \end{equation*}
\begin{proof}
    We can write $[\varpi]_G(X) = g(X^{q^n}) \in \bar \FF_q\llbr X \rrbr$ for some 
    $g(X) = c_1X + c_2X^2 + \dots$ with $c_1 \neq 0$. For each coefficient $c_i$, let
    $d_i \in \bar \FF_q$ be the unique element such that $d_i^{q^n} = c_i$. Let
    $h \in \bar \FF_q\llbr X \rrbr$ be the power series given by $d_1 X + d_2
    X^2 + \dots$. Now $(h(X))^{q^n}=[\varpi]_G(X)$, and we find that 
    \begin{equation*}
        \tilde G(S) \to \lim_{x \mapsto x^{q^n}} S^\cici: \quad
        (x_1, x_2, x_3, \dots) \mapsto (x_1, h(x_2), h(h(x_3)), \dots)
    \end{equation*}
    is a well-defined function, and (trivially) functorial in $S$. For the
    inverse, let $h^{-1}(X) \in \bar \FF_q\llbr X \rrbr$ be the inverse power
    series of $h$ as in lemma \ref{lem:inversegrouphom}, which is the unique power
    series with $h^{-1}(h(X))= h(h^{-1}(X)) = X$. The map
    \begin{equation*}
        \lim_{x \mapsto x^{q^n}} S^\cici \to \tilde G(S), \quad 
        (x_1, x_2, \dots ) \mapsto (x_1, h^{-1}(x_2), h^{-1}(h^{-1}(x_3)), \dots)
    \end{equation*}
    is well-defined as
    \begin{equation*}
        [\varpi]_G(h^{-1}(X)) = g((h^{-1}(X))^{q^n}) = (h(h^{-1}(X)))^{q^n} =
        X^{q^n}
    \end{equation*}
    and it is readily seen to be inverse to the map constructed above.
\end{proof}
\end{prop}

Now let $\breve K$ denote the completion of the maximal unramified extension of 
$K$, and write $\cO_{\breve K}$ for its ring of integers. 

\begin{prop}\label{prop:UnivCovRep}
    Now let $R$ be an adic $\cO_{\breve K}$-algebra with
    ideal of definition $I$. Let $G \in \FG_{\cO_K}(\cO_{\breve K})$ be a 
    $\varpi$-divisible formal group in the essential image of 
    $\FGL_{\cO_K}(R) \to \FG_{\cO_K}(R)$. Fix an isomorphism $G \cong
    \spf(A\llbr T \rrbr)$, and write $G_0$ for $G \otimes (R/I)$. 
    \begin{enumerate}
        \item For $R \in \Adic_A$ with ideal of definition $I$, the reduction
            map $\tilde G(R) \to \tilde G_0(R/I)$ is an isomorphism of 
            $K$-vector spaces.
        \item If $R/I$ is a perfect field, there is an isomorphism $\tilde G(R)
            \cong \Nilp^\flat(R)$.
    \end{enumerate}
\begin{proof}
    \red{This is \cite[Proposition 2.7]{weinstein2016semistable}}.
    To see injectivity of the map, write $I_R$ for the kernel of $R \to R \otimes
    (A/I)$. Let $(x_1, x_2, \dots) \in \tilde G(R)$ lie in the kernel of $G(R)
    \to G_0(R/I)$. Now $x_i \in I_R$ for all $i$. As $\varpi \in I_R$, we find 
    $$x_{i} = [\varpi]_G(x_{i+1}) = \underbrace{\varpi x_{i+1}}_{\in I_R^2} +
    \underbrace{c_2 x_{i+1}^2 + \dots}_{\in I_R^2},$$
    showing that $x_i \in \cap_{m\geq 1} I_R^m = \{0\}$ for all $I$.
    Surjectivity is a corollary of proposition \ref{prop:Teichmuellerlift}. 
    Suppose $(x_1, x_2, \dots) \in \tilde G_0(R/I)$. Let
    $\tilde x_1, \tilde x_2, \dots) \in \tilde G(R)$ be any lift. The same idea
    as in the proof of proposition \ref{prop:Teichmuellerlift} shows that the
    sequence $([\varpi]_G^{m} x_{i+m})_{m\geq 0}$ converges in $\tilde G(R)$, yielding 
    an element $z_i \in \tilde G(R)$. Now by construction, $(z_1, z_2, \dots) \in 
    G(R)$ maps to $(x_1, x_2, \dots)\in G_0(R/I)$. 
    Part two is a direct consequence of part 1 and Proposition
    \ref{prop:Teichmuellerlift}.
\end{proof}
\end{prop}

\color{darkgray}
\begin{itemize}
    \item I should probably assume that $R \in \Alg_A$ and later impose that $R$ is 
        $\pi$-complete, rather than imposing $R \in \Adic_A$ from the start. But 
        that's just a thought.
\end{itemize}

\color{black}

\subsubsection{The vertical map} % (fold)
\label{sub:The vertical map}
Let $G: \cC \to \Mod_{\cO_K}$ be a (one-dimensional) $\varpi$-divisible formal group over
$\cO_{\breve K}$ and denote $G_0 = G \otimes_{\cO_{\breve K}} \bar \FF_q$,
which we assume to be of height $n$. In this subsection we construct a is a
natural map $\cM^{(0)}_{G_0, \infty} \to \tilde G^n$. The construction of the
map $\cM_{G_0, \infty}^{(0)} \to \tilde G^n$ is canonical (except for the
choices of representing objects $\cM_{G_0,m}^{(0)} \cong A_m$).

Note that for any $m \geq 0$, there is the universal triple $U_m = (G^{(m)},
\iota^{(m)}, \phi^{(m)}) \in \cM^{(0)}_{G_0, m}(A_m)$. We get an infinite
commutative diagram
\[\begin{tikzcd}[ampersand replacement=\&]
	\&\&\& \iddots \\
	\&\& {\cM_{G_0,{m+2}}^{(0)}(A_{m+2})} \& \dots \\
	\& {\cM_{G_0,{m+1}}^{(0)}(A_{m+1})} \& {\cM_{G_0,{m+1}}^{(0)}(A_{m+2})} \& \dots \\
	{\cM_{G_0,m}^{(0)}(A_m)} \& {\cM_{G_0,m}^{(0)}(A_{m+1})} \& {\cM_{G_0,{m}}^{(0)}(A_{m+2})} \& \dots
	\arrow[from=4-1, to=4-2]
	\arrow[from=4-2, to=4-3]
	\arrow[from=3-2, to=3-3]
	\arrow[from=3-2, to=4-2]
	\arrow[from=3-3, to=4-3]
	\arrow[from=2-3, to=3-3]
	\arrow[from=4-3, to=4-4]
	\arrow[from=3-3, to=3-4]
	\arrow[from=2-3, to=2-4]
\end{tikzcd}\]
This behaves well under representing objects. Indeed, if we denote for $n > m$ the image
of $U_m$ under the (push forward along the) transition map $\tau_{m,n}:A_m \to A_n$ as
$U_{m,n}$, we find the following mappings
\[\begin{tikzcd}[ampersand replacement=\&]
	\&\&\& \iddots \\
	\&\& {U_{m+2,m+2}} \& \dots \\
	\& {U_{m,m+1}} \& {U_{m+1,m+2}} \& \dots \\
	{U_{m,m}} \& {U_{m,m+1}} \& {U_{m,m+2}} \& \dots
	\arrow[maps to, from=4-1, to=4-2]
	\arrow[from=3-2, to=3-3]
	\arrow[maps to, from=3-2, to=4-2]
	\arrow[maps to, from=3-3, to=4-3]
	\arrow[maps to, from=2-3, to=3-3]
	\arrow[maps to, from=4-3, to=4-4]
	\arrow[maps to, from=3-3, to=3-4]
	\arrow[maps to, from=2-3, to=2-4]
	\arrow[maps to, from=4-2, to=4-3]
\end{tikzcd}\]
In these diagrams, the horizontal maps correspond to base change, while the 
vertical maps are given by shrinking the level structure. 
As the transition morphisms $\tau_{m,m+1}: A_m \to A_{m+1}$ are injective, the
horizontal morphisms injective as well. Moreover, the base change morphisms leave the 
deformations invariant: If $(H,\iota_H, \phi_H)\in \cM_{G_0,m}^{(0)}(A_n)$, we have 
$H\otimes_{A_n}A_{n+1} = H$ as functors $\cC \to \Set$, and $\iota
\otimes_{A_n} A_{n+1} = \iota$ as isomorphisms (of functors) $H \otimes_{\cO_{\breve K}}
({\cO_{\breve K}}/\fm_{K}) \to G_0$. 
Now the universal object $U_m \in \cM_{G_0,m}^{(0)}(A_m)$ gives rise to a 
Drinfeld basis $$(X_1^{(m)}, \dots, X_n^{(m)}) \in G^{(m)}(A_m)[\pi^m] =
G^{(0)}(A_m)[\pi^m] \subset G^{(0)}(A_m).$$ 
Note that $[\varpi]_{G^{(0)}}(X_i^{(k+1)}) = X_i^{(k)} \in G^{(0)}(A_m)$ for any 
$0 \leq k < m$. Also,
$A_m$ injects into $A_\infty$, so in the limit, we obtain elements
\begin{equation*}
    (X^{(m)})_{m \geq 1} \in \lim_{x \mapsto [\varpi]_{G^{(0)}}(x)} \left( \colim\limits_mG^{(0)}(A_m) \right) 
\end{equation*}
Now the injective morphism 
\begin{equation*}
    \lim_{x \mapsto [\varpi]_{G^{(0)}}(x)} \left( \colim\limits_mG^{(0)}(A_m)
    \right) \inj \widetilde{G^{(0)}}(A_\infty)
\end{equation*}
yields elements $Y_1, \dots, Y_n \in \widetilde{G^{(0)}}(A_\infty)$. \red{Here
    we evaluated $\widetilde{G^{(0)}}$ at $A_\infty$, an element that does not
lie in $\cC$.} Finally, we use the canonical isomorphisms 
\begin{equation}\label{eq:isos}
    \widetilde {G^{(0)}}(A_\infty) \cong \widetilde {G^{(0)}}(A_\infty/\fm_{A_\infty})
    = \widetilde{G_0}(A_\infty/\fm_{A_\infty}) \cong \widetilde G(A_\infty/\fm_\infty)
    \cong \tilde G(A_\infty).
\end{equation}
The images of $Y_1, \dots, Y_n$ under these yield identifications yield the
desired morphisms $\cM_{G_0,\infty}^{(0)} \to \tilde G$.

Remember that by proposition \ref{prop:UnivCovRep}, $\tilde G \cong \cO_{\breve
K}\llbr X^{1/q^\infty} \rrbr$, which yields $\tilde G^n \cong \cO_{\breve
K}\llbr X_1^{1/q^{\infty}}, \dots, X_{n}^{1/q^\infty} \rrbr$. Under such an identification,
the morphism constructed above corresponds to $n$ elements $Z_1, \dots, Z_n \in 
A_\infty$ with distinguished $p$-power roots.
% subsection subsection name (end)
\color{darkgray}

\begin{itemize}
    \item Are the isomorphisms in equation \ref{eq:isos} natural?
\end{itemize}
\color{black}

% subsection The universal cover

\subsection{Dieudonn\'e modules and Zink's displays} % (fold)
\label{sub:The Dieudonne-module}

% subsection subsection name (end)

\subsection{Determinants} % (fold)
\label{sub:Determinants}
In this subsection, $G$ will only be a formal $\cO_K$-algebra over some
$R \in \cC$, and $G_0 = G \otimes_R \bar \FF_q$ is the reduction of $G$
to the residue field of $R$, which is assumed to be a formal $\cO_K$-module
of height $n$.

The following is \cite[Theorem 4.34]{hedayatzadeh2015det}.
\begin{prop}
    Let $\cO_K$ be the ring of integers of a non-Archimedean local field $K$ of
    characteristic zero. Fix a uniformizer $\varpi$ of $\cO_K$ and let $S$ be a
    locally Noetherian scheme over $\spec \cO_K$ and $G$ be a
    $\varpi$-divisible $\cO_K$-module over $S$. Then there exists a
    $\varpi$-divisible module $\bigwedge_{\cO_K}^r G$ over $S$ of height $\binom
    hr$ and an alternating morphism $\lambda: G^r \to \bigwedge^r_\cO G$
    that is universal in the following way. For every morphism $f: S' \to S$
    and every $\varpi$-divisible module $H$ over $S'$, the
    homomorphism 
    \begin{equation*}
        \Hom_{S'}(f^* \bigwedge^r_{\cO_K} G, H) \to
        \Alt_{S'}((f^*G)^r, H)
    \end{equation*}
    induced by $f^*\lambda$ is an isomorphism.
\end{prop}
The proof relies on Zink's theory of displays. 
Taking torsion, we obtain the following corollary.

\begin{cor}\label{cor:findetmaps}
    Let $G$ be a $\varpi$-divisible formal $\cO_K$-module over some Noetherian 
    base ring $A$.
    Then there are universal multilinear and alternating maps 
    \begin{equation*}
        \lambda_m: G[\varpi^m]^n \to \wedge^n G[\varpi^m].
    \end{equation*}
\end{cor}

\begin{prop}
    Let $G$ be as above, and suppose that $(X_1, \dots, X_n) \in G[\varpi^m]^n$
    is a Drinfeld basis. Then $\lambda_m(X_1, \dots, X_n) \in \wedge^nG[\varpi^m]$ 
    is a Drinfeld basis.
\begin{proof}

\end{proof}
\end{prop}


% subsection subsection name (end)

\color{darkgray}
\section*{Open Questions}

\color{black}

% subsection subsection name (end)

\end{document}
