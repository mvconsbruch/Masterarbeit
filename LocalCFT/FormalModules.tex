This section will serve as an introduction to formal groups and 
formal modules. Formal groups (or rather, formal group laws) were first
introduced by \textsc{Salomon Bochner} in 1946 as a natural means of studying Lie
Groups over fields of characteristic $0$, cf. \cite{Bochner1946FGrps}. 
The study of formal groups later became interesting for its own right, 
with pioneering works of Lazard \cite{Lazard1955FGrps}.

\subsection{Formal Modules} % (fold)
\label{sub:Formal Modules}
As promised in the introduction, we begin by defining {formal group
laws}.

\begin{defi}[Formal Group Law]
    Let $R$ be a ring. A (commututative, one-\allowbreak
    dimen\-sional) formal group law over $R$ is a power series $F(X,Y) \in
    R\llbr X, Y \rrbr$ such that $F(X,Y) \equiv X + Y$ modulo terms
    of degree $2$ and
    the following properties are satisfied:
    \begin{itemize}
        \item $F(F(X,Y),Z) = F(X,F(Y,Z))$,
        \item $F(X,Y) = F(Y,X)$,
        \item $F(X,0) = X$.
    \end{itemize}
\end{defi}

Given two formal group laws $F, G \in R\llbr X,Y\rrbr$, a morphism
$f: F\to G$ is a 
power series $f \in R\llbr T \rrbr$ such that $f(0) = 0$ and $f(F(X,Y)) =
G(f(X),f(Y))$.
Such a series is an isomorphism if there is an {inverse}, that
is, a power series $g \in R\llbr T \rrbr$ with $(f \circ g)(T) = T$.
This yields the category of formal group laws over $R$, which we notate by
$\FGLover R$.

The following statements about morphisms of formal group laws are 
useful and easily verified.
\begin{lem}\label{lem:FGLeasyfacts}
  Let $R$ be a ring and let $F,G \in R\llbr X,Y\rrbr$ be two formal
  group laws over $R$. 
  \begin{enumerate}
    \item Given two morphisms $f,g : F \to G$, the power series $G(f(T), g(T))
      \in R\llbr T \rrbr$ is a morphism of formal group laws 
      $F \to G$. In particular, $\Hom_\FGLover R(F,G)$ is an abelian
      group for any two formal group laws $F,G$.
    \item The abelian group $\End_\FGLover R(F)$ has a natural ring structure
      with multiplication given by concatenation.
    \item A morphism $f = c_1 T + c_2 T^2 + \dots \in R\llbr T \rrbr$ between
      $F$ and $G$ is an isomorphism if and only if $c_1 \in R^\times$.
  \end{enumerate}
\end{lem}

\begin{xpl} Let us introduce the following two formal group laws.
  \begin{itemize}
    \item \textit{The additive formal group law}. Write 
      $\GG_a$ for the formal group law with addition given by 
      $\GG_a(X,Y) = X + Y$. 
    \item We write $\GG_m$ for the formal group law associated with the 
      with $\GG_m(X,Y) = X + Y + XY$. Note that $\GG_m(X,Y) = (X+1)(Y+1) - 1$
  \end{itemize}
\end{xpl}

Next up is the definition of formal $A$-module laws. Naively, we'd like to say
that an $A$-module law is the same as that of a formal group law $F$
plus $A$-module structure, i.e. a morphism of rings $[\cdot]_F: A \to
\End_\FGLover R(F)$. But there is a subtlety here: Let 
\begin{equation*}
  \Lie: \FGLover R \to \Ab 
\end{equation*}
be the (constant) functor that sends $F \in \FGLover R$ to $(R,+)$, and morphisms
$f: G \to H$ given by a formal power series
$f = c_1 T + c_2 T^2 + \dots \in R\llbr T \rrbr$ to the endomorphism
of $R$ given by multiplication with $c_1$. The condition that 
$F(X,Y) \equiv X+Y$ modulo degree $2$ enforces that the induced map
$\End(F) \to \End(R)$ is a morphism of rings. Now, the $A$-module structure on $F$ 
yields an $A$-module structure on $R$, given by the concatenation\begin{equation*}
  A \xto{[\cdot]_F} \End(F) \xto{\Lie} \End(R), \quad a \mapsto \Lie([a]_F)
\end{equation*}

This is a morphism of rings, and we obtain an $A$-algebra structure on $R$. We'd
like the $A$-algebra structure on $R$ to be uniform. This
motivates the following definition.
\begin{defi}[Formal $A$-module law]
  Let $A$ be a ring and $R$ be an $A$-algebra with structure
  morphism $p: A \to R$. A (one-dimensional) $A$-module law over an $R$ is a
  pair $(F, ([a]_F)_{a \in A})$, where $F \in R\llbr X,Y \rrbr$ is 
  a formal group law and $[a]_F = p(a)X + c_2X^2 + \dots 
  \in R\llbr X \rrbr$ yield endomorphisms $F \to F$ such that the induced map
  \begin{equation*}
    A \to \End(F), \quad a \mapsto [a]_F 
  \end{equation*}
  is a morphism of rings.
\end{defi}
Similarly to above, we obtain a category of formal $A$-module laws over $R$,
which we denote by $\FMLOver A R$. Note that $\FGLover R \cong \FMLOver \Z R$.
Slightly abusing notation, we usually do not explicitely mention the $A$-structure
when referring to formal module laws, simply writing $F \in \FMLOver AR$, for
example. 

The following lemma explains a the functoriality of the assignment
$R \mapsto \FMLOver A R$.

\begin{lem}\label{lem:FMLFunc}
  The assignment $R \mapsto \FMLOver A R$ is functorial in the following sense.
  If $p: R \to R'$ is a morphism of $A$-algebras, we obtain a functor
  \begin{equation*}
    \FMLOver A R \to \FMLOver A {R'}, \quad F \mapsto p_*F,
  \end{equation*}
  where $p_*F$ is the formal $A$-module law obtained by applying $p$ to the
  coefficients of the formal power series representing addition
  and scalar multiplication of $F$. We sometimes write
  (with abuse of notation) $p_* F = F \otimes_R R'$. 
\end{lem}

Note that every formal module law $F \in \FMLOver A R$ yields a functor
\begin{equation}\label{eq:fmnilpfunc}
  \Alg R \to \Mod A, \quad S \mapsto \Nil(S),
\end{equation}
where $\Nil(S)$, the set of nilpotent elements of $S$, is equipped with
addition and scalars given by 
\begin{equation*}
  s_1 + s_2 = F(s_1, s_2) \in \Nil(S), \quad a s = [a]_F(s) \in \Nil(S).
\end{equation*}
This construction yields a functor (with slight abuse of notation)
\begin{equation}\label{eq:formfunc}
  \FMLOver A R \to \Fun(\Alg R, \Mod A),
\end{equation}
where $\Fun$ denotes the functor category.

Passing from discrete $R$-algebras to admissible $R$-algebras, this construction extends
naturally to a functor 
\begin{equation*}
  \Spf^F: \FMLOver AR \to \Fun (\Adm R, \Mod A), \quad F \mapsto \Spf R\llbr T \rrbr,
\end{equation*}
where we equip $\Spf R \llbr T \rrbr$ with the structure of an $A$-module object
using the endomorphisms coming from $F$. 
Following this line of thought leads naturally to the definition of
formal modules. 

\begin{defi}[Formal Group and Formal Module.]
  Let $X$ be an $A$-scheme, and let 
  let $\cF$ be an $A$-module object in $\FSchOver X$, the category of formal
  schemes over $X$. Suppose that there is a Zariski-covering
  $(\spec(R_i))_{i \in I}$ of $X$ with $\cF \times_{X} U_i \cong
  \spf(R_i\llbr T \rrbr).$ If for every $i\in I$ the induced $A$-module
  structure on $\spf(R_i\llbr T \rrbr)$ comes from a formal $A$-module law
  $F_i$ over $R_i$, we say that $\cF$ is a formal $A$-module. 
\end{defi}
\begin{defi}[Coordinate]
  Let $\cF$ be a formal $A$-module over $X$. The choice of a cover $\sqcup_{i
  \in I} \spec(R_i) \to X$ together with isomorphisms $\cF \times_X \spec(R_i)
  \cong \Spf(R_i\llbr T \rrbr)$ will be referred to as a coordinate of $\cF$. 
\end{defi}

Of course there is a functor 
\begin{equation*}
  \FMLOver AR \to \FMOver AR,
\end{equation*}
essentially forgetting the choice of module law. The observation of Lemma 
\ref{lem:FMLFunc} translates to formal modules, a morphism $p : R \to R'$ 
yields a functor 
\begin{equation*} p_*: \FMOver AR \to \FMOver A{R'}, \quad \cF \mapsto \cF \otimes_R R'.
\end{equation*}

\begin{xpl}
  The additive group law $\GG_a$ extends to a formal $A$-module over an affine base
  $\spec R$ by setting 
  \begin{equation*}
    [a]_{\GG_a}(T) = aT
  \end{equation*}
  for $a \in A$. More generally, we obtain a formal $A$-module over an arbitrary base
  scheme.

  The formal group associated to $\GG_m$ over $\Z_p$ is isomorphic to the functor
  \begin{equation*}
    \Adm {\Z_p} \to \Ab, \quad S \mapsto 1 + S^\cici \subset S^\times.
  \end{equation*}
  Here, we equipped $\Z_p$ with the $p$-adic topology.
  The subgroup $1 + S^\cici$ naturally carries the structure of a $\Z_p$-module.
  Indeed, for $k \in \N$, we have
  \begin{equation*}
    (1+s)^{p^k} = 1 + p^ks + \binom{p^k}2 s^2 + \dots + s^{p^k},
  \end{equation*}
  and given $s \in S^\cici$, this is of the form $1+ o(1)$ as $k$ gets large. 
  In particular, if $x = a_0 + a_1 p + a_2p^2 + \dots \in \Z_p$, expressions of the form
  \begin{equation*}
    (1+s)^x = \prod_{i = 1}^\infty (1+s)^{a_k p^k}
  \end{equation*}
  make sense by lemma \ref{lem:infiniteseriesandproducts}. This gives
  $\GG_{m,\Z_p}$ the structure of a formal $\Z_p$-module. 
  In the upcoming subsection, we discuss how this is the simplest example of a
  whole family of formal modules constructed by 
  Lubin and Tate. In section \ref{sub:Application: The Local Class Field
  Theory} we explain applications of these formal modules to local class field
  theory.
\end{xpl}

\subsubsection{Lubin-Tate Formal Module Laws} % (fold)
\label{ssub:Lubin-Tate Formal Module Laws}
Suppose that $K$ is a local field with ring of integers $\cO_K$, with uniformizer
$\pi$ and residue field $\FF_q$. 

Let $H_0$ be the formal $\cO_K$-module law defined over $\FF_q$ by setting
\begin{equation*}
  H_0(X,Y) = X + Y, \quad [\pi]_{H_0}(X) = X^q, \quad [u]_{H_0}= \bar u X.
\end{equation*}
Here, $u$ is any unit of $\cO_K$ and $\bar u \in \FF_q$ is such that 
$u \equiv \bar u$ mod $\pi$. This uniquely determines $[a]_{H_0}$ 
for $a \in \cO_K$ as $a$ may be written as $a = \pi^{\nu} u$ for a unit $u$ and
$[\pi]_{H_0}$ and $[u]_{H_0}$ commute. 

Lubin-Tate formal module laws are module laws over $H$ such that 
$H \otimes_{\cO_K} \FF_q = H_0$. 

The construction of the Lubin-Tate formal module laws now rests on the following 
lemma, which is Lemma 1 in \cite{LubinTateFormalMult}.
\begin{lem}\label{lem:LTLemma1}
  Let $f(T)$ and $g(T)$ be elements of $\cF_\pi$ and let 
  $L(X_1, \dots, X_n) = \sum_{i=1}^n a_i X_i$ be a linear form with coefficients in 
  $\cO_K$. Then there exists a unique series $F(X_1, \dots, X_n)$ with coefficients 
  in $\cO_K$ such that 
  \begin{gather*}
    F(X_1, \dots, X_n) \equiv L(X_1, \dots, X_n) \pmod {T^2}, \\ \text{and} \\
    f(F(X_1, \dots, X_n)) = F(g(X_1), \dots, g(X_n)).
  \end{gather*}
\end{lem}
Although stated in a quite narrow context, the proof only uses that $\cO_K$ is 
$\pi$-complete and flat.
Write $\cF_\pi$ for the set of power series that may arise as $[\pi]_H$, that is,
\begin{equation*}
  \cF_\pi \coloneqq \{f \in \cO_K \llbr T \rrbr \mid f \equiv \pi T \pmod {T^2}
    \text{ and } f \equiv X^q \pmod \pi\}. 
\end{equation*}
Then Lemma \ref{lem:LTLemma1} implies that there is a unique power series 
$F_f(X,Y) \in \cO_K\llbr T \rrbr$ 

% subsubsection Lubin-Tate Formal Module Laws (end)

\subsubsection{Logarithms} % (fold)
\label{ssub:Logarithms}
We return to the more general framework where $A$ is an integral domain with 
field of fractions $K$. We review results from chapter
2 and 3 of \cite{hopkins1994equivariant}. 
Suppose that $F$ is a formal $A$-module law over an $A$-algebra $R$. 

\begin{defi}[Invariant Differentials for module laws.]
  The module $\omega(F)$ of invariant differentials is the submodule of the module of 
  differentials
  \begin{equation*}
    \Omega_{R\llbr T \rrbr/R} \cong R\llbr T \rrbr \dc T,
  \end{equation*}
  cut out by the condition that all $\omega \in \omega(F)$ satisfy 
  \begin{equation}\label{eq:diffcond}
    \omega(F(X,Y)) = \omega(X) + \omega(Y)\quad \text{and} \quad \omega([a]_F(X)) = a\omega(X).
  \end{equation} 
  for all $a \in A$. 
\end{defi}

Given a formal group law $F$, it is possible to explcitely calculate the
$R$-module $\omega(F)$, which we now explain. Let $f(X,Y)$ denote
$(\partial_x F)(X,Y)$, the derivative of $F(X,Y)$ with respect to $X$. Denote
$g(Y) = f(0,Y)$. Then $g$ is a unit in $R\llbr Y \rrbr$; and we construct
$\omega_F(X) \coloneqq \frac 1{g(X)} \dc X$. Checking that $\omega_F$ is indeed
invariant is a matter of applying the chain rule.

All other invariant differentials are scalar multiples of $\omega_F$. 
\begin{prop}
  \begin{enumerate}
    \item The $R$-module $\omega(R)$ is free of rank $1$ generated by $\omega_F$ 
    \item There is a non-degenerate pairing $\omega(F) \times \Lie(F) \to R$.
  \end{enumerate}
\begin{proof}
  Part one is \cite[Proposition 2.2]{hopkins1994equivariant}. 
\end{proof}
\end{prop}

\begin{xpl}
  The invariant differentials for $\GG_m$ are spanned by the form 
  $\omega_1(X) = \frac 1{1+X} \dc X$. 
\end{xpl}

The conditions imposed on invariant differentials remind of those imposed on 
morphisms of $A$-module laws $F \to \GG_a$. And indeed, there is a map
\begin{equation} \label{eq:functorinvdifftohom}
  \dc_F : \Hom_{\FMLOver AR} (F, \GG_{a,R}) \to \omega(F), \quad f \mapsto \dc f(X)
\end{equation}
One may check that $\End(\GG_{a,R}) \supseteq R$, turning $\dc$ in a map of $R$-modules.
\begin{prop}
  \begin{enumerate}
    \item If $R$ is a flat $A$-algebra, the map $\dc_F$ is injective.
    \item If $R$ is a $K$-algebra, the map $\dc_F$ is an isomorphism.
  \end{enumerate}
\begin{proof}
  \cite[Chapter 3]{hopkins1994equivariant} \todo{PROOF}\red{ Everything is easy if 
  $K$ has characteristic $0$, as we can integrate the differential forms.
  The proof in positive characteristic is a bit tricky; First it is shown that 
  there is an isomorphism of formal goups $F \cong \GG_a$, which is immediate.
  Then that there is a unique homomorphism $f: \GG_a \to \GG_a$ that maps to $\omega_F$
  and behaves well with respect to the $A$-module structure on $F$. }
\end{proof}
\end{prop}

In particular, if $R$ is a $K$-algebra, the invariant differential 
$\omega_F(X)$ constructed above comes from a homomorphism $f(X) = X + c_2 X^2 + \dots$,
which is an isomorphism by lemma \ref{lem:FGLeasyfacts}. 
This allows us to define the logarithm attached to $F$.
\begin{defi}[Logarithm and Exponential]
  If $R$ is a flat $A$-algebra, there is a unique power series
  \begin{equation*}
    \log_F(X) = X + c_2 X^2 + \dots \in (R \otimes_A K)\llbr X \rrbr 
  \end{equation*}
  inducing an isomorphism $F \otimes (R \otimes K) \to \GG_{a,R\otimes K}$.
  This power series is called the logarithm attached to $F$. 
  The inverse of $\log_H$ is called the exponential of $F$ and will be denoted by
  $\exp_H$.
\end{defi}
% subsubsection Logarithms (end)

\subsubsection{Deformations of Formal Modules and the Standard Formal Module} % (fold)
\label{ssub:Deformations of Formal Modules and the Standard Formal Module}
If, with notation as above, $R$ is a flat $A$-module, the structure of a formal
$A$-module $F$ over $R$ is uniquely determined by its logarithm $\log_H
\in R \otimes_A K \llbr T \rrbr$. Indeed, we find
\begin{equation*}
  F(X,Y) = \exp_H(X+Y), \quad [a]_F(X) = \exp_H(a X).
\end{equation*}
It is therefore natural to wonder about conditions on power series $f \in
R\llbr T \rrbr$ that ensure that $f$ is the logarithm of some formal group law. 
Hazewinkel found such a condition in his functional equation lemma.

\begin{prop}[Hazewinkel's Functional Equation Lemma] 
  Let $p$ be a prime and $q = p^e$. Given an inclusion of rings $B \subseteq
  L$, an ideal $\fa \subseteq B$ containing $p$, an endomorphism of rings
  $\sigma: L \to L$ and elements $s_1, s_2, \dots \in L$ subject to the conditions
  that 
  \begin{equation*}
    \sigma(b) \equiv b^q \pmod \fa \text{ for all } b \in B \quad \text{and} \quad 
    \sigma^r(s_i) \fa \subset B \text{ for all } r,s \geq 1.
  \end{equation*}
  Suppose now that $f \in L\llbr T \rrbr$ satisfies the functional equation condition
  \begin{equation*}
    f(X) - \sum_{i=1}^\infty s_i (\sigma_* f)(X^{q^i}) \in B\llbr X \rrbr.
  \end{equation*}
  Then we have 
  \begin{equation*}
    F(X,Y) = f^{-1}(f(X) + f(Y)) \in B \llbr X,Y \rrbr.
  \end{equation*}
\begin{proof}
  \cite[Sections 2 and 10]{hazewinkel1978formal}
\end{proof}
\end{prop}
In the case where $B = R$, $L = R \otimes_A K$, $\sigma = \id_L$ this 


\begin{itemize}
  \item Introduction to Hazewinkel's theory of $A$-typical formal modules
  \item The standard $\cO_K$-module of height $n$: The unique formal $\cO_K$-module law $H$ over
    $\cO_K$ with logaritm 
    \begin{equation*}
      \log_H(X) = \sum_{i = 0}^\infty \frac{T^{q^{in}}}{\pi^i}.
    \end{equation*}
\end{itemize}
% subsubsection Deformations of Formal Modules and the Standard Formal Module (end)

\subsubsection{Formal DVR-Modules over Fields} % (fold)
\label{ssub:Formal DVR-Modules over Fields}
As above, let $A$ be a discrete valuation ring with uniformizer $\pi$ and finite 
residue field $k$; write $q$ for the cardinality of $k$. Let $K$ denote the
field of fractions of $A$.

We introduce the concept of height, which is an integer attached to
morphisms of formal group laws over fields. The height of a formal $A$-module
$\cF$ over $R$ will be defined as the height of it's endomorphism $[\pi]_\cF$. 

We have seen in the previous section that if $R$ is a field extension of $K$,
then any morphism of formal group laws $f: F \to G$ over $R$ is either $0$, in
which case we say it has height $\infty$, or an isomorphism, in which case we
say it has height $0$. The height becomes interesting in positive
characteristic. 

We define the height over field extensions of the residue field. 
\begin{defi}[Height of morphisms of group laws]
  Assume that $R$ is a field extension of $k$ and $f: F \to G$ is a morphism of 
  formal groups laws over $R$, given by a formal series $f(T) \in R\llbr T \rrbr$. 
  If $f = 0$, we say that $f$ has infinite height. 
  If $f \neq 0$, the height of $f$ is defined as the largest integer $h$ such that 
  $f = g(T^{q^h})$ for some power series $g(T) = c_1 T + c_2 T^2 + \dots \in
  R\llbr T \rrbr$ with $c_1 \neq 0$. 
\end{defi}
One readily checks that if $f: \cF \to \cG$ is a morphism of formal groups over
a field extension $R$ of $k$, the height of $f$ does not depend on the choices
of group laws on $\cF$ and $\cG$. This allows us to define the height function 
attached to $f$. 
\begin{defi}[Height function]
  Let $f: \cF \to \cG$ be a morphism of formal groups over a scheme $X$.
  For a (scheme-theoretic) point $x \in \abs X$, let $f_x$ denote the 
  base-change of $f$ to the residue field of $x$. 
  The height function attached to $f$ is the upper-semicontinuous function
  \begin{equation} \label{eq:defheight}
    \height(f) : \abs X \to \Z_{\geq 0} \cup \{\infty\}, \quad x \mapsto 
    \height(f_x).
  \end{equation}
\end{defi}
It is not hard to see that the height function is additive, that is, we have
\begin{equation*}
  \height(f \circ g) = \height(f) + \height(g).
\end{equation*}

Let $R$ be a local $A$-algebra and let $k'$ be a separable closure of $R/\fm_R$.
Let $F$ be a formal $A$-module law over $R$ of height $h$. Then we have the
following result on the endomorphisms of $F \otimes_R k'$.
\begin{lem}
  The $A$-module $\End_{\FMOver A{k'}}(F \otimes_R k')$ is isomorphic to the ring
  of integers of a central division algebra $D$ over $K$ of invariant $\frac 1h$.
\begin{proof}
  \cite[Proposition 1.7]{drinfel1974elliptic} \red{The proof uses techniques from
    deformations of formal modules. Hence perhaps it would make sense to have this
  in the next chapter.}
\end{proof}
\end{lem}

\begin{defi}[Isogeny]
  A morphism $f: \cF \to \cG$ of formal groups over a field $k$ is called an isogeny if
  $\ker(f)$ is a represented by a finite free $k$-scheme. More generally, a
  morphism of formal $A$-modules over a base scheme $X$ is an isogeny if and
  only if $\ker(f)$ is finite and locally free over $X$. 
\end{defi}

Isogenies can be described using the height function.

\begin{lem}
  A morphism $f: \cF \to \cG$ is a isogeny if and only if the height 
  function $\height(f)$ is locally constant with values in $\Z_{\geq 0}$. 
\end{lem}

\begin{defi}[$\pi$-divisible $A$-module]
  We say that a formal $A$-module $H$ over $X$ is $\pi$-divisible if 
  $[\pi]_H$ is an isogeny. If $X$ is connected, the height of $H$ is the
  (constant) height of the endomorphism $[\pi]_H: A \to A$. 
\end{defi}

\begin{lem}
  Over sepearbly closed fields, the formal group laws are classified by their 
  heights.
\begin{proof}
  \cite[Theorem 19.4.1]{hazewinkel1978formal}, this is originally due to Drinfeld.
  Note that this is only interesting in positive characteristic.
\end{proof}
\end{lem}

The following lemma allows us to invert quasi-isogenies.

\begin{lem}
  Let $f: \cF \to \cG$ be an isogeny of $\pi$-divisible formal $A$-modules over a
  quasi-compact \red{quasi-separated?} $A$-scheme $X$. Then there is an 
  integer $n \geq 0$ and an isogeny $g: \cG \to \cF$ with 
  \begin{equation*}
    f \circ g = [\pi^n]_\cG \quad \text{and} \quad g \circ f = [\pi^n]_\cF.
  \end{equation*}
\end{lem}
% subsubsection Formal DVR-Modules over Fields of Characteristic 0 (end)



\subsubsection{The Dieudonn\'e functor} % (fold)
\label{ssub:The Dieudonne functor}
\begin{itemize}
  \item Definition using quasi-logarithms
  \item Definition with rigidified extensions as in \cite{hopkins1994equivariant} (?)
\end{itemize}
% subsubsection The Dieudonn`e functor (end)

\subsubsection{Tate Modules and the Universal Cover} % (fold)
\label{sub:Tate Modules and the Universal Cover}
Let $A$ be a ring and $R$ be an $A$-algebra. Given $H \in \FMOver AR$ and $a \in A$,
we define the functor 
\begin{equation*}
  \Tilde H_a : \Adm R \to \Mod {A[\tfrac 1a]}, \quad
  S \mapsto \left\{(x_1, x_2, \dots) \in \prod_{\N} H(S) \mid [a]_H(x_{i+1}) =
  x_i \right\}.
\end{equation*}

From now on assume that $A$ is a discrete valuation ring with uniformizer
$\pi$, finite residue field $k$ and field of fractions $K$. 

\begin{defi}[The Universal Cover and Tate Module]
  We omit $\pi$ from notation and write $\tilde H = \tilde H_\pi$. This functor
  takes values in the category of $K$-vector spaces.
  Up to isomorphism, $\Tilde H$ is uniquely determined. We call this functor
  the universal cover of $H$. 

  The Tate-Module $T_\pi H$ is the subfunctor of $\Tilde H$ cut out out
  by the condition that $[\pi]_H(x_1) = 0$. Note that $T_\pi H$ does no longer 
  carry the structure of a $K$-vector space, it is an $A$-module. The Rational
  Tate Module $V_\pi H$ is the subfunctor of $\Tilde H$ cut out by the
  condition that $x_1$ has $[\pi]_H$-torsion. Equivalently, we have 
  \begin{equation*}
    V_\pi H (S) = T_\pi H(S) \otimes_A K.
  \end{equation*}
\end{defi}

Let $R$ be an adic $A$-algebra with finitely generated ideal of definition $I$ such
$k_R = R/I$ is a separably closed field of characteristic $p$ (in particular, $p \in I$). 
We write $\Nilp^\flat_R$ for the functor sending on admissible $R$-algebras sending
$S$ to the set $\lim_{s \mapsto s^q} S^\cici$. This functor is representable by
$\Spf R\llbr T^{\frac 1{q^\infty}} \rrbr$ (with the $T$-adic topology).

The following calculation is helpful.
\begin{lem}\label{lem:cryscalc}
  Let $S$ be a ring, let $J$ be an ideal of $S$ containing $p$, and let $q$ be
  a power of $p$. 
  Then for any two elements $s_1, s_2 \in S$ with $s_1 \equiv s_2$ mod $J$, we have 
  $s_1^{q^k} \equiv s_2^{q^k}$ mod $J^{k+1}$. 
\begin{proof}
  It suffices to check that if $s_1 \equiv s_2$ mod $J^k$, then $s_1^q \equiv s_2^q$
  mod $J^{k+1}$. Write $s_1 + r = s_2$ for some $r \in J^k$. Then 
  \begin{equation*}
    s_2^q = (s_1 + r)^q = s_1^q + \underbrace{qs_1^{q-1} r + \binom q2
    s_1^{q-2}r^2 + \dots + r^q }_{\in J^{k+1}} \in s_1^q + J^{k+1}.
  \end{equation*}
  Here we used that $p \mid \binom qm$ for all integral $1 < m < q$.
\end{proof}
\end{lem}
\begin{cor}\label{cor:crystildecalc}
  Let $S$ and $J$ be as above and let $f \in S \llbr T \rrbr$ be a power series such
  that 
  \begin{equation*}
    f(T) \equiv g(T^{q}) \pmod J
  \end{equation*}
  for some $g \in S\llbr T \rrbr$ with $g(0) = 0$. Then, if $s_1, s_2 \in S$
  are elements for which $f(s_1)$ and $f(s_2)$ exist (for example if $f$ is a
  polynomial or if $S$ is an admissible ring with $s_1, s_2 \in S^\cici$) and
  $s_1 \equiv s_2$ mod $I$, we have
  \begin{equation*}
    f^k(s_1) \equiv f^k(s_2) \pmod {J^{k+1}}.
  \end{equation*}
  Here $f^k$ is $k$-fold concatenation of $f$. 
\end{cor}

\begin{lem}[Crystalline nature of $\Nilp^\flat$.] \label{lem:crysnilp}
  Suppose that $S$ is an admissible $R$-algebra admitting an ideal of
  definition $J$ containing $p$. Then there is a bijection
  \begin{equation*}
    \Nilp^\flat(S) \cong \Nilp^\flat(S/J).
  \end{equation*}
  This bijection equips $S$ with the structure of an (additive) abelian group. 
  \begin{proof}[Proof]
    Let $(x_1, x_2, \dots)$ be an element of the right hand side. Let
    $(y_1, y_2, \dots)$ is a sequence of arbitrary lifts. From this choice of lifts
    we obtain a sequence
    $(z_1, z_2, \dots) \in \Nilp^\flat(S)$ by setting
    \begin{equation*}
      z_i = \lim_{r \to \infty} y_{i+r}^{q^r}.
    \end{equation*}
    The limit exists, because if $s \geq r$ are two non-negative integers, we calculate
    \begin{equation*}
      y_{i + s}^{q^{s-r}} \equiv x_{i+s}^{q^{s-r}} \equiv x_{i+r} \equiv y_{i+r} \pmod I,
    \end{equation*}
    implying by \ref{lem:cryscalc} that 
    \begin{equation*}
      y_{i+s}^{q^s} \equiv y_{i+r}^{q^r} \pmod {I^r}.
    \end{equation*}
    Now the limit exists by completeness of $S$.
    The sequence $(z_1, z_2, \dots)$  is readily seen to be a lift of 
    $(x_1, x_2, \dots)$. We show that this lift is unique. Suppose that 
    $(z'_1, z'_2, \dots) \in \Nil^\flat(S)$ is another lift. Now 
    \begin{equation*}
      z'_i \in \bigcap_{r \in \N}(z_{i+r} + J)^{q^r} = \bigcap_{r \in \N}(z_{i} +
      q^r z_{i+1} J + \dots + J^{q^r})=  \{z_i\}.
    \end{equation*}
    This is what we wanted to show.
  \end{proof}
\end{lem}

\begin{lem}
  Let $H$ be a $\pi$-divisible formal $A$-module over $R$ and write $H_0 = H
  \otimes_R k_R$. Now the choice of a coordinate on $H_0$ gives rise to an
  isomorphism
  \begin{equation*}
    \tilde H_0 \cong \Nilp^\flat_{k_R}.
  \end{equation*}
\begin{proof}
  We can write $[\varpi]_{H_0}(X) = g(X^{q^n}) \in k_R\llbr X \rrbr$ for some 
  $g(X) = c_1X + c_2X^2 + \dots$ with $c_1 \neq 0$. For each coefficient $c_i$, let
  $d_i \in k_R$ be the unique element such that $d_i^{q^n} = c_i$. Let
  $h \in k_R\llbr X \rrbr$ be the power series given by $d_1 X + d_2
  X^2 + \dots$. Now $(h(X))^{q^n}=[\varpi]_G(X)$, and we find that 
  \begin{equation*}
      \tilde G(S) \to \lim_{x \mapsto x^{q^n}} S^\cici: \quad
      (x_1, x_2, x_3, \dots) \mapsto (x_1, h(x_2), h(h(x_3)), \dots)
  \end{equation*}
  is a well-defined function, and (trivially) functorial in $S$. For the
  inverse, let $h^{-1}(X) \in k_R\llbr X \rrbr$ be the the unique power
  series with $h^{-1}(h(X))= h(h^{-1}(X)) = X$, see Lemma
  \ref{lem:FGLeasyfacts}. The map
  \begin{equation*}
      \lim_{x \mapsto x^{q^n}} S^\cici \to \tilde G(S), \quad 
      (x_1, x_2, \dots ) \mapsto (x_1, h^{-1}(x_2), h^{-1}(h^{-1}(x_3)), \dots)
  \end{equation*}
  is well-defined as
  \begin{equation*}
      [\varpi]_G(h^{-1}(X)) = g((h^{-1}(X))^{q^n}) = (h(h^{-1}(X)))^{q^n} =
      X^{q^n}
  \end{equation*}
  and it is readily seen to be inverse to the map constructed above.
  Note that the functors $\Nilp^\flat$ and $\lim_{x \mapsto x^{q^n}} S^\cici$
  are isomorphic.
\end{proof}
\end{lem}

We keep the assumptions on $H$.
\begin{lem}
  Suppose that $S$ is an admissible $R$-algebra admitting an ideal of definition
  $J$ such that $\pi \in J$. Then the natural reduction map
  \begin{equation*}
    \tilde H(S) \to \tilde H(S/J)
  \end{equation*}
  is an isomorphism.
  \begin{proof}[Proof]
    Given a coordinate $H \cong \spf R\llbr T \rrbr$, a sequence $(x_0, x_1,
    \dots) \in H(S/J)$, and an arbitrary lift to a sequence $(y_0, y_1, \dots)
    \in \prod_{\N} H(S)$, we may, akin to
    the calculations in Lemma \ref{lem:crysnilp} and with allusion to Corollary
    \ref{cor:crystildecalc}, define a sequence $(z_0, z_1, \dots)$ in
    $H(S)$ by setting 
    \begin{equation*}
      z_i = \lim_{r \to \infty} [\pi^r]_{H}(y_{i+r}).
    \end{equation*}
    This shows surjectivity of the reduction map. For injectivity,
    as $H(S) \to H(S/J)$ is a morphism of groups, it suffices to show that the kernel
    is trivial. But if $(z_0, z_1, \dots)$ lies in the kernel, every element 
    $z_i$ lies in $J$ and we find that $z_i \in \bigcap_{r \in \N} J^r = \{0\}$.
    This is what we had to show.
  \end{proof}
\end{lem}

We collect the results. 
\begin{prop}
  Let $A$ be a local ring with finite residue field, and let $R$ be a local, admissible 
  $A$-algebra with perfect residue field. Let $I \subset R$ be an ideal of
  definition containing $\pi$ and $p$. Let $H$ be a $\pi$-divisible formal
  $A$-module over $R$ and fix a coordinate $H_0 = H\otimes_R(R/I) \cong
  (R/I) \llbr T \rrbr$. 
  Let $S$ be an admissible $R$-algebra with ideal of definition $J$ such that 
  $\phi(I) \subseteq J$. Then there are isomorphisms (functorial in $S$)
      \begin{equation*}
        \tilde H(S) \cong \tilde H(S/J) = \tilde H_0(S/J) \cong \Nilp^\flat(S/J) \cong
        \Nilp^\flat(S).
      \end{equation*}
  In particular, $\tilde H(S)$ is, as a functor to $\Set$, representable by
  $\spf(R \llbr T^{q^{-\infty}} \rrbr)$.
\end{prop}

\begin{itemize}
  \item Results from \cite{BoyarchenkoWeinstein2011MaxVar}, section 2.5
\end{itemize}


% subsection Tate Modules and the Universal Cover (end)

\subsubsection{The Quasilogarithm Map} % (fold)
\label{ssub:The Quasilogarithm map}
\begin{itemize}
  \item See \cite{BoyarchenkoWeinstein2011MaxVar}, section 2.6.
\end{itemize}
% subsubsection The Quasilogarithm map (end)

\subsubsection{Determinants of Formal Modules} % (fold)
\label{ssub:Determinants of Formal Modules}
\begin{itemize}
  \item "Functorial" description of the determinant. Either as in
    \cite{BoyarchenkoWeinstein2011MaxVar}, or as in \cite{weinstein2016semistable}.
  \item Construction.
  \item Approximations.
\end{itemize}

% subsubsection Determinants of Formal Modules (end)
