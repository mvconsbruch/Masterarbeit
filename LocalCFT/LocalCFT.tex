%! TeX root: ../main.tex
\documentclass[../main.tex]{subfiles}

\begin{document}
\section{Explicit Aspects of Abelian Lubin--Tate Theory} % (fold)
\label{sec:Local Class Field Theory}
We review the approach to local class field theory developed by Lubin and Tate in
\cite{LubinTateFormalMult}. 

Class field theory is, in general terms, concerned with the study of 
abelian extensions of local or global fields. This area of number theory 
has a rich history, first results towards this direction date back as far as
1801, when Gauß proved the famous quadratic reciprocity law: if $\ell \neq p$
are odd prime numbers, we have the equality
\begin{equation*}
  \legendre p \ell = (-1)^{\frac{p-1}2 \frac{l-1}2} \legendre \ell p.
\end{equation*}
This unintuitive statement about the interplay of the multiplicative structures of
$\FF_p$ and $\FF_\ell$ finds a deeper and more conceptual meaning when inspected through
the lens of cyclotomic fields. This connection
was first made by Gauß himself, when he provided a proof of quadratic
reciprocity using so-called Gauß sums. Admitting theory of cyclotomic fields 
(as developed in Chapter 1 of \cite{neukirch2006algebraische}), and denoting by
$\Q(\zeta_\ell)$ the $\ell$-th cyclotomic extension,
quadratic reciprocity is essentially equivalent to the following statement.
\begin{equation*}
  \legendre \ell p = (-1)^{\frac{l-1}2 \frac{p-1}2} \iff \begin{array}c
    \text{The prime $p \in \Z$
    decomposes into an even} \\ \text{number of prime ideals in
  $\Q(\zeta_\ell)$.} \end{array}
\end{equation*}
This foreshadows class field theory. Indeed, one major achievement
of class field theory is the Artin reciprocity law, which in particular yields a 
description of the decomposition behavior of unramified primes inside abelian extensions
of number fields, cf. \cite[Theorem 7.3]{neukirch2006algebraische}. 
The appearence of cyclotomic fields above is also no coincidence.

\begin{thm}[Kronecker--Weber\footnote{despite the name, the Kronecker--Weber Theorem
wasn't (completely) proven by either Kronecker or Weber; the first correct
prove was found by Hilbert. For more on the history of class field theory, see 
\cite{Conrad2009HISTORYOC}.}] \label{thm:KroneckerWeber}
  Every finite extension of $\Q$ embeds into $\Q(\zeta_m)$ for some $m \in \Z$.
  That is, after choosing an algebraic closure $\Q \inj \bar \Q$, the maximal abelian
  subextension is given by 
  \begin{equation*}
    \Q^\ab = \bigcup_{m \in \N} \Q(\zeta_m) \subset \bar \Q.
  \end{equation*}
\end{thm}

The problem of finding similar descriptions for the maximal abelian extension of
general number fields seems to be very difficult. There is a similar description 
of Abelian extensions of imaginary quadratic fields (cf. \cite[Chapter II,
§5]{silverman1994advanced}; these extensions arise by adjoining certain values attained
by the so-called Weber function at the torion points of an elliptic curve with complex
multiplication by the field in question), but in general this problem seems to be open.

The situation simplifies a lot for non-Archimedian local fields. 
The aim of this section is to discuss the analogues of the 
Kronecker--Weber theorem and the Artin map for these fields. We shortly set the
stage. Let $E$ be a non-Archimedian local field with residue field $\FF_q$ of
characteristic $p$, and choose an embedding $E \inj \bar E$ into an algebraic
closure $\bar E$ of $E$. Let $E^\nr$ be maximal unramified subextension, with
residue field identified with $\Fqbar$.
In this situation, we have the short exact sequence
\begin{equation*}
  0 \to I_E = \Gal(\bar E/ E^\nr) \to \Gal(\bar E/E) \to \Gal(E^\nr /E) =
  \Gal(\bar \FF_q /\FF_q) \to 0.
\end{equation*}
Recall that $\Gal(\Fqbar/\FF_q)$ is topologically generated by the $q$-th power
Frobenius automorphism $\Frob_q\colon x \mapsto x^q$. Let $\Phi \in \Gal(E^\nr /E)$ 
denote the corresponding Frobenius automorphism of $E^\nr$. 
The Weil group is defined as
\begin{equation*}
  \Weil_E \coloneq \{\sigma \in \Gal(\bar E/E) : \sigma|_{E^\nr} = \Phi^m
  \text{ for some } m \in \Z\},
\end{equation*}
endowed with the unique topology such that the inertia subgroup $I_E \subset
\Weil_E$ (with its usual topology) is an open subgroup.
As above, we have the short exact sequence of topological groups
\begin{equation*}
  0 \to I_E \to \Weil_E \to \Z \to 0.
\end{equation*}
The natural map $\Weil_E^\ab \to \Gal(\bar E/E)^\ab = \Gal(E^\ab/E)$ induced by taking
(topological) abelianization can be shown to be injective and yields a bijection
\begin{equation}\label{eq:WeilgroupInjectsIntoGalois}
  \Weil_E^\ab \xto \sim \{\sigma \in \Gal(E^\ab/E) : \sigma|_{E^\nr} \in \Phi^\Z\}.
\end{equation}
We use this bijection to identify elements of $\Weil_E^\ab$ with the corresponding field
automorphism of $E^\ab$. 

In this section, we work towards the following Theorem.

\begin{thm}[Local Class Field Theory]\label{thm:LocalKroneckerWeber} \leavevmode
  \vspace{-0.5em}
  \begin{enumerate}
    \item \textnormal{The Local Kronecker--Weber Theorem.} For any choice of uniformizer $\varpi \in E^\times$, there exists a 
      totally ramified field extensions $E_\varpi$ of $E$ such that 
      the maximal abelian subextension $E^\ab \subset \bar E$ of $E$
      decomposes as $E^\ab = E_\varpi E^\nr$.     
    \item \textnormal{Local Artin Reciprocity.} There is a unique isomorphism
      (the Artin map) of topological groups
      \begin{equation*}
        \Art_E\colon E^\times \xto \sim \Weil_E^\ab
      \end{equation*}
      such that 
      \begin{enumerate}
        \item for any choice of uniformizer $\varpi \in E^\times$, we have
          $\Art_E(\varpi)|_{E^\nr} = \Phi$.
        \item for any finite Abelian extension $E'$ of $E$, the
          map $E^\times \xto{\Art_E} \Weil_E^\ab \xto{\sigma \mapsto
          \sigma|_{E'}} \Gal(E'/E)$ is surjective and has kernel given by
          $\Norm_{E'/E}(E'^\times) \subset E^\times$. 
      \end{enumerate}
  \end{enumerate}
\end{thm}

We follow the construction of $E_\varpi$ in \cite{LubinTateFormalMult}.
The construction resembles the case of imaginary quadratic fields; $E_\varpi$ is
obtained from $E$ by adjoining $\varpi$-torsion points of so-called Lubin--Tate
formal module laws. 
We sketch a proof of the equality $E^\ab = E_\varpi E^\nr$. Lubin and Tate
showed the latter
making use of previously known results in local class field theory, which ultimately
rested on results from global class field theory. 
In order to sell the theorem as a
local version of the Kronecker--Weber theorem,
we avoid arguments resting on results of local class field theory
by referring to a proof found by Gold (cf.
\cite{gold1981local}), the essential insight is that the theorem of Hasse--Arf
suffices to deduce this equality. 
Once all of this is known, the isomorphism $\Art_E$ and property
(a) arise naturally. We do not prove property (b), but we shortly note how to see
that an isomorphism satisfying properties (a) and (b) has to be unique.

\begin{proof}[Proof of uniqueness of the Artin Map]
  Let $\alpha\colon E^\times \to \Weil_E^\ab$ be
  an isomorphism satisfying properties (a) and (b). Let $\varpi \in E^\times$ be an arbitrary
  uniformizer. We show that properties (a) and (b) determine $\alpha(\varpi)$ uniquely.
  Let $E \subset E' \subset E_\varpi$ be a finite subextension, 
  with $E_\varpi$ the totally ramified Abelian extension from the local Kronecker--Weber
  theorem. Then $E'/E$ is totally ramified, hence $\varpi \in
  \Norm_{E'/E}(E'^\times)$. Now $\alpha(\varpi)|_{E'}$ is equal to the identity on $E'$
  by property (b). Varying $E'$, this readily implies $\alpha(\varpi)|_{E_\varpi} =
  \id_{E_\varpi}$. Using property (a), we find that $\alpha(\varpi)$ is the unique
  automorphism of $E^\ab$ with $\alpha(\varpi)|_{E_\varpi} = \id_{E_\varpi}$ and
  $\alpha(\varpi)|_{E^\nr} = \Phi$. As $E^\times$ is generated by the set of its
  uniformizers, it follows that $\alpha$ is uniquely determined by the properties (a) and (b).
\end{proof}

\subsection{Lubin--Tate Formal Module Laws} % (fold)
\label{sub:Lubin--Tate Formal Module Laws}
Fix a uniformizer $\varpi \in E^\times$. In this section we explain the construction
of Lubin--Tate module laws. As announced, these module laws will allow us to
construct the totally ramified extension $E_\varpi$.

We fix a positive integer $h$ and write $\cF_{\varpi,h}$ for the set of power series
\begin{equation*}
  \cF_{\varpi,h} \coloneqq \{f \in \cO_E \llbr T \rrbr \mid f \equiv \varpi T
    \ (\text{mod } {T^2}) \ \text{ and } \ 
    f \equiv T^{q^h} \ (\text{mod }\varpi)\}. 
\end{equation*}

The construction of the Lubin--Tate formal module laws is based on the following 
lemma, which is Lemma 1 in \cite{LubinTateFormalMult}.
\begin{lem}\label{lem:LTLemma1}
  Let $f(T)$ and $g(T)$ be elements of $\cF_{\varpi,h}$ and let 
  $L(X_1, \dots, X_n) = \sum_{i=1}^n a_i X_i$ be a linear form with coefficients in 
  ${\cO_E}$. Then there exists a unique series $F(X_1, \dots, X_n)$ with coefficients 
  in ${\cO_E}$ such that 
  \begin{gather*}
    F(X_1, \dots, X_n) \equiv L(X_1, \dots, X_n) \pmod {T^2}, \\ \text{and} \\
    f(F(X_1, \dots, X_n)) = F(g(X_1), \dots, g(X_n)).
  \end{gather*}
\end{lem}

As a direct consequence, we obtain the following useful result.
\begin{lem}
  Let $f \in \cF_{\varpi, h}$. Then there is a unique formal ${\cO_E}$-module law $F_f$ over ${\cO_E}$
  with $[\varpi]_F(T) = f(T)$.
\begin{proof}
  In the above lemma, set $L(X,Y) = X+Y$ and $g=f$ to uniquely determine 
  the power series $F_f$. The same Lemma yields unique power series
  $[a]_{F_f}(T) \in R\llbr T \rrbr$ by setting $L(T) = a T$, $g=f$. It is
  routine to check that $(F_f, ([a]_f)_{a \in {\cO_E}})$ is a formal ${\cO_E}$-module law, 
  cf. \cite{LubinTateFormalMult}.
\end{proof}
\end{lem}

\begin{defi}[Lubin--Tate Module Law]
  We refer to module laws arising from the construction above as Lubin--Tate module laws.
\end{defi}

Furthermore, attached to each $a \in \cO_E$ and $f,g \in \cF_{\varpi,h}$, we find
unique $[a]_{f,g}(T) \in \cO_E\llbr T \rrbr$ satisfying
\begin{equation}\label{eq:LTMoLaScaCond}
  [a]_{f,g}(T) \equiv aT \pmod {T^2} \quad \text{and} \quad
  f([a]_{f,g}(T)) = [a]_{f,g}(g(T)).
\end{equation}
We now have the following theorem.
\begin{thm}[Lubin--Tate Formal $\cO_E$-Module Laws]\label{thm:LTModLaw}
  Let $E$ be a local field with ring of integers $\cO_E$. For any choice of 
  uniformizer $\varpi \in \cO_E$ and any $f \in \cF_{\varpi,h}$, the family of power
  series $(F_f, ([a]_{f,f})_{a \in \cO_E})$
  gives rise to a formal $\cO_E$-module law over $\cO_E$. For 
  $f,g \in \cF_{\varpi,h}$, the formal $\cO_E$-module laws $F_f$ and $F_g$ are
  canonically isomorphic, via the morphism induced by $[1]_{f,g} \in \cO_E\llbr
  T \rrbr$. 
\begin{proof}
  This is Theorem 1 of \cite{LubinTateFormalMult} and the subsequent discussion.
\end{proof}
\end{thm}
In particular, up to canonical isomorphism, there is only one Lubin--Tate formal
$\cO_E$-module law over $\cO_E$ attached to the choice of the uniformizer $\varpi \in
\cO_E$. 

\begin{xpl}
  If $E = \Q_p$, this reconstructs the multiplicative formal 
  $\Z_p$ module $\Ghat$ constructed above. Indeed, we have 
  \begin{equation*}
    \cF_{p,1} = \{f \in \Z_p\llbr T \rrbr \mid f(T) \equiv T^p \text{ mod } p
    \text{ and } f(T) \equiv pT \text{ mod } (T)^2 \},
  \end{equation*}
  so that $f(T) = (1+T)^p-1$ lies in $\cF_{p,1}$.  
  One quickly checks that 
  \begin{equation*}
    F_f(X,Y) = (1+X)(1+Y) - 1 = X + Y + XY \in \Z_p \llbr X,Y \rrbr
  \end{equation*}
  is the addition law associated to $f$, and that 
  for $a \in \Z_p$, the power series
  \begin{equation*}
    [a]_{f,f} = (1+T)^{a} - 1 \in \Z_p \llbr T \rrbr
  \end{equation*}
  satisfies the condition of \eqref{eq:LTMoLaScaCond}. 
\end{xpl}
% subsubsection Lubin--Tate Formal Module Laws (end)

\subsection{Construction of the Maximal Abelian Extension} % (fold)
\label{sub:Construction of the Maximal Abelian Extension}
We now explain the interplay of Lubin--Tate formal module laws with 
local class field theory. Let us fix a uniformizer $\varpi \in E$. The main
focus will lay on the construction of the totally ramified abelian extension
$E_\varpi$ from \cref{thm:LocalKroneckerWeber}. As announced, it 
arises as an infinite union of finite extensions $E_{\varpi, m}$, which
are constructed as follows. Let $f \in \cO_E\llbr T \rrbr$ be a power series
in $\cF_{\varpi,1}$, giving rise to a
Lubin--Tate formal module law $F$. We define
\begin{equation*}
  \Lambda_{f,m} \coloneqq \{x \in \fm_{\bar E} \mid [\varpi^m]_F(x) = 0\}.
\end{equation*}
Here, we equip $\bar E$ with the canonical topology coming from $E$ by extending the
$\varpi$-adic valuation to finite subextensions. With this topology, $\bar E$
is not complete, but any $x \in \fm_{\bar E}$ lies inside some finite (hence
complete) subextension, so the power series $[\varpi]_F(x)$ converges, and the 
definition of $\Lambda_{f,m}$ makes sense.

With the $\cO_E$-module structure from $F$, the set $\Lambda_{f,m}$ is an
$\cO_E$-module. The absolute Galois group of $E$ acts $\cO_E$-linearly on
$\Lambda_{f,m}$, as the power series constituting the $\cO_E$-structure all
have coefficients in $E$. This yields a homomorphism 
\begin{equation} \label{eq:GaloisGroupToLambdaHom}
  \Gal(\bar E /E ) \to \Aut_{\Mod{\cO_E}}(\Lambda_{f,m}).
\end{equation}

We now define 
\begin{equation*}
  E_{\varpi, m} \coloneq E(\Lambda_{f, m})
\end{equation*}
and call it the Lubin--Tate extension of degree $m$. 
Omitting the choice of $f$ from notation is justified. Indeed, given 
another choice $f' \in \cF_\varpi$, Theorem \ref{thm:LTModLaw} implies the 
existence of a power series $[1]_{f,f'} \in \cO_E\llbr T \rrbr$ inducing an
isomorphism $\Lambda_{f,m} \to \Lambda_{f',m}$. As any finite extension of $E$
is complete, we obtain $E(\Lambda_{f,m}) = E(\Lambda_{f',m})$, showing the following.
\begin{lem}\label{lem:FiniteLTExtensionIndepOfF}
  The Lubin--Tate extension of degree $m$ does only depend on the choice of $\varpi$,
  not on the choice of power series $f \in \cF_\varpi$. 
\end{lem}

Hence, we shall henceforth choose $f(T) = \varpi T + T^q$. 
The fact that the corresponding formal module law is $\varpi$-divisible quickly 
implies the following.
\begin{lem}\label{lem:StructureOfLambdamf}
  For any positive integer $m$, there is an isomorphism of $\cO_E$-modules
  \begin{equation*}
    \Lambda_{f,m} \cong \cO_E/\varpi^m\cO_E.
  \end{equation*}
\begin{proof}
  It is easily seen that $\Lambda_{f,m}$ is a finite set, hence of the form
  \begin{equation*}
    \Lambda_{f,m} = \prod_{i = 1}^k \left(\cO_E/\varpi^{m_i} \cO_E \right)^{e_i}
  \end{equation*}
  by the structure theorem for finitely generated modules over principal ideal domains.
  For $m=1$ the claim is now easy to verify, we have $\# \Lambda_{f,1} = q$ and 
  hence $\Lambda_{f,1} \cong \cO_E/\varpi\cO_E$.

  Now let $m$ be arbitrary. For any $\alpha \in \Lambda_{f,m}$, the roots of
  $f(T) - \alpha$ have positive valuation, and it follows that $f$ (i.e. multiplication
  by $\varpi$) yields a
  surjective morphism $\Lambda_{f,m} \to \Lambda_{f,m-1}$. Via induction this
  implies $\# \Lambda_{f,m} = q^m$.
  Arguing inductively, we may assume that 
  $\Lambda_{f,m-1} \cong \cO_E/\varpi^{m-1}\cO_E$,  implying
  \begin{equation*}
    \Lambda_{f,m} \cong \cO_E/\varpi^{m}\cO_E \quad \text{or} \quad 
    \Lambda_{f,m} \cong \cO_E/\varpi^{m-1}\cO_E \times \cO_E/\varpi \cO_E.
  \end{equation*}
  As the multiplication-by-$\varpi$-map $\Lambda_{f,m} \to \Lambda_{f,m-1}$ is
  surjective, this enforces $\Lambda_{f,m} \cong \cO_E/\varpi^m\cO_E$, as desired.
\end{proof}
\end{lem}


\begin{lem}\label{lem:PropertiesOfFiniteLTExt}
  For any positive integer $m$, the extension $E_{\varpi,m}/E$ is totally ramified
  of degree $[E_{\varpi,m}:E] = (q-1) q^{m-1}$. Furthermore, the morphism
  \eqref{eq:GaloisGroupToLambdaHom} induces an isomorphism
  \begin{equation*}
    \Gal(E_{\varpi, m}/E) \cong \Aut_{\Mod {\cO_E}}(\Lambda_{f,m}) = 
    \cO_E^\times/(1 + \varpi^m \cO_E).
  \end{equation*}
\begin{proof}
  Given a positive integer $i \in \N$, we write $f^i(T)$ for the $i$-fold
  self-composite of $f$ (with $f^0(T) = T$) and define
  \begin{equation*}
    \eta_i(T) \coloneq \frac{f^i(T)}{f^{i-1}(T)} = \varpi + f^{i-1}(T)^{q-1}.
  \end{equation*}
  By induction, this is seen to be an Eisenstein polynomial
  of degree $(q-1) q^{i-1}$. The roots of $\eta_m(T)$ are exactly the elements
  in the difference $\Lambda_{f,m} \setminus \Lambda_{f,m-1}$, and any choice of such
  a root yields an inclusion of fields
  \begin{equation}\label{eq:LCFTInclOfFields}
    \frac{E[T]}{(\eta_m(T))} \inj E_{\varpi,m}.
  \end{equation}
  As $\Lambda_{f,m}$ generates $E_{\varpi, m}$, the homomorphism
  $\Gal(E_{\varpi,m}/E) \to \Aut(\Lambda_{f,m})$ obtained from the 
  homomorphism in \eqref{eq:GaloisGroupToLambdaHom} is injective. 
  By \eqref{eq:LCFTInclOfFields}, the source of this morphism has cardinality
  $\geq (q-1)q^{m-1}$, while the target is a set with exactly $(q-1)q^{m-1}$
  elements. Hence the homomorphism is bijective. For degree-reasons it follows that
  \eqref{eq:LCFTInclOfFields}
  is an isomorphism; in particular $E_{\varpi,m}$ is totally ramified of
  the specified degree.
\end{proof}
\end{lem}

We define the Lubin--Tate extension of $E$ as the union
\begin{equation*}
  E_\varpi \coloneq \bigcup_{m \in \N} E_{\varpi, m} = E(\Lambda_f) \quad
  \text{with} \quad \Lambda_f \coloneqq \bigcup_{m \in \N} \Lambda_{f,m}.
\end{equation*}
This is the totally ramified field from Theorem \ref{thm:LocalKroneckerWeber}.
\begin{thm}[Local Kronecker--Weber]\label{thm:LocalKW}
  Inside $\bar E$, we have $E^\ab = E_\varpi E^\nr$.
\end{thm}
We postpone the proof in order to define the Artin map. Lemma
\ref{lem:PropertiesOfFiniteLTExt} implies the existence of an isomorphism
\begin{equation*}
  \cO_E^\times \cong \lim_{m \in \N} \cO_E^\times/(1 + \varpi^m \cO_E) \cong
  \lim_{m \in \N} \Gal(E_{\varpi,m}/E) = \Gal(E_\varpi/E),
\end{equation*}
sending $u \in \cO_E^\times$ to the unique $E$-linear automorphism $\sigma$ of
$E_\varpi$ satisfying $\sigma(x) = [u]_F(x)$ for $x \in \Lambda_f$. 
As $E_\varpi$ and $E^\nr$ have trivial intersection, we have
\begin{equation*}
  \Gal(E^\ab /E) = \Gal(E_\varpi E^\nr / E) = \Gal(E_\varpi/E) \times
  \Gal(E^\nr /E) \cong \cO_E^\times \times \hat \Z.
\end{equation*}
Using \eqref{eq:WeilgroupInjectsIntoGalois}, this isomorphism
identifies $\Weil_E^\ab$ with $\cO_E^\times \times \Z$. We obtain the Artin map
\begin{equation*}
  \Art_E\colon E^\times \cong \cO_E^\times \times \Z \to 
  \Gal(E_\varpi/E) \times \Phi^\Z \cong \Weil_E^\ab.
\end{equation*}
By definition, this isomorphism satisfies 
$\Art_E(\varpi)|_{E^\nr} = \Phi$. This implies property (a) in 
Theorem \ref{thm:KroneckerWeber}. 

\begin{xpl}
  Let us consider the case $E = \Q_p$. As the polynomial 
  $f(T) = (1+T)^p - 1$ lies in $\cF_p$, we find that the extension $E_p$ is generated
  by the $p$-power torsion points of the multiplicative formal $\Z_p$-module law
  $\Ghat_m$ over $\cO_{\bar \Q_p}$.
  Hence, the maximal abelian extension of $\Q_p$ is given by 
  \begin{equation*}
    \Q_p^\ab = \bigcup_{m \in \N} \Q_p(\zeta_{p^m}) \times \Q_p^\nr
             = \bigcup_{k \in \N} \Q_p(\zeta_k),
  \end{equation*}
  with Galois group $\Gal(\Q_p^\ab /\Q_p) \cong \Z_p^\times \times \hat \Z$. 
\end{xpl}

The remainder of this section is devoted to the proof of the local Kronecker--Weber 
Theorem \ref{thm:LocalKW}.
The proof makes heavy use of the upper numbering for the ramification filtration.
We don't explain these here, instead we refer to \cite[Chapter
IV]{serre2013local} for a survey of the theory. We
do however note that in the case of Lubin--Tate extension, the 
ramification filtration can be calculated explicitely (cf. \cite[Theorem
2]{gold1981local}): we have
\begin{equation}\label{eq:CalculationOfUpperRamificationFiltration}
  \Gal(E_{\varpi, m}/E)^v \cong \begin{cases}
    \cO_E^\times/(1 + \varpi^m \cO_E) \quad &\text{if } -1 \leq v < 0 \\
    \{0\} \quad &\text{if } v \geq m \\
    (1 + \varpi^i \cO_E)/(1+\varpi^m \cO_E) \quad &\text{if } i \leq v < i+1
    \quad \text{for } i \in \N_0.
  \end{cases}
\end{equation}
In particular, this implies by Herbrand's theorem (cf. \cite[Ch. 4, Proposition
-14]{serre2013local}) that the upper numbering filtration of 
$\Gal(E_\varpi/E)$ has jumps exactly at the non-negative integers, with
\begin{equation*}
  q-1 \leq [\Gal(E_\varpi/E)^{v} : \Gal(E_\varpi/E)^{v+1}].
\end{equation*}

By the Hasse--Arf theorem, the same phenomenon occurs for any abelian extension 
(cf. \cite{arf1940untersuchungen}).
\begin{thm}[Hasse--Arf]\label{thm:Hasse--Arf}
  Suppose that $E'/E$ is an abelian extension. If $v \in \R$ is a jump in the 
  filtration $\Gal(E'/E)^v$, then $v$ is an integer.
\end{thm}

We furthermore need the following variant of Herbrand's theorem (cf. \cite[Chapter IV,
Proposition 14]{serre2013local}).
\begin{thm}[Herbrand's theorem]\label{prop:HerbrandsThm}
  Let $L$ be a Galois extension of $E$ and let $H \subseteq \Gal(L/K)$ be a normal subgroup
  corresponding to the subextension $L^H \subseteq L$. Then, for any real number $v \geq -1$, 
  we have
  \begin{equation*}
    \Gal(L/E)^v H / H \cong \Gal(L^H/K)^v = (\Gal(L/E)/H)^v.
  \end{equation*}
\end{thm}

These results suffice to show the following.

\begin{lem}\label{lem:LocalKronWebAlmostDone}
  Let $E'/E_\varpi$ be a totally ramified extension with $E'/E$ abelian.
  Then $E' = E_\varpi$. 
\begin{proof}[Sketch of Proof]
  We write $G = \Gal(E'/E)$ and $H = \Gal(E'/E_\varpi)$. We want to show that 
  $H$ is trivial. This will follow from the following two observations.
  \begin{enumerate}
    \item We have $\cap_{v \geq 0} G^v = \{0\}$. Indeed, for any open subgroup
      $J \subset G$ we have $\cap_{v \geq 0} (G/J)^v = 0$, and by definition
      $G^v = \lim_J (G/J)^v$. 
    \item Also by Herbrand's theorem, we have for integers $v \geq 0$ the equality
      \begin{equation}\label{eq:GoldEssentialEq}
        [G^v : G^{v+1}] = [(G/H)^v : (G/H)^{v+1}] [G^v \cap H : G^{v+1} \cap H].
      \end{equation}
  \end{enumerate}
  The fact that $E'/E$ is totally ramified implies that $\inf_{\varepsilon > 0}
  [G^v: G^{v+\varepsilon}] \leq q$ for any $v \geq 0$. By the Theorem of 
  Hasse--Arf, these numbers are non-zero only if $v$ is integral, so
  the left-hand side of the equation \eqref{eq:GoldEssentialEq} is bounded from
  above by $q$ (cf. \cite[Lemma 3]{gold1981local}).
  Meanwhile, by \eqref{eq:CalculationOfUpperRamificationFiltration}, the first factor
  on the right-hand side of \eqref{eq:GoldEssentialEq} is bounded from below by
  $q-1$. This implies $[G^v \cap H : G^{v+1} \cap H] = 1$ for all $v \geq 0$. As $E'/E$ is totally ramified, we have
  $H \subseteq G = G^0$, so the first observation allows us
  to deduce that $H = H \cap \left( \bigcap_{v \geq 0} G^v \right) = \{0\}$, as desired.
\end{proof}
\end{lem}

\begin{proof}[Proof of the local Kronecker--Weber theorem]
  Let $E'/E$ be an abelian extension, identified inside the 
  algebraic closure $\bar E$. We show that $E' \subset E_\varpi E^\nr$. 
  Consider the short exact sequence of Galois groups
  \begin{equation*}
    \{1\} \to \Gal(E' E_\varpi E^\nr/ E_\varpi E^\nr) \to 
    \Gal(E' E_\varpi E^\nr/ E_\varpi) \to 
    \Gal(E_\varpi E^\nr/ E_\varpi) \to  \{1\}.
  \end{equation*}
  As the intersection of  $E_\varpi$ and $E^\nr$ is trivial, the group on the right
  is isomorphic to $\Gal(E^\nr /E) \cong \hat \Z$. Hence the sequence admits 
  a splitting $\Gal(E_\varpi E^\nr/ E_\varpi) \to \Gal(E' E_\varpi E^\nr/
  E_\varpi)$. The fixed field of the image of this splitting is an extension
  $L/E_\varpi$ with $L E^\nr E_\varpi = E' E^\nr E_\varpi$ and $L/E$ abelian.
  By the Lemma above this implies $L = E_\varpi$, hence $E' \subset E^\nr E_\varpi$.
  This is what we wanted to show.
\end{proof}

% subsubsection Construction of the Maximal Abelian Extension (end)

% subsection Application: The Local Class Field Theory (end)
\end{document}
