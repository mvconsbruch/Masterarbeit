%! TeX root: ../main.tex
\documentclass[../main.tex]{subfiles}

\begin{document}


\section{Explicit Aspects of Abelian Lubin--Tate Theory} % (fold)
\label{sec:Local Class Field Theory}
We review the ideas of classical Lubin--Tate theory, which essentially give 
a recipe to construct the maximal abelian extension of a local field $E$. 

First results towards this direction date back as far as 1801, when 
Gauß proved the famous quadratic reciprocity law: if $\ell \neq p$ are odd 
prime numbers, we have the equality
\begin{equation*}
  \legendre p \ell = (-1)^{\frac{p-1}2 \frac{l-1}2} \legendre \ell p,
\end{equation*}
This unintuitive statement about the interplay of the multiplicative structures of
$\FF_p$ and $\FF_\ell$ finds a deeper, more conceptual meaning when inspected through
the lens of cyclotomic fields. This connection
was first made by Gauß himself, when he provided a proof of quadratic
reciprocity using so-called Gauß sums. Admitting theory of cyclotomic fields 
(as developed in Chapter 1 of \cite{neukirch2006algebraische}), and denoting by
$\Q(\zeta_\ell)$ the $\ell$-th cyclotomic extension,
quadratic reciprocity is essentially equivalent to the following statement.
\begin{equation*}
  \legendre \ell p = (-1)^{\frac{l-1}2 \frac{p-1}2} \iff \begin{array}c
    \text{The prime $p \in \Z$
    decomposes into an even} \\ \text{number of prime ideals in
  $\Q(\zeta_\ell)$.} \end{array}
\end{equation*}
This foreshadows class field theory. Indeed, one major achievement
of class field theory is the Artin reciprocity law, which in particular yields a 
description of the decomposition behaviour of unramified primes inside abelian extensions
of number fields, cf. \cite[Theorem 7.3]{neukirch2006algebraische}. 
The appearence of cyclotomic fields above is also no coincidence.

\begin{thm}[Kronecker--Weber\footnote{despite the name, the Kronecker--Weber Theorem
wasn't (completely) proven by either Kronecker or Weber; the first correct
prove was found by Hilbert. For more on the history of class field theory, see 
\cite{Conrad2009HISTORYOC}.}] \label{thm:KroneckerWeber}
  Every finite extension of $\Q$ embedds into $\Q(\zeta_m)$ for some $m \in \Z$.
  That is, after choosing an algebraic closure $\Q \inj \bar \Q$, the maximal abelian
  subextension is given by 
  \begin{equation*}
    \Q^\ab = \bigcup_{m \in \N} \Q(\zeta_m) \subset \bar \Q.
  \end{equation*}
\end{thm}

The problem of finding similar descriptions for the maximal abelian extension of
general number fields seems to be very difficult. There is a similar description 
of Abelian extensions of imaginary quadratic fields (cf. \cite[Chapter II,
§5]{silverman1994advanced}; these extensions arise by adjoining certain values attained
by the so-called Weber function at the torion points of an elliptic curve with complex
multiplication by the field in question), but in general this problem seems to be open.

The situation simplifies a lot for non-Archimedian local fields. 
The aim of this section is to discuss the analogues of the 
Kronecker--Weber theorem and the Artin map for these fields. We shortly set the
stage. Let $E$ be a non-Archimedian local field with residue field $\FF_q$ of
characteristic $p$, and choose an embedding $E \inj \bar E$ into an algebraic
closure $\bar E$ of $E$. Let $E^\nr$ be maximal unramified subextension, with
residue field identified with $\Fqbar$.
In this situation, we have the short exact sequence
\begin{equation*}
  0 \to I_E = \Gal(\bar E/ E^\nr) \to \Gal(\bar E/E) \to \Gal(E^\nr /E) =
  \Gal(\bar \FF_q /\FF_q) \to 0.
\end{equation*}
Recall that $\Gal(\Fqbar/\FF_q)$ is topologically generated by the $q$-th power
Frobenius automorphism $\Frob_q: x \mapsto x^q$. Let $\Phi \in \Gal(E^\nr /E)$ 
denote the corresponding Frobenius automorphism of $E^\nr$. 
The Weil group is defined as
\begin{equation*}
  \Weil_E \coloneq \{\sigma \in \Gal(\bar E/E) : \sigma|_{E^\nr} = \Phi^m
  \text{ for some } m \in \Z\}.
\end{equation*}
As above, we have the short exact sequence
\begin{equation*}
  0 \to I_E \to \Weil_E \to \Z \to 0.
\end{equation*}

We work towards the following Theorem.

\begin{thm}[Local Class Field Theory]\label{thm:LocalKroneckerWeber} \leavevmode
  \vspace{-0.8em}
  \begin{enumerate}
    \item \textnormal{The Local Kronecker--Weber Theorem.} For any choice of uniformizer $\varpi \in E^\times$, there exists a 
      totally ramified field extensions $E_\varpi$ of $E$ such that 
      the maximal abelian subextension $E^\ab \subset \bar E$ of $E$
      decomposes as $E^\ab = E_\varpi E^\nr$. The extension 
      $E_\varpi$ is obtained from $E$ by adjoining $\varpi$-torsion points of so-called
      Lubin--Tate formal module laws. 
    \item \textnormal{Local Artin Reciprocity.} There is a unique isomorphism
      (the Artin map)
      \begin{equation*}
        \Art_E: E^\times \xto \sim \Weil_E^\ab
      \end{equation*}
      such that 
      \begin{enumerate}
        \item for any choice of uniformizer $\varpi \in E^\times$, we have
          $\Art_E(\varpi)|_{E^\nr} = \Phi$.
        \item for any finite Abelian extension $E'$ of $E$, consider the
          map $E^\times \xto{\Art_E} \Weil_E \xto{\sigma \mapsto
          \sigma|_{E'}} \Gal(E'/E)$. It is surjective and has kernel given by
          $\Norm_{E'/E}(E'^\times)$. 
      \end{enumerate}
  \end{enumerate}
\end{thm}

We follow the construction of $E_\varpi$ in \cite{LubinTateFormalMult}, and we sketch
a proof of the equality $E^\ab = E_\varpi E^\nr$. Lubin and Tate showed this
making use of previously known results in local class field theory, which ultimately
rested on results from global class field theory. 
In order to stay in the local setting, and to sell the theorem as a
local version of the Kronecker--Weber theorem,
we avoid these arguments by referring to ideas of Gold (cf.
\cite{gold1981local}), who realized that the theorem of Hasse--Arf
suffices to deduce this equality. 
Once all of this is known, the isomorphism $\Art_E$ and property
(a) arise naturally. We do not prove property (b).

\subsection{Lubin--Tate Formal Module Laws} % (fold)
\label{sub:Lubin--Tate Formal Module Laws}
Fix a uniformizer $\varpi \in E^\times$. In this section we explain the construction
of Lubin--Tate module laws. As announced, these module laws will allow us to
construct the totally ramified extension $E_\varpi$.

Write $\cF_{\varpi, h}$ for the set of power series
\begin{equation*}
  \cF_\varpi \coloneqq \{f \in \cO_E \llbr T \rrbr \mid f \equiv \varpi T \pmod {T^2}
  \text{ and } f \equiv T^{q^n} \pmod \varpi\}. 
\end{equation*}

The construction of the Lubin--Tate formal module laws is based on the following 
lemma, which is Lemma 1 in \cite{LubinTateFormalMult}.
\begin{lem}\label{lem:LTLemma1}
  Let $f(T)$ and $g(T)$ be elements of $\cF_{\varpi,h}$ and let 
  $L(X_1, \dots, X_n) = \sum_{i=1}^n a_i X_i$ be a linear form with coefficients in 
  ${\cO_E}$. Then there exists a unique series $F(X_1, \dots, X_n)$ with coefficients 
  in ${\cO_E}$ such that 
  \begin{gather*}
    F(X_1, \dots, X_n) \equiv L(X_1, \dots, X_n) \pmod {T^2}, \\ \text{and} \\
    f(F(X_1, \dots, X_n)) = F(g(X_1), \dots, g(X_n)).
  \end{gather*}
\end{lem}

As a direct consequence, we obtain the following useful result.
\begin{lem}
  Let $f \in \cF_{\varpi, h}$. Then there is a unique formal ${\cO_E}$-module law $F_f$ over ${\cO_E}$
  with $[\varpi]_F(T) = f(T)$.
\begin{proof}
  In the above lemma, set $L(X,Y) = X+Y$ and $g=f$ to uniquely determine 
  the power series $F_f$. The same Lemma yields unique power series
  $[a]_{F_f}(T) \in R\llbr T \rrbr$ by setting $L(T) = a T$, $g=f$. It is
  routine to check that $(F_f, ([a]_f)_{a \in {\cO_E}})$ is a formal ${\cO_E}$-module law, 
  cf. \cite{LubinTateFormalMult}.
\end{proof}
\end{lem}

\begin{defi}[Lubin--Tate Module Law]
  We refer to module laws arising from the construction above as Lubin--Tate module laws.
\end{defi}

Furthermore, attached to each $a \in \cO_E$ and $f,g \in \cF_{\varpi,h}$, we find
unique $[a]_{f,g}(T) \in \cO_E\llbr T \rrbr$ satisfying
\begin{equation}\label{eq:LTMoLaScaCond}
  [a]_{f,g}(T) \equiv aT \pmod {(T)^2} \quad \text{and} \quad
  f([a]_{f,g}(T)) = [a]_{f,g}(g(T)).
\end{equation}
We now have the following theorem.
\begin{thm}[Lubin--Tate Formal $\cO_E$-Module Laws]\label{thm:LTModLaw}
  Let $E$ be a local field with ring of integers $\cO_E$. For any choice of 
  uniformizer $\varpi \in \cO_E$ and any $f \in \cF_{\varpi,h}$, the family of power
  series $(F_f, ([a]_{f,f})_{a \in \cO_E})$
  gives rise to a formal $\cO_E$-module law over $\cO_E$. For 
  $f,g \in \cF_{\varpi,h}$, the formal $\cO_E$-module laws $F_f$ and $F_g$ are
  canonically isomorphic, via the morphism induced by $[1]_{f,g} \in \cO_E\llbr
  T \rrbr$. 
\begin{proof}
  See Theorem 1 of \cite{LubinTateFormalMult} and the subsequent discussion.
\end{proof}
\end{thm}
In particular, up to canonical isomorphism, there is only one Lubin--Tate formal
$\cO_E$-module law over $\cO_E$ attached to the choice of the uniformizer $\varpi \in
\cO_E$. 

\begin{xpl}
  If $E = \Q_p$, this reconstructs the multiplicative formal 
  $\Z_p$ module $\Ghat$ constructed above. Indeed, we have 
  \begin{equation*}
    \cF_p = \{f \in \Z_p\llbr T \rrbr \mid f(T) \equiv T^p \text{ mod } p
    \text{ and } f(T) \equiv pT \text{ mod } (T)^2 \},
  \end{equation*}
  implying that $f(T) = (1+T)^p-1$ lies in $\cF_p$.  
  One quickly checks that 
  \begin{equation*}
    F_f(X,Y) = (1+X)(1+Y) - 1 = X + Y + XY \in \Z_p \llbr X,Y \rrbr
  \end{equation*}
  is the addition law associated to $f$, and that 
  for $a \in \Z_p$, the power series
  \begin{equation*}
    [a]_{f,f} = (1+T)^{a} - 1 \in \Z_p \llbr T \rrbr
  \end{equation*}
  satisfies the condition of \eqref{eq:LTMoLaScaCond}. 
\end{xpl}
% subsubsection Lubin--Tate Formal Module Laws (end)

\subsection{Construction of the Maximal Abelian Extension} % (fold)
\label{sub:Construction of the Maximal Abelian Extension}
We now come to the construction of $E_\varpi$. We first construct its
finite parts $E_{\varpi, m}$ as follows. Let $\varpi$ be a
uniformizer of $E$ and let $f \in \cF_\varpi$ with corresponding
Lubin--Tate formal module law $F$. We define
\begin{equation*}
  \Lambda_{f,m} \coloneqq \{x \in \fm_{\bar E} \mid [\varpi^m]_F(x) = 0\}.
\end{equation*}
Here, we equip $\bar E$ with the canonical topology coming from $E$ by extending the
$\varpi$-adic valuation to finite subextensions. With this topology, $\bar E$
is not complete, but any $x \in \fm_{\bar E}$ lies inside some finite (hence
complete) subextension, so the power series $[\varpi]_F(x)$ converges, and the 
definition of $\Lambda_{f,m}$ makes sense.

With the $\cO_E$-module structure from $F$, the set $\Lambda_{f,m}$ is a
$\cO_E$-module. The absolute Galois group of $E$ acts $\cO_E$-linearly on
$\Lambda_{f,m}$, as the power series constituting the $\cO_E$-structure all
have coefficients in $E$. This yields a homomorphism 
\begin{equation} \label{eq:GaloisGroupToLambdaHom}
  \Gal(\bar E /E ) \to \Aut_{\Mod{\cO_E}}(\Lambda_{f,m}).
\end{equation}

We now define 
\begin{equation*}
  E_{\varpi, m} \coloneq E(\Lambda_{f, m})
\end{equation*}
and call it the Lubin--Tate extension of degree $m$. Omitting the choice of $f$ 
in notation is justified, by the following key observation.
\begin{lem}\label{lem:FiniteLTExtensionIndepOfF}
  The Lubin--Tate extension of degree $m$ does only depend on the choice of $\varpi$,
  not on the choice of power series $f \in \cF_\varpi$. 
\begin{proof}
  Given another choice $f' \in \cF_\varpi$ yielding a Lubin--Tate module law
  $F'$, write $E'_{\varpi, m}$ for the Lubin--Tate extension of degree $m$ 
  obtained by adjoining torsion points of $f'$. Then the power series 
  $[1]_{f,f'}$ yields an isomorphism $\Lambda_{f,m} \to \Lambda_{f',m}$.
  As $x \in E$ implies $[1]_{f,f'}(x) \in E$, we find $E' \subseteq E$, and the
  claim follows by symmetry.
\end{proof}
\end{lem}

Hence, we shall henceforth choose $f(T) = \varpi T + T^q$. From the fact that 
$F$ is $\varpi$-divisible of height $1$, it is not hard to deduce that
$\Lambda_{f,m}$ is isomorphic to $\cO_E/(\varpi^m \cO_E)$ as $\cO_E$-module,
cf. \cite[Theorem 2]{LubinTateFormalMult}. 

\begin{lem}\label{lem:PropertiesOfFiniteLTExt}
  For any positive integer $m$, the extension $E_{\varpi,m}/E$ is totally ramified
  and of degree $[E_{\varpi,m}:E] = (q-1) q^{m-1}$. Furthermore, the morphism
  \eqref{eq:GaloisGroupToLambdaHom} induces an isomorphism
  \begin{equation*}
    \Gal(E_{\varpi, m}/E) \cong \Aut_{\Mod {\cO_E}}(\Lambda_{f,m}) = 
    \cO_E^\times/(1 + \varpi^m \cO_E).
  \end{equation*}
\begin{proof}
  Given a positive integer $i \in \N$, we write $f^i(T)$ for the $i$-fold
  self-composite of $f$ and define
  \begin{equation*}
    \eta_i(T) \coloneq \frac{f^i(T)}{f^{i-1}(T)} = \varpi + f^{i-1}(T)^{q-1}.
  \end{equation*}
  By induction, this is seen to be an Eisenstein polynomial
  of degree $(q-1) q^{i-1}$. The roots of $\eta_m(T)$ are in bijection with the 
  elements in $\Lambda_{f,m} \setminus \Lambda_{f,m-1}$, and we may deduce that 
  \begin{equation*}
    E_{\varpi,m} \cong \frac{E[T]}{(\eta_m(T))}.
  \end{equation*}
  Hence $E_{\varpi,m}$ is totally ramified of the specified degree. As
  $\Lambda_{f,m}$ generates $E_{\varpi, m}$, the morphism
  $\Gal(E_{\varpi,m}/E) \to \Lambda_{f,m}$ is injective. This implies that it
  is bijective since both source and target of this morphism have the same cardinality. 
\end{proof}
\end{lem}

We define the Lubin--Tate extension of $E$ as the union
\begin{equation*}
  E_\varpi \coloneq \bigcup_{m \in \N} E_{\varpi, m} = E(\Lambda_f) \quad
  \text{with} \quad \Lambda_f \coloneqq \bigcup_{m \in \N} \Lambda_{f,m}.
\end{equation*}
The lemma above implies the existence of an isomorphism
\begin{equation*}
  \cO_E^\times \cong \lim_{m \in \N} \cO_E^\times/(1 + \varpi^m \cO_E) \cong
  \lim_{m \in \N} \Gal(E_{\varpi,m}/E) = \Gal(E_\varpi/E),
\end{equation*}
sending $u \in \cO_E^\times$ to the unique $E$-linear automorphism $\sigma$ of
$E_\varpi$ satisfying $\sigma(x) = [u]_F(x)$ for $x \in \Lambda_f$. 

The remainder of this section is devoted to the proof of the fact that $E^\ab =
E_\varpi E^\nr$. 
The proof makes heavy use of the upper numbering for the ramification filtration.
We introduce either of these notions here; instead we refer to \cite[Chapter
IV]{serre2013local} for their definition and a survey of their properties. We
do however note that in the case of Lubin--Tate extension, the ramification
filtration can be calculated explicitely (cf. \cite[Theorem 2]{gold1981local}): we have
\begin{equation}\label{eq:CalculationOfUpperRamificationFiltration}
  \Gal(E_{\varpi, m}/E)^v \cong \begin{cases}
    \cO_E^\times/(1 + \varpi^m \cO_E) \quad &\text{if } -1 \leq v < 0 \\
    \{0\} \quad &\text{if } v \geq m \\
    (1 + \varpi^i \cO_E)/(1+\varpi^m \cO_E) \quad &\text{if } i \leq v < i+1
    \quad \text{for } i \in \N_0.
  \end{cases}
\end{equation}
In particular, this implies by Herbrand's theorem (cf. \cite[Ch. 4, Proposition
14]{serre2013local}) that the upper numbering filtration of 
$\Gal(E_\varpi/E)$ has jumps exactly at the non-negative integers, with
\begin{equation*}
  q-1 \leq [\Gal(E_\varpi/E)^{v}: \Gal(E_\varpi/E)^{v+1}] \leq q.
\end{equation*}

By the theorem of Hasse--Arf, the same phenomenon occurs for any abelian extension.
\begin{thm}[Hasse--Arf]\label{thm:Hasse--Arf}
  Suppose that $E'/E$ is an abelian extension. If $v \in \R$ is a jump in the 
  filtration $\Gal(E'/E)^v$, then $v$ is an integer.
\end{thm}

This quickly implies the following.

\begin{lem}\label{lem:LocalKronWebAlmostDone}
  Let $E'/E_\varpi$ be a totally ramified extension with $E'/E$ abelian.
  Then $E' = E_\varpi$. 
\begin{proof}
  We write $G = \Gal(E'/E)$ and $H = \Gal(E'/E_\varpi)$. We want to show that 
  $H$ is trivial. We make the following two observations.
  \begin{enumerate}
    \item We have $\cap_{v \geq 0} G^v = \{0\}$. Indeed, for any open subgroup
      $J \subset G$ we have $\cap_{v \geq 0} (G/J)^v = 0$, and by Herbrand's theorem
      $G^v = \lim_J (G/J)^v$. 
    \item Also by Herbrand's theorem, we have for integers $v \geq 0$ the equality
      \begin{equation*}
        [G^v : G^{v+1}] = [(G/H)^v : (G/H)^{v+1}] [G^v \cap H : G^{v+1} \cap H].
      \end{equation*}
  \end{enumerate}
  The fact that $E'/E$ is totally ramified implies that the left-hand side of
  the equation above is bounded from above by $q$ (cf. \cite[Lemma 3]{gold1981local}).
  Meanwhile, by \eqref{eq:CalculationOfUpperRamificationFiltration}, the first factor
  of the right-hand side is bounded from below by $q-1$. This implies $[G^v \cap H :
  G^{v+1} \cap H] = 1$ for all $v \geq 0$. Hence, the first observation allows us
  to deduce that $H = H \cap \left( \bigcap_{v \geq 0} G^v \right) = \{0\}$, as desired.
\end{proof}
\end{lem}

\begin{proof}[Proof of the local Kronecker--Weber theorem]
  Let $E'/E$ be an abelian extension, identified inside the 
  algebraic closure $\bar E$. We show that $E' \subset E_\varpi E^\nr$. 
  Consider the short exact sequence of Galois groups
  \begin{equation*}
    \{1\} \to \Gal(E' E_\varpi E^\nr/ E_\varpi E^\nr) \to 
    \Gal(E' E_\varpi E^\nr/ E_\varpi) \to 
    \Gal(E_\varpi E^\nr/ E_\varpi) \to  \{1\}.
  \end{equation*}
  As the intersection of  $E_\varpi$ and $E^\nr$ is trivial, the group on the right
  is isomorphic to $\Gal(E^\nr /E) \cong \hat \Z$. Hence the sequence admits 
  a splitting $\Gal(E_\varpi E^\nr/ E_\varpi) \to \Gal(E' E_\varpi E^\nr/
  E_\varpi)$. The fixed field of the image of this splitting is an extension
  $L/E_\varpi$ with $L E^\nr E_\varpi = E' E^\nr E_\varpi$ and $L/E$ abelian.
  By the Lemma above this implies $L = E_\varpi$, hence $E' \subset E^\nr E_\varpi$. 
\end{proof}

% subsubsection Construction of the Maximal Abelian Extension (end)

% subsection Application: The Local Class Field Theory (end)
\end{document}
