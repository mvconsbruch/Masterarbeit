%! TeX root: ../main.tex
\documentclass[../main.tex]{subfiles}

\begin{document}
\subsection{Extensions of Formal Modules} % (fold)
\label{sub:Exact Categories Extensions of Formal Modules}

In this section, we equip the category $\FMArbOver AS$, where 
$A$ is any ring and $S$ is a $A$-scheme,
with a notion of short exact sequences. We show that this gives $\FMArbOver AS$ 
the structure of an exact category in the sense of 
Quillen \cite[Appendix A]{keller1990chain}. We introduce functors
\begin{align*}
  \Ext(-,-)&: \FMArbOver AS^\opp \times \FMArbOver AS \to \Set \\
  \RigExt(-,-)&: \FMArbOver AS^\opp \times \FMArbOver AS \to \Set,
\end{align*}
which send a pair $(\cF, \cF')$ to the set of equivalence classes of 
extensions (respectively rigidified extensions) of $\cF$ by $\cF'$. 

\subsubsection{The Category of Formal Modules is Exact} % (fold)
\label{ssub:The Category of Formal Modules is Exact}
Before turning our attention to formal modules, we introduce the notion of 
exact categories, following \cite[Appendix A]{keller1990chain}.

\begin{defi}[Exact Category]
  Let $\cA$ be an additive category, and let $\cE$ be a class whose
  members are exact triples of objects connected by arrows 
  \begin{equation*}
    X \xto i Y \xto d Z,
  \end{equation*}
  where $i$ is a kernel of $d$ and $d$ is a co-kernel of $i$. 
  We call a morphism $i:X \to Y$ an inflation if it appears as first component
  of some $(i,d) \in E$, second components of such pairs are called deflations.
  We say that the pair $(\cA, \cE)$ is an exact category if 
  $\cE$ is closed under isomorphisms and satisfies the following properties.
  \begin{enumerate}
    \item The identity $\id_0: 0 \to 0$ is a deflation.
    \item The composition of two deflations is a deflation.
    \item For each $f \in \Hom_\cA(Z', Z)$, there is a cartesian square
    \begin{equation*}
        \begin{tikzcd}[ampersand replacement=\&]
        	{Y'} \& {Z'} \\
        	Y \& Z
        	\arrow["{d'}", from=1-1, to=1-2]
        	\arrow["{f'}"', from=1-1, to=2-1]
        	\arrow["f", from=1-2, to=2-2]
        	\arrow["d"', from=2-1, to=2-2]
        \end{tikzcd}
    \end{equation*}
    such that $d'$ is a deflation.
  \item[3\textsuperscript{op}.] For each $f \in \Hom_\cA(X, X')$, there is a
    co-cartesian square
      \begin{equation*}
        \begin{tikzcd}[ampersand replacement=\&]
        	X \& Y \\
        	{X'} \& {Y'}
        	\arrow["i", from=1-1, to=1-2]
        	\arrow["f"', from=1-1, to=2-1]
        	\arrow["{f'}", from=1-2, to=2-2]
        	\arrow["{i'}"', from=2-1, to=2-2]
        \end{tikzcd}
      \end{equation*}
      such that $i'$ is an inflation.
  \end{enumerate}
\end{defi}

As above, suppose that $A$ is any ring and $S$ is an $A$-scheme. Let $\cF$, $\cE$ and $\cF'$ be formal $A$-modules
over $S$. The category $\FMArbOver AS$ is additive by \cref{lem:FGLAdditive}.

\begin{defi}[Short Exact Sequence]
  A pair of composable morphisms $\cF' \to \cE \to \cF$ in 
  $\FMArbOver A S$ is called 
  a short exact sequence if the induced sequence 
  \begin{equation*}
    0 \to \Lie(\cF') \to \Lie(\cE) \to \Lie(\cF) \to 0
  \end{equation*}
  is a short exact sequence of $\cO_S$-modules. In this case, we write
  \begin{equation*}
    0 \to \cF' \to \cE \to \cF' \to 0.
  \end{equation*}
  A pair of composable morphisms $F' \to E \to F$ in $\FMLArbOver AR$
  is called an exact sequence if it is exact after passing to 
  the respective formal modules. 
\end{defi}

\begin{lem}%\label{lem:SESStandardForm}
  Let $R$ be an $A$-algebra and let $F, F' \in \FMLArbOver AR$ be 
  formal $A$-module laws of dimensions $m$ and $n$, respectively. Write 
  $\cF', \cF \in \FMArbOver AR$ for the associated formal $A$-modules,
  and suppose that they fit into a exact sequence
  \begin{equation*}
    0 \to \cF' \xto \alpha \cE \xto \beta \cF \to 0.
  \end{equation*}
  Write $\bX$ for the variables of $F'$ and $\bZ$ for those of $F$. Then 
  there exists a (non-canonical) coordinate on $\cE$ giving rise to a formal
  $A$-module law $E$ in the variables $(\bX, \bZ)$ such that the induced morphisms
  of formal module laws are of the form 
  $\alpha(\bX) = (\bX, 0)$, $\beta(\bX, \bZ) = \bZ$. Furthermore, 
  the formal $A$-module law $E$ is of the form
  \begin{equation}%\label{eq:SESStandardFormModuleLaw}
  \begin{gathered}
    E((\bX_1, \bZ_1), (\bX_2, \bZ_2)) = (F'(\bX_1, \bX_2) +_{F'} \Delta(\bZ_1,
    \bZ_2) , F(\bZ_1, \bZ_2)) \\
    \text{and} \\
    [a]_E(\bX, \bZ) = ([a]_{F'}(\bX) +_{F'} \delta_a(\bZ), [a]_F(\bZ)).
  \end{gathered}
  \end{equation}
  for some $m$-tuple of power series $\Delta \in (R\llbr \bZ_1, \bZ_2 \rrbr)^m$,
  $\delta_a \in (R\llbr \bZ \rrbr)^m$. 
\begin{proof}
  The construction of $E$ is sketched in \cite[Proposition
  6.5]{hopkins1994equivariant}. We know that $\cE \cong \spf R\llbr M \rrbr$ 
  for some free $R$-module $M$ of rank $m+n$. As we have a short 
  exact sequence on Lie-algebras, we may apply the formal implicit function
  theorem \todo{reference/proof} to obtain a section 
  $\sigma: \cF \to \cE$ of $\beta:\cE \to \cF$. 
  The datum of the morphisms $\alpha$ and $\sigma$ is equivalent to 
  morphisms
  \begin{equation*}
    \alpha^\flat: R \llbr M \rrbr \to R \llbr \bX \rrbr \quad \text{and} \quad
    \sigma^\flat: R \llbr M \rrbr \to R \llbr \bZ \rrbr
  \end{equation*}
  on affines. Taking their sum, we obtain a morphism $R \llbr M \rrbr
  \to R\llbr \bX, \bT \rrbr$. On Lie-algebras, this morphism recovers the
  isomorphism $\Lie(\cE) \cong \Lie(\cF') \oplus \Lie(\cF)$ induced by 
  $\Lie(\sigma)$. In particular, $\sigma^\flat + \alpha^\flat$ is an isomorphism
  in degree $1$, hence an isomorphism. This yields the desired
  coordinate $\cE \cong \spf R\llbr \bX, \bZ\rrbr$. The fact about the 
  structure of the formal $A$-module law $E$ follows quickly from the fact that 
  $\alpha$ and $\beta$ are morphisms of formal $A$-module laws. 
\end{proof}
\end{lem}

Let's turn our attention to the power series $(\Delta, (\delta_a)_{a\in A})$
appearing in the above Lemma. They satisfy certain conditions.
\begin{defi}[Symmetric 2-cocycles]%\label{def:SymCoc2}
  Let $\SymCoc^2(F, F')$ be the set of collections of power series $(\Delta,
  (\delta_a)_{a \in A})$ satisfying the following properties
  \begin{itemize}
    \item $\Delta(\bZ_1,\bZ_2) = \Delta(\bZ_2,\bZ_1)$
    \item $\Delta(\bZ_2,\bZ_3) +_{F'} \Delta(\bZ_1, F(\bZ_2,\bZ_3)) =
      \Delta(F(\bZ_1,\bZ_2), \bZ_3) +_{F'} \Delta(\bZ_1,\bZ_2)$
    \item $\delta_a(\bZ_1) +_{F'} \delta_a(\bZ_2) +_{F'} \Delta([a]_F(\bZ_1),
      [a]_F(\bZ_2)) = 
      [a]_{F'}\Delta(\bZ_1,\bZ_2) +_{F'} \delta_a(F(\bZ_1,\bZ_2))$
    \item $\delta_a(\bZ_1) +_{F'} \delta_b(\bZ_1) +_{F'} \Delta([a]_F(\bZ_1),
      [b]_F(\bZ_1)) =
      \delta_{a+b}(\bZ_1)$
    \item $[a]_{F'}\delta_b(\bZ_1) +_{F'} \delta_a([b]_F(\bZ_1)) = \delta_{ab}(\bZ_1)$.
  \end{itemize}
  These objects are called symmetric $2$-cocycles. 
\end{defi}
\begin{prop}%\label{prop:ClassOfFGLitoSymCoc}
  There is a bijection 
  \begin{equation*}
    \SymCoc^2(F, F') \xto \sim \left\{
      \begin{gathered}
        \text{$A$-module laws $E$ on $R\llbr \bX, \bZ \rrbr$ fitting into an
        exact sequence}
        \\ 0 \to F' \xto \alpha E \xto \beta F \to 0 \\
        \text{where $\alpha(\bX) = (\bX, 0)$ and $\beta(\bX, \bZ) = \bZ$. }
      \end{gathered}
      \right\}
  \end{equation*}
  The map sends a pair $\{\Delta, (\delta_a)_a\}$ to the $A$-module law with
  structure defined following \eqref{eq:SESStandardFormModuleLaw}. 
  \begin{proof}[Proof]
    This is only a matter of calculation, cf. \cite[Section
    6]{hopkins1994equivariant}.
  \end{proof}
\end{prop}

\begin{lem}
  If $\cF'$, $\cE$ and $\cF$ are formal $A$-modules over an $A$-scheme $S$,
  and $\alpha$ and $\beta$ are morphisms such that 
  $0 \to \cF' \xto\alpha \cE \xto\beta \cF \to 0$ 
  is a short exact sequence of formal $A$-modules, $\alpha$ is a kernel of
  $\beta$ and $\beta$ is a cokernel of $\alpha$. 
\begin{proof}
  Let $\psi: \cG \to \cE$ be a morphism of formal $A$-modules such that
  the composition $\cG \xto\psi \cE \xto\beta \cF$ is trivial. 
  We have to show that there is a unique morphism $\bar \psi: \cG \to 
  \cF'$ making the following diagram commute.
  \begin{equation*}
    \begin{tikzcd}[ampersand replacement=\&]
    	0 \& {\cF'} \& \cE \& \cF \& 0 \\
    	\&\& \cG
    	\arrow[from=1-1, to=1-2]
    	\arrow["\alpha", from=1-2, to=1-3]
    	\arrow["\beta", from=1-3, to=1-4]
    	\arrow[from=1-4, to=1-5]
    	\arrow["{\exists!\overline \psi}", dashed, from=2-3, to=1-2]
    	\arrow["{\psi}"', from=2-3, to=1-3]
    	\arrow["0"', from=2-3, to=1-4]
    \end{tikzcd}
  \end{equation*}
  As $\overline \psi$ is unique, we may work locally $S$ and assume that $S =
  \spec R$ is affine and 
  $\cF'$, $\cF$ and $\cG$ all come from formal $A$-module laws. We may
  now assume that the short exact sequence is in the form of
  \cref{lem:SESStandardForm}. Write $E$, $F$, $F'$, $G$ for the 
  formal $A$-module laws corresponding to $\cE$, $\cF$, $\cF'$ and 
  $\cG$. Write $\bY$ for the variables of $G$. Now, as 
  $\beta \circ \psi = 0$, the induced morphism of
  formal $A$-module laws $\psi: G \to E$ is of the form $\psi(\bY) =
  (\psi_1(\bY), 0)$, and we find that $\psi_1(\bY) \in (R\llbr \bY \rrbr)^m$
  yields a morphism of formal $A$-modules $G \to F'$. It is clearly unique. 
  
  Similar ideas show that $\beta$ is a cokernel of $\alpha$. 
\end{proof}

\end{lem}

Similar ideas lead to the following result.

\begin{lem}\label{lem:ClassificationOfInflAndDefl}
  A morphism $\psi\colon \cF \to \cG$ of formal module laws is a  deflation
  if and only if $\Lie(\psi)$ is surjective. It is an inflation if and only
  if $\Lie(\psi)$ is injective with free cokernel.
  \begin{proof}[Proof]
    We prove the deflation result, the result about inflations follows along the
    same lines but is a bit more tedious. We may choose 
    coordinates on $\cF$ and $\cG$ such that $G$ is a formal $\cO_E$-module law
    in variables $\bX$ and $F$ is a formal $\cO_E$-module law in variables
    $(\bX, \bZ)$, and the morphism $F \to G$ is given by $(\bX, \bZ) \mapsto
    \bZ$. Now, writing $F = (F_1, F_2)$ and $[a]_F = ([a]_{F,1}, [a]_{F,2})$, we find
    that $(F_1(\bX, 0), ([a]_{F,1}(\bX, 0))_{a \in \cO_E})$ yields a formal
    $\cO_E$-module law $F'$ in the variables $\bX$, and that the map
    $\bX \mapsto (\bX, 0)$ gives a morphism $F' \to F$ that fits into an exact sequence
      $0 \to F' \to F \to G \to 0$. This is what we wanted to show.
  \end{proof}
\end{lem}

\begin{lem}
  The composition of two deflations of formal $A$-modules is a deflation.
\begin{proof}
  \red{Proof is simple application of \cref{lem:SESStandardForm} but no
  time to write down} 
\end{proof}
\end{lem}

\begin{lem}
  Let $0 \to \cF' \xto\alpha \cE \xto\beta \cF \to 0$ be a short exact sequence in 
  $\FMLArbOver AS$. If $f \in \Hom_{\FMArbOver AS} \cG \to \cF$ is a morphism 
  of formal $A$-modules, then there is a formal $A$-module $f^*\cE$ 
  and a deflation $f^*\cE \to \cG$ fitting into a diagram with short exact 
  sequences as rows
  \begin{equation*}
    \begin{tikzcd}[ampersand replacement=\&]
	    0 \& {\cF'} \& {f^*\cE} \& \cG \& 0 \\
	    0 \& {\cF'} \& \cE \& \cF \& 0
	    \arrow[from=1-1, to=1-2]
	    \arrow["{\alpha'}", from=1-2, to=1-3]
	    \arrow[Rightarrow, no head, from=1-2, to=2-2]
	    \arrow["{\beta'}", from=1-3, to=1-4]
	    \arrow[from=1-3, to=2-3]
	    \arrow[from=1-4, to=1-5]
	    \arrow["f", from=1-4, to=2-4]
	    \arrow[from=2-1, to=2-2]
	    \arrow["\alpha", from=2-2, to=2-3]
	    \arrow["\beta", from=2-3, to=2-4]
	    \arrow[from=2-4, to=2-5]
    \end{tikzcd}
  \end{equation*}
  The square on the right is cartesian.
\begin{proof}
  Assume first that $S = \spec R$ is affine and that $\cF$, $\cF'$ and $\cG$ 
  come from formal $A$-module laws over $R$. Then we assume to be in the
  situation of \cref{lem:SESStandardForm}, with
  $\cE$ coming from a formal $A$-module law $E$. Using the induced morphism 
  $f: G \to F$ of formal $A$-module laws, define the $A$-module law
  law $f^* E$ via
  \begin{gather*}
    f^*E((\bX_1, \bY_1), (\bX_2, \bY_2)) = (F'(\bX_1, \bX_2) +_{F'}
    \Delta(f(\bY_1), f(\bY_2)) , G(\bY_1, \bY_2)) \\
    \text{and } \\
    [a]_{f^*E}(\bX, \bY) = ([a]_{F'}(\bX) +_{F'} \delta_a(f(\bY)), [a]_F(\bY)).
  \end{gather*}
  Here, $\Delta$ and $\delta_a$ are the power series coming from $E$ (cf.
  \cref{lem:SESStandardForm}).
  Now the top-row is exact with $\alpha'(\bX) = (\bX, 0)$ and 
  $\beta'(\bX, \bY) = \bY$. 
  The morphism of $A$-module laws $f^*E \to E$ is given by 
  $(\bX, \bY)\mapsto (\bX,f(\bY))$. One readily checks that 
  \begin{equation*}
    \begin{tikzcd}[ampersand replacement=\&]
    	{f^*E} \& G \\
    	E \& F
    	\arrow["{\beta'}", from=1-1, to=1-2]
    	\arrow[from=1-1, to=2-1]
    	\arrow["f", from=1-2, to=2-2]
    	\arrow["\beta", from=2-1, to=2-2]
    \end{tikzcd}
  \end{equation*}
  is cartesian in the category of formal $A$-module laws over $R$. 
  Hence the construction is independent of the choice of coordinate,
  and we may define $f^* \cE$ as the formal module with underlying formal
  scheme given by $\cE \times_{\cF} \cG$ and module law structure given by the
  maps admitting the local description above.
  \end{proof}
\end{lem}

The dual statement is also true. 
\begin{lem}
  Let $0 \to \cF' \xto \alpha \cE \xto\beta \cF \to 0$ be as above, and let
  $g \in \Hom_{\FMArbOver AS}(\cF', \cG')$ be a morphism of formal 
  $A$ modules. There is a formal $A$-module $g_*\cE$ over $S$ and an
  inflation $\alpha':\cG' \to g_*\cE$ fitting into a diagram with short 
  exact sequences
  \begin{equation*}
    \begin{tikzcd}[ampersand replacement=\&]
    	0 \& {\cF'} \& \cE \& \cF \& 0 \\
    	0 \& {\cG'} \& {g_* \cE} \& \cF \& 0
    	\arrow[from=1-1, to=1-2]
    	\arrow["\alpha", from=1-2, to=1-3]
    	\arrow["g"', from=1-2, to=2-2]
    	\arrow["\beta", from=1-3, to=1-4]
    	\arrow[from=1-3, to=2-3]
    	\arrow[from=1-4, to=1-5]
    	\arrow[Rightarrow, no head, from=1-4, to=2-4]
    	\arrow[from=2-1, to=2-2]
    	\arrow["{\alpha'}", from=2-2, to=2-3]
    	\arrow["{\beta'}", from=2-3, to=2-4]
    	\arrow[from=2-4, to=2-5]
    \end{tikzcd}
  \end{equation*}
  The square on the left is co-cartesian.
\begin{proof}
  We proceed as in the proof of the previous lemma and assume that 
  $S = \spec R$ and that $\cF'$, $\cF$ and $\cG$ come from formal $A$-module
  laws over $R$. Now $E$ is a formal $A$-module law over $R$ of the form 
  described in \cref{lem:SESStandardForm}, and 
  using the power series $\Delta$ and $\delta_a$ we define $g_* E$ via 
  \begin{gather*}
    g_*E((\bY_1, \bZ_1), (\bY_2, \bZ_2)) = (G'(\bY_1, \bY_2) +_{G'}
    g(\Delta(\bZ_1, \bZ_2)) , F(\bZ_1, \bZ_2)) \\
    \text{and } \\
    [a]_{g_*E}(\bX, \bY) = ([a]_{G'}(\bX) +_{G'} g(\delta_a(\bZ)), [a]_F(\bZ)).
  \end{gather*}
  The morphism $E \to g_*E$ is given by $(\bX, \bZ) \mapsto (g(\bX), \bZ)$. 
  These data glue and give rise to a formal $A$-module
  $g_*\cE$ over $S$ satisfying the desired properties.
\end{proof}
\end{lem}

\begin{rmk} 
  The constructions above show that $\SymCoc^2(F, F')$ is functorial in 
  both entries; contravariant in the first, covariant in the second.
\end{rmk}

As a consequence of the previous lemmas, we obtain the following result.
\begin{prop}\label{prop:FMArbIsExact}
  Let $S$ be an $A$-scheme. Then
  the category $\FMArbOver AS$, equipped with the notion of exact sequences
  from \cref{def:SESofFormalModules}, is an exact category. 
\end{prop}

The following calculation is convenient.
\begin{lem}%\label{lem:LieAlgebrasOfPBandPFW}
  We have natural isomorpisms 
  \begin{equation*}
    \Lie(f^* \cE) \cong \Lie(\cE) \times_{\Lie(\cF)} \Lie(\cG) \quad \text{and}
    \quad \Lie(g_* \cE) \cong \Lie(\cG') \sqcup_{\Lie(\cF')} \Lie(\cE).
  \end{equation*}
\begin{proof}
  This is true locally, and the local descriptions descent to $S$.
\end{proof}
\end{lem}

% subsubsection The Category of Formal Modules is Exact (end)

\subsubsection{Extensions and Rigidified Extensions} % (fold)
\label{ssub:Extensions and Rigidified Extensions}
We now introduce the functors $\Ext$ and $\RigExt$. 
Let $\cF$ and $\cF'$ be formal $A$-modules over an $A$-scheme $S$.
\begin{defi}[Extension]
  An extension of $\cF$ by $\cF'$ is a short exact sequence 
  \begin{equation*}
    0 \to \cF' \to \cE \to \cF \to 0.
  \end{equation*}
  We say that this extension is equivalent to another extension 
  \begin{equation*}
    0 \to \cF' \to \cE' \to \cF \to 0
  \end{equation*}
  if and only if there is an isomorphism $\cE \to \cE'$ making the diagram 
  \begin{equation*}
    \begin{tikzcd}[ampersand replacement=\&]
    	0 \& {\cF'} \& \cE \& \cF \& 0 \\
    	0 \& {\cF'} \& {\cE'} \& \cF \& 0
    	\arrow[from=1-1, to=1-2]
    	\arrow[from=1-2, to=1-3]
    	\arrow[Rightarrow, no head, from=1-2, to=2-2]
    	\arrow[from=1-3, to=1-4]
    	\arrow[from=1-3, to=2-3]
    	\arrow[from=1-4, to=1-5]
    	\arrow[Rightarrow, no head, from=1-4, to=2-4]
    	\arrow[from=2-1, to=2-2]
    	\arrow[from=2-2, to=2-3]
    	\arrow[from=2-3, to=2-4]
    	\arrow[from=2-4, to=2-5]
    \end{tikzcd}
  \end{equation*}
  commute. We denote the set of equivalence classes of extensions of $\cF$ by
  $\cF'$ as $\Ext(\cF, \cF')$. 
\end{defi}
\cref{prop:FMArbIsExact} turns $\Ext(-,-)$ into a functor.
In particular, $\Ext(\cF,\cF')$ carries the structure of a
left-$\End(\cF')$-module,
with zero-object given by the canonical extension $\cF \oplus \cF'$.

\begin{defi}[Rigifidied Extension]
  A rigidified extension of $\cF$ by $\cF'$ is a pair consisting of an extension
  \begin{equation*}
    0 \to \cF' \to \cE \to \cF \to 0
  \end{equation*}
  and a splitting $s$ of the short exact sequence
  \begin{equation*}
    \begin{tikzcd}[ampersand replacement=\&]
      0 \& {\Lie(\cF')} \& {\Lie(\cE)} \& {\Lie(\cF)} \& 0.
  	  \arrow[from=1-1, to=1-2]
  	  \arrow["", from=1-2, to=1-3]
  	  \arrow["", from=1-3, to=1-4]
  	  \arrow["s", curve={height=-12pt}, from=1-4, to=1-3]
  	  \arrow[from=1-4, to=1-5]
    \end{tikzcd}
  \end{equation*}
  We say that two rigidified extensions $(\cE,s)$ and $(\cE',s')$ are
  isomorphic if there is an isomorphism $i: \cE \to \cE'$ of extensions such
  that $s' = \Lie(i) \circ s$. We denote the set of isomorphism classes of
  rigidified extensions by $\RigExt(\cF, \cF')$. 
\end{defi}
\begin{lem}
  The assignment $(\cF, \cF') \mapsto \RigExt(\cF,\cF')$ is functorial in both
  entries (contravariant in the first, covariant in the
  second). 
\begin{proof}
  Given a morphism 
  $f: \cG \to \cF$, the induced morphism $\RigExt(\cF, \cF') \to 
  \RigExt(\cG, \cF')$ is given by sending the pair $(\cE,s)$ to
  the pair $(f^*\cE, s')$, where 
  \begin{equation*}
    s': \Lie(\cG) \to \Lie(f^*\cE) \cong \Lie(\cE) \times_{\Lie(\cF)} \Lie(\cG),
    \quad x \mapsto ((s \circ \Lie(f)) (x), x). 
  \end{equation*}
  Here we used the description of $\Lie(f^*\cE)$ from 
  \cref{lem:LieAlgebrasOfPBandPFW}.
  Similarly, given a morphism $g: \cF' \to \cG'$, the induced morphism 
  $\RigExt(\cF, \cF') \to \RigExt(\cF, \cG')$ sends $(\cE,s)$ to
  $(g_*\cE, \Lie(g') \circ s)$, where $g': \cE \to g_* \cE$ is the 
  canonical morphism.
\end{proof}
\end{lem}
In particular, $\RigExt(-,\cF')$ carries the structure of an 
$\End(\cF')$-module, the zero-object is given by the equivalence class of 
the pair
$(\cF' \oplus \cF, s_{\mathrm{triv}})$, where 
$s_{\mathrm{triv}}: \Lie(\cF) \to \Lie(\cF') \oplus \Lie(\cF)$ is the canonical
inclusion.

Of course there is a natural transformation $\RigExt(-,-) \to \Ext(-,-)$,
forgetting the splitting. It appears as the right-most
term of an interesting exact sequence.
\begin{prop}%\label{prop:InterestingES}
  There is an exact sequence of Abelian groups, functorial in
  $\cF$ and $\cF'$
  \begin{equation*}
    \Hom_{\FMArbOver AS}(\cF, \cF') \xto{\Lie} \Hom_{\QCoh {\cO_S}}(\Lie(\cF), \Lie(\cF'))
    \to \RigExt(\cF, \cF') \to \Ext(\cF, \cF'). 
  \end{equation*}
\begin{proof}
  The kernel of $\RigExt(\cF,\cF') \to \Ext(\cF,\cF')$ is given (up to
  equivalence) by 
  pairs of the form $(\cF' \oplus \cF, s)$, where $s$ is a morphism
  of quasi-coherent $\cO_S$-modules such that 
  $$ \Lie(\cF) \xto s \Lie(\cF') \oplus \Lie(\cF) \to  \Lie(\cF)$$
  is the identity. It is clear that these morphisms $s$ correspond to morphisms 
  $\Lie(\cF) \to \Lie(\cF')$. 

  For any $\lambda: \Lie(\cF) \to \Lie(\cF')$, denote by $s_\lambda : 
  \Lie(\cF) \to \Lie(\cF') \oplus \Lie(\cF)$ the splitting given by 
  $y \mapsto (\lambda(y) , y)$.
  The kernel of $\Hom_{\QCoh {\cO_S}}(\Lie(\cF), \Lie(\cF')) \to \RigExt(\cF,
  \cF')$ is spanned by those $\lambda: \Lie(\cF) \to \Lie(\cF')$ such that 
  $(\cF' \oplus \cF, s_\lambda)$ is in the same equivalence
  class as $(\cF' \oplus \cF, s_{\mathrm{triv}})$. Any such pair fits into
  a commutative diagram 
  \begin{equation*}
    \begin{tikzcd}[ampersand replacement=\&]
    	0 \& {\cF'} \& {\cF' \oplus \cF} \& \cF \& 0 \\
    	0 \& {\cF'} \& {\cF' \oplus \cF} \& \cF \& 0,
    	\arrow[from=1-1, to=1-2]
    	\arrow[from=1-2, to=1-3]
    	\arrow[Rightarrow, no head, from=1-2, to=2-2]
    	\arrow[from=1-3, to=1-4]
    	\arrow["\psi", from=1-3, to=2-3]
    	\arrow[from=1-4, to=1-5]
    	\arrow[Rightarrow, no head, from=1-4, to=2-4]
    	\arrow[from=2-1, to=2-2]
    	\arrow[ from=2-2, to=2-3]
    	\arrow["\beta", from=2-3, to=2-4]
    	\arrow[from=2-4, to=2-5]
    \end{tikzcd}
  \end{equation*}
  where $\psi$ is an isomorphism satisfying $\Lie(\psi)(x,y) = (x + \lambda(y), y)$.
  This shows that the composition 
  \begin{equation*}
    \Hom_{\FMArbOver AS}(\cF, \cF') \xto{\Lie} \Hom_{\QCoh {\cO_S}}(\Lie(\cF),
    \Lie(\cF')) \to \RigExt(\cF, \cF')
  \end{equation*}
  is the zero-morphism. Indeed, given $g \in  \Hom_{\FMArbOver AS}(\cF, \cF')$,
  we may choose $\psi$ as the map $(x, y) \mapsto (x +_{\cF'} g(y), y)$.
  Conversely, any morphism $\psi$ fitting into the diagram above is necessarily
  of this form. This proves exactness on the left.
\end{proof}
\end{prop}
% subsubsection Extensions and Rigidified Extensions (end)

% section (end)

\end{document}

