%! TeX root: ../main.tex
\documentclass[../main.tex]{subfiles}

\begin{document}
\section{Topological Rings}
To deal with the topological rings showing up, the notion of admissible rings
will be convenient (taken from \cite[Tag 07E8]{stacks-project}).
\begin{defi}\label{def:admring}
  Let $A$ be a topological ring. We say that $A$ is admissible if
  \begin{itemize}
    \item The element $0 \in A$ has a fundamental system of neighbourhoods
      consisting of ideals.
    \item There exists an ideal of definition, that is, an ideal $I \subset A$ such
      that every open neighbourhood of $0$ contains $I^n$ for some $n$.
    \item It is complete, that is, the natural map
      \begin{equation*}
        A \to \lim_{J \subset A \text{ open ideal}} A/J
      \end{equation*}
      is an isomorphism.
  \end{itemize}
  We say that $A$ is adic if it admits an open ideal of definition.
  Given a topological ring $A$, we denote the category of admissible 
  and adic $A$-algebras (algebras $S$ with continuous morphism $A \to S$) by
  $\Adm A$ and $\Adic A$, respectively.
\end{defi}

\begin{lem}\label{lem:infiniteseriesandproducts}
  Let $S$ be an admissible ring, and let $(s_1, s_2, \dots)$ be a sequence with
  elements in $S$. Then $\sum_{i = 1}^\infty s_i$ converges if and only if 
  $\prod_{i=1}^\infty (1 + s_i)$ converges if and only if $\lim_{i \to \infty} s_i
  = 0$. 
\begin{proof}
  If sum and product converge, $(s_i)_{i \in \N}$ has to converge to zero. The
  reverse implication follows after writing $S \cong \lim_{J} S/J$ for a system
  of open ideals $J \subset S$.
\end{proof}
\end{lem}

\end{document}

