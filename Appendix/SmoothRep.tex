\documentclass[../main.tex]{subfiles}

\begin{document}
\section{Smooth Representations of Locally Profinite Groups}
We review some aspects of the representation theory (over complex vector spaces)
of locally profinite groups. 
If $G$ is an arbitrary group, we denote the category of complex representations,
(that is, morphisms $G \to \GL(V)$, where $V$ is a $\C$-vector space) as
$\Rep G$. At the slight cost of precision, we allow ourselves to refer to an element of 
$\pi: G \to \GL(V) \in \Rep G$ by the underlying vector space $V$, or the 
pair $(\pi, V)$. 

Recall the following basic definitions.
\begin{defi}[Locally Profinite Group]
  A locally profinite group is a Hausdorff topological group such that there exists 
  a neighbourhood of $1 \in G$ consisting of compact open subgroups.
\end{defi}
Throughout this section, if not stated otherwise, $G$ is a locally profinite
group and $H \subset G$ is a closed subgroup of $G$.

\begin{defi}[Smooth Representation]
  A smooth representation of $G$ is a
  representation $\pi: G \to \GL(V) \in \Rep G$,
  such that for any $v \in V$, the stabilizer $G_v$ of $v$ is an open subgroup
  of $G$. We define $\RepSm G$, the category of smooth $G$-representations, as
  the full subcategory of $\Rep G$ with objects given by smooth
  $G$-representations.
\end{defi}

\begin{defi}[Smooth Part of a Representation]
  Let $\pi:G \to \GL(V) \in \Rep G$. We write 
  \begin{equation*}
    V^\sm = \bigcup_{K \subseteq G} V^K,
  \end{equation*}
  where $K$ runs over the compact open subgroups of $G$ and $V^K \subseteq V$ denotes the 
  subspace of elements fixed by $K$. We write $\pi^\sm$ for 
  the induced subrepresentation $G \to \GL(V^\sm)$ of $\pi$.
\end{defi}

\begin{defi}[Algebraic Induction]
  Let $G$ be any group and let $H$ be a subgroup of $G$. We define the Algebraic
  Induction Functor $\algInd_H^G: \Rep H \to \Rep G$ as follows. Given an
  $H$-representation $\pi: H \to \GL(V)$, consider the vector space
  \begin{equation*}
    \algInd_H^G(V) = \{\phi: G \to V \mid \phi(hg) = \pi(h) g\}.
  \end{equation*}
  Now $G$ acts naturally on $\algInd_H^G(V)$ by right-translation (that is, 
  $g.\phi(x) = \phi(xg)$), and we write
  $(\algInd_H^G(\pi),\algInd_H^G(V))$ for the corresponding representation of $G$.
\end{defi}

\begin{rmk} 
  We have $\algInd_H^G(V) = \Hom_{\Mod{\C[H]}}(\C[G], V)$. As
  $\C[G]$ has a natural $(\C[H], \C[G])$-bimodule structure, we obtain a natural
  left-$G$-action on $\algInd_H^G(V)$. This action is precisely the one
  described above.
\end{rmk}

\begin{defi}[Smooth Induction]\label{def:SmoothInduction}
  We define the smooth indction functor $\Ind_H^G: \RepSm H \to \RepSm G$ as 
  the smooth part of the algebraic induction functor. That is, for any smooth
  representation $\pi: G \to \GL(V)$, we define
  \begin{equation*}
    \Ind_H^G(\pi) \coloneqq \algInd_H^G(\pi)^\sm.
  \end{equation*}
  \end{defi}

\begin{defi}[Compact Induction]\label{def:CompactInduction}
  Let $\pi: H \to \GL(V)$ be a smooth representation
  of $H$. Then we define $\cInd_H^G(\pi)$, the compactly induced representation of $\pi$
  as the subrepresentation of $\Ind_H^G(\pi)$ with underlying vector space
  \begin{equation*}
    \{\phi \in \Ind_H^G(\pi) \mid \supp(\phi) \subseteq G \text{ is compact in }
    H\backslash G\}.
  \end{equation*}
  This construction yields a functor $\cInd_H^G: \RepSm H \to \RepSm G$. 
\end{defi}
Note that if $H$ is co-compact in $G$, we have $\cInd_H^G = \Ind_H^G$. 

\begin{rmk} 
  If $H$ is an open subgroup of $G$, the quotient $H \backslash G$ is 
  discrete. Now given $(\pi, V) \in \RepSm H$, an element $\phi \in \Ind_H^G(\pi)$
  lies in $\cInd_H^G(\pi)$ if and only if 
  the image of $\supp(\phi)$ is finite in $H \backslash G$. 
  In this case there is an isomorphism
  \begin{equation}\label{eq:cIndTensorExpr}
    \Psi: \cInd_H^G(V) \cong \C[G] \otimes_{\C[H]} V, \quad \phi \mapsto \sum_{[g]
    \in \supp(\phi)} g^{-1} \otimes \phi(g)
  \end{equation}
  which does not depend on the choice of representative $g \in [g]$ as 
  $(hg)^{-1} \otimes \phi(hg) = g^{-1} \otimes \phi(g)$. 
  Giving $\C[G]$ the structure of an $(\C[G], \C[H])$-bimodule, the natural
  left-$G$-action on $\C[G] \otimes_{\C[H]} V$ agrees with the one 
  of $\cInd_H^G(V)$ under the isomorphism $\Psi$. 
\end{rmk}

\begin{defi}[Restriction Functor]\label{def:RestrictionFunctor}
  Let $G$ be any group and let $H$ be a subgroup of $G$.
  If $\pi: G \to \GL(V)$ is a representation of $G$, we define the
  restriction of $\pi$ from $G$ to $H$ as
  \begin{equation*}
    \Res_H^G(\pi): H \inj G \to \GL(V)
  \end{equation*}
  and call $\Res_H^G: \Rep G \to \Rep H$ the restriction functor.  
If $G$ is locally profinite and $H$ is a closed subgroup, this construction restricts
to a functor
\begin{equation*}
  \Res_H^G(\pi): \RepSm G \to \RepSm H.
\end{equation*}
\end{defi}

\begin{thm}[Frobenius Reciprocity]\label{thm:Frobenius Reciprocity}
  Let $G$ be a locally profinite group and let $H \subseteq G$ be an open subgroup.
  Then there are adjunctions
  \begin{equation*}
    \cInd_H^G \ladj \Res_H^G \ladj \Ind_H^G.
  \end{equation*}
  In particular, if $H$ is additionally co-compact, $\Ind_H^G$ is both left-
  and right-adjoint to $\Res_H^G$.
\begin{proof}
  Making use of the remarks above, both adjunctions are the
  Tensor-Hom-Adjunction in disguise. 
\end{proof}
\end{thm}

\begin{lem}\label{lem:FrobRecUnitsAreInjective}
  Let $H$ be an open subgroup of $G$. 
  The functors $\cInd_H^G$ and $\Res_H^G$ are faithful. 
  Equivalently, the components of the units
  \begin{equation*}
    \id_{\RepSm H} \to \Res_H^G \circ \cInd_H^G \quad \text{ and } \quad 
    \id_{\RepSm G} \to \Ind_H^G \circ \Res_H^G
  \end{equation*}
  coming from the adjunctions in Theorem \ref{thm:Frobenius Reciprocity} are
  injective. 
\begin{proof}
  Faithfulness of $\Res_H^G$ is clear. For faithfulness of $\cInd_H^G$, note that 
  the unit of the adjunction $\cInd_H^G \ladj \Res_H^G$ is given 
  on components $(\pi, V) \in \RepSm H$ by the map 
  $v \mapsto \phi_v$, where $\phi_v \in \cInd_H^G(V)$ is defined as
  \begin{equation*}
    \phi_v: G \to V, \quad g \mapsto \begin{cases}
      \pi(g)v &\text{ if } g \in H,\\
      0       &\text{ otherwise.}
    \end{cases}
  \end{equation*}
  The resulting morphism $V \to \Res_H^G ( \cInd_H^G(V))$ is injective. Now all
  claims follow since faithfulness of the left-adjoint is 
  equivalent to the unit being a monomorphism on components, cf. \cite[Lemma
  4.5.13]{riehl2017category}.
\end{proof}
\end{lem}


\end{document}
