\documentclass[../main.tex]{subfiles}

\begin{document}
\section{Smooth Representations of Locally Profinite Groups}
We review some aspects of the representation theory (over complex vector spaces)
of locally profinite groups. 
If $G$ is an arbitrary group, we denote the category of complex representations,
(that is, morphisms $G \to \GL(V)$, where $V$ is a $\C$-vector space) as
$\Rep G$. At the slight cost of precision, we also allow ourselves to refer to
an element of $\pi\colon G \to \GL(V) \in \Rep G$ by the underlying vector space
$V$, or the pair $(\pi, V)$. 

\begin{defi}[Locally Profinite Group]
  A locally profinite group is a topologically group $G$ such that
  every open neighbourhood of the identity in $G$ contains a compact open
  subgroup of $G$.
\end{defi}
For example, any discrete group is locally profinite and $\GL_n(E)$ is locally 
profinite for any non-archimedian local field $E$. Quotients and closed
subgroups of locally profinite groups are again locally profinite.

Throughout this section, if not stated otherwise, $G$ is a locally profinite
group and $H \subset G$ is a closed subgroup of $G$.

\begin{defi}[Smooth Representation]
  A smooth representation of $G$ is a
  representation $\pi\colon G \to \GL(V) \in \Rep G$,
  such that for any $v \in V$, the stabilizer $G_v$ of $v$ is an open subgroup
  of $G$. We define $\RepSm G$, the category of smooth $G$-representations, as
  the full subcategory of $\Rep G$ with objects given by smooth $G$-representations.
\end{defi}

The forgetful functor $\RepSm G \to \Rep G$ has a left adjoint, given by taking
smooth parts. 
\begin{defi}[Smooth Part of a Representation]
  Let $(\pi,V) \in \Rep G$. We write 
  \begin{equation*}
    V^\sm = \bigcup_{K \subseteq G} V^K,
  \end{equation*}
  where $K$ runs over the compact open subgroups of $G$ and $V^K \subseteq V$
  denotes the subspace of elements fixed by $K$. 
  Now $V^\sm$ is a $G$-stable subspace of $V$, and we write $(\pi^\sm, V^\sm)$ 
  for the induced representation $G \to \GL(V^\sm)$ of $\pi$.
  We call $\pi^\sm$ the smooth part of $\pi$.
\end{defi}

\begin{defi}[Algebraic Induction]
  Let $G$ be any group and let $H$ be a subgroup of $G$. We define the Algebraic
  Induction Functor $\algInd_H^G\colon \Rep H \to \Rep G$ as follows. Given an
  $H$-representation $(\pi, V)\in \Rep H$, consider the vector space
  \begin{equation*}
    \algInd_H^G(V) = \{\phi\colon G \to V \mid \phi(hg) = \pi(h) g\}.
  \end{equation*}
  Now $G$ acts naturally on $\algInd_H^G(V)$ by right-translation (that is, 
  $g.\phi(x) = \phi(xg)$), and we write
  $(\algInd_H^G(\pi),\algInd_H^G(V))$ for the corresponding representation of $G$.
\end{defi}

\begin{rmk} 
  We have $\algInd_H^G(V) = \Hom_{\Mod{\C[H]}}(\C[G], V)$. As
  $\C[G]$ has a natural $(\C[H], \C[G])$-bimodule structure, we obtain a natural
  left-$G$-action on $\algInd_H^G(V)$. This action is precisely the one
  described above.
\end{rmk}

\begin{defi}[Restriction Functor]\label{def:RestrictionFunctor}
  Let $G$ be any group and let $H$ be a subgroup of $G$.
  If $\pi\colon G \to \GL(V)$ is a representation of $G$, we define the
  restriction of $\pi$ from $G$ to $H$ as
  \begin{equation*}
    \Res_H^G(\pi)\colon H \inj G \xto\pi \GL(V)
  \end{equation*}
  and call $\Res_H^G\colon \Rep G \to \Rep H$ the restriction functor.  
\end{defi}

\begin{lem}\label{lem:AlgFrobRec}
  Let $G$ be any group and let $H$ be any subgroup of $G$. Then 
  $\Res_H^G$ is left-adjoint to $\algInd_H^G$.
\begin{proof}
  By the Remark above, this statement readily reduces to the Tensor-Hom-Adjunction.
\end{proof}
\end{lem}

\begin{lem}\label{lem:SmoothRestriction}
  If $G$ is locally profinite and $H$ is a closed subgroup, for any 
  $(\pi,V) \in \Rep G$ we have an $H$-equivariant split inclusion
  \begin{equation*}
    \Res_H^G (V^\sm) \subseteq \left( \Res_H^G (V) \right)^\sm,
  \end{equation*}
  with equality if $H$ is open. 
  In particular, $\Res_H^G$ restricts to a functor
  \begin{equation*}
    \Res_H^G\colon \RepSm G \to \RepSm H.
  \end{equation*}
\begin{proof}
  The first part follows from
  \begin{equation*}
    \Res_H^G(V^\sm) = \bigcup_{K \subset G} V^K \subseteq \bigcup_{K \subset G}
    V^{K \cap H} = \left(\Res_H^G(V)\right)^\sm,
  \end{equation*}
  where $K$ runs over the compact open subsets of $G$. This is an equality if
  $H$ is open. The canonical splitting (sending everything outside the image to 
  zero) is $H$-equivariant.
\end{proof}
\end{lem}

\begin{defi}[Smooth Induction]\label{def:SmoothInduction}
  We define the smooth indction functor 
  $$\Ind_H^G\colon \RepSm H \to \RepSm G$$ 
  as the smooth part of the algebraic induction functor. That is, for any smooth
  representation $\pi\colon G \to \GL(V)$, we set
  \begin{equation*}
    \Ind_H^G(\pi) \coloneqq \left(\algInd_H^G(\pi)\right)^\sm.
  \end{equation*}
  \end{defi}

\begin{defi}[Compact Induction]\label{def:CompactInduction}
  Let $\pi\colon H \to \GL(V)$ be a smooth representation
  of $H$. Then we define $\cInd_H^G(\pi)$, the compactly induced representation
  of $\pi$, as the subrepresentation of $\Ind_H^G(\pi)$ with underlying vector
  space
  \begin{equation*}
    \{\phi \in \Ind_H^G(\pi) \mid \supp(\phi) \subseteq G \text{ is compact in }
    H\backslash G\}.
  \end{equation*}
  This construction yields a functor $\cInd_H^G\colon \RepSm H \to \RepSm G$. 
\end{defi}
Note that if $H$ is co-compact in $G$, we have $\cInd_H^G = \Ind_H^G$. 

\begin{rmk} 
  If $H$ is an open subgroup of $G$, the quotient $H \backslash G$ is 
  discrete. Now given $(\pi, V) \in \RepSm H$, an element $\phi \in \Ind_H^G(\pi)$
  lies in $\cInd_H^G(\pi)$ if and only if 
  the image of $\supp(\phi)$ is finite in $H \backslash G$. 
  In this case there is an isomorphism
  \begin{equation}\label{eq:cIndTensorExpr}
    \Psi\colon \cInd_H^G(V) \cong \C[G] \otimes_{\C[H]} V, \quad \phi \mapsto \sum_{[g]
    \in \supp(\phi)} g^{-1} \otimes \phi(g)
  \end{equation}
  which does not depend on the choice of representative $g \in [g]$ as 
  $(hg)^{-1} \otimes \phi(hg) = g^{-1} \otimes \phi(g)$. 
  Giving $\C[G]$ the structure of an $(\C[G], \C[H])$-bimodule, the natural
  left-$G$-action on $\C[G] \otimes_{\C[H]} V$ is compatible with the one 
  on $\cInd_H^G(V)$ under the isomorphism $\Psi$. 
\end{rmk}

\begin{prop}\label{prop:InducedRepresentationExact}
  For $H$ a closed subgroup of $G$, the functors $\algInd_H^G$, $\Ind_H^G$ and
  $\cInd_H^G$  are exact. 
  \begin{proof}
    The statement for $\algInd_H^G$ is implied by the fact that 
    $\C[G]$ is a free (thereby projective) $\C[H]$-module. For the remaining statements,
    see \cite[p. 18f]{bushnell2006local}.
  \end{proof}
\end{prop}



\begin{thm}[Smooth Frobenius Reciprocity]\label{thm:SmFrobRec}
  Let $G$ be a locally profinite group and let $H \subseteq G$ be a closed subgroup.
  Then there is an adjunction
  \begin{equation*}
    \Res_H^G \ladj \Ind_H^G.
  \end{equation*}
  If $H$ is additionally assumed to be open in $G$, there is an adjunction
  $$\cInd_H^G \ladj \Res_H^G.$$
  In particular, if $H$ is co-compact and open in $G$, the functor $\Ind_H^G$
  is both left- and right-adjoint to $\Res_H^G$.
\begin{proof}
  Making use of the remarks above, both adjunctions are the
  Tensor-Hom-Adjunction in disguise. For the adjunction $\cInd_H^G \ladj
  \Res_H^G$, this is immediate. For the second we observe that
  \begin{equation*}
    \Hom_{\RepSm H}(\Res_H^G V, W) \cong \Hom_{\Rep G}(V, \algInd_H^G(W)) = 
    \Hom_{\RepSm G}(V, \Ind_H^G(W)).
  \end{equation*}
  Here the first isomorphism is by Tensor-Hom-adjunction, the second 
  equality uses that $V$ is a smooth representation of $G$.
\end{proof}
\end{thm}

\begin{prop}\label{prop:InductionOnTower}
  Let $I$ be a closed subgroup of $H$. There is a natural isomorphism $\Ind_H^G
  \circ \Ind_I^H \xto{\sim} \Ind_I^G$. 
  The same statement is true for compact and algebraic induction.
  \begin{proof}
    Trivially, $\Res_I^G = \Res_I^H \circ \Res_H^G$. The claim follows as the functors
    in question are adjoints to the left or the right hand side of this equation,
    thereby isomorphic.
  \end{proof}
\end{prop}

\begin{lem}\label{lem:FrobRecUnitsAreInjective}
  Let $H$ be a closed subgroup of $G$. The functor $\Res_H^G$ is faithful.
  Equivalently, the unit $\id_{\RepSm G} \to \Ind_H^G \circ \Res_H^G$ of 
  the adjunction $\Res_H^G \ladj \Ind_H^G$ is injective on components.
  If $H$ is additionally assumed to be an open subgroup of $G$,
  The functor $\cInd_H^G$ is faithful. 
  Equivalently, the components of the unit
    $\id_{\RepSm H} \to \Res_H^G \circ \cInd_H^G$
  coming from the adjunction $\cInd_H^G \ladj \Res_H^G$ are injective. 
\begin{proof}
  Faithfulness of $\Res_H^G$ is clear. For faithfulness of $\cInd_H^G$, note that 
  the unit of the adjunction $\cInd_H^G \ladj \Res_H^G$ is given 
  on components $(\pi, V) \in \RepSm H$ by the map 
  $v \mapsto \phi_v$, where $\phi_v \in \cInd_H^G(V)$ is defined as
  \begin{equation*}
    \phi_v\colon G \to V, \quad g \mapsto \begin{cases}
      \pi(g)v &\text{ if } g \in H,\\
      0       &\text{ otherwise.}
    \end{cases}
  \end{equation*}
  The resulting morphism $V \to \Res_H^G ( \cInd_H^G(V))$ is injective. Now all
  claims follow since faithfulness of the left-adjoint is 
  equivalent to the unit being a monomorphism on components, cf. \cite[Lemma
  4.5.13]{riehl2017category}.
\end{proof}
\end{lem}

\begin{rmk} 
  For the sake of completeness, we note that the unit of the adjunction 
  $\Res_H^G \ladj \Ind_H^G$ is given on components $(\pi, V) \in \RepSm G$ by
  \begin{equation*}
    V \to \Ind_H^G( \Res_H^G(V)), \quad v \mapsto \psi_v; \quad
    \text{where}\quad \psi_v(g) = \pi(g) v.
  \end{equation*}
\end{rmk}

The following Lemma is an instance of base-change.
\begin{lem}\label{lem:BaseChangeForResInd}
  Let $H$ and $N$ be closed subgroups of $G$ satisfying $NH = HN$. Let $(\pi,
  V)$ be a smooth representation of $H$.
  Then there is a natural split monomorphism of $N$-representations 
  \begin{equation*}
    \Res_N^{HN} (\Ind_H^{HN} \pi) \to \Ind_{H \cap N}^N(\Res_{H\cap N}^H \pi).
  \end{equation*}
  If $N$ is open in $HN$, this map is an isomorphism.
\begin{proof}
  One quickly checks that the map
  \begin{equation*}
    \Res^{HN}_{N}(\algInd_{H}^{HN} V) \to \algInd_{H\cap N}^{N}(\Res_{H \cap N}^H V)
  \end{equation*}
  given by sending $\phi\colon HN \to V$ to its restriction $\phi|_N$, is an
  isomorphism. Now the claim follows by taking smooth parts and applying Lemma
  \ref{lem:SmoothRestriction}.
\end{proof}
\end{lem}
\begin{rmk} 
  There are multiple ways to construct the map above.
  Applying $\Ind_H^{HN}(-)$ to the unit of the adjunction $\Res_{H\cap N}^H \ladj
  \Ind_{H\cap N}^H$ yields for any $\pi \in \RepSm H$ a natural morphism
  \begin{equation*}
    \Ind_H^{HN}(\pi) \to \Ind_{H\cap N}^H \Res_{H \cap N}^H (\pi) \cong
    \Ind_{N}^{HN} \Ind_{H \cap N}^N \Res_{H \cap N}^H (\pi),
  \end{equation*}
  which is equivalent to a map 
  \begin{equation*}
    \Res_N^{HN} (\Ind_H^{HN} \pi) \to \Ind_{H \cap N}^N(\Res_{H\cap N}^H \pi).
  \end{equation*}
  This gives the same map as in the proof. The dual construction (starting with the
  co-unit) also yields the same map.
\end{rmk}


\end{document}
