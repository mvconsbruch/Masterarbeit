\subsubsection{General Theory} % (fold)
\label{ssub:General Theory}

We recall a few basic notions from the theory of algebraic groups. Milne's book
\cite{milne2017reductive} is the main reference. 

\begin{defi}[Borel Subgroup]\label{def:BorelSubgrp}
  A smooth connected solvable sub-groupscheme $B \subset G$ is called Borel if it 
  is maximal among the smooth connected solvable sub-groupschemes of $G$. 
  A pair of sub-groupschemes $(B,T)$ is called a Borel pair if $B \subset G$ is 
  Borel and $T$ is a maximal Torus of $G$ contained in $B$.
\end{defi}
For the remainder of this subsection, fix a Borel pair $(B,T)$. By an application
of Chevalley's theorem, the quotient $G/B$ (in the sense of \cite{milne2017algebraic})
exists. This quotient is the unique scheme (up to unique isomorphism) representing
the \'etale sheafification of the presheaf $T \mapsto G(T)/B(T)$. In particular,
we have $(G/B)(k) = G(k)/B(k)$. Quotients of the form $G/H$ exist generally for
smooth algebraic subgroups $H \subset G$. 
\begin{defi}[Weyl Group]\label{def:WeylGroup}
  The Weyl group associated to $T \subset G$ is defined as $\Weyl(G,T) =
  \Norm_G(T)/T$, where $\Norm_G(T)$ is the normalizer of $T$ in $G$. 
  If $T'$ is another maximal torus inside $G$, $T$ and $T'$ are conjugate,
  and there is a canonical isomorphism $\Weyl(G,T) \cong \Weyl(G,T')$. 
  This allows us to omit $T$ from notation, and talk about the (canonical)
  Weyl group.
\end{defi}

\begin{thm}[Bruhat Decomposition]\label{thm:BruhatDecomposition}
  There is an isomorphism of smooth algebraic varieties
  \begin{equation*}
    G / B = \bigcup_{w \in \Weyl(G,T)} B w B/B.
  \end{equation*}
  In particular, there is a natural bijection
  \begin{equation*}
    G \backslash ( G / B \times G / B ) \to B \backslash G / B.
  \end{equation*}
\begin{proof}
  See \cite[Section 20.h]{milne2017algebraic}. The bijection is given (on
  rational points) by $(gB, g'B) \mapsto  B g^{-1} g' B$. 
\end{proof}
\end{thm}




% subsection General Theory (end)

Given a Borel sub-groupscheme $B$ of $G$, the quotient $G/B$ (to be taken as 
the fppf-sheafification \red{if we only care about smooth, etale should
suffice} of the presheaf $S \mapsto G(S)/B(S)$ for 
suitable $k$-schemes $S$) exists and is representable by a complete algebraic
variety \red{todo: See MilneBook Ch. 18 and ConradBook discussion at beginning
of Ch. 1}. Following notation in \cite{delignelusztig1976}, we write 
$X = X_G = G/B$. The $k$-valued points of $X$ are in bijection with the 
Borel sub-groupschemes of $G$. Indeed, it is easy to see that for any $g \in G(k)$,
the subgroup $g B g^{-1}$ of $G$ is Borel. The induced action
\begin{equation*}
  G(k) \lacts \{B \subset G \mid B \text{ Borel}\}, \quad g.B = gBg^{-1}
\end{equation*}
is transitive by \red{MilneBook, 18.17} and any Borel sub-groupscheme $B$ is
stabilized by its $k$-rational points. 

\begin{rmk} 
  In the case where $G = \GL_n$, the Borel subgroups are given by those subgroups
  that are diagonalizable. In particular, a valid choice for $B$
  is the subgroup of invertible upper triangular matrices, which contains the 
  Torus $T$ of diagonal matrices. In this case the quotient $\GL_n/B$ has
  another description; it is isomorphic to the complete flag variety. 
  \red{expand, cf. MilneBook, p 149}.
\end{rmk}

\begin{defi}[Weyl Group]\label{def:WeylGrp}
  If $T$ is a maximal Torus in $G$, the group $\Weyl(G,T) = \Norm_G(T)/T$ does - up to
  canonical isomorphism - only depend on $G$. It is therefore justified to omit
  $T$ from notation, writing $\Weyl(G) = \Weyl(G,T)$. We call $\Weyl(G)$ the
  Weyl group of $G$.
  \red{I think, technically speaking, the Weyl group is also a group scheme?}
\end{defi}

For the remainder of this section we fix a Borel pair $(B,T)$. 

\begin{prop}\label{prop:WeylGroupActsOnBorel}
  There is a simply transitive action $\Weyl(G) \lacts X$. 
\begin{proof}
  This is essentially the Bruhat decomposition, cf. \red{Milne, Prop. 18.59}
\end{proof}
\end{prop}

\begin{defi}[Relative Position]\label{def:BorelsInRelPos}
  Let $w \in \Weyl(G, T)$. 
  We say that two Borel subgroups $B', B''$ are in relative position $w$ if 
  $w.B' = B''$. 
\end{defi}

We denote by $O(w)(k) \subset X(k) \times_{\spec(k)} X$ the subset
of $X(k) \times X(k)$ given by pairs $(B', B'')$ in relative position $w$. 
More generally, $O(w)$ is the locally closed subscheme of $X \times_{\spec \bar
\FF_q} X$ given by the orbit of the element $(B, w. B) \in (X \times_k X)(k)$. 

For $w \in \Weyl(G)$, we denote by $X(w)$ the intersection of $O(w)$ with the graph
of the Frobenius $\Gamma_F \subset X \times_k X$. The $k$-rational points of $X(w)$ 
are given by those Borel subgroups $B' \subset G$ that are in relative 
position $w$ to their Frobenius twist $F(B')$. It can be shown that 
$X(w)$ is smooth of dimension $l(w)$, where $l(w)$ denotes the Coxeter-length
of $w$. In addition, $X(w)$ admits an action of $G^F$. 

\textbf{Example.}
Again, we fix the Borel subgroup $B \subset \GL_n$ of upper triangular matrices
and $T \subset B$ the torus of diagonal matrices. These choices provide an isomorphism
$\Weyl(\GL_n,T) \cong \Sigma_n$, the symmetric group with $n$ elements.
Let $w = (1\ 2\ \dots\ n) \in \Sigma_n$ be the $n$-cycle. Then $X(w)$ is an
$n-1$-dimensional
smooth projective subvariety of $X$. 


