%! TeX root: ../main.tex
\documentclass[../main.tex]{subfiles}

\begin{document}

\section{Formal Modules}
This section will serve as an introduction to formal groups and 
formal modules. Formal groups (or rather, formal group laws) were first
introduced by \textsc{Salomon Bochner} in 1946 as a natural means of studying Lie
Groups over fields of characteristic $0$, cf. \cite{Bochner1946FGrps}. 
The study of formal groups later became interesting for its own right, 
with pioneering works of Lazard \cite{Lazard1955FGrps}. \todo{blabla}

\subsection{Basic Notions} % (fold)
\label{sub:Basic Notions}
As promised in the introduction, we begin by defining {formal group
laws}.
For now, let $R$ be any ring.
\begin{defi}[Formal Group Laws of arbitrary dimension]
  A (commutative) formal group law  of dimension
  $n$ over $R$ is a tuple of
  power series $F = (F_1, \dots, F_n)$ with $$F_i(X_1, \dots, X_n,Y_1,
  \dots, Y_n) \in R\llbr X_1, \dots,
  X_n, Y_1, \dots, Y_n\rrbr, \quad 1 \leq i \leq n$$
  such that $F_i(\mathbf X, \mathbf Y) \equiv X_i + Y_i $ modulo degree $\geq 2$
  and the following equalities are satisfied:
  \begin{enumerate}
    \item $F(F (\mathbf X, \mathbf Y), \mathbf Z) = 
      F(\mathbf X, F(\mathbf Y, \mathbf Z))$.
    \item $F( \mathbf X, \mathbf 0) = \mathbf X$.
    \item $F( \mathbf X, \mathbf Y) = F(\mathbf Y, \mathbf X).$
  \end{enumerate}
  Here, and in the sequel, we abbreviate $\mathbf X =
  (X_1, \dots, X_n)$, et cetera.
  Given a formal group $F$ of dimension $n$ and a formal group law
  $G$ of dimension $m$,
  a morphism $F \to G$ is a $m$-tuple $f = (f_1, \dots, f_m)$ 
  of power series $f_i \in R\llbr X_1, \dots, X_n \rrbr$ such that $\bff(0)=0$ and
  \begin{equation*}
    G(f( \mathbf X), f( \mathbf Y) ) = f(F( \mathbf X, \mathbf Y)).
  \end{equation*}
  For any $n$-dimensional formal module $F$, the identity is given by 
  the morphism $\id_{F}$ with components $\id_{F,i}( \mathbf X) = X_i$. 
  Composition of morphisms is given by composition of tuples of power-series.
  This yields the category of formal group laws of arbitrary dimension over $R$,
  which we denote by $\FGLArbOver R$. We will mostly be concerned with the full
  subcategory of one-dimensional formal groups, which we denote by $\FGLOver R$. 
\end{defi}

\begin{lem}\label{lem:FGLAdditive}
  \begin{enumerate}
    \item Attached to any $F \in \FGLArbOver R$ of dimension $n$, there exists
      a unique power series 
      $$[-1]_F(X_1, \dots, X_n) \in R\llbr X_1, \dots, X_n\rrbr$$
      satisfying $F(\mathbf X, [-1]_F(\mathbf X)) = 0$. This is automatically
      an endomorphism of $F$. 
    \item The set $\Hom_{\FGLArbOver R} (F,G)$
      is an abelian group with addition $f+g = G(f,g)$. 
      In particular, $\FGLArbOver R$ is pre-additive (cf.
      \cite[\href{https://stacks.math.columbia.edu/tag/00ZY}{Tag
      00ZY}]{stacks-project}).     
    \item Furthermore, $\FGLArbOver R$ admits finite products. Thereby it is 
      an additive category (cf.
      \cite[\href{https://stacks.math.columbia.edu/tag/0104}{Tag
      0104}]{stacks-project}).
      The unique final and initial object of $\FGLArbOver R$ is the unique
      $0$-dimensional formal $A$-module law.
    \item In particular $\End_{\FGLArbOver R}(F)$ is a (possibly non-commutative)
      ring.
  \end{enumerate}
  \begin{proof}
    The first statement is an application of the formal implicit function theorem,
    cf. \cite[Theorem A.4.7]{hazewinkel1978formal}. The remaining
    statements are a matter of direct computation.
  \end{proof}
\end{lem}

\begin{xpl} Let us introduce the following two formal group laws.
  \begin{itemize}
    \item \textit{The additive formal group law}. Over any ring $R$, we write 
      $\Ghat_a$ for the formal group law with addition given by 
      $\Ghat_a(X,Y) = X + Y$. 
    \item \textit{The multiplicative formal group law}. We write $\Ghat_m$ for
      the formal group law associated with the with $\Ghat_m(X,Y) = X + Y +
      XY$. Note that $\Ghat_m(X,Y) = (X+1)(Y+1) - 1$. 
  \end{itemize}
\end{xpl}

\begin{defi}[Lie-algebra of formal group law]
Let $\Lie\colon \FGLArbOver R \to \Ab$ be the functor taking an $n$-dimensional 
formal group law $F$ to the $R$-module
\begin{equation*}
  \Lie(\bF) = \Hom_{\Mod R} \left( \frac{(X_1, \dots, X_n)}{(X_1, \dots,
  X_n)^2}, R \right) 
\end{equation*}
Given an $m$-dimensional group law $\bG$ and a morphism $\bff\colon \bF \to
\bG$, $\Lie(\bff)$ is the induced morphism
\begin{equation*}
  \Lie(\bF) \to \Lie(\bG), \quad \psi \mapsto \left( S_j \mapsto \psi
  (\overline{f_j})\right) \in 
  \Hom_\Mod R\left(\frac{(X_1, \dots, X_n)}{(X_1, \dots, X_n)^2}, R \right),
\end{equation*}
where $\overline{ f_j}$ is the reduction of $f_j$ mod $(\bX)^2$. 
\end{defi}
We have a canonical basis on both sides, and writing $\Lie(\bF) = R^n$, 
$\Lie(\bG) \cong R^m$, the induced map $\Lie(\bff) \colon R^n \to R^m$ is
given by multiplication with the matrix 
\begin{equation*}
  \left( \frac {\partial f_i}{\partial X_j} (0) \right)_{\substack{1 \leq i \leq m \\ 1 \leq j \leq n}}.
\end{equation*}

Now let $A$ be any other ring. We define formal $A$-module laws. Naively, we
would like to define these objects as formal group laws $F$
with $A$-module structure, i.e. a morphism of rings $[\cdot]_F\colon A \to
\End_\FGLArbOver R(F)$. However, given a one-dimensional group law $F \in
\FGLOver R$, the condition that $F(X,Y) \equiv X+Y$ modulo degree $\geq 2$
enforces that the induced map $\End(F) \xto{\Lie} \End(R)$ is a morphism of rings. 
If we are given $[-]_F\colon  A \to \End_{\FGLOver R}(F)$, this $A$-module structure on $F$ 
yields an $A$-module structure on $R$, given by the composition
\begin{equation*}
  A \xto{[\cdot]_F} \End(F) \xto{\Lie} \End(R), \quad a \mapsto \Lie([a]_F).
\end{equation*}
This is a morphism of rings, and we obtain an $A$-algebra structure on $R$. 
This motivates the following definition.
\begin{defi}[Formal $A$-Module Law of arbitrary dimension]\label{def:formalmodulelaw}
  Let $R$ be an $A$-algebra with structure morphism $j\colon  A \to R$. A formal
  $A$-module law over $R$ of dimension $n$ is given by the data of 
  a formal $n$-dimensional group law $F$ over $R$ and a morphism of rings
  \begin{equation*}
    A \to \End_{\FGLArbOver R} (F), \quad a \mapsto ([a]_{F,i}
    )_{1 \leq i \leq n} \in (R \llbr X_1, \dots, X_n \rrbr )^n
  \end{equation*}
  such that $[a]_{F,i}(\mathbf X) \equiv j(a) X_i$ modulo terms of degree 
  $\geq 2$. Morphisms between formal $A$-modules of arbitrary dimension are 
  morphisms of formal groups respecting the $A$-module structure. 
  The resulting category is denoted $\FMLArbOver A R$. The full
  subcategory of one-dimensional formal $A$-module laws over $R$ is denoted 
  by $\FMLOver A R$.
\end{defi}

Note that $\FGLOver R \cong \FMLOver \Z R$.
At the slight cost of precision, we usually do not explicitely mention the
$A$-structure when referring to formal module laws, simply writing $F \in
\FMLOver AR$ for example. 

Given an additional $A$-algebra $R'$, a morphsim
$i\colon R \to R'$ yields a functor 
\begin{equation*}
    \FMLArbOver AR \to \FMLArbOver A{R'}, \quad
    F \mapsto F \otimes_R R',
\end{equation*}
where $F \otimes_R R'$ denotes the formal $A$-module law obtained by applying $i$
to the coefficients of the formal power series representing the $A$-module structure of $F$. 
This turns the assignment $R \mapsto \FMLArbOver AR$ into a 
functor $\Alg A \to \Set$. The corresponding presheaf on $\AffSchOver {A}^\opp$ is  separated for the Zariski topology, 
but it fails to be a sheaf in general: the Lie algebra of any
formal group law over $R$ has to be free, and this condition cannot be 
ensured Zariski-locally on $\spec R$. It may therefore be 
advisable to consider the sheafification $R \mapsto \FMLArbOver AR^\#$ instead, but in our applications
$R$ will almost always be a local $A$-algebra, so considerations
of this type will not play a large role.

Every $n$-dimensional formal module law $F \in \FMLArbOver A R$
yields a functor
\begin{equation}\label{eq:fmnilpfunc}
  \Alg R \to \Mod A, \quad S \mapsto \Nil(S)^n,
\end{equation}
where $\Nil(S)^n$, the set of $n$-tuples of nilpotent elements of $S$, is
equipped with addition and scalars given by 
\begin{equation*}
  s_1 + s_2 = F(s_1, s_2) \in \Nil(S)^n, \quad a s = [a]_F(s) \in \Nil(S)^n.
\end{equation*}
This construction yields a functor 
\begin{equation}\label{eq:formfunc}
  \FMLOver A R \to \Fun(\Alg R, \Mod A),
\end{equation}
where $\Fun$ denotes the functor category.

Passing from discrete $R$-algebras to admissible $R$-algebras (cf. Definition
\ref{def:admring}), this construction extends naturally to a functor 
\begin{equation*}
  \FMLOver AR \to \Fun (\Adm R, \Mod A), \quad F \mapsto \Spf R\llbr \bT \rrbr,
\end{equation*}
where we equip $\Spf R \llbr \bT \rrbr$ with the structure of an $A$-module object
using the endomorphisms coming from $F$. 



Following this line of thought leads naturally to the definition of
formal modules. 

\begin{defi}[Formal Groups and Formal Modules.]
  Given an $A$-scheme $X$, we define the category
  $\FMArbOver A X$ as follows. 
  Objects are $A$-module objects $\cF$ in the category
  of formal schemes over $X$, having the property that 
  there is a cover of $X$ by Zariski-open affine subsets $U_i = \spec (R_i)$
  such that $\cF \times_X U_i$ is isomorphic to $\Spf R_i\llbr X_1, \dots,
  X_n\rrbr$ and the induced $A$-module structure on $\spf R_i\llbr X_1, \dots, X_n\rrbr$
  yields a formal $A$-module law on $R_i$. Given $\cF, \cG \in \FMLArbOver AX$,
  a morphism $\phi\colon  \cF \to \cG$ is the same as a morphism of $A$-module objects 
  in the category of formal schemes over $X$.
  Again, we denote the full subcategory of one-dimensional formal $A$-modules
  over $X$ by $\FMOver AX$. 
\end{defi}

% \begin{rmk} 
%   Formal schemes (over a base an $A$-scheme $X$, say) locally isomorphic to 
%   $\spf \cO_X(U)\llbr \bT \rrbr$ are sometimes called Formal
%   Lie Varieties \todo{reference}. Equivalently to the definition above, we could
%   have defined formal $A$-modules as $A$-module objects in the category of
%   Formal Lie Varieties, such that the $A$-module structure
%   on the tangent space at the identity agr\textbf{}ees with the usual one.
% \end{rmk}

\begin{defi}[Coordinate]
  Let $\cF$ be a formal $A$-module over $X$. The choice of a cover $\sqcup_{i
  \in I} \spec(R_i) \to X$ together with maps $\spf(R_i \llbr T \rrbr) \to \cF$
  furnishing for each $i$ a cartesian square
  \begin{equation*}
  \begin{tikzcd}[ampersand replacement=\&]
  	{\spf(R_i \llbr T \rrbr)} \& \cF \\
  	{\spec(R_i)} \& X
  	\arrow[from=1-1, to=1-2]
  	\arrow[from=1-1, to=2-1]
  	\arrow[from=1-2, to=2-2]
  	\arrow[from=2-1, to=2-2]
  \end{tikzcd}
  \end{equation*}
  will be referred to as a coordinate for $\cF$. 
\end{defi}

Of course there is a functor 
\begin{equation*}
  \FGG\colon  \FMLArbOver AR \to \FMArbOver AR,
\end{equation*}
essentially forgetting the choice of module law. Just as in the case of formal 
module laws, a morphism $R \to R'$ yields a functor 
\begin{equation*} 
  \FMOver AR \to \FMOver A{R'}, \quad \cF \mapsto \cF \otimes_R R'.
\end{equation*}

\begin{defi}[Lie functor]
  The functor $\Lie$ descends to a functor 
  \begin{equation*}
    \Lie\colon  \FMArbOver A X \to \QCoh {\cO_X}, 
  \end{equation*}
  given by locally describing a formal $A$-module $\cF$ via formal group laws and gluing the local data. Alternatively, it arises from sending 
  sending a formal $A$-module $\cF$ to $(\cI / \cI^2)^\vee$, where $\cI$ is the
  ideal associated to the closed immersion $[0]_\cF\colon  X \to \cF$. 
\end{defi}

\begin{lem}\label{lem:IsosCheckOnLie}
  A map $\phi\colon  \cF \to \cG$ of formal $A$-modules (of arbitrary dimension) over $X$ is an isomorphism if and only if the induced morphism of Lie algebras $\Lie(\phi)\colon  \Lie(\cF) \to \Lie(\cG)$ is an isomorphism.
\begin{proof}
  Choosing coordinates, we may assume that $\cF$ and $\cG$ come from formal $A$-module
  laws $F$ and $G$. Now the claim is an easy consequence of the formal implicit
  function theorem.
\end{proof}
\end{lem}

\begin{xpl}
  % For any scheme $X$, the additive group law $\Ghat_a$ 

  Consider the formal group law $\Ghat_m$ over $\Z_p$. It 
  admits the structure of a $\Z_p$-module law as follows. 
  As formal group, $\Ghat_m$ is isomorphic to the assignment
  \begin{equation*}
    \Adm {\Z_p} \to \Ab, \quad S \mapsto 1 + S^\cici \subset S^\times.
  \end{equation*}
  The subgroup $1 + S^\cici$ naturally carries the structure of a $\Z_p$-module.
  Indeed, for $k \in \N$, we have
  \begin{equation*}
    (1+s)^{p^k} = 1 + p^ks + \binom{p^k}2 s^2 + \dots + s^{p^k},
  \end{equation*}
  and given $s \in S^\cici$, this is of the form $1+ o(1)$ as $k$ gets large. 
  In particular, if $x = a_0 + a_1 p + a_2p^2 + \dots \in \Z_p$, expressions of  the form
  \begin{equation*}
    (1+s)^x = \prod_{i = 1}^\infty (1+s)^{a_k p^k}
  \end{equation*}
  make sense by Lemma \ref{lem:infiniteseriesandproducts}. This gives
  $\Ghat_{m,\Z_p}$ the structure of a formal $\Z_p$-module law, with
  $[x]_{\Ghat_m}(T) = (1+T)^x -1$. 
  This is the simplest example of a whole family of formal modules constructed by 
  Lubin and Tate. In Section \ref{sec:Local Class Field Theory} we explain
  applications of these formal modules to local class field theory.
\end{xpl}

\begin{defi}[Formal Module associated to $R$-module]
  \label{def:additiveformalmoduleassociatedtomodule}
  Suppose that $M$ is a finite projective $R$-module. Then we write
  $\Ghat_a \otimes M$ for the additive formal $A$-module associated to $M$ over $R$.
  As a formal scheme, this formal module is given by
  \begin{equation*}
    \Ghat_a \otimes M \cong \spf R \llbr M^\vee \rrbr,
  \end{equation*}
  where $R \llbr M^\vee \rrbr$ denotes the completion of $\Sym_R(M^\vee)$ with respect
  to the ideal generated by $M^\vee$. The (formal) $A$-module structure is the
  canonical additive one. 
  Note that $\Lie(\Ghat_a \otimes M) = M$ by design. 
  More generally, if $X$ is a \red{quasi-compact and quasi-separated} $A$-scheme
  and $\cM$ is a finite locally free quasi-coherent 
  $\cO_X$-module, this construction yields a formal $A$-module
  $\Ghat_a \otimes \cM$ over $X$.
\end{defi}
\begin{rmk} 
  If $R\to R'$ is a ring morphism that turns $R'$ into a (say) finite free
$R$ -algebra, the above definition overloads the expression $\Ghat_a \otimes_R R'$.
  In order to disambiguate, we usually denote the additive formal $A$-module
  over $R'$ by $\Ghat_{a, R'}$. 
\end{rmk}

% subsection Basic Notions (end)

\subsection{Invariant Differentials and Logarithms} % (fold)
\label{sub:Logarithms}
Again, $A$ is a complete discrete valuation ring with uniformizing parameter 
$\varpi$ and finite residue field $k = A/\varpi A$. We write $q$ for the cardinality of 
$k$ and $E$ for the field of fractions of $A$. Let $R$ be a local,
admissible $A$-algebra with structure map $i\colon  A \to R$.

We review results from Sections
2 and 3 of \cite{hopkins1994equivariant}. 
Suppose that $\bF = (F_1, \dots, F_n)$ is a $n$-dimensional formal $A$-module
law over $R$. We abbreviate $\bX = (X_1, \dots, X_n)$, $\bY = (Y_1,
\dots, Y_n)$, etc.

\begin{defi}[Invariant Differentials]
  The module $\omega(\bF)$ of invariant differentials is the submodule of the
  $R$-module of differentials
  \begin{equation*}
    \Omega_{R\llbr T_1, \dots, T_n \rrbr/R} \cong \bigoplus_{i=1}^n R\llbr T_1, \dots, T_n
    \rrbr \dc T_i,
  \end{equation*}
  consisting of those $\omega \in \omega(\bF)$ satisfying
  \begin{equation}\label{eq:diffcond}
    \omega(\bF(\bX,\bY)) = \omega(\bX) + \omega(\bY)\quad \text{and} \quad
    \omega([a]_\bF(\bX)) = a\omega(\bX).
  \end{equation} 
  for all $a \in A$. 
\end{defi}

It is possible to explicitly calculate a basis for the $R$-module
$\omega(\bF)$, which we now explain. Let 
$$A(\bX, \bY) \in \Mat_{n \times n} (R\llbr \bX, \bY \rrbr)$$ 
denote the matrix $\left((\partial/\partial X_j)F_i (\bX,\bY)\right)_{i,j}$,
the derivative of $\bF(\bX,\bY)$ with respect to $\bX$. Set 
$B(\bY) = A(0,\bY)$. Then $B$ is a unit in $\Mat_{n \times n} R\llbr \bY \rrbr$; 
and we write $(C_{ij}(\bY))$ for the components of 
$B(\bY)^{-1}$. We now construct 
$$\omega_{i} \coloneqq \sum_{j=1}^n C_{ij}(\bX) \dc X_j \in \Omega_{R\llbr \bX \rrbr/R}$$ 
for $1 \leq i \leq n$. By definition we have 
\begin{equation}\label{eq:coeffofcanonicaldiff}
  C_{ij}(0) = \begin{cases}
    1 &\text{ if }i = j,\\
    0 &\text{ otherwise.}
  \end{cases}
\end{equation}
Checking that $\omega_{i}$ is an invariant differential is a matter of 
applying the chain rule. 
\begin{prop}
    The $R$-module $\omega(\bF)$ is free of rank $n$ generated by invariant differentials
    $\omega_{1}, \omega_{2}, \dots, \omega_{n}$.
\begin{proof}
  This is \cite[Proposition 1.1]{honda1970formalgroups}. 
\end{proof}
\end{prop}
\begin{xpl}
  The invariant differentials for $\Ghat_a$ are spanned by the form $\dc X$. 
  The invariant differentials for $\Ghat_m$ are spanned by the form 
  $\omega_1(X) = \frac 1{1+X} \dc X$.
\end{xpl}
By the Proposition above and Equation \eqref{eq:coeffofcanonicaldiff}, we may
define a pairing
\begin{equation*}
  \omega(\bF) \times \Lie(\bF) \to R, \quad \langle X_i, \omega_j \rangle =
  \begin{cases}
    1 &\text{ if } i = j,\\
    0 &\text{ otherwise.}
  \end{cases}
\end{equation*}
This pairing is independent of the parameterization of $\bF$. In particular, it
descends to a pairing defined for formal modules $\cF \in \FMArbOver A R$, and
we have a natural isomorphism $\omega(\cF) \cong \Hom_R(R, \Lie(\cF))$.

Let $\Ghat_a$ be the additive formal $A$-module over $R$. There is a map
\begin{equation} \label{eq:functorinvdifftohom}
  \dc_\bF \colon  \Hom_{\FMLOver AR} (\bF, \Ghat_{a,R}) \to \omega(\bF), \quad f \mapsto \dc f(\bX)
\end{equation}
which is a map of $R$-modules if we equip the left hand side with the $R$-module
structure coming from the natural action of $R \subset \End(\Ghat_a)$. 
\begin{prop}\label{prop:loginvdiff}
  \begin{enumerate}
    \item If $R$ is a flat $A$-algebra, the map $\dc_F$ is injective.
    \item If $R$ is a $E$-algebra, the map $\dc_F$ is an isomorphism.
  \end{enumerate}
\begin{proof}
  This is \cite[Proposition 3.2]{hopkins1994equivariant}.
\end{proof}
\end{prop}

Suppose now that $F \in \FMLArbOver A R$ is a formal module law of dimension $n$
over a flat $A$-algebra $R$. 
Let $\omega_1, \dots, \omega_n$ be the distinguished basis for $\omega(F)$
constructed above. 
By the previous proposition, there are unique power series 
$f_i(\bX) \in (R \otimes_A E)\llbr \bX \rrbr$ that furnish homomorphisms
$F \otimes (R \otimes_A E) \to \Ghat_{a,R\otimes_AE}$ of formal $A$-module laws
and satisfying
\begin{equation*}
  \dc_F f_i(\bX) = \omega_i(\bX) \in \omega(F).
\end{equation*}
\begin{defi}[Logarithm and Exponential]
  The induced morphism of formal group laws
  \begin{equation*}
    f = (f_1, \dots, f_n) \colon  F \otimes (R \otimes_A E) \to \Ghat_a^n \otimes_R (R \otimes E)
  \end{equation*}
  is called the logarithm attached to $F$, we write 
  $\log_F(\bX) \in ((R \otimes_A E)\llbr \bX \rrbr)^n$ for the corresponding collection
  of power series. The inverse of $\log_F(\bX)$ is called the exponential 
  attached to $F$, denoted $\exp_F(\bX)$. We have $\Lie(\log_F) = \Lie(\exp_F) = \id$,
  so $\log_F$ and $\exp_F$ are isomorphisms.
\end{defi}

\begin{xpl} \leavevmode
  \begin{itemize}
    \item The logarithm for the formal $\Z_p$-module law $\Ghat_m$ over 
      $\Z_p$ is given by the integral of $\frac{1}{1+T} \dc T$, which is 
      simply the usual logarithm
      \begin{equation*}
        \log_{\Ghat_m}(T) = - \sum_{i = 1}^\infty \frac{(-T)^{i}}i. 
      \end{equation*}
    \item If $F$ is the formal $\FF_q\llbr \varpi \rrbr$-module with $F(X,Y) = X+Y$,
      $[\zeta]_F(T) = \zeta T$ for $\zeta \in \FF_q$ and $[\varpi]_F(T) = T^q$, 
      the logarithm for $F$ needs to satisfy $\log_F(\zeta T) = \zeta \log_F(T)$.
      Hence, it is necessarily of the form
      \begin{equation*}
        \log_F(T) = \sum_{i=0}^\infty a_i T^{q^i}
      \end{equation*}
      with $a_0 = 1$ and the remaining coefficients uniquely determined by the 
      equation
      \begin{equation*}
         \log_F(T^q) = \log_F([\varpi]_F(T)) = \varpi \log_F(T).
      \end{equation*}
      This example foreshadows some of the results in \cref{sub:Hazewinkels
      FuncEq and the Standard Formal Module}.
  \end{itemize}
\end{xpl}

\begin{lem}
  Let $F$ and $G$ be formal $A$-module laws over $R$, with $\dim F = n$ and
  $\dim G = m$. 
  Let $\phi\colon  F \to G$ be a morphism. Then the diagram 
  \begin{equation*}
  \begin{tikzcd}[ampersand replacement=\&]
    {F \otimes(R \otimes_A E)} \& {\Ghat_{a, R\otimes_A E} \otimes (\Lie(F)}
    \otimes_A E) =
    \Ghat_{a, R\otimes_AE}^n \\
    {G \otimes(R \otimes_A E)} \& {\Ghat_{a, R\otimes_A E} \otimes (\Lie(G)} \otimes_A E) =
    \Ghat_{a, R\otimes_AE}^m
  	\arrow["{\log_F}", from=1-1, to=1-2]
  	\arrow["\phi"', from=1-1, to=2-1]
  	\arrow["{\Lie(\phi)}", from=1-2, to=2-2]
  	\arrow["{\log_G}", from=2-1, to=2-2]
  \end{tikzcd}
  \end{equation*}
  commutes. In particular, attached to any $\cF \in \FMArbOver AR$ comes a 
  natural morphism 
  $$\log_\cF\colon  \cF \otimes (R \otimes_A E) \to \Ghat_{a, R\otimes_A E} \otimes
  \left(\Lie(\cF) \otimes_R (R \otimes_A E)\right).$$
  \begin{proof}
    The square commutes because of the equality
    $$\Hom(\Ghat_{a, R\otimes_AE}^n, \Ghat_{a, R\otimes_AE}^m) = \Hom_{R \otimes_A E}((R\otimes_A E)^n, (R\otimes_A E)^m),$$
    and the fact that $\Lie(\log_G \circ \phi \circ \exp_F) = \Lie(\phi).$
  \end{proof}
\end{lem}

\begin{lem}\label{lem:KernelOfLog}
  Let $E$ be a local field with integers $\cO_E$ and a choice of uniformizer $\varpi \in \cO_E$, 
  and let $F$ be a Lubin-Tate $\cO_E$-module law corresponding to some 
  $f \in \cF_\varpi$, cf. Theorem \ref{thm:LTModLaw}. Let $S$ be an admissible 
  $\cO_E$-algebra and let $s \in S^\cici$ be an element such that the
  series $\log_{\cF}(s)$ 
  converges. Then we have $\log_F(s) = 0$ if and only if $[\varpi]^r_F(s) = 0$ for
  some $r \in \N$. 
\begin{proof}
  Up to canonical isomorphism, $F$ is a $\cO_E$-module law with $[\varpi]_F(T) =
  \varpi T + T^q$. Now one may check that 
  \begin{equation*}
    \log_F(T) = \lim_{r \to \infty} \frac{ [\varpi]_F^r(T) }{\varpi^r} = \prod_{i=1}^\infty 
    \frac{[\varpi]^i_F(T)}{\varpi [\varpi]^{i-1}_F(T)},
  \end{equation*}
  where convergence is taken coefficient-wise. After inserting
  $s \in S^\cici$, we see that the product vanishes if and only if $[\varpi]_F^r(s) = 0$
  for some $r \in \N$. 
\end{proof}
\end{lem}
% subsection Logarithms (end)

\subsection{Formal Modules over Extensions of the Residue Field} % (fold)
\label{sub:Formal DVR-Modules over Fields}
Let $R$ be a $\cO_E$-algebra, and let 
$\FF_q$ be the residue field of $\cO_E$. We have seen in the previous section
that if $R$ is a field extension of $E$,
then any morphism of formal group laws $f\colon  F \to G$ over $R$ is either $0$
or an isomorphism, which makes the theory of formal $\cO_E$-modules over $R$ rather
simple. This situation changes if $R$ is a field extension of $\FF_q$: there are
homomorphisms of formal group laws $f\colon F \to G$ corresponding to power
series $f(T) \in R \llbr T \rrbr$ with vanishing differential. The prototype of
such homomorphisms is the relative Frobenius homomorphism $f \colon F \to
F^{(q)}$ corresponding to the monomial $f(T) = T^q$. Here, $F^{(q)}$ is the
formal group law obtained by raising each coefficient of $F(X,Y) \in R\llbr X,
Y\rrbr$ to the $q$-th power.

We introduce the concept of height, which is in a sense an attempt to quantify
the disorder introduced by the Frobenius homomorphisms. This leads to
interesting invariants of formal $\cO_E$-modules over local $\cO_E$-algebras $R$.

\begin{defi}[Height of morphisms of group laws]
  Assume that $k$ is a field extension of $\Fqbar$ and $f\colon  F \to G$ is a
  morphism of 
  formal groups laws over $k$, given by a formal series $f(T) \in k\llbr T \rrbr$. 
  We define the height of $f$, denoted $\height(f)$, as follows.
  If $f = 0$, we say that $f$ has infinite height. 
  If $f \neq 0$, the height of $f$ is defined as the largest integer $h$ such that 
  $f = g(T^{q^h})$ for some power series $g(T) = c_1 T + c_2 T^2 + \dots \in
  R\llbr T \rrbr$ with $c_1 \neq 0$. 
\end{defi}

One readily checks that if $f\colon  \cF \to \cG$ is a morphism of formal groups over
a field extension $k$ of $\Fqbar$, the height of $f$ does not depend on the choices
of group laws on $\cF$ and $\cG$. This allows us to define the height of $f$. 

%\begin{defi}[Height function]
%  Let $R$ be a $\cO_E$-algebra, and let $\cF$ and $\cG$ be formal
%  $\cO_E$-modules over $R$. Let $f\colon  \cF \to \cG$ be a morphism of
%  formal $\cO_E$-modules.
%  For a primes ideal $x \in \abs X$, let $f_x$ denote the base change
%  of $f$ to the residue field of $x$. 
%  The height function attached to $f$ is the upper semicontinuous function.
%  \begin{equation} \label{eq:defheight}
%    \height(f) \colon  \abs X \to \Z_{\geq 0} \cup \{\infty\}, \quad x \mapsto 
%    \height(f_x).
%  \end{equation}
%\end{defi}

It is not hard to see that the height function is additive, that is, we have
\begin{equation*}
  \height(f \circ g) = \height(f) + \height(g).
\end{equation*}

\begin{defi}[Isogeny]
  A morphism $f\colon  \cF \to \cG$ of formal groups over a field $k$ is called
  an isogeny if $\ker(f)$ is a representable by a finite free $k$-scheme. 
\end{defi}

It is not difficult to verify the following result. 
\begin{lem}
  A morphism $f\colon  \cF \to \cG$ of formal groups over $k$
  is an isogeny if and only if the height $\height(f)$ is finite. 
\end{lem}

\begin{defi}[$\varpi$-divisible $A$-module]
  Let $R$ be a local $\cO_E$-algebra with maximal ideal containing the 
  image of $\varpi$.
  We say that a formal $\cO_E$-module $\cF$ over $R$ is $\varpi$-divisible if 
  $[\varpi]_\cF$ is an isogeny. In this case, we define the height of $\cF$
  as the height of the isogeny $[\varpi]_\cF$.
\end{defi}

The additivity of the height function quickly implies the following.

\begin{lem}\label{lem:NoNonzeroHomsBetweenDiffHeight}
  There are no non-zero homomorphisms between formal $\cO_E$-modules 
  over $R$ of different height.
\end{lem}

\begin{prop}\label{prop:ReductionOfHomsIsInjective}
  Let $F$ and $G$ be one-dimensional formal $\cO_E$-module laws over $R \in \cC$,
  and suppose that $F$ is $\varpi$-divisible.
  The $\cO_E$-module homomorphism
  \begin{equation*}
    \Hom_{\FMLOver {\cO_E}R} (F,G) \to \Hom_{\FMLOver{\cO_E}{\Fqbar}}(F \otimes \Fqbar
    , G \otimes \Fqbar),
  \end{equation*}
  given by sending a polynomial $f \in R\llbr T \rrbr$ to its reduction modulo
  $\fm_R$, is injective.
\begin{proof}
  This is \cite[Proposition 4.2]{hopkins1994equivariant}.
\end{proof}
\end{prop}

We close this subsection with a discussion about the structure of 
formal $\cO_E$-modules over an algebraic closure $\FF_q \inj \Fqbar$. 

\begin{prop}\label{prop:classificationofmodulelawsbyheight}
  Over $\Fqbar$, any two formal $\cO_E$-module laws of the same
  height are isomorphic.
\begin{proof}
  This is part of \cite[Proposition 1.7]{drinfel1974elliptic}.
\end{proof}
\end{prop}

In particular, any formal $\cO_E$-module of height $h$ is isomorphic to 
the normalized formal $\cO_E$-module $\XX$, defined as the
reduction of the standard formal $\cO_E$-module law $H$ of height $h$, cf. 
\cref{sub:Hazewinkels FuncEq and the Standard Formal Module}.
For now, it is only important that $[\varpi]_\XX(T) = T^{q^h}$.

\begin{lem}\label{lem:inverseqisog}
  Let $f\colon  F \to G$ be an isogeny of $\varpi$-divisible formal $\cO_E$-module laws
  over $\Fqbar$. Then there is an 
  integer $n \geq 0$ and an isogeny $g\colon  G \to F$ with 
  \begin{equation*}
    f \circ g = [\varpi^n]_\cG \quad \text{and} \quad g \circ f = [\varpi^n]_\cF.
  \end{equation*}
  \begin{proof}[Proof]
    As the height is additive, we necessarily have 
    $\height(F) = \height(G)$, thus by Lemma
    \ref{prop:classificationofmodulelawsbyheight}, we may assume that $F$ and
    $G$ are given by the normalized formal $\cO_E$-module
    $\XX$.
    Write $f(T) = g(T^{q^n})$ for some power series $h(T) = c_1 T + c_2T^2 +
    \dots$, where $c_1 \neq 0$ is a unit in $\XX$, and let $g(T) = h^{-1}(T)$ be the 
    formal inverse of $h$. Now $g$ is a morphism of formal $\cO_E$-module laws
    satisfying $f \circ g(T) = g\circ f(T) = T^{q^n}$. The claim follows.
  \end{proof}
\end{lem}


\begin{prop}\label{prop:EndomorphismsOfFormalModulesOverksep}
  Suppose that $F \in \FMLOver {\cO_E}{\Fqbar}$ is $\varpi$-divisible
  of height $h$. Then 
  $\End_{\FMLOver {\cO_E}{\Fqbar}}(F)$ is isomorphic to the maximal order $\cO_D$
  of the central division algebra $D$ over $E$ of rank $h^2$ and invariant
  $\frac 1h$.
\begin{proof}
  This is part of \cite[Proposition 1.7]{drinfel1974elliptic}.
\end{proof}
\end{prop}

We give a more explicit description of $\cO_D$ in the case that $F = \XX$. 
Then $\cO_D = \End(\XX)$ admits the following description. Let 
$E_h$ be the degree $h$ unramified extension of $\cO_E$, and denote
the residue field of $\cO_{E_h}$ with $\FF_{q^h}$. Let 
$\Phi\colon \cO_{E_h} \to \cO_{E_h}$ denote the lift of the 
$q$-th power frobenius on $\FF_{q^h}$. Then 
\begin{equation}\label{eq:DescripitionOfEnds}
 \cO_D \cong \frac{\cO_{E_h}\{\Pi\}}{(\Pi^h = \varpi, \Pi a = \Phi(a) \Pi)}.
\end{equation}
Indeed, by \cref{prop:ReductionOfHomsIsInjective} and 
\cref{lem:MultByROUForStandardModule}, we see that $\cO_{E_h}$ embedds into
$\End(\XX)$. Furthermore, as $\XX$ is defined over $\FF_q$, the monomial
$\Pi(T) = T^q$ furnishes an endomorphism of $\XX$, and one readily checks that
$\Pi a = \Phi(a) \Pi$. The claim follows as the right-hand side
of \eqref{eq:DescripitionOfEnds} is of rank $h^2$ over $\cO_{E}$. 
In particular, the isomorphism \eqref{eq:DescripitionOfEnds} yields a 
reduction map $\cO_D \to \FF_{q^h}$ by sending $a \in \cO_{E_h}$ to
its residue modulo $\varpi$ and $\Pi$ to $0$. 


% subsubsection Formal DVR-Modules over Fields of Characteristic 0 (end)

\subsection{Hazewinkel's Functional Equation Lemma and the Standard Formal Module Law} % (fold)
\label{sub:Hazewinkels FuncEq and the Standard Formal Module}
Let $R$ be a flat $\cO_E$-algebra. By the results of \cref{sub:Logarithms}, the
structure of a formal 
$\cO_E$-module law $F$ over $R$ is uniquely determined by its logarithm $\log_F
\in R [\tfrac 1\varpi] \llbr T \rrbr$. Indeed, we find
\begin{equation*}
  F(X,Y) = \exp_F(\log_F(X)+\log_F(Y)), \quad [a]_F(X) = \exp_F(a \log_F(X)).
\end{equation*}
It is therefore natural to wonder if it is possible to reverse this process, that
is, if it is possible to classify the power series arising as logarithms of 
formal $\cO_E$-module laws.  Hazewinkel's 
functional equation lemma is a result in this direction. It provides a
sufficient condition on power series $f \in R[\tfrac 1 \varpi]\llbr T \rrbr$
that ensures that $f$ arises as the logarithm of some formal $\cO_E$-module law.

\begin{thm}[Hazewinkel's Functional Equation Lemma]\label{thm:HazewinkelIntegrality}
  Let $f \in R[\tfrac 1\varpi]\llbr T \rrbr$ be a 
  power series with $f'(0) \in R[\tfrac 1\varpi]^\times$. Let 
  $\sigma\colon  R[\tfrac 1\varpi] \to R[\tfrac 1\varpi]$ be an endomorphism
  of rings that restricts to an endomorphism of $R$ and suppose that 
  there are elements $s_1, s_2, \dots \in R[\tfrac 1\varpi]$ such that
  \begin{equation*}
    f(X) - \sum_{i=1}^\infty s_i (\sigma^i_* f)(X^{q^i}) \in R\llbr X \rrbr.
  \end{equation*}
  Let $\fa \subset R$ be an ideal and suppose that that the conditions
  \begin{equation*}
    \sigma(b) \equiv b^q \pmod \fa \text{ for all } b \in R \quad \text{and} \quad 
    \sigma^r(s_i) \fa \subset R \text{ for all } r,i \geq 1
  \end{equation*}
  are satsified.
  Then we have 
  \begin{equation*}
    F(X,Y) = f^{-1}(f(X) + f(Y)) \in R \llbr X,Y \rrbr,
  \end{equation*}
  where $f^{-1}$ is the inverse power series as in Lemma \ref{lem:IsosCheckOnLie}.
  Also, if $g(Z) \in R[\tfrac 1\varpi]\llbr Z \rrbr$ is another power series
  satisfying the same condition
  \begin{equation*}
    g(Z) - \sum_{i=1}^\infty s_i (\sigma^i_* f)(Z^{q^i}) \in R\llbr Z \rrbr,
  \end{equation*}
  then $f^{-1}(g(Z)) \in R\llbr Z \rrbr$. 
  Furthermore, if $\alpha(T) \in R\llbr T \rrbr$ and $\beta(T) \in R \llbr T \rrbr$, then
  \begin{equation} \label{eq:funceqlemcongruence}
    \alpha(T) \equiv \beta(T) \pmod {\fa^r} \iff f(\alpha(T)) \equiv f(\beta(T))
    \pmod {\fa^r}
  \end{equation}

  \begin{proof}
    A more general statement can be found in \cite[Section
    2]{hazewinkel1979funceqexp}. Proofs can be found in \cite[Sections 2 and
    10]{hazewinkel1978formal}.
  \end{proof}
\end{thm}
Note that by construction, $F(X,Y)$ as defined above yields a (commutative)
formal group law over $R$. 
If $\sigma$ restricts to the identity on $\cO_E \subseteq R$, then 
the second part of the Functional Equation Lemma implies that we even obtain
formal $\cO_E$-modules with $[b]_F(T) = f^{-1}(b f(T))$. Indeed, $bf(T)$
satisfies the same functional equation if $b \in \cO_E$. 

We now consider a special family of 
formal $\cO_E$-module laws, so called $\cO_E$-typical formal module laws.

\begin{defi}[$\cO_E$-typical Formal Module Law]\label{def:ATypicFormalModuleLaw}
  We say that a formal module law $F \in \FMLOver {\cO_E}{R}$ is 
  $\cO_E$-typical, if its logarithm is of the form
  \begin{equation*}
    \log_F(T) = \sum_{i=0}^\infty b_i X^{q^i}
  \end{equation*}
  for elements $b_0, b_1, \dots \in R \otimes_A E$.
\end{defi}

The family of $\cO_E$-typical formal module laws is quite exhaustive.
The result following is \cite[21.5.6]{hazewinkel1978formal}.

\begin{lem}\label{lem:AnyOModuleIsIsomorphicToAtypic}
  Any formal $\cO_E$-module law over $R \in \cC$ is isomorphic to an
  $\cO_E$-typical one.
\end{lem}

The Functional Equation Lemma allows us to classify $\cO_E$-typical module
laws. Let us write $\cO_E[\underline v] = \cO_E[v_1, v_2, \dots]$ and let
$f_v(T) \in \cO_E[\underline v][\tfrac 1\varpi]\llbr T \rrbr$ be the unique power series
satisfying the functional equation
\begin{equation*}
  f_v(T) = T + \sum_{i \geq 1} \frac{v_i}{\varpi} \sigma^i(f_v)(X^{q^i}),
\end{equation*}
where $\sigma$ is the endomorphism of $\cO_E[\underline v]$ obtained by
sending $v_j$ to $v_j^q$ for all $j$. By the Functional Equation Lemma
this gives rise to a $\cO_E$-module law $F_v$ over $\cO_E[\underline v]$.
By \cite[Definition 21.5.5 and Criterion 21.5.9]{hazewinkel1978formal},
we have the following.

\begin{lem}\label{lem:ClassificationOfAtypicalOnes}
  A formal $\cO_E$-module law $F$ over $R$ is $A$-typical if and only if there is
  a homomorphism $\cO_E[\underline v] \to R$ such that $F = F_v \otimes R$. 
\end{lem}

A special role will play the $A$-typical formal module law $H$
with logarithm given by the power series
\begin{equation*}
  f(T) = \sum_{i=1}^\infty \frac{T^{q^{in}}}{\varpi^i} \in E\llbr T \rrbr.
\end{equation*}
It satisfies the functional equation
\begin{equation*}
  f(T) = T + \frac 1\varpi f(T^{q^n})
\end{equation*}
and arises from $F_v$ via the homomorphism $\cO_E[\underline v] \to \cO_\br E$
sending $v_n$ to $1$ and $v_j$ to $0$ for $j \neq n$. 

The fact that $f^{-1}(X) = X - \frac 1\varpi X^{q^n} + \dots$ reveals
$[\varpi]_H(T) \equiv \varpi T$ mod $(T^2)$. Additionally, note that 
\begin{equation*}
  f([\varpi]_H(T)) = \varpi f(T) = \varpi T + f(T^{q^n}) \equiv f(T^{q^n}) \pmod \varpi.
\end{equation*}
Hence, the equivalence in \eqref{eq:funceqlemcongruence} implies that 
$[\varpi]_H(T) \equiv T^{q^n}$ mod $\varpi$. So $H$ is a Lubin--Tate formal
$\cO_E$-module law of height $n$, we call it the standard Lubin--Tate formal
module law of height $n$. Although the coefficients of $H$ lie inside $\cO_E$,
we usually consider it as a formal $\cO_E$-module over $\cO_\br E$.

We note the following. 
\begin{lem}\label{lem:MultByROUForStandardModule}
  Let $\zeta \in \br E$ be a $(q^n-1)$-th root of unity. Then 
  $[\zeta]_H(T) = \zeta T$ is an automorphism of $H$. In particular,
  $\End(H)$ naturally carries the structure of a $\cO_{E_n}$-algebra,
  where $E_n$ is the unramified extension of $E$ with residue field
  $\FF_{q^n}$. 
\begin{proof}
  This is an immediate consequence of the equality $\zeta \log_H(T)
  = \log_H(\zeta T)$. We have
  \begin{equation*}
    \zeta T = \exp_H(\log_H(\zeta T)) = \exp_H(\zeta \log_H(T)) = \exp_H
    ([\zeta]_{\Ghat_{a, \br E}}(\log_H(T))).
  \end{equation*}
  As $\exp_H\colon  \Ghat_{a, \br E} \to H \otimes_{\cO_\br E} \br E$ is an
  isomorphism of 
  formal modules, the claim follows.
\end{proof}
\end{lem}

% subsubsection Deformations of Formal Modules and the Standard Formal Module (end)

\subsection{Deformations of Formal Modules} % (fold)
\label{sub:Deformations of Formal Modules}
We shortly discuss deformations of formal $\cO_E$-modules.
Let $\cC$ denote the category of complete, local, Noetherian
$\cO_\br F$-algebras with residue field $\bar \FF_q$. Morphisms in $\cC$ are continuous
homomorphisms, a homomorphism of rings $\phi: R \to S$ lies in 
$\Hom_{\cC}(R,S)$ if and only if $\phi(\fm_R) \subset \fm_S$.

Let $\XX = H \otimes \Fqbar$ be the formal $\cO_E$-module law of height $n$
over $\Fqbar$ obtained by reduction of the standard formal $\cO_E$-module $H$ 
and let 
$R \in \cC$ be a $\cO_\br E$-algebra with maximal ideal $\fm_R$. A deformation
of $\XX$ to $R$ is a pair $(\cF, \iota)$ consisting of a formal $\cO_\br
E$-module $\cF$ over $R$ and an isomorphism $\iota\colon \FGG(\XX) \xto \sim
\cF \otimes \Fqbar$. We say that two deformations $(\cF, \iota)$ and $(\cF',
\iota')$ are isomorphic if there is an isomorphism $\cF \to \cF'$ whose
reduction to $\Fqbar$ translates $\iota$ into $\iota'$. We define the 
deformation functor

\begin{equation*}
  \cM^{(0)}_0 \colon \cC \to \Set, \quad R \mapsto \{\text{Deformations $(\cF,
  \iota)$ of $\XX$ to $R$}\}/\sim.
\end{equation*}

In the literature one encounters a variant of this functor, defined on the
level of formal
module laws rather than formal modules. Let $F$ and $G$ be formal $\cO_E$-modules over
$R \in \cC$. We say that $F$ and $G$ are $\star$-isomorphic if there exists a 
$\star$-isomorphism $F \xto \sim G$, that is, an isomorphism given by a power series
$f \in R\llbr T \rrbr$ satisfying $f(T) \equiv T$ modulo $\fm_R$.
The functor in question is now defined as 
\begin{equation*}
  \cM^{(0)}_{0, \text{FML}} \colon \cC \to \Set, \quad R \mapsto \{F \in \FMLOver {\cO_E} \Fqbar \mid
  F \otimes \Fqbar = \XX\} / \star\text{-isom.}
\end{equation*}

\begin{lem}\label{lem:FunctorsAreIsomorphic}
  The functors $\cM^{(0)}_0$ and $\cM^{(0)}_{0, \textnormal{FML}}$ are
  naturally isomorphic.
\end{lem}
\begin{proof}
  There is a natural map $\cM^{(0)}_{0, \text{FML}} \to \cM^{(0)}_0$, given on
  any component
  $R \in \cC$ by sending a formal 
  $\cO_E$-module law $F \in \cM^{(0)}_{0, \text{FML}}(R)$ to its associated formal module
  $(\FGG(F), \FGG(\id))$. We show that this map is bijective. 

  \textit{Surjectivity.} Let $(\cF, \iota) \in \cM^{(0)}_0(R)$, and choose any coordinate 
  on $\cF$ so that $\cF = \FGG(F)$ for some $\cO_E$-module law $F$. Then
  $\iota$ gives rise to an isomorphism of formal laws $\XX \iso F \otimes
  \Fqbar$ which we also denote by $\iota$. Let $\tilde \iota \in R\llbr T
  \rrbr$ be an arbitrary power series reducing to $\iota$ modulo $\fm_R$. Now
  the constant coefficient of $\tilde \iota$ is invertible, hence $\tilde
  \iota$ has a formal inverse power series $\tilde \iota^{-1} \in R \llbr T
  \rrbr$. Let $F'$ be the formal module law with addition given by 
  $\tilde \iota^{-1} F(\tilde \iota(S), \tilde \iota(T)) \in R\llbr S,T \rrbr$
  and similarly defined $\cO_E$-module structure.
  Now by construction, $\tilde \iota^{-1}$ is an isomorphism of module laws $F \to F'$,
  and we find that $F'$ satisfies $F' \otimes \Fqbar = \XX$. Hence, $F'$ 
  maps to the isomorphism class of $(\cF, \iota)$. This shows surjectivity.

  \textit{Injectivity.} Let $F$ and $G$ be formal $\cO_E$-module laws inside
  $\cM^{(0)}_{0, \text{FML}}(R)$ with isomorphic associated deformations
  $(\cF, \iota_\cF) \sim (\cG, \iota_\cG) \in \cM^{(0)}_0(R)$. 
  This isomorphism yields an isomorphism of formal $\cO_E$-module laws
  $\alpha\colon F \to G$ whose reduction modulo $\fm_R$ fits into the following
  commutative triangle.
  \begin{equation*}
\begin{tikzcd}[ampersand replacement=\&]
	{F\otimes\Fqbar} \& {G\otimes\Fqbar} \\
	{} \& {}
	\arrow["{\alpha_0}", from=1-1, to=1-2]
  \arrow[""{name=0, anchor=center, inner sep=0}, "{\text{$\XX$}}"{marking, allow upside down}, draw=none, from=2-1, to=2-2]
	\arrow[shorten >=7pt, Rightarrow, no head, from=1-1, to=0]
	\arrow[shorten >=7pt, Rightarrow, no head, from=1-2, to=0]
\end{tikzcd}
  \end{equation*}
  Hence, $\alpha$ is a $\star$-isomorphism, implying that 
  $F$ and $G$ lie in the same equivalence class. This proves injectivity.
\end{proof}

\begin{thm}[Representability of $\cM^{(0)}_0$]\label{thm:RepresentabilityOfDefSpaceWOLevel}
  The functor $\cM_0^{(0)}$ is representable by a ring $A_0 \in \cC$, non-canonically 
  isomorphic to
  \begin{equation*}
    \cO_\br E\llbr u_1, \dots, u_{n-1} \rrbr \in \cC.
  \end{equation*}
\begin{proof}
  This statement is due to Lubin--Tate in the case $\cO_E = \Z_p$, cf.
  \cite{LubinTate1966FormalModuli}. For the general case, cf. 
  \cite[Proposition 4.2]{drinfel1974elliptic}.
  \end{proof}
\end{thm}

We may use the universal $\cO_E$-typical formal module law
$F_v$ over $\cO_E\llbr v_1, v_2, \dots \rrbr$ to construct an
isomorphism $\Hom_\cC(\cO_\br E\llbr u_1, \dots, u_{n-1} \rrbr, R) \cong
\cM_0^{(0)}(R)$. Any homomorphism $\alpha \in \Hom_\cC(\cO_\br E\llbr u_1,
\dots, u_{n-1} \rrbr, R)$ gives rise to elements 
$\alpha(u_1), \dots, \alpha(u_{n-1}) \in \fm_R$.
Consider the homorphism of $\cO_E$-algebras
\begin{equation*}
  \tilde \alpha\colon \cO[v_1, v_2, \dots] \to \cO_\br E\llbr u_1, \dots, u_{n-1} \rrbr,
  \quad v_i \mapsto \begin{cases}
    \alpha(u_i), &\text{ if } 1 \leq i \leq n-1 \\
    1, &\text{ if } i = n \\
    0, &\text{ if } i \geq n+1.
  \end{cases}
\end{equation*}
and let $F_\alpha = F_v \otimes R$ be the formal $\cO_E$-module law over
$R$ obtained by applying $\tilde \alpha$ to the coefficients of $F_v$. 
By the Functional Equation Lemma (\cref{thm:HazewinkelIntegrality}), we find
that $F_\alpha$ is $\star$-isomorphic to $\XX$. Hence 
$F_\alpha \in \cM_0^{(0)}(R)$, and the mapping $\alpha \mapsto F_\alpha$
can be shown to be an isomorphism, cf. \cite[Section 12]{hopkins1994equivariant}. 


% subsection Deformations of Formal Modules (end)

\subsection{Quotients of Formal Modules by Finite Submodules} % (fold)
\label{sub:Quotients of Formal Modules by Finite Submodules}

Let $R \in \cC$ be a regular, local $\cO_E$-algebra without zero-divisors and let $F \in
\FMLArbOver {\cO_E}R$ be a formal $\cO_E$-module law over $R$. 
Then $F$ equips $\fm_R$ with the structure
of an $\cO_E$-module. Let $Q \subset \fm_R$ be a finite subset of cardinality
$c \in \N$ that is a submodule with respect to this $\cO_E$-module structure. Note
that $c$ is a power of $q$.
We write
\begin{equation*}
  f_Q(T) = \prod_{a \in Q}\left(T -_F a\right),
\end{equation*}
where the subscript $F$ denotes subtraction with respect to $F$. 

We now formulate the main result of this section. Let $\Frac(R)$ denote the field of 
fractions of $R$, which contains $R$ as a subring as $R$ has no zero divisors.
Note that $f_Q'(0) = \pm u(0) \prod_{a \in Q} a \neq 0$, so there exists
a formal inverse power series $f_Q^{-1}(T) \in \Frac(R) \llbr T \rrbr$.
Denote by $(F/Q)$ the formal $\cO_E$-module law over $\Frac(R)$ with 
\begin{equation*}
  (F/Q)(X,Y) = f_Q ( F( f_Q^{-1}(X), f_Q^{-1}(Y)))\quad \text{and} \quad
  [a]_{(F/Q)}(T) = f_Q([a]_F(f_Q^{-1}(T)))\text{ for $a \in \cO_E$}.
\end{equation*}
The power series $f_Q$ yields a morphism $F \to F/Q$ of formal module
laws over $\Frac(R)$.

\begin{thm}[Quotients of Formal Module Laws]\label{thm:Quotients}\leavevmode
  \begin{enumerate}
    \item The power series constituting the formal $\cO_E$-module law $(F/Q)$
      all have coefficients in $R$. That is, $(F/Q)$ is a formal module law over
      $R$, and $f_Q\colon F \to (F/Q)$ is a morphism of formal module laws over $R$. 
    \item Denote by $\cF$ the formal module corresponding to $F$. The finite group
      $Q$ acts on the formal scheme $\cF = \Spf(R\llbr T \rrbr)$, and the 
      map $\cF \xto{f_Q} (\cF/Q)$, where $(\cF/Q)$ denotes the formal module
      corresponding to $(F/Q)$, is a categorical quotient 
      (in the category of formal Lie-varieties of dimension $1$, 
      i.e., formal schemes isomorphic to $\spf R\llbr T \rrbr$) for this
      action. 
    \item Furthermore, if $\cH$ is another formal $\cO_E$-module law over $R$ and
      $g\colon \cF \to \cH$ is $Q$-invariant, the induced morphism
      $\tilde g \colon (\cF/Q) \to \cH$ is a morphism of formal $\cO_E$-modules.
  \end{enumerate}
\end{thm}
The proof is rather elementary and only uses commutative ring theory of 
$R\llbr T \rrbr$ and $R\llbr X,Y \rrbr$. These are again noetherian regular
local rings by \cite[Proposition 19.5]{matsumura1989commutative}, implying 
that they are unique factorization domains by the theorem of
Auslander--Buchsbaum, cf.
\cite[\href{https://stacks.math.columbia.edu/tag/0AG0}{Tag 0AG0}]{stacks-project}.
Furthermore, they are complete with respect to its maximal ideal.

The first important observation is the following lemma.
For the notion of Weierstraß degree and the Weierstraß Preparation Theorem, we refer
to Appendix A.3 of \cite{hazewinkel1978formal}.
\begin{lem}\label{lem:fQIsWeierstraß} \leavevmode
  \begin{enumerate}
    \item The power series $f_Q$ has Weierstraß degree $c$. That is, it is of
      the form 
      \begin{equation*} 
        f_Q(T) = \sum_{i=0}^\infty a_i T^i
      \end{equation*} 
      with $a_i \in \fm_R$ for $i < c$ and $a_c \in R^\times$. 
    \item There exists a unit $u(T) \in R\llbr T \rrbr$ such that $f_Q$ admits
      the factorization
      \begin{equation*}
        f_Q(T) = u(T) g(T),\quad \text{where} \quad 
        g_Q(T) = \prod_{a \in Q} \left(T - a \right).
      \end{equation*}
  \end{enumerate}
  \end{lem}
\begin{proof}
  The first statement follows directly from the fact that for $a \in Q$, we have
  \begin{equation*}
    F(T, a) \in T + a + T \fm_R.
  \end{equation*}
  The second is an application of the Weierstraß Preparation Theorem. This theorem
  implies a decomposition
  \begin{equation*}
    f_Q(T) = u(T)  g(T)
  \end{equation*}
  with $g(T) = a_0 + a_1T + \dots + a_{n-1}T^{n-1} + T^n \in R[T]$ of degree $n$
  and $a_i \in \fm_R$. One quickly checks that for $a \in \fm_R$, the
  elements $(T-a) \in R\llbr T \rrbr$ are irreducible. For $a \in Q$, we find 
  $f_Q(a) = 0$, implying that each factor $(T-a)$ divides $f_Q(T)$. As 
  $R\llbr T \rrbr$ is a unique factorization domain, this implies that 
  each factor $(T-a)$ divides $g(T)$. For degree-reasons we obtain $g(T) =
  \prod_{a \in Q}\left(T-a \right)$, as desired.
\end{proof}

We define an action of $Q$ on $R\llbr T \rrbr$ as follows. For 
$a \in Q$ and $h(T) \in R\llbr T \rrbr$, we put
\begin{equation*}
  a.h(T) = h(T +_F a) \in R\llbr T \rrbr.
\end{equation*}
As $T +_F a$ is a topologically nilpotent element of $R\llbr T \rrbr$, this definition
makes sense. The power series $f_Q(T)$ is $Q$-invariant, as
\begin{equation*}
  f_Q(T+_F a)  = \prod_{b \in Q}\left(T +_F a -_F b \right) 
  = \prod_{b \in Q}\left(T -_F b \right) = f_Q(T).
\end{equation*}
Note that given any $h \in R\llbr T \rrbr$, the power series
$h(f(T)) \in R\llbr T \rrbr$ is $Q$-invariant.
The main observation in the proof of \cref{thm:Quotients} is
the realization that every invariant power series arises this way.

\begin{prop}\label{prop:DescriptionOfInvariantsUnderQAction}
  Let $R\llbr T \rrbr^Q$ denote the invariants under the action of 
  $Q$ on $R\llbr T \rrbr$. Then any element in $R\llbr T \rrbr^Q$ arises as a 
  (unique) power series in $f_Q(T)$, i.e.,
  \begin{equation*}
    R\llbr T \rrbr^Q = R\llbr f_Q(T) \rrbr \subset R\llbr T \rrbr.
  \end{equation*}
\end{prop}
\begin{proof}
  By the short calculation above, only one inclusion remains. Let $h(T) \eqcolon
  h_0(T) \in R\llbr T \rrbr^Q$ be an arbitrary $Q$-invariant power series. Put 
  $c_0 = h_0(0)$ and write $v_0(T) = h_0(T) - c_0$. Then $v_0$ is $Q$-invariant,
  hence we find $v_0(a) = v_0(a +_F 0) = 0$ for $a \in Q$. In particular
  we find $g_Q(T) \mid v_0(T)$, where $g_Q(T)$ is the polynomial from 
  Lemma \ref{lem:fQIsWeierstraß}. As $g_Q$ differs from $f_Q$ by a unit, 
  we find $v_0(T) = f_Q(T) h_1(T)$ for some power series $h_1(T) \in R\llbr T \rrbr$.
  As $R$ is a domain, $h_1(T)$ is again $Q$-invariant.

  We can now iterate this process, writing $h_1(T) = c_1 + v_1(T)$, 
  $v_1(T) = f_Q(T) h_2(T)$, $h_2(T) = c_2 + v_2(T)$, et cetera. This results
  for any positive integer $m$ in a continued product
  \begin{align*}
    h(T) &= c_0 + f_Q(T) (c_1 + f_Q(T)(c_2 + \dots + f_Q(T)(c_m + v_m(T)) \dots ))\\
         &= c_0 + c_1 f_Q(T) + c_2f_Q(T)^2 + \dots + f_Q(T)^m v_m(T).
  \end{align*}
  In the limit $m \to \infty$, we obtain the desired power series description
  of $h(T)$.
\end{proof}

We remark that virtually the same argument yields the equality
\begin{equation*}
  R\llbr X,Y \rrbr^{Q \times Q} = R\llbr f(X), f(Y) \rrbr,
\end{equation*}
where $Q\times Q$ acts on $R\llbr X,Y \rrbr$ separately for each variable.

\begin{proof}[Proof of \cref{thm:Quotients}]
  \textit{Part 1}. One readily checks that the power series 
  $f(F(X,Y)) \in R\llbr X,Y \rrbr$ is $Q\times Q$-invariant, and that 
  $[a]_F(T) \in R\llbr T \rrbr$ is $Q$-invariant. Hence, 
  by \cref{prop:DescriptionOfInvariantsUnderQAction}, there exist unique power
  series
  $F'(X,Y) \in R\llbr X,Y \rrbr$ and $[a]_{F'}(T) \in R\llbr T \rrbr$ such that 
  \begin{equation*}
    f_Q(F(X,Y)) = F'(f_Q(X),f_Q(Y)) \quad \text{and} \quad 
    f_Q([a]_F(T)) = [a]_{F'}(f_Q(T)) \text{ for }a\in \cO_E.
  \end{equation*}
  But a set power series with coefficients in $\Frac(R)$ is uniquely determined
  by these properties, namely, these power series have to be equal to those
  defining the formal $\cO_E$-module law $(F/Q)$. As $R$ injects 
  into $\Frac(R)$, we find $F' = (F/Q)$, and the first claim follows.

  \textit{Part 2.} This is a reformulation of
  \cref{prop:DescriptionOfInvariantsUnderQAction}.

  \textit{Part 3.} This follows formally from the description the formal 
  $\cO_E$-module law $(F/Q)$ in terms of $F$ and $f_Q$.
\end{proof}


% subsection Quotients of Formal Modules by Finite Submodules (end)

\subsection{Explicit Dieudonné Theory} % (fold)
\label{sub:Explicit Dieudonne Theory}
Let $\cF$ and $\cF'$ be formal $A$-modules of dimension $m$ and $n$ respectively,
over an affine base $\spec R$, coming from formal module laws $F$ and $F'$. We
give an explicit description of $\Ext(\cF, \cF')$ in terms of terms of the
Symmetric 2-Cocycles associated with $F$ and $F'$ (cf. Definition
\ref{def:SymCoc2}). 
We also give a related explicit description of $\RigExt(F, \Ghat_a)$ in terms of 
Quasi-Logarithms, cf. Definition \ref{def:QuasiLogarithm}. 

Write $\bX$ for the variables of $F'$ and $\bZ$ for the variables of $F$.
\begin{defi}[Symmetric $1$-Cochain] \label{def:SymCoc1}
  A symmetric $1$-cochain for the pair $(F,F')$ is an $n$-tuple of power 
  series $\bgg = (g_1, \dots, g_m)$, such that $g_i(\bZ) \in R\llbr \bZ \rrbr$
  satisfying $g_i(0) = 0$ for all $i$. We write $\delta \bgg$ 
  for the coboundary of $\bgg$, that is, the pair $(\Delta \bgg, (\delta_a
  \bgg)_{a \in A})$,
  where
  \begin{equation*}
    \Delta \bgg = \bgg(\bZ_1) -_{F'} \bgg(F(\bZ_1,\bZ_2)) +_{F'} \bgg(\bZ_2)
    \in (R\llbr \bZ_1, \bZ_2 \rrbr)^m
  \end{equation*}
  and 
  \begin{equation*}
    \delta_a \bgg = [a]_{F'} \bgg(\bZ) -_{F'} \bgg([a]_F(\bZ)) \in (R \llbr \bZ
    \rrbr)^m.
  \end{equation*}
  One readily checks that $\delta \bgg \in \SymCoc^2(F, F')$. 
\end{defi}

\begin{prop}\label{prop:ExtInTermsOfSymCoc}
  Given two extensions $\cE,\cE' \in \Ext(\cF, \cF')$, write 
  $E$, $E'$ for the respective formal $A$-module laws coming from Lemma
  $\ref{lem:SESStandardForm}$, and write
  $\Delta_E$ and $\Delta_{E'}$ for the associated symmetric $2$-cocycles (cf.
  Proposition \ref{prop:ClassOfFGLitoSymCoc}). There is a bijection
  \begin{equation*}
    \{\bgg \in (R\llbr \bZ \rrbr)^m \mid \bgg(0) = 0 \text{ and } \delta \bgg = \Delta_{E'} - \Delta_E\}
    \xto \sim \{\text{Isomorphisms of extensions } E \to E' \}.
  \end{equation*}
  Explicitly, this bijection is given by sending $\bgg$ to the morphism
  $i_\bgg \in \Hom_{\FMLArbOver A R}(E, E')$, given by 
  $i_\bgg (\bX, \bZ) = (\bX +_{F'} \bgg(\bZ), \bZ)$. In particular, there is a bijection
  \begin{equation*}
    \Ext(\cF, \cF') \cong \frac{\SymCoc^2(F, F')}{\{\delta \bgg 
    \mid \bgg \in (R\llbr \bZ \rrbr)^m \text{ with } \bgg(0) = 0\}}.
  \end{equation*}
  This bijection is an isomorphism of $\End(\cF')$-modules.
\end{prop}
For now, this finishes the study of $\Ext(\cF, \cF')$. 

Assume now that $\cF' = \Ghat_a$, and that 
$\cF$ comes from a one-dimensional formal $A$-module $F \in \FMLOver AR$. For
the remainder of this subsection, we will be concerned with the $R$-module
$\RigExt(\cF, \Ghat_a)$. The notion of Quasi-Logarithms will play a major role.

\begin{defi}[Quasi-Logarithms]\label{def:QuasiLogarithm}
  A power series $g(T) \in (R \otimes_A E) \llbr T \rrbr$ is called a 
  Quasi-Logarithm for $F$, if $g(0) = 0$ and $g'(T)$, as well as all 
  of the power series appearing in $\delta g$ (with $F' = \Ghat_a$, cf.
  Definition \ref{def:SymCoc1})
  have coefficients in $R$. We define the $R$-module
  \begin{equation*}
    \QLog(F) = \frac{\{g(T) \in (R \otimes_A E) \llbr T \rrbr \mid g \text{ is a 
    quasi-logarithm for } F\}}{\{g(T) \in R\llbr T \rrbr \mid g(0) = 0\}}
  \end{equation*}
\end{defi}

Let $(\cE,s) \in \RigExt(F,\Ghat_a)$ be a rigidified extension. 
The splitting $s$ yields an isomorphism $\omega(\cE) \cong \omega(\Ghat_a) \oplus
\omega(\cF)$ on duals, giving an invariant differential $\omega_\cE \in
\omega(\cE)$ pulling back to $\dc X$ on $\Ghat_a$. Conversely, any such invariant
differential $\omega_\cE$ yields a splitting, so the choice of $s$ is
equivalent to the choice of $\omega_E$, and we will henceforth write 
$(\cE, \omega_\cE) \in \RigExt(\cF, \Ghat_a)$. 

\begin{thm}[Classification of Rigidified Extensions in terms of Quasi-Logarithms]
  \label{thm:RigExtStr}
  There is a bijection
  \begin{equation}\label{eq:QLogBijection}
    \{\text{Quasi-logarithms for $F$}\} \xto \sim 
      \left\{
      \begin{gathered}
        \text{Pairs $(E, \omega_E)$, where $E$ is an $A$-module law} \\
        \text{fitting into  an exact sequence}
        \\ 0 \to \Ghat_a \xto \alpha E \xto \beta F \to 0 \\
        \text{with $\alpha(X) = (X, 0)$ and $\beta(X, T) = T$ and $\omega_E$} \\
        \text{is an invariant differential on $E$ with 
        $\alpha^* \omega_E = \dc X$.} \\
      \end{gathered}
      \right\}
  \end{equation}
The map sends any quasi-logarithm $g(T) \in (R\otimes_A E)\llbr T \rrbr$
to the pair $(E_{\delta g}, \dc (X + g(T)) \in \RigExt(F, \Ghat_a)$. Here
$E_{\delta g} \in \Ext(F, \Ghat_a)$ is the extension corresponding to 
$\delta g \in \SymCoc^2(F,\Ghat_a)$. 

Furthermore, given two rigidified extensions
$(E, \omega_E), (D, \omega_{D})$ with associated quasi-logarithms
$g(T)$ and $h(T)$, there is a (unique) isomorphism $(E, \omega_E) \to (D, \omega_{D})$
if and only if $h(T)-g(T) =\colon  f(T)$ has coefficients in $R\llbr T \rrbr$. 
In this case, the isomorphism $i_f(X,T) \in \Hom_{\FMLArbOver A R}(E, D)$ is
given by $i_f(X,T) = (X+f(T), T)$. In particular, there is a canonical bijection
\begin{equation*}
  \QLog(F) \xto \sim \RigExt(F, \Ghat_a).
\end{equation*}
This bijection is an isomorphism of $R$-modules.
\begin{proof}[Proof]
  We construct an inverse of the map in \eqref{eq:QLogBijection}. Let $(E,
  \omega_E)$ be an element of the set on the right
  and let $(\Delta, (\delta_a)_{a \in A}) \in
  \SymCoc^2(F, \Ghat_a)$ be the symmetric 2-cochain corresponding to $E$.
  Following Proposition \ref{prop:loginvdiff}, the datum of 
  $\omega_E \in \omega(E)$ is equivalent to a morphism 
  $$f_E \in \Hom_{\FMLOver A {R \otimes E}}(E \otimes_R (R \otimes_A E), \Ghat_a)
  \quad \text{satisfying} \quad f_E(X,T) = X + g(T)$$
  for some $g(T) \in (R \otimes_A E)\llbr T \rrbr$. The fact that 
  $f_E$ is a homomorphism implies that 
  \begin{multline*}
    X_1 + X_2 + \Delta(T_1, T_2) + g(F(T_1,T_2)) = f_E(E((X_1, T_1), (X_2, T_2))) = \\
    = f_E(X_1, T_1) + f_E(X_2, T_2)) = X_1 + g(T_1) + X_2 + g(T_2),
  \end{multline*}
  thereby $\Delta g = \Delta(T_1, T_2) \in R\llbr T_1, T_2\rrbr$. Similarly, 
  we find $\delta_a g = \delta_a \in R\llbr T \rrbr$. Hence, $g(T)$ is a
  quasi-logarithm with $\delta g = (\Delta, (\delta_a)_a)$. 
  This construction yields the desired inverse.
  The remaining statements are verified directly, also cf. \cite[Section
  8]{hopkins1994equivariant}.
\end{proof}
\end{thm}

Now, let $A$ be a complete, discrete valuation ring with 
uniformizing parameter $\varpi$ and finite residue field $k$. 
\begin{prop} \label{prop:ExplicitInterestingES}
  Let $\cF$ be a one-dimensional formal $A$-module law over a flat, local
  $A$-algebra $R$, and suppose that $\cF' = \Ghat_a$.
  The short exact sequence of Proposition \ref{prop:InterestingES}
  fits into a commutative diagram with 
  exact rows and vertical maps isomorphisms induced by 
  any choice of coordinate $\cF = \FGG(F)$.
\begin{equation*}
\begin{tikzcd}[ampersand replacement=\&]
  {\Hom(\cF,\Ghat_a)} \& {\omega(\cF)} \& {\RigExt(\cF,\Ghat_a)} \&
  {\Ext(\cF,\Ghat_a)} \\ 
  \begin{array}{c} \left\{\begin{gathered} f \in TR\llbr T \rrbr \colon  \\  \delta f
    = 0\end{gathered}  \right\} \end{array} \& \begin{array}{c}
    \left\{\begin{gathered} f \in (R\otimes_A E)\llbr T \rrbr : \\ 
  \text{$\delta f = 0$, $f(0) = 0$} \\
    \text{and $f'(T) \in R\llbr T \rrbr$}\end{gathered} \right\} \end{array} \& {\QLog(F)} \&
    {\frac{\SymCoc^2(F, \Ghat_a)}{\{\delta g \mid g \in T R\llbr T \rrbr\}}}
	\arrow[hook, "{\dc_F}", from=1-1, to=1-2]
	\arrow[from=1-1, to=2-1]
	\arrow[from=1-2, to=1-3]
	\arrow[from=1-2, to=2-2]
	\arrow[two heads, from=1-3, to=1-4]
	\arrow[from=1-3, to=2-3]
	\arrow[from=1-4, to=2-4]
	\arrow[hook, from=2-1, to=2-2]
	\arrow[from=2-2, to=2-3]
	\arrow["\delta", two heads, from=2-3, to=2-4]
\end{tikzcd}
\end{equation*}
\begin{proof}
  Injectivity of $\dc_F$ is provided by Proposition \ref{prop:loginvdiff},
  and related to the original exact sequence as $\Hom_R(\Lie(\cF), \Lie(\Ghat_a)) = 
  \omega(\cF)$.
  Surjectivity of $\RigExt(\cF, \Ghat_a) \to \Ext(\cF, \Ghat_a)$ comes from the
  fact that $\Lie(\cF)$ is projective. 
  The first vertical map is an isomorphism by definition. 
  The vertical arrow describing $\omega(F)$ is obtained by
  identifying the preimage of $\omega(F) \subseteq \omega(F \otimes_R (R
  \otimes_AE))$ under the isomorphism 
  \begin{equation*}
    \{f \in T(R \otimes_A E)\llbr T \rrbr \mid \delta f = 0\} = \Hom_{\FMLOver A
    {R \otimes_A E}} (F \otimes (R \otimes_A E), \Ghat_a) \xto{\dc_F} \omega(F
    \otimes_R (R \otimes_AE)).
  \end{equation*}
  All squares commute by construction.
\end{proof}
\end{prop}

We admit the following facts from Section 9 of \cite{hopkins1994equivariant}.
\begin{prop}\label{prop:InterestingSequenceStdMod}
  Let $\cF$ be a formal $A$-module of height $h$ over $R \in \cC$. 
  Then $\Ext(\cF, \Ghat_a)$ is a free $R$-module of rank $n-1$, 
  and $\RigExt(\cF, \Ghat_a)$ is a free $R$-module of rank $n$. 
\begin{proof}
  This is Proposition 9.8 in \cite{hopkins1994equivariant}. \end{proof}
\end{prop}

In the case that $R$ is flat over $\cO_E$, and $\cF$ comes from a 
$\cO_E$-typical formal module, \cite[Section 9]{hopkins1994equivariant} also
gives an explicit description of a basis for the modules in question.
If $\cF$ is isomorphic to the standard formal $\cO_E$-module $H$ of height $n$,
which allows for an explicit
description of the exact sequence in \cref{prop:InterestingES}.  
There is no non-trivial map $H \to \Ghat_a$, so the sequence becomes
\begin{equation*}
\begin{tikzcd}[ampersand replacement=\&]
	0 \& {\omega(H)} \& {\RigExt(H,\Ghat_a)} \& {\Ext(H,\Ghat_a)} \& 0 \\
	0 \& \begin{array}{c} \left\{\begin{gathered} {g \in TE\llbr T \rrbr : \delta g = 0} \\          \text{ and $g'(T) \in \cO_E\llbr T \rrbr$}\end{gathered} \right\} \end{array} \& {\QLog(H)} \& {\frac{\SymCoc^2(H, \Ghat_a)}{\{\delta g \mid g \in T \cO_E\llbr T \rrbr\}}} \& 0.
	\arrow[from=1-1, to=1-2]
	\arrow[from=1-2, to=1-3]
	\arrow[from=1-2, to=2-2]
	\arrow[from=1-3, to=1-4]
	\arrow[from=1-3, to=2-3]
	\arrow[from=1-4, to=1-5]
	\arrow[from=1-4, to=2-4]
	\arrow["", from=2-1, to=2-2]
	\arrow[from=2-2, to=2-3]
	\arrow["\delta", from=2-3, to=2-4]
	\arrow[from=2-4, to=2-5]
\end{tikzcd}   
\end{equation*}
Now \cite[Proposition 9.8]{hopkins1994equivariant} implies the following result.
\begin{prop}
  The $\cO_\br E$-module $\omega(H)$ is free of rank $1$, generated by 
  $f(T) = \log_H(T)$. $\QLog(H)$ is free of rank $n$, generated by the classes of
  $(f(T), \frac 1\varpi f(T^q), \dots, \frac 1\varpi f(T^{q^{n-1}}))$. Consequently,
  the short exact sequence above is given by 
  \begin{equation*}
    0 \to \left \langle f(T) \right \rangle \to \left \langle f(T), \frac 1\varpi
      f(T^q) , \dots,
    \frac 1\varpi f(T^{q^{n-1}}) \right \rangle \xto \delta 
    \left \langle \delta \left(\frac 1\varpi f(T^q)\right),\dots, \delta
    \left(\frac 1\varpi f(T^{q^{n-1}})\right ) \right \rangle \to 0.
  \end{equation*}
\end{prop}

We remark that the functors $\Ext_R(-,\Ghat_a)$ and $\RigExt_R(-,\Ghat_a)$ are
also functorial in $R$.
\begin{lem}
  If $R \to R'$ is a homomorphism of local $A$-algebras, the induced maps 
  of free $R'$-modules
  \begin{gather*}
    \Ext_R(\cF, \Ghat_a) \otimes_R R' \to \Ext_{R'}(\cF, \Ghat_a) \\
    \RigExt_R(\cF, \Ghat_a) \otimes_R R' \to \RigExt_{R'}(\cF, \Ghat_a)
  \end{gather*}
  are isomorphisms.
\begin{proof}
  \cite[Corollary 9.13]{hopkins1994equivariant}.
\end{proof}
\end{lem}

\begin{defi}[The Dieudonn\'e module of a formal $A$-module]\label{def:DioModule}
  Given $\cF \in \FMOver AR$, we define 
  $$\Dio(\cF) \coloneqq \Hom_R(\RigExt(\cF, \Ghat_a), R).$$ 
  We call $\Dio(\cF)$ the (covariant) Dieudonn\'e-module of $\cF$. 
\end{defi}

\begin{prop}[Crystalline Nature of $\Dio(-)$]\label{prop:DioCrystalline}
  The assignment $\cF \mapsto \Dio(\cF)$ yields a functor
  \begin{equation*}
    \FMOver AR \to \Mod R.
  \end{equation*}
  Given two formal $A$-modules $\cF, \cG \in \FMOver AR$ and two morphisms
  $\phi, \psi$ from $\cF$ to $\cG$ such that the induced morphisms of their reductions
  to $R/I$ agree, the induced morphisms $\Dio(\cF) \to \Dio(\cG)$ agree.
  \begin{proof}
    \todo{!!!} 
  \end{proof}
\end{prop}
% subsubsection Explicit Dieudonne Theory (end)

\subsection{The Universal Additive Extension} % (fold)
\label{sub:The Universal Additive Extension}
We follow \cite[Section 11]{hopkins1994equivariant}, 
and specialize to the situation where $A$ is a complete discrete valuation ring
with uniformizer $\varpi$ and finite residue field of characteristic $p$ and
$R$ is a local admissible $A$-algebra with residue field $\bar \FF_q$.

\begin{lem}
  Let $M$ be a finite free module over $R$. Then there is a natural bijection,
  functorial in $M$ and $\cF$
  \begin{equation*}
    \Ext(\cF, \Ghat_a \otimes M) \cong \Ext(\cF, \Ghat_a) \otimes_R M.
  \end{equation*}
\begin{proof}
  After choosing coordinates on $\cF$, this follows directly from the 
  description of $\Ext$ in terms of symmetric 2-cocycles, cf. Propositions
  \ref{prop:ClassOfFGLitoSymCoc} and \ref{prop:ExtInTermsOfSymCoc}.
\end{proof}
\end{lem}

Let $\cF$ be a one-dimensional formal $A$-module over $R$.
We put $\mathrm M(\cF) \coloneqq \Hom_R(\Ext(\cF, \Ghat_a), R)$, which is free of
rank $n-1$, and
write $\cV = \Ghat_a \otimes \mathrm M(\cF)$. Now, by the previous lemma,
\begin{equation*}
  \Ext(\cF,\cV) = \Ext(\cF, \Ghat_a \otimes \mathrm M(\cF)) = \End_R(\Ext(\cF, \Ghat_a)). 
\end{equation*}
Let $0 \to \cV \to \cE \to \cF \to 0$ be the extension corresponding to the identity 
on the right. This class is unique up to unique isomorphism. Indeed, 
as $R$ is a local ring we may choose formal module laws $F$ and $V$ giving rise
to $\cF$ and $\cV$, and let $E$ be the module law obtained from 
Lemma \ref{lem:SESStandardForm}. 
If $0 \to V \to E' \to F \to 0$ is another extension in this class, we
have by construction a commutative square
\begin{equation*}
\begin{tikzcd}[ampersand replacement=\&]
	0 \& {V} \& E \& F \& 0 \\
	0 \& {V} \& {E'} \& F \& 0,
	\arrow[from=1-1, to=1-2]
	\arrow[from=1-2, to=1-3]
	\arrow[Rightarrow, no head, from=1-2, to=2-2]
	\arrow[from=1-3, to=1-4]
	\arrow["i", from=1-3, to=2-3]
	\arrow[from=1-4, to=1-5]
	\arrow[Rightarrow, no head, from=1-4, to=2-4]
	\arrow[from=2-1, to=2-2]
	\arrow[from=2-2, to=2-3]
	\arrow[from=2-3, to=2-4]
	\arrow[from=2-4, to=2-5]
\end{tikzcd}
\end{equation*}
and by Proposition \ref{prop:ExtInTermsOfSymCoc} we see that 
any other isomorphism $i'$ making the diagram above commute differs from
$i$ by an element in $\Hom(F, V) = 0$. 

\begin{defi}[Universal Additive Extension]
  The extension
  \begin{equation*}
    0 \to \cV \to \cE \to \cF \to 0
  \end{equation*}
  constructed above is called the universal additive extension of $\cF$.
\end{defi}

\begin{prop}
  If $N$ is a finite, free $R$-module, $\cG' = \Ghat_a \otimes N$ and 
  \begin{equation*}
    0 \to \cG' \to \cE' \to F \to 0
  \end{equation*}
  is an extension of $\cF$ by $\cG'$, there are unique homomorphisms
    $i\colon  \cE \to \cE'$ and $g'\colon  \cV \to \cG'$ making the diagram
  \begin{equation*}
\begin{tikzcd}[ampersand replacement=\&]
	0 \& {\cV} \& \cE \& \cF \& 0 \\
	0 \& {\cG'} \& {\cE'} \& \cF \& 0
	\arrow[from=1-1, to=1-2]
	\arrow[from=1-2, to=1-3]
	\arrow["{g'}"', from=1-2, to=2-2]
	\arrow[from=1-3, to=1-4]
	\arrow["i", from=1-3, to=2-3]
	\arrow[from=1-4, to=1-5]
	\arrow[Rightarrow, no head, from=1-4, to=2-4]
	\arrow[from=2-1, to=2-2]
	\arrow[from=2-2, to=2-3]
	\arrow[from=2-3, to=2-4]
	\arrow[from=2-4, to=2-5]
\end{tikzcd}
  \end{equation*}
  commute. In particular, we have $\cE' = g'_* \cE$. 
\begin{proof}
  As $\cV$ and $\cG'$ are additive, we have
  \begin{equation*}
    \Hom(\cV, \cG') = \Hom_R(\mathrm M(\cF), N) = \Ext(\cF, \Ghat_a) \otimes N =
    \Ext(\cF, \cG').
  \end{equation*}
  This yields $g'$. Again, $i$ is unique as by observations similar to 
  Proposition \ref{prop:ExtInTermsOfSymCoc}, the difference of two
  morphisms $i,i'\colon \cE \to \cE'$ is given a morphism $\cF \to \cG'$, which has
  to be trivial.
\end{proof}
\end{prop}

\begin{lem}\label{lem:LieAlgOfUnivAddExt}
  There is a natural isomorphism $\Lie(\cE) \xto \sim \Hom(\RigExt(\cF, \Ghat_a), R) 
  = \Dio(\cF)$.
\begin{proof}
  We show the equivalent statement $\omega(\cE) = \RigExt(\cF, \Ghat_a)$. 
  Let $(\cE', \omega_{\cE'}) \in \RigExt(\cF, \Ghat_a)$. Then by universality of 
  $\cE$, we obtain a unique homomorphism $i\colon  \cE \to \cE'$. This yields a 
  homomorphism of $R$-modules $\RigExt(\cF, \Ghat_a) \to \omega(\cE)$, 
  sending a pair $(\cE', \omega_{\cE'})$ to $i^*\omega_{\cE'}$. This morphism
  fits into the following commutative diagram, where the top row is 
  the short exact sequence from Proposition \ref{prop:ExplicitInterestingES}
  and the bottom row is the dual short exact sequence of 
  $0 \to \Lie(\cV) \to \Lie(\cE) \to \Lie(\cF) \to 0$.
  \begin{equation*}
  \begin{tikzcd}[ampersand replacement=\&]
  	0 \& {\omega(\cF)} \& {\RigExt(\cF, \Ghat_a)} \& {\Ext(\cF, \Ghat_a)} \& 0 \\
  	0 \& {\omega(\cF)} \& {\omega(\cE)} \& {\omega(\cV)} \& 0
  	\arrow[from=1-1, to=1-2]
  	\arrow[from=1-2, to=1-3]
  	\arrow[Rightarrow, no head, from=1-2, to=2-2]
  	\arrow[from=1-3, to=1-4]
  	\arrow[from=1-3, to=2-3]
  	\arrow[from=1-4, to=1-5]
  	\arrow[Rightarrow, no head, from=1-4, to=2-4]
  	\arrow[from=2-1, to=2-2]
  	\arrow[from=2-2, to=2-3]
  	\arrow[from=2-3, to=2-4]
  	\arrow[from=2-4, to=2-5]
  \end{tikzcd}
  \end{equation*}
  Thereby, $\RigExt(\cF, \Ghat_a) \to \omega(\cE)$ is a natural isomorphism.
\end{proof}
\end{lem}
% subsection The Universal Additive Extension (end)

\subsection{Determinants of Formal modules} % (fold)
\label{sub:Determinants of Formal modules}
In \cite{hedayatzadeh2015det}, Hedayatzadeh constructs determinants of 
$\varpi$-divisible formal $\cO_E$-modules over \red{properties} rings. 
We cite the result of main importance for us.
\begin{thm}[Determinants of Formal Modules]\label{thm:HedayatzadehsResult}
  ABCDE
\begin{proof}
\end{proof}
\end{thm}

Let $\cF_0 \in \FMOver {\cO_E}{\Fqbar}$ be a formal module of height $n$ and 
write $\wedge^n\cF_0$ for the associated determinant module, that is, the 
formal $\cO_E$ module with $\Dio(\wedge^n\cF_0) = \wedge^n\Dio(\cF_0)$. Write
$\cM_m$ for the Deformation space of $\cF_0$ with Drinfeld level
$\varpi^m$-structure, and write $\cM_{m, \wedge}$ for the deformation space of
$\wedge^n\cF$. 
Following \cite{weinstein2016semistable}, we sketch how this result can be
used to construct a functor $\cM_{m} \to \cM_{m, \wedge}$. 

For $R\in\cC$ and $(\cF, \iota) \in \cM_{0}(R)$, we write $\delta_m$ for the
induced universal multilinear and alternating morphisms
\begin{equation}\label{eq:determinantontorsionpoints}
  \delta_m \colon \cF[\varpi^m]^n \to \wedge^n\cF[\varpi^m].
\end{equation}

We now have the following result.

\begin{lem}\label{lem:DeterminantOfDrinfeldStructure}
  Let $(x_1, \dots, x_n) \in \cF[\varpi^{m}]^n(R)$ be a Drinfeld level
  $\varpi^m$ structure. Then 
  \begin{equation*}
    \delta_m(x_1, \dots, x_n) \in \wedge^n\cF[\varpi^m](R) 
  \end{equation*}
  is a Drinfeld level $\varpi^m$ structure.
\begin{proof}
  This is \cite[Proposition 2.11]{weinstein2016semistable}.
\end{proof}
\end{lem}

In particular, we obtain the desired map

\begin{equation*}
  \cM_m^{(0)}(R) \to \cM^{(0)}_{m, \wedge}(R), \quad (\cF, \iota, \phi)
  \mapsto (\wedge^n \cF, \wedge^n \iota, \delta_m \circ \phi).
\end{equation*}

We also need the following result.

\begin{lem}\label{lem:WeinsteinDeterminantAndNorm}
  Let $L/E$ be a separable extension of degree $n$ and suppose that there 
  is an action $\cO_L \inj \End(\cF)$ turning $\cF$ into a formal
  $\cO_L$-module of height $1$. Then, for all $m \geq 1$, the identity
  \begin{equation*}
    \delta_m(\alpha x_1, \dots, \alpha x_n) = \Norm_{L/E}(\alpha) \delta_m
    (x_1, \dots, x_n)
  \end{equation*}
  holds. 
  \begin{proof}
    This is \cite[Lemma 2.12]{weinstein2016semistable}.
  \end{proof}
\end{lem}

We remark that the Lemma above in particular applies to the standard formal 
$\cO_E$-module $H$ over $\cO_\br E$. We turn our attention to the determinant of the standard
formal $\cO_E$-module in the following example.

\begin{xpl}
  The determinant $\wedge^n H$ is the formal $\cO_E$-module law over 
  $\cO_\br E$ with logarithm given by the power series
  \begin{equation*}
    f_\wedge(T) = \sum_{i=0}^\infty (-1)^{(n-1)i} \frac{T^{q^i}}{\varpi^i}.
  \end{equation*}
  This is to be understood in the following way. If $\cH$ is a formal module
  equipped with a coordinate $\cH \cong \spf(\cO_\br E \llbr T \rrbr)$ inducing an 
  isomorphism $\cH \cong \FGG(H)$, then the same coordinate yields 
  an isomorphism $\wedge^n \cH \cong \FGG(\wedge H)$. We do not prove this,
  but we note that it can be witnessed on the
  corresponding Dieuodonn\'e-modules: $\Dio(\wedge^n H)$ and 
  $\wedge^n \Dio(H)$ are naturally isomorphic. The module $\QLog(\wedge^n H)$
  is
  generated by $f_\wedge(T)$, and $\wedge^n \QLog(H)$ is generated by the element
  \begin{equation*}
    w(T) = \log_H(T) \wedge \frac 1\varpi \log_H(T^q) \wedge \dots \wedge \frac 1\varpi \log_H(T^{q^{n-1}}) \in \wedge^n \Dio(H).
  \end{equation*}
  We have $f_\wedge(T^q) = (-1)^{n-1} \varpi f(T) - \varpi T$, which equals
  $(-1)^{n-1}\varpi f_\wedge(T)$ in $\QLog(\wedge^n H) \cong \Dio(\wedge^n
  H)^\vee$. 
  This readily implies
  \begin{equation*}
    \phi(w) = w(T^q) = (-1)^{n-1} \varpi w(T) \quad \text{and} \quad
    \phi(f_\wedge(T)) = f_\wedge(T^q) = (-1)^{n-1}\varpi f_\wedge(T).
  \end{equation*}
  Thereby, $\Dio(\wedge^n H)$ and $\wedge^n \Dio(H)$ are free of rank one
  with isomorphic Frobenius structure, hence isomorphic as Dieudonn\'e-modules.
  We also note that by Theorem \ref{thm:HazewinkelIntegrality}, one finds
  \begin{equation} \label{eq:detformalmodulevarpiexpl}
    [\varpi]_{(\wedge^n H)_0}(T) = (-1)^{n-1} T^q.
  \end{equation}
\end{xpl}

% subsubsection Determinants of Formal modules (end)

\subsection{The Universal Cover} % (fold)
\label{sub:Tate Modules and the Universal Cover} 

% Subintro (fold)
Assume that $A$ is a discrete valuation ring with uniformizer
$\varpi$, finite residue field $k$ and fraction field $K$. 
Write $q = \# k$. Let $R$ be an admissible $A$-algebra admitting an ideal of definition
$I$ with $R/I = \Fqbar$. Let $\cF$ be a formal
$\varpi$-divisible $A$-module over $R$ of height $n$. 
\begin{defi}[The Universal Cover]
  We denote by $\tilde \cF$ the functor
  \begin{equation*}
    \Tilde \cF \colon \Adm R \to \Vec {K}, \quad
    S \mapsto \left\{(x_1, x_2, \dots) \in \prod_{\N} \cF(S) \mid \varpi (x_{i+1}) =
    x_i \right\}.
  \end{equation*}
  Note that for $S \in \Adm R$, the set $\Tilde \cF(S)$ has a natural
  $A$-module structure with multiplication by $\varpi$ an isomorphism, making
  it a $K$-vector space.
\end{defi}

We remark that the Tate module
\begin{equation}\label{eq:TateModuleDef}
  T_\varpi \cF \colon \Adm R \to \Mod A, \quad S \mapsto \{(x_1, x_2, \dots)
  \in \tilde \cF \mid \varpi x_1 = 0\}
\end{equation}
as well as the rational Tate module 
\begin{equation}\label{eq:RationalTateModuleDef}
  V_\varpi \cF \colon \Adm R \to \Vec K, \quad S \mapsto \{(x_1, x_2, \dots)
  \in \tilde \cF \mid \exists n \in \N : \varpi^n x_1 = 0\}
\end{equation}
arise as subfunctors of $\Tilde \cF$. 
% Subintro (end)

\subsubsection{Useful Calculations} % (fold)
\label{ssub:Useful Calculations}
Let $p$ be a prime. Let $R$ be a Noetherian local ring with maximal ideal $I$
such that $p \in I$, $R$ is complete with respect to the $I$-adic topology and
$k_R \coloneqq R/I$ is an algebraically closed field (necessarily of
characteristic $p$). If $q$ is a power of $p$, we write $\cP_{R,q}$ for the set
of power series $f \in R\llbr T \rrbr$ satisfying 
\begin{equation} \label{eq:condonpowerseries}
  f(T) \equiv g(T^q) \pmod I
\end{equation}
for some power series $g(T) = c_1 T + c_2 T^2 + \dots \in R\llbr T \rrbr$ with 
$c_1 \in R^\times$. 
If $q'>q$ is another power of $p$, we have injections $\cP_{R,q} \inj \cP_{R,q'}$
given by sending $f(T)$ to its $(q'/q)$-fold self-composite $f^{q'/q}(T)$. 
Making use of these transition maps, we define
\begin{equation*}
  \cP_R \coloneqq \colim_{n \in \N} \cP_{R, p^n},
\end{equation*}
identifying any power series $f \in \cP_{R,q}$ with its image in $\cP_{R,q'}$ for 
higher $p$-powers $q'$. 
For any $f \in \cP_{R,q}$, we define the functor
\begin{equation*}
  U_f: \Adm R \to \Set, \quad S \mapsto \left\{(x_0,x_1, \dots) \in \prod_\N S^\cici 
                                          \mid f(x_{i+1}) = x_i\right\}.
\end{equation*}
This functor does, up to canonical isomorphism, only depend on the equivalence
class of $f$ in $\cP_R$. 
We write $U_{0,f}$ for the base change of $U_f$ to $k_R$, that is
\begin{equation*}
  U_{0,f}: \Adm {k_R} \to \Set, \quad S \mapsto \left\{(x_0,x_1, \dots) \in
                              \prod_\N S^\cici \mid \bar f(x_{i+1}) = x_i\right\}.
\end{equation*}
Here, $\bar f$ is the image of $f$ under the reduction map $R\llbr T \rrbr \to
k_R\llbr T \rrbr$. 

In the sequel, we denote $R$-algebras by $S$ and write $J$ for an ideal
of definition containing the image of $I$ (provided, for example, by \ref{lem:iodimage}).
Given an element $f\in \cP_R$, we do not distinguish between $f$ and a choice of a 
representative $\tilde f \in \cP_{R,q}$ for some sufficiently large $p$-power.

The following observation lays the groundwork for many of the upcoming results.
\begin{lem}\label{lem:cryscalc}
  Let $f$ be any power series in $\cP_R$. For any two elements $s_1,s_2 \in S$ 
  with $s_1 \equiv s_2 \mod J$ such that $f(s_1)$ and $f(s_2)$ exist (for
  example if $f$ is a polynomial or $s_1, s_2 \in S^\cici$), we have 
  \begin{equation*}
    f^k(s_1) \equiv f^k(s_2) \pmod {J^{k+1}}.
  \end{equation*}
  Here, $f^k$ denotes $k$-fold composition of $f$.
\begin{proof}
  We will show that if $s_1 \equiv s_2$ mod $J^k$, then $f(s_1) \equiv f(s_2)$ mod 
  $J^{k+1}$, which suffices to prove the claim. 
  We may write $s_2 = s_1 + r$ for some $r\in J^k$. By the assumptions on $f$
  there exist power series 
  $g,h \in R\llbr T \rrbr$ such that $h$ only
  has coefficients in $I$ and $f(T) = g(T^q) + h(T)$. As $I$ is finitely generated,
  say by elements $(r_1, \dots, r_l)$, we obtain a representation 
  \begin{equation*}
    f(s_1) - f(s_2) = g(s_1^{q})-g(s_2^{q}) + \sum_{i=1}^l r_i \left(h_i(s_1) -
    h_i(s_2)\right).
  \end{equation*}
  As $r$ divides $\left(h_i(s_1) - h_i(s_2)\right)$, we find
  $r_i(h_i(s_1) - h_i(s_2)) \in (r_i r) \subseteq J^{k+1}$. Also note that 
  for any $s\in S$ and $n \in \N$, 
  $$(s+r)^{nq} = s^{nq} + nqrs^{nq-1}r + \dots + r^{nq},$$
  so after cancellation, all monomials of $g(s_1^q) - g(s_2^q)$ lie in
  $(qr)$ or $(r^2)$. This implies
  \begin{equation*}
    g(s_1^q) - g\left((s_1+r)^q\right) \in (qr) + (r^2) \subseteq J^{k+1},
  \end{equation*}
  and we are done.
\end{proof}
\end{lem}

\begin{lem}\label{lem:reductioniso}
  The natural reduction map 
  \begin{equation*}
    U_f(S) \to U_{f}(S/J) = U_{0,f}(S/J) 
  \end{equation*}
  is bijective.
\begin{proof}
  We first show surjectivity. Given a sequence $(x_0, x_1, \dots) \in U_{f}(S/J)$, 
  we can choose a sequence of arbitrary lifts $(y_0, y_1, \dots ) \in \prod_\N
  S^\cici$ and set 
  \begin{equation*}
    z_i = \lim_{r \to \infty} f^r(y_{i+r}).
  \end{equation*}
  The limit exists, because if $s\geq r$ are two non-negative integers, we calculate
  \begin{equation*}
    f^{s-r}(y_{i+s}) \equiv \bar f^{s-r}(x_{i+s}) = x_{i+r} \equiv y_{i+r}
    \pmod J,
  \end{equation*}
  implying by Lemma \ref{lem:cryscalc} that 
  \begin{equation*}
    f^{s}(y_{i+s}) \equiv f^r(y_{i+r}) \pmod{J^r}.
  \end{equation*}
  This shows that $(f^{r}(y_{i+r}))_{r \in \N}$ is a Cauchy-sequence for the 
  $J$-adic topology on $S$, thereby convergent (cf. Lemma
  \ref{lem:AdmAdicComp}). The sequence $(z_0, z_1, \dots)$ now lies in $U_f(S)$
  and lifts $(x_0, x_1, \dots)$. It remains to show that the lift is unique.
  Suppose that $(z'_0, z'_1, \dots)$ is another lift. Then, for any $i,k \in\N$
  we have $z_{i+k} \equiv z'_{i+k}$ mod $J$, and another application of Lemma
  \ref{lem:cryscalc} shows that 
  \begin{equation*}
    z_i = f^k(z_{i+k}) \equiv f^k(z'_{i+k}) = z'_i \pmod {J^k}.
  \end{equation*}
  Thereby $(z_i - z'_i) \in \bigcap_{k \in \N} J^k = \{0\}$. Hence,
  the lift is unique.
\end{proof}
\end{lem}

We write $\Nilp^\flat$ for the functor $U_{T^q}$. That is, 
$\Nilp^\flat(S) = \lim_{x \mapsto x^q}S^\cici$ is 
the set of $q$-power compatible sequences with values in $S^\cici$. 

\begin{lem}\label{lem:nilp0iso}
  For any $f \in \cP_R$, there is a canonical \todo{Use different $S$}
  bijection $U_{0,f}(S/J) \to \Nilp^\flat(S/J)$. This bijection is functorial in 
  $S$.
\begin{proof}
  By assumption on $f$ we have $f(T) = g(T^{q}) \in k_R\llbr T \rrbr$ for some 
  $g(T) = c_1T + c_2T^2 + \dots$ with $c_1 \neq 0$. For each coefficient $c_i$, let
  $d_i \in k_R$ be the unique element such that $d_i^{q} = c_i$. Let
  $h(T) \in k_R\llbr T \rrbr$ be the power series given by $d_1 T + d_2
  T^2 + \dots$. Now $(h(T))^{q}=f(T)$, and we find that 
  \begin{equation*}
      U_f(S/J) \to \Nilp^\flat(S/J): \quad
      (x_1, x_2, x_3, \dots) \mapsto (x_1, h(x_2), h(h(x_3)), \dots)
  \end{equation*}
  is a well-defined function, and functorial in $S$. For the
  inverse, let $h^{-1}(T) \in k_R\llbr T \rrbr$ be the unique power
  series with $h^{-1}(h(T))= h(h^{-1}(T)) = T$, see Lemma
  \ref{lem:IsosCheckOnLie}. The map
  \begin{equation*}
      \Nilp^\flat(S/J) \to U_f(S/J), \quad 
      (x_1, x_2, \dots ) \mapsto (x_1, h^{-1}(x_2), h^{-1}(h^{-1}(x_3)), \dots)
  \end{equation*}
  is well-defined as
  \begin{equation*}
      f(h^{-1}(T)) = g((h^{-1}(T))^{q}) = (h(h^{-1}(T)))^{q} =
      T^{q},
  \end{equation*}
  and it is readily seen to be inverse to the map constructed above.
\end{proof}
\end{lem}

We collect results.
\begin{prop}\label{prop:pHTcalc}
  Given $f,g \in \cP_R$, we have bijections, functorial in $S$,
  \begin{equation} 
    U_f(S) \to U_f(S/J) \to \Nilp^\flat(S/J) \to U_g(S/J) \to U_g(S).
  \end{equation}
  Explicitly, the bijection $U_f(S) \to U_g(S)$ can be described as follows.
  Suppose that $f,g \in \cP_{R,q}$ for some sufficiently large $q$. 
  Let $h_f(T)$ and $h_g(T)$ be power series with coefficients in $A$ such that 
  $$h_f(T)^q \equiv f(T) \pmod I\quad\text{and}\quad h_g(T)^q \equiv g(T) \pmod
  I.$$
  Write $h_g^{-1}(T)$ for the (formal) inverse power series of $h_g$. 
  Now the isomorphism is given by the mapping
  \begin{equation*}
    (x_0, x_1, \dots) \mapsto (y_0, y_1, \dots), \quad \text{where} \quad y_i =
    \lim_{r \to \infty} g^r(h_g^{-(r+i)}(h_f^{r+i} (x_{i+r}))).
  \end{equation*}
  Here, the exponents are to be interpreted as iterated composition.
\begin{proof}
  The first part follows directly from repeated application of the previous
  two Lemmas. The second part follows by tracing through the previous lemmas.  
\end{proof}
\end{prop}
% subsubsection Useful Calculations (end)

\subsubsection{Applications to the Universal Cover} % (fold)
\label{ssub:The Universal Cover}

Fix a coordinate $\cF \cong \spf(R\llbr T \rrbr)$ so that $\cF = \FGG(F)$ for
some $A$-module law $F \in \FMLOver AR$. Then $[\varpi]_F(T) \in \cP_R$, and
we obtain an isomorphism $\tilde \cF \cong U_{[\varpi]_F} \eqqcolon \tilde F$.
Write $F_0 = F \otimes k_R$, and $\tilde F_0 = U_{0, [\varpi]_F}$. 

\begin{lem}
  We have an isomorphism 
  \begin{equation*}
    \tilde \cF_0 \cong \Nilp^\flat_{k_R}
  \end{equation*}
  of functors $\Adm {k_R} \to \Set$
  \begin{proof}
    Any lift of $[\varpi]_{F_0}(T) \in k_R\llbr T \rrbr$ lies inside $\cP_R$. Hence,
    the statement is an application of Lemma \ref{lem:nilp0iso}.
  \end{proof}
\end{lem}

\begin{lem}
  Suppose that $S$ is an admissible $R$-algebra admitting an ideal of definition
  $J$ such that $\varpi \in J$. Then the natural reduction map
  \begin{equation*}
    \tilde \cF(S) \to \tilde \cF(S/J) = \tilde \cF_0(S/J)
  \end{equation*}
  is an isomorphism.
  \begin{proof}[Proof]
    After choosing a coordinate $\cF = \FGG(F)$, we have $[\varpi]_F \in \cP_R$
    and hence $\tilde \cF(S) \cong U_{[\varpi]_F}$. Thereby the statement is
    given by Lemma \ref{lem:reductioniso}.
  \end{proof}
\end{lem}

The following is analogous to Proposition \ref{prop:pHTcalc}.
\begin{prop}\label{prop:UnivCoverReductionIso}
  Let $S$ be an admissible $R$-algebra with ideal of definition $J$ such that 
  $\phi(I) \subseteq J$. Then there are canonical isomorphisms (of sets)
  \begin{equation*}
    \tilde \cF(S) \cong \tilde \cF(S/J) = \tilde \cF_0(S/J) \cong \Nilp^\flat(S/J) \cong
    \Nilp^\flat(S).
  \end{equation*}
  In particular, $\tilde \cF(S)$ is, as a functor to $\Set$, representable by
  $\spf(R \llbr T^{q^{-\infty}} \rrbr)$.
\end{prop}
We write $\lambda$ for the isomorphism $\Tilde \cF \to \Nilp^\flat$, and
$\lambda_i: \Tilde \cF \to (-)^\cici$ for projection on the $i$-th component.
Similarly, we write $\mu: \Nilp^\flat \to \Tilde \cF$ for the inverse of 
$\lambda$ and $\mu_i$ for the $i$-th component of $\mu$. 

By the proposition above, quasi-isogenies on $\cF_0$ induce isomorphisms on 
$\tilde \cF$. This will be used to construct an action of $D^\times$ on 
$\tilde \cF$ below. 
The relative Frobenius morphism lifts as well.
\begin{defi}[Relative Frobenius on $\Tilde \cF$] \label{def:FrobOnUnivCov}
  Write $\Pi: \Tilde \cF \to \Phi^{-1,*}\Tilde \cF$ for the isomorphism coming from
  the Frobenius quasi-isogeny 
  \begin{equation*}
    \Frob_q: \cF_0 \to \cF_0^{(q)} = \Phi^{-1,*} \cF_0.
  \end{equation*}
\end{defi}

We finally note the following auxiliary result.
\begin{lem}\label{lem:UnivCoverIsRatTateifDiscrete}
  Let $S$ be a discrete admissible $R$-algebra. Then
  \begin{equation*}
    V_\varpi \cF(S) = \tilde \cF(S).
  \end{equation*}
\begin{proof}
  As $S$ is discrete, each $s \in S^\cici$ is nilpotent. From here it is easy to 
  see that there is some $m \in \N$ such that $[\varpi^m]_F(s) = 0$. The desired 
  statement follows directly.
\end{proof}
\end{lem}

% subsubsection The Universal Cover (end)
\subsection{The Quasilogarithm Map} % (fold)
\label{sub:The Quasilogarithm map}
We keep the assumptions on $A$, $R$ and $S$ from the previous subsection. That is,
$A$ is a local ring with finite residue field and uniformizer $\varpi$, 
$R$ is a local $A$-algebra with maximal ideal $I$ complete with respect to
the $I$-adic topology and algebraically closed residue field $k_R$, and 
$S$ denotes an admissible $R$-algebra (where $R \to S$ is continuous with
the $I$-adic topology on $R$) with ideal of definition $J \subseteq S$ containing
the image of $I$. 

The aim of this subsection is to define, attached to any $\varpi$-divisible formal
$A$-module $\cF$ over $R$, a map
\begin{equation*}
\qlog_\cF: \tilde \cF(S) \to \Dio(\cF) \otimes_R (S \otimes_A K),
\end{equation*}
called the quasi-logarithm map.
We give an explicit description of this map if $\cF = \FGG(H)$ is the standard
$\cO_K$-module over $\cO_{\breve K}$. 

The construction of $\qlog_\cF$ is as follows.  
Let $0 \to \cV \xto\psi \cE \xto\phi \cF \to 0$ be the universal additive
extension of $\cF$. For any sequence $(x_1, x_2, \dots) \in \tilde \cF(S)$, choose an arbitrary sequence $(y_1, y_2, \dots) \in
\tilde \cE(S)$ such that $y_i$ is a lift of $x_i$ under the map $\cE(S) \to \cF(S)$. 
Let $y$ be the limit $y = \lim_{i \to \infty} [\varpi]_{\cE}^i(y_i)$ and put 
$$\qlog_\cF((x_1, x_2, \dots)) = \log_\cE(y) \in \Dio(\cF) \otimes_R(S \otimes_A K).$$ 

\begin{prop}
  This construction yields a well-defined map. 
\begin{proof}
  We may assume that $\cF$ and $\cV$ come from formal module laws $F$ and 
  $V$, and we may furthermore assume that $\cE = \FGG(E)$ 
  for an $\cO_K$-module law $E$ obtained by Lemma \ref{lem:SESStandardForm}. Now
  $(x_1, x_2, \dots)$ is a sequence in $S^\cici$ and $(y_1, y_2, \dots)$ is a
  sequence of elements in $(S^\cici)^n$.

  It suffices to show that $y = \lim_{i\to \infty} [\varpi]^i_E(y_i)$ exists and 
  that it is independent of the choice of lifts $(y_1, y_2, \dots)$. 
  Both claims follow from the additivity of $\cV$, implying that 
  $[\varpi]_{V}(T) = \varpi T$. The sequence 
  $([\varpi^i]_{E}(y_i))$ converges, as for positive integers $i\leq j$, we have 
  \begin{equation*}
    [\varpi^i](y_i) - [\varpi^j](y_j) = [\varpi^i]([\varpi^{i-j}]y_j - y_i)
    \in \psi(\varpi^i (S^\cici)^{n-1}) \subseteq J^i (S^\cici)^{n}. 
  \end{equation*}

  If $(y_1', y'_2, \dots)$ is another sequence of lifts, put
  $y' = \lim_{i \to \infty} [\varpi^i]_E (y_i') \in S^\cici$. Now there exists some
  $z \in \cV(S)$ such that 
  $y - y' = \psi(z)$. But by construction
  $z \in \bigcap_{i \in \N} \varpi^i(S^{\cici})^{n-1} = 0$.
\end{proof}
\end{prop}

Let us now consider the case where $\cF = \FGG(H)$ comes from the standard
formal $\cO_K$-module of height $n$ over $\cO_{\breve K}$. Then 
from Proposition \ref{prop:InterestingSequenceStdMod} we have the distinguished
basis elements of $\Ext(H,\Ghat_a)$ corresponding to the symmetric $2$-cocycles
$\delta f_i$, $1 \leq i \leq n-1$ where $f_i(T) = \frac 1 \varpi
\log_H(T^{q^i})$. Also recall that, setting $f_0(T) = \log_H(T)$, the elements 
$(f_0, f_1, \dots, f_{n-1})$ freely generate $\QLog(H)$. The universal
additive extension now corresponds to the symmetric $2$-cocycle $(\delta f_1,
\dots, \delta f_{n-1}) \in \SymCoc^2(H,V)$.
We can make the quasi-logarithm map explicit.

\begin{prop}\label{prop:qlogmapExplicit}
  Let $x = (x_0, x_1, \dots) \in \Tilde H(S)$. With respect to the basis
  $(\log_H(T),\allowbreak \log_H(T^q), \allowbreak \dots, \allowbreak \log_H(T^{q^{n-1}}))$ of 
  $\QLog(H) \otimes_{\cO_K} K$, the quasi-logarithm map is given by
  \begin{equation*}
    \qlog_H(x) = \left(\log_H(x_0), \log_H((\Pi x)_0), \dots, \log_H((\Pi^{n-1}
    x)_0)\right) \in (S\otimes K)^n.
  \end{equation*}
  Here, $\Pi x = ((\Pi x)_0, (\Pi x)_1, \dots)$ is the image of $x$ under
  $\Pi$, the automorphism of $\Tilde
  H(S)$ induced by the (relative) Frobenius quasi-isogeny on $H_0$, cf. Definition
  \ref{def:FrobOnUnivCov}.
\end{prop}
We postpone the proof to state the following auxiliary result.
\begin{lem}\label{lem:FrobOnTildeHExpl}
  Let $x = (x_0 ,x_1, \dots) \in \Tilde H(S)$. For positive integers $i$ and
  $j$ we have
  \begin{equation*}
    \log_H((\Pi^j x)_i) = \lim_{r \to \infty} \varpi^r \log_H(x_{r+i}^{q^j}).
  \end{equation*}
\begin{proof}
  Tracing through the commutative square (with $\lambda$ and $\mu$ the
  isomorphisms from the previous subsection)
    \begin{equation*}
    \begin{tikzcd}[ampersand replacement=\&]
    	{\Tilde H(S)} \& {\Nilp^\flat(S)} \\
    	{\Tilde H(S)} \& {\Nilp^\flat(S),}
    	\arrow["\lambda", from=1-1, to=1-2]
    	\arrow["\Pi", from=1-1, to=2-1]
    	\arrow["{(y_i)_i \mapsto (y_i^{q})_i}", from=1-2, to=2-2]
    	\arrow["\mu"', from=2-2, to=2-1]
    \end{tikzcd}
    \end{equation*}
    we find 
    \begin{equation} \label{eq:CompsOfPiExplicit}
      (\Pi^j x)_i = \lim_{s \to \infty} \lim_{r\to\infty} \left([\varpi]_H^s
      (x_{r+s+i}^{q^{nr + j}})\right).
    \end{equation}
    The claim follows after applying $\log_H$ and making repeated use of the
    functional equation $\log_H(T^{q^n}) = \varpi \log_H(T) + \varpi T$.
\end{proof}
\end{lem}
\begin{proof}[Proof of Proposition \ref{prop:qlogmapExplicit}]
  Using the coordinates provided by $(\delta f_1, \dots, \delta f_{n-1})$, the
  universal additive extension of $H$ is isomorphic to 
  \begin{equation*}
    0 \to \Ghat_a^{n-1} \to E \to H \to 0,
  \end{equation*}
  where $E$ is a module law with 
  \begin{equation*}
    [\varpi]_E(\bX, T) = \big(\varpi X_1 + (\delta_{\varpi}f_1)(T), \dots, \varpi X_{n-1} + 
    (\delta_{\varpi}f_{n-1})(T), [\varpi]_H(T)\big).
  \end{equation*}
  Beginning with $x = (x_0, x_1, \dots) \in \Tilde H (S)$, lifting to $(y_0,
  y_1, \dots) \in E(S)^\N$ 
  and writing $y = \lim_{i\to\infty} [\varpi]_E^i (y_i)$, we find
  \begin{equation*}
    y = \left(\lim_{r\to\infty} (\delta_{\varpi^r} f_1)(x_r), \dots, \lim_{r\to \infty}
    (\delta_{\varpi^r}f_{n-1})(x_r), x_0\right) \in E(S).
  \end{equation*}
  Now, Lemma \ref{lem:FrobOnTildeHExpl} provides the equality
  $$\lim_{r\to\infty}\delta_{\varpi^r} f_i(x_r) = \frac 1\varpi \lim_{r \to \infty}
  \varpi^r \log_H(x_r^{q^{nr+i}}) - \frac 1\varpi \log_H\left(x_0^{q^i}\right) = 
  \frac 1\varpi \left(\log_H((\Pi^i x)_0)- \log_H(x_0^{q^i})\right).$$
  We need to calculate $\log_E(y)$, which calls for an explicit description of 
  $\log_E \colon E \otimes (R \otimes_AK) \to (\Ghat_a \otimes(R \otimes_A K))^n$. 
  Tracing through the procedure provided in Subsection \ref{sub:Logarithms},
  we find
  \begin{equation*}
    \log_E(\bX, T) = \left(X_1 + \tfrac 1\varpi \log_H(T^q), \dots, 
    X_{n-1} + \tfrac 1\varpi \log_H(T^{q^{n-1}}), \log_H(T)\right).
  \end{equation*}
  This representation is with respect to the basis $(f_1, \dots, f_{n-1}, f_0)$. 
  The claim follows.
\end{proof}

% subsubsection The Quasilogarithm map (end)
\subsection{An Approximation of the Determinant Morphism} % (fold)
\label{sub:Determinants}
Let $H$ be the standard formal $\cO_K$-module over $\cO_{\breve K}$ of height
$n$. Write $\wedge H$ for the formal $\cO_K$-module over $\cO_{\breve K}$ with
logarithm
\begin{equation*}
  \log_{\wedge H}(T) = \sum_{i = 0}^\infty (-1)^{(n-1)i} \frac{T^{qi}}{\varpi^i}.
\end{equation*}
By Hazewinkel's integrality Lemma (cf. Theorem \ref{thm:HazewinkelIntegrality}), 
such a module law exists. We have $\Dio(\wedge H) = \wedge^n \Dio(H)$. 
We follow \cite[Theorem 2.10.3]{BoyarchenkoWeinstein2011MaxVar} to describe a map $\delta:
\Tilde H^n \to \Tilde {\wedge H}$ making the square
\begin{equation}\label{diag:qlogsquare}
\begin{tikzcd}[ampersand replacement=\&]
	{\Tilde H^n(S)} \& {\Tilde{\wedge H}(S)} \\
	{\Dio(H)^n \otimes (S\otimes_{\cO_K} K)} \& {\Dio(\wedge H) \otimes(S\otimes_{\cO_K}K)}
	\arrow["\delta", from=1-1, to=1-2]
	\arrow["\qlog_H \times \dots \times \qlog_H"', from=1-1, to=2-1]
  \arrow["\qlog_{\wedge H}", from=1-2, to=2-2]
	\arrow["\det", from=2-1, to=2-2]
\end{tikzcd}
\end{equation}
commute. 

Let $(s_1, \dots, s_n) \in \Tilde H(S)^n$, and write $x_i = \lambda(s_i) \in
\Nilp^\flat(S)$, which are elements in $S^\cici$ with distinguished $q$-power
roots. Here $\lambda: \Tilde H \to \Nilp^\flat$ is the isomorphism from Section
\ref{sub:Tate Modules and the Universal Cover} with inverse $\mu = (\mu_0,
\mu_1, \dots)$. 
We set
\begin{equation*}
  \delta_0(s_1, \dots, s_n) = \sum_{(a_1, \dots, a_n)} \varepsilon(a_1, \dots
  a_n) \mu_0(x_1^{q^{a_1}} \cdots x_n^{q^{a_n}}) \in \wedge H(S),
\end{equation*}
where 
\begin{itemize}
  \item The sum takes place in ${\wedge H}(S)$.
  \item The sum ranges over $n$-tuples $(a_1, \dots, a_n)$ of (possibly negative) integers 
    satisfying $a_1 + \dots + a_n = n (n-1)/2$, subject to the
    condition that each $a_i$ occupies a distinct residue class modulo $n$.
  \item The expression $\varepsilon(a_1, \dots, a_n)$ denotes the sign of the 
    permutation $i \mapsto a_{i+1}$ (mod $n$) of $(0, \dots, n-1)$.
\end{itemize}

\begin{prop}\label{prop:commutativityofdeterminantqlogdiag}
  The map $\delta_0$ makes the diagram 
  \begin{equation*}
    \begin{tikzcd}[ampersand replacement=\&]
    	{\tilde H^n(S)} \& {\wedge H(S)} \\
    	{\Dio(H)^n \otimes (S \otimes K)} \& {\Dio(\wedge H) \otimes(S \otimes K)}
    	\arrow["{\delta_0}", from=1-1, to=1-2]
    	\arrow["{\qlog_H^n}"', from=1-1, to=2-1]
    	\arrow["{\log_{\wedge H}}", from=1-2, to=2-2]
    	\arrow["\det", from=2-1, to=2-2]
    \end{tikzcd}
  \end{equation*}
  commute. It is $\cO_K$-multilinear and alternating.
\begin{proof}
  This is part of the proof of \cite[Theorem
  2.10.3]{BoyarchenkoWeinstein2011MaxVar}.
  Commutativity follows from 
  \begin{multline*}
    \log_{\wedge H}(\delta_0(s_1, \dots, s_n)) = \sum_{(a_1, \dots, a_n)} 
    \varepsilon(\mathbf a) \log_{\wedge H} \mu_0( x_1^{q^{a_1}}\cdots x_n^{q^{a_n}})
     \\ = \sum_{(a_1, \dots, a_n)} \varepsilon(\mathbf a) \sum_{m \in \Z} (-1)^{(n-1)m}
    \frac{x_1^{q^{a_1+m}} \cdots x_n^{q^{a_n+m}}}{\varpi^m} 
    = \det \left( \sum_{m \in \Z} \frac{x_i^{q^{mn + j-1}}}{\varpi^m} \right)_{\substack{
      1 \leq i \leq n,\\
      1 \leq j \leq n}},
  \end{multline*}
  which is equal to $\det(\qlog_H^n(s_1, \dots, s_n))$ by Proposition 
  \ref{prop:qlogmapExplicit} and Lemma \ref{lem:LogInTermsOfNil}. 
  The fact that $\delta_0$ is multilinear and alternating ultimately follows
  from the corresponding properties of $\det$, the fact that $\ker(\log_H) =
  \wedge H[\varpi^\infty]$ (cf. Lemma \ref{lem:KernelOfLog}) and topological
  considerations in the induced diagram in the category of adic spaces over
  $(\breve K, \cO_{\breve K})$.
\end{proof}
\end{prop}

This allows us to define the sought for morphism of functors
$\delta: \Tilde H^n \to \Tilde {\wedge H}$. 
\begin{defi}\label{def:DeltaMap}
  Put $\delta_i(s_1, \dots, s_n) = \delta_0(\varpi^{-i} s_1, \dots, s_n)$. Then 
  $\delta = (\delta_0, \delta_1, \dots)$ yields a map 
  $\Tilde H^n \to \Tilde{\wedge H}$. It is $K$-multilinear and alternating.
\end{defi}

Using the canonical identifications $\Tilde {H}^n \cong (\Nilp^\flat)^n$
and $\Tilde{\wedge H} \cong \Nilp^\flat$, the morphism $\delta$ yields a 
map $(\Nilp^\flat)^n \to \Nilp^\flat$, which in turn is the same as a power series 
\begin{equation*}
  \Delta(X_1, \dots, X_n) \in \cO_{\breve K}\llbr X_1^{q^{-\infty}}, \dots,
  X_n^{q^{-\infty}}\rrbr
\end{equation*}
together with distinguished $q$-th power roots. 
We have the following approximation of $\Delta$, cf. \cite[Lemma 2.10.4]{BoyarchenkoWeinstein2011MaxVar}. 
\begin{lem}\label{lem:DeltaApprox}
  We have 
  \begin{equation*}
    \Delta(X_1, \dots, X_n) \equiv \det(X_i^{q^{j-1}})_{\substack{1 \leq
  i \leq n,\\ 1 \leq j \leq n}} 
  \end{equation*}
  modulo terms of degree greater than $1 + q + \dots + q^{n-1}$.
\begin{proof}
  By Proposition \ref{prop:commutativityofdeterminantqlogdiag} and 
  the explicit description of the quasi-logarithm map in Proposition
  \ref{prop:qlogmapExplicit}, we have the equality
  \begin{equation*}
    \sum_{k = -\infty}^\infty (-1)^{(n-1) k} \frac{\Delta(X_1, \dots,
    X_n)^{q^k}}{\varpi^k} = 
    \det \left( \sum_{k = -\infty}^\infty
    \frac{X_i^{q^{nk + j-1}}}{\varpi^k} \right)_{1 \leq i, j \leq n}
  \end{equation*}
  of elements inside $\br E \llbr X_1^{q^{-\infty}}, \dots, X_n^{q^{-\infty}} \rrbr$
  (equipped with the topology induced from the 
  $(\varpi, X_1, \allowbreak \dots, X_n)$-adic topology on $\cO_\br E\llbr
  X_1^{q^{-\infty}}, \dots, X_n^{q^{-\infty}} \rrbr$).
  The claim follows after comparing coefficients of the respective series. 
\end{proof}
\end{lem}

% subsubsection Determinants of Formal Modules (end)


% subsection Determinants of Formal Modules (end)

\end{document}
