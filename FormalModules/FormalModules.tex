%! TeX root: ../main.tex
\documentclass[../main.tex]{subfiles}

\begin{document}

\section{Formal Modules}
This section will serve as an introduction to formal groups and 
formal modules. Formal groups (or rather, formal group laws) were first
introduced by \textsc{Salomon Bochner} in 1946 as a natural means of studying Lie
Groups over fields of characteristic $0$, cf. \cite{Bochner1946FGrps}. 
The study of formal groups later became interesting for its own right, 
with pioneering works of Lazard \cite{Lazard1955FGrps}. \todo{blabla}

\subsection{BasicNotions.tex} 
\label{sub:Basic Notions}
As promised in the introduction, we begin by defining {formal group
laws}.

\begin{defi}[Formal Group Law]
    Let $R$ be a ring. A (commututative, one-\allowbreak
    dimen\-sional) formal group law over $R$ is a power series $F(X,Y) \in
    R\llbr X, Y \rrbr$ such that $F(X,Y) \equiv X + Y$ modulo terms
    of degree $\geq 2$ and
    the following properties are satisfied:
    \begin{itemize}
        \item $F(F(X,Y),Z) = F(X,F(Y,Z))$,
        \item $F(X,Y) = F(Y,X)$,
        \item $F(X,0) = X$.
    \end{itemize}
\end{defi}

Given two formal group laws $F, G \in R\llbr X,Y\rrbr$, a morphism
$f: F\to G$ is a 
power series $f \in R\llbr T \rrbr$ such that $f(0) = 0$ and $f(F(X,Y)) =
G(f(X),f(Y))$.
To any Such a series is an isomorphism if there is an {inverse}, that
is, a power series $g \in R\llbr T \rrbr$ with $(f \circ g)(T) = T$.
This yields the category of formal group laws over $R$, which we notate by
$\FGLOver R$.

The following statements about morphisms of formal group laws are 
useful and easily verified.
\begin{lem}\label{lem:FGLeasyfacts}
  Let $R$ be a ring and let $F,G \in R\llbr X,Y\rrbr$ be two formal
  group laws over $R$. 
  \begin{enumerate}
    \item Given two morphisms $f,g : F \to G$, the power series $G(f(T), g(T))
      \in R\llbr T \rrbr$ is a morphism of formal group laws 
      $F \to G$. In particular, $\Hom_\FGLOver R(F,G)$ is an abelian
      group for any two formal group laws $F,G$.
    \item The abelian group $\End_\FGLOver R(F)$ has a natural ring structure
      with multiplication given by composition.
    \item A morphism $f = c_1 T + c_2 T^2 + \dots \in R\llbr T \rrbr$ between
      $F$ and $G$ is an isomorphism if and only if $c_1 \in R^\times$.
  \end{enumerate}
\end{lem}

\begin{xpl} Let us introduce the following two formal group laws.
  \begin{itemize}
    \item \textit{The additive formal group law}. Write 
      $\GG_a$ for the formal group law with addition given by 
      $\GG_a(X,Y) = X + Y$. 
    \item We write $\GG_m$ for the formal group law associated with the 
      with $\GG_m(X,Y) = X + Y + XY$. Note that $\GG_m(X,Y) = (X+1)(Y+1) - 1$
  \end{itemize}
\end{xpl}

Next up is the definition of formal $A$-module laws. Naively, we would like to
say
that an $A$-module law is the same as that of a formal group law $F$
plus $A$-module structure, i.e. a morphism of rings $[\cdot]_F: A \to
\End_\FGLOver R(F)$. But there is a subtlety here: let 
\begin{equation*}
  \Lie: \FGLOver R \to \Ab 
\end{equation*}
be the (constant) functor that sends $F \in \FGLOver R$ to $(R,+)$, and morphisms
$f: G \to H$ given by a formal power series
$f = c_1 T + c_2 T^2 + \dots \in R\llbr T \rrbr$ to the endomorphism
of $R$ given by multiplication with $c_1$. The condition that 
$F(X,Y) \equiv X+Y$ modulo degree $2$ enforces that the induced map
$\End(F) \to \End(R)$ is a morphism of rings. Now, the $A$-module structure on $F$ 
yields an $A$-module structure on $R$, given by the composition\begin{equation*}
  A \xto{[\cdot]_F} \End(F) \xto{\Lie} \End(R), \quad a \mapsto \Lie([a]_F)
\end{equation*}

This is a morphism of rings, and we obtain an $A$-algebra structure on $R$. We would
like the $A$-algebra structure on $R$ to be uniform. This
motivates the following definition.
\begin{defi}[Formal $A$-module law]
  Let $A$ be a ring and $R$ be an $A$-algebra with structure
  morphism $p: A \to R$. A (one-dimensional) $A$-module law over $R$ is a
  pair $(F, ([a]_F)_{a \in A})$, where $F \in R\llbr X,Y \rrbr$ is 
  a formal group law and $[a]_F = p(a)X + c_2X^2 + \dots 
  \in R\llbr X \rrbr$ yield endomorphisms $F \to F$ such that the induced map
  \begin{equation*}
    A \to \End(F), \quad a \mapsto [a]_F 
  \end{equation*}
  is a morphism of rings.
\end{defi}
Similarly to above, we obtain a category of formal $A$-module laws over $R$,
which we denote by $\FMLOver A R$. Note that $\FGLOver R \cong \FMLOver \Z R$.
Slightly abusing notation, we usually do not explicitely mention the $A$-structure
when referring to formal module laws, simply writing $F \in \FMLOver AR$, for
example. 

The following lemma explains a the functoriality of the assignment
$R \mapsto \FMLOver A R$.

\begin{lem}\label{lem:FMLFunc}
  The assignment $R \mapsto \FMLOver A R$ is functorial in the following sense.
  If $p: R \to R'$ is a morphism of $A$-algebras, we obtain a functor
  \begin{equation*}
    \FMLOver A R \to \FMLOver A {R'}, \quad F \mapsto p_*F,
  \end{equation*}
  where $p_*F$ is the formal $A$-module law obtained by applying $p$ to the
  coefficients of the formal power series representing addition
  and scalar multiplication of $F$. We sometimes write
  (with abuse of notation) $p_* F = F \otimes_R R'$. 
\end{lem}

Note that every formal module law $F \in \FMLOver A R$ yields a functor
\begin{equation}\label{eq:fmnilpfunc}
  \Alg R \to \Mod A, \quad S \mapsto \Nil(S),
\end{equation}
where $\Nil(S)$, the set of nilpotent elements of $S$, is equipped with
addition and scalars given by 
\begin{equation*}
  s_1 + s_2 = F(s_1, s_2) \in \Nil(S), \quad a s = [a]_F(s) \in \Nil(S).
\end{equation*}
This construction yields a functor (with slight abuse of notation)
\begin{equation}\label{eq:formfunc}
  \FMLOver A R \to \Fun(\Alg R, \Mod A),
\end{equation}
where $\Fun$ denotes the functor category.

Passing from discrete $R$-algebras to admissible $R$-algebras, this construction extends
naturally to a functor 
\begin{equation*}
  \Spf^F: \FMLOver AR \to \Fun (\Adm R, \Mod A), \quad F \mapsto \Spf R\llbr T \rrbr,
\end{equation*}
where we equip $\Spf R \llbr T \rrbr$ with the structure of an $A$-module object
using the endomorphisms coming from $F$. 
Following this line of thought leads naturally to the definition of
formal modules. 

\begin{defi}[Formal Group and Formal Module.]
  Let $X$ be an $A$-scheme, and let 
  let $\cF$ be an $A$-module object in $\FSchOver X$, the category of formal
  schemes over $X$. Suppose that there is a Zariski-covering
  $(\spec(R_i))_{i \in I}$ of $X$ with $\cF \times_{X} U_i \cong
  \spf(R_i\llbr T \rrbr).$ If for every $i\in I$ the induced $A$-module
  structure on $\spf(R_i\llbr T \rrbr)$ comes from a formal $A$-module law
  $F_i$ over $R_i$, we say that $\cF$ is a formal $A$-module. 
\end{defi}

\begin{rmk} 
  Formal schemes (over a base an $A$-scheme $X$, say) locally isomorphic to 
  $\spf \cO_X(U)\llbr T \rrbr$ are sometimes called (one-dimensional) Formal
  Lie Varieties \todo{reference}. Equivalently to the definition above, we could
  have defined formal $A$-modules as $A$-module objects (of relative dimension one
  over $X$) in the category of Formal Lie Varieties, such that the $A$-module structure
  on the tangent space at the identity agrees with the usual one.
\end{rmk}

\begin{defi}[Coordinate]
  Let $\cF$ be a formal $A$-module over $X$. The choice of a cover $\sqcup_{i
  \in I} \spec(R_i) \to X$ together with isomorphisms $\cF \times_X \spec(R_i)
  \cong \Spf(R_i\llbr T \rrbr)$ will be referred to as a coordinate of $\cF$. 
\end{defi}

Of course there is a functor 
\begin{equation*}
  \FMLOver AR \to \FMOver AR,
\end{equation*}
essentially forgetting the choice of module law. The observation of Lemma 
\ref{lem:FMLFunc} translates to formal modules, a morphism $p : R \to R'$ 
yields a functor 
\begin{equation*} 
  p_*: \FMOver AR \to \FMOver A{R'}, \quad \cF \mapsto \cF \otimes_R R'.
\end{equation*}

\begin{xpl}
  The additive group law $\GG_a$ extends to a formal $A$-module over an affine base
  $\spec R$ by setting 
  \begin{equation*}
    [a]_{\GG_a}(T) = aT
  \end{equation*}
  for $a \in A$. More generally, we obtain a formal $A$-module over an arbitrary base
  scheme.

  The formal group associated to $\GG_m$ over $\Z_p$ is isomorphic to the functor
  \begin{equation*}
    \Adm {\Z_p} \to \Ab, \quad S \mapsto 1 + S^\cici \subset S^\times.
  \end{equation*}
  Here, we equipped $\Z_p$ with the $p$-adic topology.
  The subgroup $1 + S^\cici$ naturally carries the structure of a $\Z_p$-module.
  Indeed, for $k \in \N$, we have
  \begin{equation*}
    (1+s)^{p^k} = 1 + p^ks + \binom{p^k}2 s^2 + \dots + s^{p^k},
  \end{equation*}
  and given $s \in S^\cici$, this is of the form $1+ o(1)$ as $k$ gets large. 
  In particular, if $x = a_0 + a_1 p + a_2p^2 + \dots \in \Z_p$, expressions of the form
  \begin{equation*}
    (1+s)^x = \prod_{i = 1}^\infty (1+s)^{a_k p^k}
  \end{equation*}
  make sense by lemma \ref{lem:infiniteseriesandproducts}. This gives
  $\GG_{m,\Z_p}$ the structure of a formal $\Z_p$-module. 
  In the upcoming subsection, we discuss how this is the simplest example of a
  whole family of formal modules constructed by 
  Lubin and Tate. In section \ref{sec:Local Class Field Theory} we explain applications of these formal modules to local class field theory.
\end{xpl}



test

\subsection{Lubin--Tate Formal Module Laws} % (fold)
\label{sub:Lubin--Tate Formal Module Laws}
We sketch the construction of a family of formal modules introduced by 
Lubin and Tate in \cite{LubinTateFormalMult}.

Let $A$ be a complete discrete valuation ring with 
finite residue field $k$, set $q = \# k$ and let $\pi \in A$ be a choice of a
uniformizer.
Write $\cF_{\pi, h}$ for the following set of power series
\begin{equation*}
  \cF_\pi \coloneqq \{f \in \cO_K \llbr T \rrbr \mid f \equiv \pi T \pmod {T^2}
  \text{ and } f \equiv T^{q^n} \pmod \pi\}. 
\end{equation*}

The construction of Lubin--Tate formal module laws rests on the following 
lemma, which is Lemma 1 in \cite{LubinTateFormalMult}.
\begin{lem}\label{lem:LTLemma1}
  Let $f(T)$ and $g(T)$ be elements of $\cF_{\pi,h}$ and let 
  $L(X_1, \dots, X_n) = \sum_{i=1}^n a_i X_i$ be a linear form with coefficients in 
  $A$. Then there exists a unique series $F(X_1, \dots, X_n)$ with coefficients 
  in $A$ such that 
  \begin{gather*}
    F(X_1, \dots, X_n) \equiv L(X_1, \dots, X_n) \pmod {T^2}, \\ \text{and} \\
    f(F(X_1, \dots, X_n)) = F(g(X_1), \dots, g(X_n)).
  \end{gather*}
\end{lem}

As a direct consequence, we obtain the following useful result.
\begin{lem}
  Let $f \in \cF_{\pi, h}$. Then there is a unique formal $A$-module law $F_f$ over $A$
  with $[\pi]_F(T) = f(T)$.
\begin{proof}
  In the above Lemma, set $L(X,Y) = X+Y$ and $g=f$ to uniquely determine 
  the power series $F_f$. The same Lemma yields unique power series
  $[a]_{F_f}(T) \in R\llbr T \rrbr$ by setting $L(T) = a T$, $g=f$. It is
  routine to check that $(F_f, ([a]_f)_{a \in A})$ is a formal $A$-module law, 
  cf. \cite{LubinTateFormalMult}.
\end{proof}
\end{lem}

\begin{defi}[Lubin--Tate Module Law]
  We refer to module laws arising by the construction above as Lubin--Tate module laws.
\end{defi}

Furthermore, attached to each $a \in \cO_K$ and $f,g \in \cF_{\pi,h}$, we find
unique $[a]_{f,g}(T) \in \cO_K\llbr T \rrbr$ satisfying
\begin{equation}\label{eq:LTMoLaScaCond}
  [a]_{f,g}(T) \equiv aT \pmod {(T)^2} \quad \text{and} \quad
  f([a]_{f,g}(T)) = [a]_{f,g}(g(T)).
\end{equation}
We now have the following theorem.
\begin{thm}[Lubin--Tate Formal $\cO_K$-Module Laws]\label{thm:LTModLaw}
  Let $K$ be a local field with ring of integers $\cO_K$. For any choice of 
  uniformizer $\pi \in \cO_K$ and any $f \in \cF_{\pi,h}$, the family of power
  series $(F_f, ([a]_{f,f})_{a \in \cO_K})$
  gives rise to a formal $\cO_K$-module law over $\cO_K$. For 
  $f,g \in \cF_{\pi,h}$, the formal $\cO_K$-module laws $F_f$ and $F_g$ are
  canonically isomorphic, via the morphism induced by $[1]_{f,g} \in \cO_K\llbr
  T \rrbr$. 
\begin{proof}
  See Theorem 1 of \cite{LubinTateFormalMult} and the succeeding discussion.
\end{proof}
\end{thm}
In particular, up to canonical isomorphism, there is only one Lubin--Tate formal
$\cO_K$-module law over $\cO_K$ attached to the choice of the uniformizer $\pi \in
\cO_K$. 

\begin{xpl}
  If $K = \Q_p$, this reconstructs the multiplicative formal 
  $\Z_p$ module $\GG_m$ constructed above. Indeed, we have 
  \begin{equation*}
    \cF_p = \{f \in \Z_p\llbr T \rrbr \mid f(T) \equiv T^p \text{ mod } p
    \text{ and } f(T) \equiv pT \text{ mod } (T)^2 \},
  \end{equation*}
  implying that $f(T) = (1+T)^p-1$ lies in $\cF_p$.  
  One quickly checks that 
  \begin{equation*}
    F_f(X,Y) = (1+X)(1+Y) - 1 = X + Y + XY \in \Z_p \llbr X,Y \rrbr
  \end{equation*}
  is the addition law associated to $f$, and that 
  for $a \in \Z_p$, the power series
  \begin{equation*}
    [a]_{f,f} = (1+T)^{a} - 1 \in \Z_p \llbr T \rrbr
  \end{equation*}
  satisfies the condition of \eqref{eq:LTMoLaScaCond}. 
\end{xpl}

% subsubsection Lubin--Tate Formal Module Laws (end)




%\subsection{Logarithms} % (fold)
%\label{sub:Logarithms}
%\blue{
Again, $A$ is a complete discrete valuation ring with uniformizing parameter 
$\pi$ and finite residue field $k = A/\pi A$. We write $q$ for the cardinality of 
$k$ and $K$ for the residue field of $A$. Let $R$ be a (commutative) $A$-algebra
with structure map $i: A \to R$.}

We review results from section
2 and 3 of \cite{hopkins1994equivariant}. 
Suppose that $\bF = (F_1, \dots, F_n)$ is an $n$-dimensional formal $A$-module law over an
$A$-algebra $R$. Write $\bX = (X_1, \dots, X_n)$, $\bY = (Y_1, \dots, Y_n)$, etc.

\begin{defi}[Invariant Differentials for module laws.]
  The module $\omega(\bF)$ of invariant differentials is the submodule of the
  $R$-module of differentials
  \begin{equation*}
    \Omega_{R\llbr T_1, \dots, T_n \rrbr/R} \cong \bigoplus_{i=1}^n R\llbr T_1, \dots, T_n
    \rrbr \dc T_i,
  \end{equation*}
  cut out by the condition that any $\omega \in \omega(\bF)$ satisfies
  \begin{equation}\label{eq:diffcond}
    \omega(\bF(\bX,\bY)) = \omega(\bX) + \omega(\bY)\quad \text{and} \quad
    \omega([a]_\bF(\bX)) = a\omega(\bX).
  \end{equation} 
  for all $a \in A$. 
\end{defi}

It is possible to explicitly calculate a basis for the $R$-module
$\omega(\bF)$, which we now explain. Let 
$$A(\bX, \bY) \in \Mat_{n \times n} (R\llbr \bX, \bY \rrbr)$$ 
denote the matrix $\left((\partial/\partial X_j)F_i (\bX,\bY)\right)_{i,j}$,
the derivative of $\bF(\bX,\bY)$ with respect to $\bX$. Write 
$B(\bY) = A(0,\bY)$. Then $B$ is a unit in $\Mat_{n \times n} R\llbr \bY \rrbr$; 
and we write $(C_{ij}(\bY))$ for the components of 
$B(\bY)^{-1}$. We now construct 
$$\omega_{i} \coloneqq \sum_{j=1}^n C_{ij}(\bX) \dc X_j \in \Omega_{R\llbr \bX \rrbr/R}$$ 
for $1 \leq i \leq n$. By definition we have 
\begin{equation}\label{eq:coeffofcanonicaldiff}
  C_{ij}(0) = \begin{cases}
    1 &\text{ if }i = j,\\
    0 &\text{ otherwise.}
  \end{cases}
\end{equation}
Checking that $\omega_{i}$ is an invariant differential is a matter of
applying the chain rule, and we have
\begin{prop}
    The $R$-module $\omega(\bF)$ is free of rank $n$ generated by invariant differentials
    $\omega_{1}, \omega_{2}, \dots, \omega_{n}$.
\begin{proof}
  This is \cite[Proposition 1.1]{1970HondaFormalGroups}. 
\end{proof}
\end{prop}
\begin{xpl}
  The invariant differentials for $\GG_a$ are spanned by the form $\dc X$. 
  The invariant differentials for $\GG_m$ are spanned by the form 
  $\omega_1(X) = \frac 1{1+X} \dc X$.
\end{xpl}
By the Proposition above and Equation \eqref{eq:coeffofcanonicaldiff}, we may
define a pairing
\begin{equation*}
  \omega(\bF) \times \Lie(\bF) \to R, \quad \langle X_i, \omega_j \rangle =
  \begin{cases}
    1 &\text{ if } i = j,\\
    0 &\text{ otherwise.}
  \end{cases}
\end{equation*}
This pairing is independent of the parametrization of $\bF$. In particular, it
descents to a pairing defined for formal modules $\cF \in \FMArbOver A R$, and
we have a natural isomorphism $\omega(\cF) \cong \Hom_R(R, \Lie(\cF))$.

Let $\GG_a$ be the additive formal $A$-module over $R$. There is a map
\begin{equation} \label{eq:functorinvdifftohom}
  \dc_\bF : \Hom_{\FMLOver AR} (\bF, \GG_{a,R}) \to \omega(\bF), \quad f \mapsto \dc f(\bX)
\end{equation}
which is a map of $R$-modules if we equip the left hand side with the $R$-module
structure coming from the natural action of $R \subset \End(\GG_a)$. 
\begin{prop}\label{prop:loginvdiff}
  \begin{enumerate}
    \item If $R$ is a flat $A$-algebra, the map $\dc_F$ is injective.
    \item If $R$ is a $K$-algebra, the map $\dc_F$ is an isomorphism.
  \end{enumerate}
\begin{proof}
  \cite[Proposition 3.2]{hopkins1994equivariant} \todo{PROOF}\red{ Everything is easy if 
  $K$ has characteristic $0$, as we can integrate the differential forms.
  The proof in positive characteristic is a bit tricky; First it is shown that 
  there is an isomorphism of formal goups $F \cong \GG_a$, which is immediate.
  Then that there is a unique homomorphism $f: \GG_a \to \GG_a$ that maps to $\omega_F$
  and behaves well with respect to the $A$-module structure on $F$. }
\end{proof}
\end{prop}

Suppose now that $F \in \FMLOver A R$ is a one-dimensional formal module. 
The previous Proposition shows that if $R$ is a $K$-algebra, the invariant differential 
$\omega_1(X)$ constructed above comes from a homomorphism $f(X) = X + c_2 X^2 + \dots$,
which is an isomorphism by Lemma \ref{lem:FGLeasyfacts}. 
This allows us to define the logarithm attached to $F$.
\begin{defi}[Logarithm and Exponential]
  If $R$ is a flat $A$-algebra, there is a unique power series
  \begin{equation*}
    \log_F(X) = X + c_2 X^2 + \dots \in (R \otimes_A K)\llbr X \rrbr 
  \end{equation*}
  inducing an isomorphism $F \otimes (R \otimes K) \to \GG_{a,R\otimes K}$.
  This power series is called the logarithm attached to $F$.   The inverse of $\log_F$ is called the exponential of $F$ and will be denoted by
  $\exp_F$.
\end{defi}
\begin{rmk} \leavevmode
  \begin{enumerate}
    \item We have $\dc \log_F(X) = \omega_1(X)$, hence $\log'_F(X) \in R\llbr X \rrbr$. 
    \item These definitions do not descent to formal modules. Given $\cF \in \FMOver A
  R$, there is no natural choice of Logarithms and Exponentials, these
  definitions depend on the choice of parametrization
  $\cF \cong \Spf R \llbr T \rrbr$. 
  \end{enumerate}
  \end{rmk}





\subsection{Hazewinkel's Functional Equation Lemma and the Standard Formal
Module Law} % (fold)
\label{sub:Hazewinkels FuncEq and the Standard Formal Module}
If, $A$ is an integral domain and $R$ is a flat $A$-module, the structure of a formal
$A$-module $F$ over $R$ is uniquely determined by its logarithm $\log_H
\in R \otimes_A K \llbr T \rrbr$. Indeed, we find
\begin{equation*}
  F(X,Y) = \exp_H(X+Y), \quad [a]_F(X) = \exp_H(a X).
\end{equation*}
It is therefore natural to wonder about conditions on power series $f \in
(R\otimes_A K) \llbr T \rrbr$ ensuring that $f$ is the logarithm of some
formal group law. Hazewinkel found such a condition in his functional equation
lemma.

\begin{prop}[Hazewinkel's Functional Equation Lemma] 
  Let $p$ be a prime and $q = p^e$. Given an inclusion of rings $B \subseteq
  L$, an ideal $\fa \subseteq B$ containing $p$, an endomorphism of rings
  $\sigma: L \to L$ and elements $s_1, s_2, \dots \in L$ subject to the conditions
  that 
  \begin{equation*}
    \sigma(b) \equiv b^q \pmod \fa \text{ for all } b \in B \quad \text{and} \quad 
    \sigma^r(s_i) \fa \subset B \text{ for all } r,s \geq 1.
  \end{equation*}
  Suppose now that $f \in L\llbr T \rrbr$ has $f'(0) \in L^\times$ and
  satisfies the functional equation condition
  \begin{equation*}
    f(X) - \sum_{i=1}^\infty s_i (\sigma^i_* f)(X^{q^i}) \in B\llbr X \rrbr.
  \end{equation*}
  Then we have 
  \begin{equation*}
    F(X,Y) = f^{-1}(f(X) + f(Y)) \in B \llbr X,Y \rrbr,
  \end{equation*}
  where $f^{-1}$ is the inverse power series as in Lemma \ref{lem:IsosCheckOnLie}.
  Also, if $g(Z) \in L\llbr Z \rrbr$ is another power series satisfying the 
  same condition
  \begin{equation*}
    g(Z) - \sum_{i=1}^\infty s_i (\sigma^i_* f)(Z^{q^i}) \in B\llbr Z \rrbr,
  \end{equation*}
  then $f^{-1}(g(Z)) \in B\llbr Z \rrbr$. 
  Furthermore, if $\alpha(T) \in B\llbr T \rrbr$ and $\beta(T) \in B \llbr T \rrbr$, then
  \begin{equation} \label{eq:funceqlemcongruence}
    \alpha(T) \equiv \beta(T) \pmod {\fa^r} \iff f(\alpha(T)) \equiv f(\beta(T))
    \pmod {\fa^r}
  \end{equation}

  \begin{proof}
    A more general statement can be found in \cite[
    2]{hazewinkel1979funceqexp}. Proofs can be found in \cite[s 2 and
    10]{hazewinkel1978formal}.
  \end{proof}
\end{prop}
Note that by construction, $F(X,Y)$ as defined above yields a (commutative)
formal group law over $B$. 
Let $B^\sigma$ denote the subring of elements in $B$ fixed by $\sigma$. Then 
the second part of the Functional Equation Lemma implies that we even obtain
formal $B^\sigma$-modules with $[b]_F(T) = f^{-1}(b f(T))$, as $bf(T)$
satisfies the same functional equation if $b \in B^\sigma$. 

We now enter the situation where $K$ is a local field with ring of integers
$\cO_K$ and uniformizer $\varpi$ and 
use the Functional Equation Lemma to construct Lubin--Tate Formal Group Laws. 
A special role will play the power series
\begin{equation*}
  f(T) = \sum_{i=1}^\infty \frac{T^{q^{in}}}{\varpi^i} \in K\llbr T \rrbr.
\end{equation*}
It satisfies the functional equation
\begin{equation*}
  f(T) = T + \frac 1\varpi f(T^{q^n}),
\end{equation*}
which is a functional equation of the form above, with 
$B = \cO_K$, $\fa = (\varpi)$, $L = K$, $s_1 = \varpi^{-1}$, $s_2 = s_3 = \dots = 0$,
$\sigma = \id_L$. 
Hence $f$ arises as the logarithm of a formal $\cO_K$-module law $H$ over $\cO_K$.
The fact that $f^{-1}(X) = X - \frac 1\varpi X^{q^n} + \dots$ reveals
$[\varpi]_H(T) \equiv \varpi T$ mod $(T^2)$. Additionally, note that 
\begin{equation*}
  f([\varpi]_H(T)) = \varpi f(T) = \varpi T + f(T^{q^n}) \equiv f(T^{q^n}) \pmod \varpi.
\end{equation*}
Hence, the equivalence in \eqref{eq:funceqlemcongruence} implies that 
$[\varpi]_H(T) \equiv T^{q^n}$ mod $\varpi$. So $H$ is a Lubin--Tate formal $\cO_K$-module
law of height $n$, we call it the standard Lubin--Tate formal module law of
height $n$. 
\begin{rmk} 
  The formal $\cO_K$-module $H$ is a member of the set of so called $A$-typical
  formal modules - formal $A$-modules $F$ with logarithm of the 
  form
  \begin{equation*}
    \log_F(T) = \sum_{i=0}^\infty b_i X^{q^i}
  \end{equation*}
  for elements $b_0, b_1, \dots \in R \otimes_A K$ (cf. \cite[Definition
  21.5.5 and Criterion 21.5.9]{hazewinkel1978formal}). If $R$ is flat over $A$,
  every formal $A$-module over $R$ is isomorphic to an $A$-typical one
  (cf. \cite[21.5.6]{hazewinkel1978formal}). The following discussion remains
  valid for $\cO_K$-typical formal modules.
\end{rmk}

It will be convenient to make the terms in the exact sequence of Proposition 
\ref{prop:ExplicitInterestingES} explicit for $\cF = \FGG(H)$. As $H$ is of height $n>0$, 
there is no non-trivial map $H \to \GG_a$ and the sequence becomes
\begin{equation*}
\begin{tikzcd}[ampersand replacement=\&]
	0 \& {\omega(H)} \& {\RigExt(H,\GG_a)} \& {\Ext(H,\GG_a)} \& 0 \\
	0 \& \begin{array}{c} \left\{\begin{gathered} {g \in TK\llbr T \rrbr : \delta g = 0} \\          \text{ and $g'(T) \in \cO_K\llbr T \rrbr$}\end{gathered} \right\} \end{array} \& {\QLog(H)} \& {\frac{\SymCoc^2(H, \GG_a)}{\{\delta g \mid g \in T \cO_K\llbr T \rrbr\}}} \& 0.
	\arrow[from=1-1, to=1-2]
	\arrow[from=1-2, to=1-3]
	\arrow[from=1-2, to=2-2]
	\arrow[from=1-3, to=1-4]
	\arrow[from=1-3, to=2-3]
	\arrow[from=1-4, to=1-5]
	\arrow[from=1-4, to=2-4]
	\arrow["", from=2-1, to=2-2]
	\arrow[from=2-2, to=2-3]
	\arrow["\delta", from=2-3, to=2-4]
	\arrow[from=2-4, to=2-5]
\end{tikzcd}   
\end{equation*}
We now have 
\begin{prop}
  The $R$-module $\omega(H)$ is free of rank $1$, generated by 
  $f(T) = \log_H(T)$. $\QLog(H)$ is free of rank $n$, generated by the classes of
  $(f(T), \frac 1\varpi f(T^q), \dots, \frac 1\varpi f(T^{q^{n-1}}))$. Consequently,
  the short exact sequence above is given by 
  \begin{equation*}
    0 \to \left \langle f(T) \right \rangle \to \left \langle f(T), \frac 1\varpi
      f(T^q) , \dots,
    \frac 1\varpi f(T^{q^{n-1}}) \right \rangle \xto \delta 
    \left \langle \delta \left(\frac 1\varpi f(T^q)\right),\dots, \delta
    \left(\frac 1\varpi f(T^{q^{n-1}})\right ) \right \rangle \to 0.
  \end{equation*}
\begin{proof}
  An easy calculation shows that $\frac 1\varpi f(T^{q^k})$ is a quasi-logarithm for 
  $1 \leq k \leq n-1$. As $\delta f = 0$, we have $f(T) \in \QLog(F)$ as well. 
  The claim is \cite[Proposition 13.8]{hopkins1994equivariant} which is a 
  special case of [ibid., Proposition 9.8].
\end{proof}
\end{prop}
% subsubsection Deformations of Formal Modules and the Standard Formal Module (end)




\subsection{Formal DVR-Modules over Fields} % (fold)
\label{sub:Formal DVR-Modules over Fields}
As above, let $A$ be a discrete valuation ring with uniformizer $\pi$ and finite 
residue field $k$; write $q$ for the cardinality of $k$. Let $K$ denote the
field of fractions of $A$.

We introduce the concept of height, which is an integer attached to
morphisms of formal group laws over fields. The height of a formal $A$-module
$\cF$ over $R$ will be defined as the height of it's endomorphism $[\pi]_\cF$. 

We have seen in the previous section that if $R$ is a field extension of $K$,
then any morphism of formal group laws $f: F \to G$ over $R$ is either $0$, in
which case we say it has height $\infty$, or an isomorphism, in which case we
say it has height $0$. The height becomes interesting in positive
characteristic. 

We define the height over field extensions of the residue field. 
\begin{defi}[Height of morphisms of group laws]
  Assume that $R$ is a field extension of $k$ and $f: F \to G$ is a morphism of 
  formal groups laws over $R$, given by a formal series $f(T) \in R\llbr T \rrbr$. 
  If $f = 0$, we say that $f$ has infinite height. 
  If $f \neq 0$, the height of $f$ is defined as the largest integer $h$ such that 
  $f = g(T^{q^h})$ for some power series $g(T) = c_1 T + c_2 T^2 + \dots \in
  R\llbr T \rrbr$ with $c_1 \neq 0$. 
\end{defi}
One readily checks that if $f: \cF \to \cG$ is a morphism of formal groups over
a field extension $R$ of $k$, the height of $f$ does not depend on the choices
of group laws on $\cF$ and $\cG$. This allows us to define the height function 
attached to $f$. 
\begin{defi}[Height function]
  Let $f: \cF \to \cG$ be a morphism of formal groups over a scheme $X$.
  For a (scheme-theoretic) point $x \in \abs X$, let $f_x$ denote the 
  base-change of $f$ to the residue field of $x$. 
  The height function attached to $f$ is the upper-semicontinuous function
  \begin{equation} \label{eq:defheight}
    \height(f) : \abs X \to \Z_{\geq 0} \cup \{\infty\}, \quad x \mapsto 
    \height(f_x).
  \end{equation}
\end{defi}
It is not hard to see that the height function is additive, that is, we have
\begin{equation*}
  \height(f \circ g) = \height(f) + \height(g).
\end{equation*}

Let $R$ be a local $A$-algebra and let $k'$ be a separable closure of $R/\fm_R$.
Let $F$ be a formal $A$-module law over $R$ of height $h$. Then we have the
following result on the endomorphisms of $F \otimes_R k'$.
\begin{lem}
  The $A$-module $\End_{\FMOver A{k'}}(F \otimes_R k')$ is isomorphic to the ring
  of integers of a central division algebra $D$ over $K$ of invariant $\frac 1h$.
\begin{proof}
  \cite[Proposition 1.7]{drinfel1974elliptic} \red{The proof uses techniques from
    deformations of formal modules. Hence perhaps it would make sense to have this
  in the next chapter.}
\end{proof}
\end{lem}

\begin{defi}[Isogeny]
  A morphism $f: \cF \to \cG$ of formal groups over a field $k$ is called an isogeny if
  $\ker(f)$ is a represented by a finite free $k$-scheme. More generally, a
  morphism of formal $A$-modules over a base scheme $X$ is an isogeny if and
  only if $\ker(f)$ is finite and locally free over $X$. 
\end{defi}

Isogenies can be described using the height function.

\begin{lem}
  A morphism $f: \cF \to \cG$ is a isogeny if and only if the height 
  function $\height(f)$ is locally constant with values in $\Z_{\geq 0}$. 
\end{lem}

\begin{defi}[$\pi$-divisible $A$-module]
  We say that a formal $A$-module $H$ over $X$ is $\pi$-divisible if 
  $[\pi]_H$ is an isogeny. If $X$ is connected, the height of $H$ is the
  (constant) height of the endomorphism $[\pi]_H: A \to A$. 
\end{defi}

\begin{lem}
  Over sepearbly closed fields, the formal group laws are classified by their 
  heights.
\begin{proof}
  \cite[Theorem 19.4.1]{hazewinkel1978formal}, this is originally due to Drinfeld.
  Note that this is only interesting in positive characteristic.
\end{proof}
\end{lem}

The following lemma allows us to invert quasi-isogenies.

\begin{lem}
  Let $f: \cF \to \cG$ be an isogeny of $\pi$-divisible formal $A$-modules over a
  quasi-compact \red{quasi-separated?} $A$-scheme $X$. Then there is an 
  integer $n \geq 0$ and an isogeny $g: \cG \to \cF$ with 
  \begin{equation*}
    f \circ g = [\pi^n]_\cG \quad \text{and} \quad g \circ f = [\pi^n]_\cF.
  \end{equation*}
\end{lem}
% subsubsection Formal DVR-Modules over Fields of Characteristic 0 (end)




\subsection{The Dieudonn\'e functor} % (fold)
\label{sub:The Dieudonne functor}
\subsubsection{Rigidified Extensions and Quasi-Logarithms} % (fold)
The aim of this subsection is to give a description of $\RigExt(\cF, \cF')$ 
similarly to the one given for $\Ext(\cF, \cF')$ in the previous 
subsection, at least in the case where $\cF' = \GG_a$ is the (one-dimensional)
additive formal group law and $\cF$ comes from a one-dimensional group law $F$. 
This description will be in terms of quasi-logarithms, which we shall now
define.


By Theorem \ref{thm:ExtClassCoc} and Theorem \ref{thm:RigExtStr}, we may equip the 
sets $\Ext(F, \GG_a)$ and $\RigExt(F, \GG_a)$ with the structure of an $R$-module. 
There is a natural map of $R$-modules $\RigExt(F, \GG_a) \to \Ext(F, \GG_a)$
sending $(E, \omega_E)$ to $E$. This map is surjective. The kernel is given by
pairs $(\GG_a \oplus F, \omega) \in \RigExt(F, \GG_a)$, where $\omega$ is an
invariant differential on
$\GG_a \oplus F$ pulling back to $\dc X$ on $\GG_a$. As $\omega(\GG_a \oplus F)
\cong \omega(\GG_a) \times \omega(F)$, this datum is equivalent to an invariant
differential $\nu \in \omega(F)$. 


% subsubsection Rigidified Extensions (end)



\subsection{Tate Modules and the Universal Cover} % (fold)
\label{sub:Tate Modules and the Universal Cover}
\subsubsection{Useful Calculations} % (fold)
\label{ssub:Useful Calculations}
Let $p$ be a prime.
Let $R$ be a Noetherian local ring with maximal ideal $I$ such that 
$p \in I$, $R$ is complete with respect to the $I$-adic topology and $k_R
\coloneqq R/I$ is an algebraically closed field (necessarily of characteristic
$p$). 
If $q$ is a power of $p$, we write $\cF_{R,q}$ for the set of power series $f
\in R\llbr T \rrbr$ satisfying 
\begin{equation} \label{eq:condonpowerseries}
  f(T) \equiv g(T^q) \pmod I
\end{equation}
for some power series $g(T) = c_1 T + c_2 T^2 + \dots \in R\llbr T \rrbr$ with 
$c_1 \in R^\times$. 
If $q'>q$ is another power of $p$, we have injections $\cF_{R,q} \inj \cF_{R,q'}$
given by sending $f(T)$ to its $(q'/q)$-fold self-composite $f^{q'/q}(T)$. 
Making use of these transition maps, we define
\begin{equation*}
  \cF_R \coloneqq \colim_{n \in \N} \cF_{R, p^n},
\end{equation*}
identifying any power series $f \in \cF_{R,q}$ with its image in $\cF_{R,q'}$ for 
higher $p$-powers $q'$. 
For any $f \in \cF_{R,q}$, we define the functor
\begin{equation*}
  U_f: \Adm R \to \Set, \quad S \mapsto \left\{(x_0,x_1, \dots) \in \prod_\N S^\cici 
                                          \mid f(x_{i+1}) = x_i\right\}.
\end{equation*}
This functor does, up to canonical isomorphism, only depend on the equivalence
class of $f$ in $\cF_{R}$. 
We write $U_{0,f}$ for the base change of $U_f$ to $k_R$, that is
\begin{equation*}
  U_{0,f}: \Adm {k_R} \to \Set, \quad S \mapsto \left\{(x_0,x_1, \dots) \in
                              \prod_\N S^\cici \mid \bar f(x_{i+1}) = x_i\right\}.
\end{equation*}
Here, $\bar f$ is the image of $f$ under the reduction map $R\llbr T \rrbr \to
k_R\llbr T \rrbr$. 

In the sequel, we denote $R$-algebras by $S$ and write $J$ for an ideal
of definition containing the image of $I$ (provided, for example, by \ref{lem:iodimage}). 
Given an element $f\in \cF_R$, we do not distinguish between $f$ and a choice of a 
representative $\tilde f \in \cF_{R,q}$ for some sufficiently large $p$-power.

The following observation lays the groundwork for many of the upcoming results.
\begin{lem}\label{lem:cryscalc}
  Let $f$ be any power series in $\cF_R$. For any two elements $s_1,s_2 \in S$ 
  with $s_1 \equiv s_2 \mod J$ such that $f(s_1)$ and $f(s_2)$ exist (for
  example if $f$ is a polynomial or $s_1, s_2 \in S^\cici$), we have 
  \begin{equation*}
    f^k(s_1) \equiv f^k(s_2) \pmod {J^{k+1}}.
  \end{equation*}
  Here, $f^k$ denotes $k$-fold composition of $f$.
\begin{proof}
  We will show that if $s_1 \equiv s_2$ mod $J^k$, then $f(s_1) \equiv f(s_2)$ mod 
  $J^{k+1}$, which suffices to prove the claim. 
  We may write $s_2 = s_1 + r$ for some $r\in J^k$. By the assumptions on $f$
  there exist power series 
  $g,h \in R\llbr T \rrbr$ such that $h$ only
  has coefficients in $I$ and $f(T) = g(T^q) + h(T)$. As $I$ is finitely generated,
  say by elements $(r_1, \dots, r_l)$, we obtain a representation 
  \begin{equation*}
    f(s_1) - f(s_2) \in g(s_1^{q})-g(s_2^{q}) + \sum_{i=1}^l r_i \left(h_i(s_1) -
    h_i(s_2)\right).
  \end{equation*}
  As $r$ divides $\left(h_i(s_1) - h_i(s_2)\right)$, we find
  $r_i(h_i(s_1) - h_i(s_2)) \in (r_i r) \subseteq J^{k+1}$. Also note that 
  for any $s\in S$ and $n \in \N$, 
  $$(s+r)^{nq} = s^{nq} + nqrs^{nq-1}r + \dots + r^{nq},$$
  so after cancellation, all monomials of $g(s_1^q) - g(s_2^q)$ lie in
  $(qr)$ or $(r^2)$. This implies
  \begin{equation*}
    g(s_1^q) - g\left((s_1+r)^q\right) \in (qr) + (r^2) \subseteq J^{k+1},
  \end{equation*}
  and we are done.
\end{proof}
\end{lem}

\begin{lem}\label{lem:reductioniso}
  The natural reduction map 
  \begin{equation*}
    U_f(S) \to U_{f}(S/J) = U_{0,f}(S/J)
  \end{equation*}
  is bijective.
\begin{proof}
  We first show surjectivity. Given a sequence $(x_0, x_1, \dots) \in U_{f}(S/J)$, 
  we can choose a sequence of arbitrary lifts $(y_0, y_1, \dots ) \in \prod_\N
  S^\cici$ and set 
  \begin{equation*}
    z_i = \lim_{r \to \infty} f^r(y_{i+r}).
  \end{equation*}
  The limit exists, because if $s\geq r$ are two non-negative integers, we calculate
  \begin{equation*}
    f^{s-r}(y_{i+s}) \equiv \bar f^{s-r}(x_{i+s}) = x_{i+r} \equiv y_{i+r}
    \pmod J,
  \end{equation*}
  implying by Lemma \ref{lem:cryscalc} that 
  \begin{equation*}
    f^{s}(y_{i+s}) \equiv f^r(y_{i+r}) \pmod{J^r}.
  \end{equation*}
  This shows that $(f^{r}(y_{i+r}))_{r \in \N}$ is a Cauchy-sequence for the 
  $J$-adic topology on $S$, thereby convergent (cf. Lemma
  \ref{lem:AdmAdicComp}). The sequence $(z_0, z_1, \dots)$ now lies in $U_f(S)$
  and lifts $(x_0, x_1, \dots)$. It remains to show that the lift is unique.
  Suppose that $(z'_0, z'_1, \dots)$ is another lift. Then, for any $i,k \in\N$
  we have $z_{i+k} \equiv z'_{i+k}$ mod $J$, and another application of Lemma
  \ref{lem:cryscalc} shows that 
  \begin{equation*}
    z_i = f^k(z_{i+k}) \equiv f^k(z'_{i+k}) = z'_i \pmod {J^k}.
  \end{equation*}
  Thereby $(z_i - z'_i) \in \bigcap_{k \in \N} J^k = \{0\}$. Hence,
  the lift is unique.
\end{proof}
\end{lem}

We write $\Nilp^\flat$ for the functor $U_{T^q}$. That is, 
$\Nilp^\flat(S) = \lim_{x \mapsto x^q}S^\cici$ is 
the set of $q$-power compatible sequences with values in $S^\cici$. 

\begin{lem}\label{lem:nilp0iso}
  For any $f \in \cF_R$, there is a canonical \todo{Use different $S$}
  bijection $U_{0,f}(S/J) \to \Nilp^\flat(S/J)$. This bijection is functorial in 
  $S$.
\begin{proof}
  By assumption on $f$ we have $f(T) = g(T^{q}) \in k_R\llbr T \rrbr$ for some 
  $g(T) = c_1T + c_2T^2 + \dots$ with $c_1 \neq 0$. For each coefficient $c_i$, let
  $d_i \in k_R$ be the unique element such that $d_i^{q} = c_i$. Let
  $h(T) \in k_R\llbr T \rrbr$ be the power series given by $d_1 T + d_2
  T^2 + \dots$. Now $(h(T))^{q}=f(T)$, and we find that 
  \begin{equation*}
      U_f(S/J) \to \Nilp^\flat(S/J): \quad
      (x_1, x_2, x_3, \dots) \mapsto (x_1, h(x_2), h(h(x_3)), \dots)
  \end{equation*}
  is a well-defined function, and (trivially) functorial in $S$. For the
  inverse, let $h^{-1}(T) \in k_R\llbr T \rrbr$ be the the unique power
  series with $h^{-1}(h(T))= h(h^{-1}(T)) = T$, see Lemma
  \ref{lem:FGLeasyfacts}. The map
  \begin{equation*}
      \Nilp^\flat(S/J) \to U_f(S/J), \quad 
      (x_1, x_2, \dots ) \mapsto (x_1, h^{-1}(x_2), h^{-1}(h^{-1}(x_3)), \dots)
  \end{equation*}
  is well-defined as
  \begin{equation*}
      f(h^{-1}(T)) = g((h^{-1}(T))^{q}) = (h(h^{-1}(T)))^{q} =
      T^{q}
  \end{equation*}
  and it is readily seen to be inverse to the map constructed above.
\end{proof}
\end{lem}

We collect results.
\begin{prop}\label{prop:pHTcalc}
  Given $f,g \in \cF_R$, we have bijections, functorial in $S$,
  \begin{equation} 
    U_f(S) \to U_f(S/J) \to \Nilp^\flat(S/J) \to U_g(S/J) \to U_g(S).
  \end{equation}
  Explicitely, the bijection $U_f(S) \to U_g(S)$ can be described as follows.
  Suppose that $f,g \in \cF_{R,q}$ for some sufficiently large $q$. 
  Let $h_f(T)$ and $h_g(T)$ be power series with coefficients in $A$ such that 
  $$h_f(T)^q \equiv f(T) \pmod I\quad\text{and}\quad h_g(T)^q \equiv g(T) \pmod
  I.$$
  Write $h_g^{-1}(T)$ for the (formal) inverse power series of $h_g$. 
  Now the isomorphism is given by the mapping
  \begin{equation*}
    (x_0, x_1, \dots) \mapsto (y_0, y_1, \dots), \quad \text{where} \quad y_i =
    \lim_{r \to \infty} g^r(h_g^{-(r+i)}(h_f^{r+i} (x_{i+r}))).
  \end{equation*}
  Here, the exponents are to be interpreted as iterated composition.
\begin{proof}
  The first part follows directly from repeated application of the previous
  two Lemmas. The second part follows by tracing through the previous lemmas.  
\end{proof}
\end{prop}
% subsubsection Useful Calculations (end)
\subsubsection{The Universal Cover} % (fold)
\label{ssub:The Universal Cover}
Let $A$ be an integral domain and $R$ be an $A$-algebra. Given $H \in \FMOver
AR$ and $a \in A$,
we define the functor 
\begin{equation*}
  \Tilde H_a : \Adm R \to \Mod {A}, \quad
  S \mapsto \left\{(x_1, x_2, \dots) \in \prod_{\N} H(S) \mid [a]_H(x_{i+1}) =
  x_i \right\}.
\end{equation*}
Here, the $A$-module structure is given by $b.(x_1, x_2,\dots) = ([b]_H(x_1), [b]_(x_2),
\dots)$. Note that multiplication by $a$ on $\tilde H_a(S)$ is an automorphism
(it sends $(x_1, x_2, \dots)$ to $([a]_H x_1, x_1, x_2, \dots)$, which has inverse given
by shifting to the left)  so that $\tilde H_a(S)$ is naturally an
$A[\frac1a]$-module.

From now on assume that $A$ is a discrete valuation ring with uniformizer
$\pi$, finite residue field $k$ and field of fractions $K$. Let $R$ be 
a local $A$-algebra with maximal ideal $I$ and algebraically closed residue field 
$k_R = R/I$. Let $H$ be a formal $A$-module over $R$. 
\begin{defi}[The Universal Cover and Tate Module]
  We write $\tilde H = \tilde H_\pi$. This functor
  takes values in the category of $K$-vector spaces.
  Up to natural isomorphism, $\tilde H$ does not depend on the choice of 
  $\pi$. We call this functor the universal cover of $H$. 

  The Tate-Module $T_\pi H$ is the subfunctor of $\Tilde H$ cut out out
  by the condition that $[\pi]_H(x_1) = 0$. Note that $T_\pi H$ does no longer 
  carry the structure of a $K$-vector space, it is an $A$-module. The Rational
  Tate Module $V_\pi H$ is the subfunctor of $\Tilde H$ cut out by the
  condition that $x_1$ has $[\pi]_H$-torsion. Equivalently, we have 
  \begin{equation*}
    V_\pi H (S) = T_\pi H(S) \otimes_A K.
  \end{equation*}
\end{defi}

\begin{lem}
  Let $H$ be a $\pi$-divisible formal $A$-module over $R$ and write $H_0 = H
  \otimes_R k_R$. Now the choice of a coordinate on $H_0$ gives rise to a
  an isomorphism 
  \begin{equation*}
    \tilde H_0 \cong \Nilp^\flat_{k_R}
  \end{equation*}
  of functors $\Adm {k_R} \to \Set$
  \begin{proof}
    Note that given any coordinate on $H$, we have $[\pi]_H(T) \in \cF_R$. Hence,
    the statement is an application of Lemma \ref{lem:nilp0iso}.
  \end{proof}
\end{lem}

\begin{lem}
  Suppose that $S$ is an admissible $R$-algebra admitting an ideal of definition
  $J$ such that $\pi \in J$. Then the natural reduction map
  \begin{equation*}
    \tilde H(S) \to \tilde H(S/J)
  \end{equation*}
  is an isomorphism.
  \begin{proof}[Proof]
    After choosing a coordinate on $H$, we have $[\pi]_H \in \cF_R$ and 
    $\tilde H(S) \cong U_{[\pi]_H}$, and the statement is given by 
    Lemma \ref{lem:reductioniso}.
  \end{proof}
\end{lem}

The following is analogous to Proposition \ref{prop:pHTcalc}.
\begin{prop}
  Let $S$ be an admissible $R$-algebra with ideal of definition $J$ such that 
  $\phi(I) \subseteq J$. Then there are canonical isomorphisms (of sets)
      \begin{equation*}
        \tilde H(S) \cong \tilde H(S/J) = \tilde H_0(S/J) \cong \Nilp^\flat(S/J) \cong
        \Nilp^\flat(S).
      \end{equation*}
  In particular, $\tilde H(S)$ is, as a functor to $\Set$, representable by
  $\spf(R \llbr T^{q^{-\infty}} \rrbr)$.
\end{prop}

\begin{rmk} 
  In case where $H$ comes from a Lubin--Tate $\cO_K$-module law, the bijections
  \begin{equation*}
    \tilde H(S) \rightleftarrows \Nilp^\flat(S), \quad (x_0, x_1, \dots) \mapsspamto (y_0, y_1, \dots)
  \end{equation*}
  are, in either direction, given by the equations
  \begin{equation*}
    y_i = \lim_{r \to \infty} x_{i+r}^{q^r} \quad \text{and} \quad 
    x_i = \lim_{s \to \infty} [\pi^s]_H(y_{i+s}).
  \end{equation*}
  This follows directly from the explicit description of the isomorphism in
  Proposition \ref{prop:pHTcalc}, as we may choose $h_{[\pi]_H}(T) = h_{T^q}(T)
  = T$.
\end{rmk}
% subsubsection The Universal Cover (end)

% subsection Tate Modules and the Universal Cover (end)




\subsection{The Quasilogarithm Map} % (fold)
\label{sub:The Quasilogarithm map}
We keep the assumptions on $A$, $R$ and $S$ from the previous subsection. That is,
$A$ is a local ring with finite residue field and uniformizer $\pi$, 
$R$ is a local $A$-algebra with maximal ideal $I$ complete with respect to
the $I$-adic topology and algebraically closed residue field $k_R$, and 
$S$ denotes an admissible $R$-algebra (where $R \to S$ is continuous with
the $I$-adic topology on $R$) with ideal of definition $J \subseteq S$ containing
the image of $I$. 

The aim of this subsection is to define, attached to any $\pi$-divisible formal
$A$-module $H$ over $R$, a quasi-logarithm map
\begin{equation*}
  \qlog_H: \tilde H(S) \to (M(H_0) \otimes \GG_a)(S)
\end{equation*}
and give an explicit description of this map if $H$ is the standard $\cO_K$-module
over $\cO_{\breve K}$. 
% subsubsection The Quasilogarithm map (end)

\subsection{Determinants of Formal Modules} % (fold)
\label{sub:Determinants of Formal Modules}
\begin{itemize}
  \item "Functorial" description of the determinant. Either as in
    \cite{BoyarchenkoWeinstein2011MaxVar}, or as in \cite{weinstein2016semistable}.
  \item Construction.
  \item Approximations.
\end{itemize}

% subsubsection Determinants of Formal Modules (end)
% subsection Application: The Local Class Field Theory (end)



\section{Sandbox} % (fold)
\label{sec:Sandbox}
\subsection{Invariant Differentials and Logarithms} % (fold)
\label{sub:Invariant Differentials and Logarithms}
\blue{
Again, $A$ is a complete discrete valuation ring with uniformizing parameter 
$\pi$ and finite residue field $k = A/\pi A$. We write $q$ for the cardinality of 
$k$ and $K$ for the residue field of $A$. Let $R$ be a (commutative) $A$-algebra
with structure map $i: A \to R$.}

We review results from section
2 and 3 of \cite{hopkins1994equivariant}. 
Suppose that $\bF = (F_1, \dots, F_n)$ is an $n$-dimensional formal $A$-module law over an
$A$-algebra $R$. Write $\bX = (X_1, \dots, X_n)$, $\bY = (Y_1, \dots, Y_n)$, etc.

\begin{defi}[Invariant Differentials for module laws.]
  The module $\omega(\bF)$ of invariant differentials is the submodule of the
  $R$-module of differentials
  \begin{equation*}
    \Omega_{R\llbr T_1, \dots, T_n \rrbr/R} \cong \bigoplus_{i=1}^n R\llbr T_1, \dots, T_n
    \rrbr \dc T_i,
  \end{equation*}
  cut out by the condition that any $\omega \in \omega(\bF)$ satisfies
  \begin{equation}\label{eq:diffcond}
    \omega(\bF(\bX,\bY)) = \omega(\bX) + \omega(\bY)\quad \text{and} \quad
    \omega([a]_\bF(\bX)) = a\omega(\bX).
  \end{equation} 
  for all $a \in A$. 
\end{defi}

It is possible to explicitly calculate a basis for the $R$-module
$\omega(\bF)$, which we now explain. Let 
$$A(\bX, \bY) \in \Mat_{n \times n} (R\llbr \bX, \bY \rrbr)$$ 
denote the matrix $\left((\partial/\partial X_j)F_i (\bX,\bY)\right)_{i,j}$,
the derivative of $\bF(\bX,\bY)$ with respect to $\bX$. Write 
$B(\bY) = A(0,\bY)$. Then $B$ is a unit in $\Mat_{n \times n} R\llbr \bY \rrbr$; 
and we write $(C_{ij}(\bY))$ for the components of 
$B(\bY)^{-1}$. We now construct 
$$\omega_{i} \coloneqq \sum_{j=1}^n C_{ij}(\bX) \dc X_j \in \Omega_{R\llbr \bX \rrbr/R}$$ 
for $1 \leq i \leq n$. By definition we have 
\begin{equation}\label{eq:coeffofcanonicaldiff}
  C_{ij}(0) = \begin{cases}
    1 &\text{ if }i = j,\\
    0 &\text{ otherwise.}
  \end{cases}
\end{equation}
Checking that $\omega_{i}$ is an invariant differential is a matter of
applying the chain rule, and we have
\begin{prop}
    The $R$-module $\omega(\bF)$ is free of rank $n$ generated by invariant differentials
    $\omega_{1}, \omega_{2}, \dots, \omega_{n}$.
\begin{proof}
  This is \cite[Proposition 1.1]{1970HondaFormalGroups}. 
\end{proof}
\end{prop}
\begin{xpl}
  The invariant differentials for $\GG_a$ are spanned by the form $\dc X$. 
  The invariant differentials for $\GG_m$ are spanned by the form 
  $\omega_1(X) = \frac 1{1+X} \dc X$.
\end{xpl}
By the Proposition above and Equation \eqref{eq:coeffofcanonicaldiff}, we may
define a pairing
\begin{equation*}
  \omega(\bF) \times \Lie(\bF) \to R, \quad \langle X_i, \omega_j \rangle =
  \begin{cases}
    1 &\text{ if } i = j,\\
    0 &\text{ otherwise.}
  \end{cases}
\end{equation*}
This pairing is independent of the parametrization of $\bF$. In particular, it
descents to a pairing defined for formal modules $\cF \in \FMArbOver A R$, and
we have a natural isomorphism $\omega(\cF) \cong \Hom_R(R, \Lie(\cF))$.

Let $\GG_a$ be the additive formal $A$-module over $R$. There is a map
\begin{equation} \label{eq:functorinvdifftohom}
  \dc_\bF : \Hom_{\FMLOver AR} (\bF, \GG_{a,R}) \to \omega(\bF), \quad f \mapsto \dc f(\bX)
\end{equation}
which is a map of $R$-modules if we equip the left hand side with the $R$-module
structure coming from the natural action of $R \subset \End(\GG_a)$. 
\begin{prop}
  \begin{enumerate}
    \item If $R$ is a flat $A$-algebra, the map $\dc_F$ is injective.
    \item If $R$ is a $K$-algebra, the map $\dc_F$ is an isomorphism.
  \end{enumerate}
\begin{proof}
  \cite[Proposition 3.2]{hopkins1994equivariant} \todo{PROOF}\red{ Everything is easy if 
  $K$ has characteristic $0$, as we can integrate the differential forms.
  The proof in positive characteristic is a bit tricky; First it is shown that 
  there is an isomorphism of formal goups $F \cong \GG_a$, which is immediate.
  Then that there is a unique homomorphism $f: \GG_a \to \GG_a$ that maps to $\omega_F$
  and behaves well with respect to the $A$-module structure on $F$. }
\end{proof}
\end{prop}

Suppose now that $F \in \FMLOver A R$ is a one-dimensional formal module. 
The previous Proposition shows that if $R$ is a $K$-algebra, the invariant differential 
$\omega_1(X)$ constructed above comes from a homomorphism $f(X) = X + c_2 X^2 + \dots$,
which is an isomorphism by Lemma \ref{lem:FGLeasyfacts}. 
This allows us to define the logarithm attached to $F$.
\begin{defi}[Logarithm and Exponential]
  If $R$ is a flat $A$-algebra, there is a unique power series
  \begin{equation*}
    \log_F(X) = X + c_2 X^2 + \dots \in (R \otimes_A K)\llbr X \rrbr 
  \end{equation*}
  inducing an isomorphism $F \otimes (R \otimes K) \to \GG_{a,R\otimes K}$.
  This power series is called the logarithm attached to $F$. 
  The inverse of $\log_F$ is called the exponential of $F$ and will be denoted by
  $\exp_F$.
\end{defi}
\begin{rmk} 
  These definitions do not descent to formal modules. Given $\cF \in \FMOver A
  R$, there is no natural choice of Logarithms and Exponentials, these
  definitions depend on the choice of parametrization
  $\cF \cong \Spf R \llbr T \rrbr$. 
\end{rmk}
% subsubsection Logarithms (end)


% subsection Invariant Differentials and Logarithms (end)
% section Sandbox (end)

\end{document}
