We sketch the construction of a family of formal modules introduced by 
Lubin and Tate in \cite{LubinTateFormalMult}.

Let $A$ be a complete discrete valuation ring with 
finite residue field $k$, set $q = \# k$ and let $\pi \in A$ be a choice of a
uniformizer.
Write $\cF_{\pi, h}$ for the following set of power series
\begin{equation*}
  \cF_\pi \coloneqq \{f \in \cO_K \llbr T \rrbr \mid f \equiv \pi T \pmod {T^2}
  \text{ and } f \equiv T^{q^n} \pmod \pi\}. 
\end{equation*}

The construction of Lubin--Tate formal module laws rests on the following 
lemma, which is Lemma 1 in \cite{LubinTateFormalMult}.
\begin{lem}\label{lem:LTLemma1}
  Let $f(T)$ and $g(T)$ be elements of $\cF_{\pi,h}$ and let 
  $L(X_1, \dots, X_n) = \sum_{i=1}^n a_i X_i$ be a linear form with coefficients in 
  $A$. Then there exists a unique series $F(X_1, \dots, X_n)$ with coefficients 
  in $A$ such that 
  \begin{gather*}
    F(X_1, \dots, X_n) \equiv L(X_1, \dots, X_n) \pmod {T^2}, \\ \text{and} \\
    f(F(X_1, \dots, X_n)) = F(g(X_1), \dots, g(X_n)).
  \end{gather*}
\end{lem}

As a direct consequence, we obtain the following useful result.
\begin{lem}
  Let $f \in \cF_{\pi, h}$. Then there is a unique formal $A$-module law $F_f$ over $A$
  with $[\pi]_F(T) = f(T)$.
\begin{proof}
  In the above Lemma, set $L(X,Y) = X+Y$ and $g=f$ to uniquely determine 
  the power series $F_f$. The same Lemma yields unique power series
  $[a]_{F_f}(T) \in R\llbr T \rrbr$ by setting $L(T) = a T$, $g=f$. It is
  routine to check that $(F_f, ([a]_f)_{a \in A})$ is a formal $A$-module law, 
  cf. \cite{LubinTateFormalMult}.
\end{proof}
\end{lem}

\begin{defi}[Lubin--Tate Module Law]
  We refer to module laws arising by the construction above as Lubin--Tate module laws.
\end{defi}

Furthermore, attached to each $a \in \cO_K$ and $f,g \in \cF_{\pi,h}$, we find
unique $[a]_{f,g}(T) \in \cO_K\llbr T \rrbr$ satisfying
\begin{equation}\label{eq:LTMoLaScaCond}
  [a]_{f,g}(T) \equiv aT \pmod {(T)^2} \quad \text{and} \quad
  f([a]_{f,g}(T)) = [a]_{f,g}(g(T)).
\end{equation}
We now have the following theorem.
\begin{thm}[Lubin--Tate Formal $\cO_K$-Module Laws]\label{thm:LTModLaw}
  Let $K$ be a local field with ring of integers $\cO_K$. For any choice of 
  uniformizer $\pi \in \cO_K$ and any $f \in \cF_{\pi,h}$, the family of power
  series $(F_f, ([a]_{f,f})_{a \in \cO_K})$
  gives rise to a formal $\cO_K$-module law over $\cO_K$. For 
  $f,g \in \cF_{\pi,h}$, the formal $\cO_K$-module laws $F_f$ and $F_g$ are
  canonically isomorphic, via the morphism induced by $[1]_{f,g} \in \cO_K\llbr
  T \rrbr$. 
\begin{proof}
  See Theorem 1 of \cite{LubinTateFormalMult} and the succeeding discussion.
\end{proof}
\end{thm}
In particular, up to canonical isomorphism, there is only one Lubin--Tate formal
$\cO_K$-module law over $\cO_K$ attached to the choice of the uniformizer $\pi \in
\cO_K$. 

\begin{xpl}
  If $K = \Q_p$, this reconstructs the multiplicative formal 
  $\Z_p$ module $\GG_m$ constructed above. Indeed, we have 
  \begin{equation*}
    \cF_p = \{f \in \Z_p\llbr T \rrbr \mid f(T) \equiv T^p \text{ mod } p
    \text{ and } f(T) \equiv pT \text{ mod } (T)^2 \},
  \end{equation*}
  implying that $f(T) = (1+T)^p-1$ lies in $\cF_p$.  
  One quickly checks that 
  \begin{equation*}
    F_f(X,Y) = (1+X)(1+Y) - 1 = X + Y + XY \in \Z_p \llbr X,Y \rrbr
  \end{equation*}
  is the addition law associated to $f$, and that 
  for $a \in \Z_p$, the power series
  \begin{equation*}
    [a]_{f,f} = (1+T)^{a} - 1 \in \Z_p \llbr T \rrbr
  \end{equation*}
  satisfies the condition of \eqref{eq:LTMoLaScaCond}. 
\end{xpl}

% subsubsection Lubin--Tate Formal Module Laws (end)


