Suppose that $K$ is a local field with ring of integers $\cO_K$, with uniformizer
$\pi$ and residue field $\FF_q$. 

Let $H_0$ be the formal $\cO_K$-module law defined over $\FF_q$ by setting
\begin{equation*}
  H_0(X,Y) = X + Y, \quad [\pi]_{H_0}(X) = X^q, \quad [u]_{H_0}= \bar u X.
\end{equation*}
Here, $u$ runs over the units of $\cO_K$ and $\bar u \in \FF_q$ is such that 
$u \equiv \bar u$ mod $\pi$. This uniquely determines $[a]_{H_0}$ 
for $a \in \cO_K$ as $a$ may be written as $a = \pi^{\nu} u$ for a unit $u$ and
$[\pi]_{H_0}$ and $[u]_{H_0}$ commute. 

Lubin--Tate formal module laws are $\cO_K$-module laws $H$ over $\cO_K$ such that 
$H \otimes_{\cO_K} \FF_q = H_0$. 

The construction of the Lubin--Tate formal module laws rests on the following 
lemma, which is Lemma 1 in \cite{LubinTateFormalMult}.
\begin{lem}\label{lem:LTLemma1}
  \todo{Restate in more general terms, for arbitrary powers of $q$, and perhaps
  for general discrete valuation rings with finite residue field.}
  Let $f(T)$ and $g(T)$ be elements of $\cF_\pi$ and let 
  $L(X_1, \dots, X_n) = \sum_{i=1}^n a_i X_i$ be a linear form with coefficients in 
  $\cO_K$. Then there exists a unique series $F(X_1, \dots, X_n)$ with coefficients 
  in $\cO_K$ such that 
  \begin{gather*}
    F(X_1, \dots, X_n) \equiv L(X_1, \dots, X_n) \pmod {T^2}, \\ \text{and} \\
    f(F(X_1, \dots, X_n)) = F(g(X_1), \dots, g(X_n)).
  \end{gather*}
\end{lem}
Although stated only for the rings of integers of a local field, the proof only
uses that $\cO_K$ is complete with respect to the $\pi$-adic topology, $\pi$ is
not a zero divisor and the map $x \mapsto x^q$ restricts to the identity mod $\pi$.
In particular, the statement remains true if working over the integers of the
completion of a maximal unramified extension $\br K$ of $K$.

Write $\cF_\pi$ for the set of power series that may arise as $[\pi]_H$, that is,
\begin{equation*}
  \cF_\pi \coloneqq \{f \in \cO_K \llbr T \rrbr \mid f \equiv \pi T \pmod {T^2}
    \text{ and } f \equiv T^q \pmod \pi\}. 
\end{equation*}
Using Lemma \ref{lem:LTLemma1}, we can construct formal $\cO_K$-modules over 
$\cO_K$ as follows. Attached to $f \in \cF_\pi$, we find a unique power series
$F_f(X,Y) \in \cO_K\llbr X,Y\rrbr$ satisfying
\begin{equation}\label{eq:LTMoLaAddCond}
  F_f(X,Y) \equiv X+Y \pmod{(X,Y)^2} \quad \text{and} \quad
  F_f(f(X),f(Y)) = f(F_f(X,Y)).
\end{equation}
Furthermore, attached to each $a \in \cO_K$ and $f,g \in \cF_\pi$, we find
unique $[a]_{f,g}(T) \in \cO_K\llbr T \rrbr$ satisfying
\begin{equation}\label{eq:LTMoLaScaCond}
  [a]_{f,g}(T) \equiv aT \pmod {(T)^2} \quad \text{and} \quad
  f([a]_{f,g}(T)) = [a]_{f,g}(g(T)).
\end{equation}
We now have
\begin{thm}[Lubin--Tate Formal $\cO_K$-Module Laws]\label{thm:LTModLaw}
  For $f \in \cF_\pi$, the family of power series $(F_f, ([a]_{f,f})_{a \in \cO_K})$
  gives rise to a formal $\cO_K$-module law over $\cO_K$. For 
  $f,g \in \cF_\pi$, the formal $\cO_K$-module laws $F_f$ and $F_g$ are isomorphic,
  via the morphism induced by $[1]_{f,g} \in \cO_K\llbr T \rrbr$. 

\begin{proof}
  See Theorem 1 of \cite{LubinTateFormalMult} and the succeeding discussion.
\end{proof}
\end{thm}
In particular, up to isomorphism, the construction of a Lubin--Tate formal
$\cO_K$-module law only depends on the choice of the uniformizer $\pi \in
\cO_K$, not on the choice of $f \in \cF_\pi$. 

\begin{xpl}
  If $K = \Q_p$, this reconstructs the multiplicative formal 
  $\Z_p$ module $\GG_m$ constructed above. Indeed, we have 
  \begin{equation*}
    \cF_p = \{f \in \Z_p\llbr T \rrbr \mid f(T) \equiv T^p \text{ mod } p
    \text{ and } f(T) \equiv pT \text{ mod } (T)^2 \},
  \end{equation*}
  implying that $f(T) = (1+T)^p-1$ lies in $\cF_p$.  
  One quickly checks that 
  \begin{equation*}
    F_f(X,Y) = (1+X)(1+Y) - 1 = X + Y + XY \in \Z_p \llbr X,Y \rrbr
  \end{equation*}
  satisfies the conditions of \eqref{eq:LTMoLaAddCond}, and that 
  for $a \in \Z_p$, the power series
  \begin{equation*}
    [a]_{f,f} = (1+T)^{a} - 1 \in \Z_p \llbr T \rrbr
  \end{equation*}
  satisfies the condition of \eqref{eq:LTMoLaScaCond}. 
\end{xpl}

% subsubsection Lubin--Tate Formal Module Laws (end)


