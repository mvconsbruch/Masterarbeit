Again, $A$ is a complete discrete valuation ring with uniformizing parameter 
$\pi$ and finite residue field $k = A/\pi A$. We write $q$ for the cardinality of 
$k$ and $K$ for the field of fractions of $A$. Let $R$ be a local,
admissible $A$-algebra with structure map $i: A \to R$.

We review results from section
2 and 3 of \cite{hopkins1994equivariant}. 
Suppose that $\bF = (F_1, \dots, F_n)$ is an $n$-dimensional formal $A$-module
law over a $R$. We abbreviate $\bX = (X_1, \dots, X_n)$, $\bY = (Y_1,
\dots, Y_n)$, etc.

\begin{defi}[Invariant Differentials for module laws.]
  The module $\omega(\bF)$ of invariant differentials is the submodule of the
  $R$-module of differentials
  \begin{equation*}
    \Omega_{R\llbr T_1, \dots, T_n \rrbr/R} \cong \bigoplus_{i=1}^n R\llbr T_1, \dots, T_n
    \rrbr \dc T_i,
  \end{equation*}
  consisting of those $\omega \in \omega(\bF)$ satisfying
  \begin{equation}\label{eq:diffcond}
    \omega(\bF(\bX,\bY)) = \omega(\bX) + \omega(\bY)\quad \text{and} \quad
    \omega([a]_\bF(\bX)) = a\omega(\bX).
  \end{equation} 
  for all $a \in A$. 
\end{defi}

It is possible to explicitly calculate a basis for the $R$-module
$\omega(\bF)$, which we now explain. Let 
$$A(\bX, \bY) \in \Mat_{n \times n} (R\llbr \bX, \bY \rrbr)$$ 
denote the matrix $\left((\partial/\partial X_j)F_i (\bX,\bY)\right)_{i,j}$,
the derivative of $\bF(\bX,\bY)$ with respect to $\bX$. Set 
$B(\bY) = A(0,\bY)$. Then $B$ is a unit in $\Mat_{n \times n} R\llbr \bY \rrbr$; 
and we write $(C_{ij}(\bY))$ for the components of 
$B(\bY)^{-1}$. We now construct 
$$\omega_{i} \coloneqq \sum_{j=1}^n C_{ij}(\bX) \dc X_j \in \Omega_{R\llbr \bX \rrbr/R}$$ 
for $1 \leq i \leq n$. By definition we have 
\begin{equation}\label{eq:coeffofcanonicaldiff}
  C_{ij}(0) = \begin{cases}
    1 &\text{ if }i = j,\\
    0 &\text{ otherwise.}
  \end{cases}
\end{equation}
Checking that $\omega_{i}$ is an invariant differential is a matter of
applying the chain rule, and we have
\begin{prop}
    The $R$-module $\omega(\bF)$ is free of rank $n$ generated by invariant differentials
    $\omega_{1}, \omega_{2}, \dots, \omega_{n}$.
\begin{proof}
  This is \cite[Proposition 1.1]{1970HondaFormalGroups}. 
\end{proof}
\end{prop}
\begin{xpl}
  The invariant differentials for $\GG_a$ are spanned by the form $\dc X$. 
  The invariant differentials for $\GG_m$ are spanned by the form 
  $\omega_1(X) = \frac 1{1+X} \dc X$.
\end{xpl}
By the Proposition above and Equation \eqref{eq:coeffofcanonicaldiff}, we may
define a pairing
\begin{equation*}
  \omega(\bF) \times \Lie(\bF) \to R, \quad \langle X_i, \omega_j \rangle =
  \begin{cases}
    1 &\text{ if } i = j,\\
    0 &\text{ otherwise.}
  \end{cases}
\end{equation*}
This pairing is independent of the parametrization of $\bF$. In particular, it
descents to a pairing defined for formal modules $\cF \in \FMArbOver A R$, and
we have a natural isomorphism $\omega(\cF) \cong \Hom_R(R, \Lie(\cF))$.

Let $\GG_a$ be the additive formal $A$-module over $R$. There is a map
\begin{equation} \label{eq:functorinvdifftohom}
  \dc_\bF : \Hom_{\FMLOver AR} (\bF, \GG_{a,R}) \to \omega(\bF), \quad f \mapsto \dc f(\bX)
\end{equation}
which is a map of $R$-modules if we equip the left hand side with the $R$-module
structure coming from the natural action of $R \subset \End(\GG_a)$. 
\begin{prop}\label{prop:loginvdiff}
  \begin{enumerate}
    \item If $R$ is a flat $A$-algebra, the map $\dc_F$ is injective.
    \item If $R$ is a $K$-algebra, the map $\dc_F$ is an isomorphism.
  \end{enumerate}
\begin{proof}
  This is \cite[Proposition 3.2]{hopkins1994equivariant}.
\end{proof}
\end{prop}

Suppose now that $F \in \FMLArbOver A R$ is a formal module law of dimension $n$
over a flat $A$-algebra $R$. 
Let $\omega_1, \dots, \omega_n$ be the distinguished basis for $\omega(F)$
constructed above. 
By the previous proposition, there are unique power series 
$f_i(\bX) \in (R \otimes_A K)\llbr \bX \rrbr$ furnishing homomorphisms
$F \otimes (R \otimes_A K) \to \GG_{a,R\otimes_AK}$ of formal $A$-module laws
and satisfying
\begin{equation*}
  \dc_F f_i(\bX) = \omega_i(\bX) \in \omega(F).
\end{equation*}
\begin{defi}[Logarithm and Exponential]
  The induced morphism of formal group laws
  \begin{equation*}
    f = (f_1, \dots, f_n) : F \otimes (R \otimes_A K) \to \GG_a
  \end{equation*}
  is called the logarithm attached to $F$, we write 
  $\log_F(\bX) \in ((R \otimes_A K)\llbr \bX \rrbr)^n$ for the corresponding collection
  of power series. The inverse of $\log_F(\bX)$ is called the exponential 
  attached to $F$, denoted $\exp_F(\bX)$. We have $\Lie(\log_F) = \Lie(\exp_F) = \id$,
  so $\log_F$ and $\exp_F$ are isomorphisms.
\end{defi}

\begin{lem}
  Let $F$ and $G$ be formal $A$-module laws over $R$, with $\dim F = n$ and
  $\dim G = m$. 
  Let $\phi: F \to G$ be a morphism. Then the diagram 
  \begin{equation*}
  \begin{tikzcd}[ampersand replacement=\&]
    {F \otimes(R \otimes_A K)} \& {\GG_a \otimes (\Lie(F)} \otimes_A K) =
    \GG_{a, R\otimes_AK}^n \\
    {G \otimes(R \otimes_A K)} \& {\GG_a \otimes (\Lie(G)} \otimes_A K) =
    \GG_{a, R\otimes_AK}^m
  	\arrow["{\log_F}", from=1-1, to=1-2]
  	\arrow["\phi"', from=1-1, to=2-1]
  	\arrow["{\Lie(\phi)}", from=1-2, to=2-2]
  	\arrow["{\log_G}", from=2-1, to=2-2]
  \end{tikzcd}
  \end{equation*}
  commutes. In particular, attached to any $\cF \in \FMArbOver AR$ comes a 
  natural morphism 
  $$\log_\cF: \cF \otimes (R \otimes_A K) \to \GG_a \otimes (\Lie(\cF) \otimes_A K).$$
  \begin{proof}
    The square commutes because $\Hom(\GG_{a, R\otimes_AK}^n, \GG_{a,
    R\otimes_AK}^m) = \Hom_{R \otimes_A K}((R\otimes_A K)^n, (R\otimes_A K)^m)$
    and $\Lie(\log_G \circ \phi \circ \exp_H) = \Lie(\phi).$
  \end{proof}
\end{lem}

\begin{lem}\label{lem:KernelOfLog}
  Let $K$ be a local field with integers $\cO_K$ and a choice of uniformizer $\pi \in \cO_K$, 
  and let $F$ be a Lubin-Tate $\cO_K$-module law corresponding to some 
  $f \in \cF_\pi$, cf. Theorem \ref{thm:LTModLaw}. Let $S$ be an admissible 
  $\cO_K$-algebra, and let $s \in S^\cici$ be an element such that the
  series $\log_{\cF}(s)$ 
  converges. Then we have $\log_F(s) = 0$ if and only if $[\pi]^r_F(s) = 0$ for
  some $r \in \N$. 
\begin{proof}
  Up to canonical isomorphism, $F$ is a $\cO_K$-module law with $[\pi]_F(T) =
  \pi T + T^q$. Now one may check that 
  \begin{equation*}
    \log_F(T) = \lim_{r \to \infty} \frac{ [\pi]_F^r(T) }{\pi^r} = \prod_{i=1}^\infty 
    \frac{[\pi]^i_F(T)}{\pi [\pi]^{i-1}_F(T)},
  \end{equation*}
  where convergence is to be taken coefficient-wise. After plugging in 
  $s \in S^\cici$, we see that the product vanishes if and only if $[\pi]_F^r(s) = 0$
  for some $r \in \N$. 
\end{proof}
\end{lem}
