We return to the more general framework where $A$ is an integral domain with 
field of fractions $K$. We review results from section
2 and 3 of \cite{hopkins1994equivariant}. 
Suppose that $F$ is a formal $A$-module law over an $A$-algebra $R$. 

\begin{defi}[Invariant Differentials for module laws.]
  The module $\omega(F)$ of invariant differentials is the submodule of the module of 
  differentials
  \begin{equation*}
    \Omega_{R\llbr T \rrbr/R} \cong R\llbr T \rrbr \dc T,
  \end{equation*}
  cut out by the condition that all $\omega \in \omega(F)$ satisfy 
  \begin{equation}\label{eq:diffcond}
    \omega(F(X,Y)) = \omega(X) + \omega(Y)\quad \text{and} \quad \omega([a]_F(X)) = a\omega(X).
  \end{equation} 
  for all $a \in A$. 
\end{defi}

Given a formal group law $F$, it is possible to explcitely calculate the
$R$-module $\omega(F)$, which we now explain. Let $f(X,Y)$ denote
$(\partial_x F)(X,Y)$, the derivative of $F(X,Y)$ with respect to $X$. Denote
$g(Y) = f(0,Y)$. Then $g$ is a unit in $R\llbr Y \rrbr$; and we construct
$\omega_F(X) \coloneqq \frac 1{g(X)} \dc X$. Checking that $\omega_F$ is indeed
invariant is a matter of applying the chain rule.

All other invariant differentials are scalar multiples of $\omega_F$. 
\begin{prop}
  \begin{enumerate}
    \item The $R$-module $\omega(R)$ is free of rank $1$ generated by $\omega_F$ 
    \item There is a non-degenerate pairing $\omega(F) \times \Lie(F) \to R$.
  \end{enumerate}
\begin{proof}
  Part one is \cite[Proposition 2.2]{hopkins1994equivariant}. 
\end{proof}
\end{prop}

\begin{xpl}
  The invariant differentials for $\GG_m$ are spanned by the form 
  $\omega_1(X) = \frac 1{1+X} \dc X$. 
\end{xpl}

The conditions imposed on invariant differentials remind of those imposed on 
morphisms of $A$-module laws $F \to \GG_a$. And indeed, there is a map
\begin{equation} \label{eq:functorinvdifftohom}
  \dc_F : \Hom_{\FMLOver AR} (F, \GG_{a,R}) \to \omega(F), \quad f \mapsto \dc f(X)
\end{equation}
One may check that $\End(\GG_{a,R}) \supseteq R$, turning $\dc$ in a map of $R$-modules.
\begin{prop}
  \begin{enumerate}
    \item If $R$ is a flat $A$-algebra, the map $\dc_F$ is injective.
    \item If $R$ is a $K$-algebra, the map $\dc_F$ is an isomorphism.
  \end{enumerate}
\begin{proof}
  \cite[Chapter 3]{hopkins1994equivariant} \todo{PROOF}\red{ Everything is easy if 
  $K$ has characteristic $0$, as we can integrate the differential forms.
  The proof in positive characteristic is a bit tricky; First it is shown that 
  there is an isomorphism of formal goups $F \cong \GG_a$, which is immediate.
  Then that there is a unique homomorphism $f: \GG_a \to \GG_a$ that maps to $\omega_F$
  and behaves well with respect to the $A$-module structure on $F$. }
\end{proof}
\end{prop}

In particular, if $R$ is a $K$-algebra, the invariant differential 
$\omega_F(X)$ constructed above comes from a homomorphism $f(X) = X + c_2 X^2 + \dots$,
which is an isomorphism by lemma \ref{lem:FGLeasyfacts}. 
This allows us to define the logarithm attached to $F$.
\begin{defi}[Logarithm and Exponential]
  If $R$ is a flat $A$-algebra, there is a unique power series
  \begin{equation*}
    \log_F(X) = X + c_2 X^2 + \dots \in (R \otimes_A K)\llbr X \rrbr 
  \end{equation*}
  inducing an isomorphism $F \otimes (R \otimes K) \to \GG_{a,R\otimes K}$.
  This power series is called the logarithm attached to $F$. 
  The inverse of $\log_H$ is called the exponential of $F$ and will be denoted by
  $\exp_H$.
\end{defi}
% subsubsection Logarithms (end)


