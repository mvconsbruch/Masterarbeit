\blue{
Again, $A$ is a complete discrete valuation ring with uniformizing parameter 
$\pi$ and finite residue field $k = A/\pi A$. We write $q$ for the cardinality of 
$k$ and $K$ for the residue field of $A$. Let $R$ be a (commutative) $A$-algebra
with structure map $i: A \to R$.}

We review results from section
2 and 3 of \cite{hopkins1994equivariant}. 
Suppose that $\bF = (F_1, \dots, F_n)$ is an $n$-dimensional formal $A$-module law over an
$A$-algebra $R$. Write $\bX = (X_1, \dots, X_n)$, $\bY = (Y_1, \dots, Y_n)$, etc.

\begin{defi}[Invariant Differentials for module laws.]
  The module $\omega(\bF)$ of invariant differentials is the submodule of the
  $R$-module of differentials
  \begin{equation*}
    \Omega_{R\llbr T_1, \dots, T_n \rrbr/R} \cong \bigoplus_{i=1}^n R\llbr T_1, \dots, T_n
    \rrbr \dc T_i,
  \end{equation*}
  cut out by the condition that any $\omega \in \omega(\bF)$ satisfies
  \begin{equation}\label{eq:diffcond}
    \omega(\bF(\bX,\bY)) = \omega(\bX) + \omega(\bY)\quad \text{and} \quad
    \omega([a]_\bF(\bX)) = a\omega(\bX).
  \end{equation} 
  for all $a \in A$. 
\end{defi}

It is possible to explicitly calculate a basis for the $R$-module
$\omega(\bF)$, which we now explain. Let 
$$A(\bX, \bY) \in \Mat_{n \times n} (R\llbr \bX, \bY \rrbr)$$ 
denote the matrix $\left((\partial/\partial X_j)F_i (\bX,\bY)\right)_{i,j}$,
the derivative of $\bF(\bX,\bY)$ with respect to $\bX$. Write 
$B(\bY) = A(0,\bY)$. Then $B$ is a unit in $\Mat_{n \times n} R\llbr \bY \rrbr$; 
and we write $(C_{ij}(\bY))$ for the components of 
$B(\bY)^{-1}$. We now construct 
$$\omega_{i} \coloneqq \sum_{j=1}^n C_{ij}(\bX) \dc X_j \in \Omega_{R\llbr \bX \rrbr/R}$$ 
for $1 \leq i \leq n$. By definition we have 
\begin{equation}\label{eq:coeffofcanonicaldiff}
  C_{ij}(0) = \begin{cases}
    1 &\text{ if }i = j,\\
    0 &\text{ otherwise.}
  \end{cases}
\end{equation}
Checking that $\omega_{i}$ is an invariant differential is a matter of
applying the chain rule, and we have
\begin{prop}
    The $R$-module $\omega(\bF)$ is free of rank $n$ generated by invariant differentials
    $\omega_{1}, \omega_{2}, \dots, \omega_{n}$.
\begin{proof}
  This is \cite[Proposition 1.1]{1970HondaFormalGroups}. 
\end{proof}
\end{prop}
\begin{xpl}
  The invariant differentials for $\GG_a$ are spanned by the form $\dc X$. 
  The invariant differentials for $\GG_m$ are spanned by the form 
  $\omega_1(X) = \frac 1{1+X} \dc X$.
\end{xpl}
By the Proposition above and Equation \eqref{eq:coeffofcanonicaldiff}, we may
define a pairing
\begin{equation*}
  \omega(\bF) \times \Lie(\bF) \to R, \quad \langle X_i, \omega_j \rangle =
  \begin{cases}
    1 &\text{ if } i = j,\\
    0 &\text{ otherwise.}
  \end{cases}
\end{equation*}
This pairing is independent of the parametrization of $\bF$. In particular, it
descents to a pairing defined for formal modules $\cF \in \FMArbOver A R$, and
we have a natural isomorphism $\omega(\cF) \cong \Hom_R(R, \Lie(\cF))$.

Let $\GG_a$ be the additive formal $A$-module over $R$. There is a map
\begin{equation} \label{eq:functorinvdifftohom}
  \dc_\bF : \Hom_{\FMLOver AR} (\bF, \GG_{a,R}) \to \omega(\bF), \quad f \mapsto \dc f(\bX)
\end{equation}
which is a map of $R$-modules if we equip the left hand side with the $R$-module
structure coming from the natural action of $R \subset \End(\GG_a)$. 
\begin{prop}\label{prop:loginvdiff}
  \begin{enumerate}
    \item If $R$ is a flat $A$-algebra, the map $\dc_F$ is injective.
    \item If $R$ is a $K$-algebra, the map $\dc_F$ is an isomorphism.
  \end{enumerate}
\begin{proof}
  \cite[Proposition 3.2]{hopkins1994equivariant} \todo{PROOF}\red{ Everything is easy if 
  $K$ has characteristic $0$, as we can integrate the differential forms.
  The proof in positive characteristic is a bit tricky; First it is shown that 
  there is an isomorphism of formal goups $F \cong \GG_a$, which is immediate.
  Then that there is a unique homomorphism $f: \GG_a \to \GG_a$ that maps to $\omega_F$
  and behaves well with respect to the $A$-module structure on $F$. }
\end{proof}
\end{prop}

Suppose now that $F \in \FMLOver A R$ is a one-dimensional formal module. 
The previous Proposition shows that if $R$ is a $K$-algebra, the invariant differential 
$\omega_1(X)$ constructed above comes from a homomorphism $f(X) = X + c_2 X^2 + \dots$,
which is an isomorphism by Lemma \ref{lem:FGLeasyfacts}. 
This allows us to define the logarithm attached to $F$.
\begin{defi}[Logarithm and Exponential]
  If $R$ is a flat $A$-algebra, there is a unique power series
  \begin{equation*}
    \log_F(X) = X + c_2 X^2 + \dots \in (R \otimes_A K)\llbr X \rrbr 
  \end{equation*}
  inducing an isomorphism $F \otimes (R \otimes K) \to \GG_{a,R\otimes K}$.
  This power series is called the logarithm attached to $F$.   The inverse of $\log_F$ is called the exponential of $F$ and will be denoted by
  $\exp_F$.
\end{defi}
\begin{rmk} \leavevmode
  \begin{enumerate}
    \item We have $\dc \log_F(X) = \omega_1(X)$, hence $\log'_F(X) \in R\llbr X \rrbr$. 
    \item These definitions do not descent to formal modules. Given $\cF \in \FMOver A
  R$, there is no natural choice of Logarithms and Exponentials, these
  definitions depend on the choice of parametrization
  $\cF \cong \Spf R \llbr T \rrbr$. 
  \end{enumerate}
  \end{rmk}



