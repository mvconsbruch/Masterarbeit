As promised in the introduction, we begin by defining {formal group
laws}.
For now, let $A$ be any ring.
\begin{defi}[Formal Group Laws of arbitrary dimension]
  A (commutative, one-dimensional) formal group law  of dimension
  $n$ over $R$ is a power series $ F(X,Y) \in R\llbr X ,Y \rrbr $ 
  such that $F( X,  Y) \equiv X + Y $ modulo degree $\geq 2$
  and the following equalities are satisfied:
  \begin{enumerate}
    \item $ F( F ( X,  Y),  Z) = 
       F( X,  F( Y,  Z))$.
    \item $ F(  X,  0) =  X$.
    \item $ F(  X,  Y) =  F( Y,  X).$
  \end{enumerate}
  Given two formal group laws $F$ and $G$
  a morphism $ F \to  G$ is a power series $f \in R\llbr T \rrbr$ such that
  $f(0)=0$ and
  \begin{equation*}
    G( f(  X),  f(  Y) ) = 
    f( F(  X,  Y)).
  \end{equation*}
  For any formal module $F$, the identity is given by 
  the morphism $\id_{F}(X) = X$. 
  Composition of morphisms is given by composition of power-series.
  This yields the category of formal modules of dimension $1$ over $R$,
  which we denote by $\FGLOver R$. 
\end{defi}
We encounter higher-dimensional group laws later in this section, but they only
play an auxilliary role, see Definition \ref{def:FGLArb}.

The following statements about morphisms of formal group laws are 
useful and easily verified.
\begin{lem}\label{lem:FGLeasyfacts}
  Let $R$ be a ring and let $F,G \in R\llbr X,Y\rrbr$ be two formal
  (one-dimensional) group laws over $R$. 
  \begin{enumerate}
    \item The set $\Hom_{\FGLOver R} (F,G)$
      is an abelian group with addition $f+g = G(f,g)$, and $\End_\FGLOver R(F)$
      has a natural ring structure with multiplication given by composition.
    \item A morphism $f = c_1 T + c_2 T^2 + \dots \in R\llbr T \rrbr$ between
      $F$ and $G$ is an isomorphism if and only if $c_1 \in R^\times$.
  \end{enumerate}
\end{lem}

\begin{xpl} Let us introduce the following two formal group laws.
  \begin{itemize}
    \item \textit{The additive formal group law}. Write 
      $\GG_a$ for the formal group law with addition given by 
      $\GG_a(X,Y) = X + Y$. 
    \item We write $\GG_m$ for the formal group law associated with the 
      with $\GG_m(X,Y) = X + Y + XY$. Note that $\GG_m(X,Y) = (X+1)(Y+1) - 1$
  \end{itemize}
\end{xpl}

Next up is the definition of formal $A$-module laws. Naively, we would like to 
define formal $A$-module laws as formal group laws $F$
with $A$-module structure, i.e. a morphism of rings $[\cdot]_F: A \to
\End_\FGLOver R(F)$. But there is a subtlety: let 
\begin{equation*}
  \Lie: \FGLOver R \to \Ab 
\end{equation*}
be the (constant) functor that sends $F \in \FGLOver R$ to $(R,+)$, and morphisms
$f: G \to H$ given by a formal power series
$f = c_1 T + c_2 T^2 + \dots \in R\llbr T \rrbr$ to the endomorphism
of $R$ given by multiplication with $c_1$. The condition that 
$F(X,Y) \equiv X+Y$ modulo degree $2$ enforces that the induced map
$\End(F) \to \End(R)$ is a morphism of rings. Now, the $A$-module structure on $F$ 
yields an $A$-module structure on $R$, given by the composition
\begin{equation*}
  A \xto{[\cdot]_F} \End(F) \xto{\Lie} \End(R), \quad a \mapsto \Lie([a]_F)
\end{equation*}
This is a morphism of rings, and we obtain an $A$-algebra structure on $R$. 
This motivates the following definition.
\begin{defi}[Formal $A$-module law]\label{def:formalmodulelaw}
  Let $A$ be a ring and $R$ be an $A$-algebra with structure
  morphism $i: A \to R$. A (one-dimensional) $A$-module law over $R$ is a
  pair $(F, ([a]_F)_{a \in A})$, where $F \in R\llbr X,Y \rrbr$ is 
  a formal group law and $[a]_F = i(a)X + c_2X^2 + \dots 
  \in R\llbr X \rrbr$ yield endomorphisms $F \to F$ such that the induced map
  \begin{equation*}
    A \to \End(F), \quad a \mapsto [a]_F 
  \end{equation*}
  is a morphism of rings. A morphism $f: F\to G$ between formal $A$-module laws
  $F$ and $G$ is a morphism 
  $f(T) \in \Hom_{\FGLOver A R}(F,G)$ satisfying the additional constraint
  $f([a]_F(T)) = [a]_G(f(T))$ for all $a \in A$.
\end{defi}
Similarly to above, we obtain a category of formal $A$-module laws over $R$,
which we denote by $\FMLOver A R$. Note that $\FGLOver R \cong \FMLOver \Z R$.
Slightly abusing notation, we usually do not explicitely mention the $A$-structure
when referring to formal module laws, simply writing $F \in \FMLOver AR$, for
example. 

The following lemma explains a the functoriality of the assignment
$R \mapsto \FMLOver A R$.

\begin{lem}\label{lem:FMLFunc}
  The assignment $R \mapsto \FMLOver A R$ is functorial in the following sense.
  If $i: R \to R'$ is a morphism of $A$-algebras, we obtain a functor
  \begin{equation*}
    \FMLOver A R \to \FMLOver A {R'}, \quad F \mapsto F \otimes_R R',
  \end{equation*}
  where $F \otimes_R R'$ is the formal $A$-module law obtained by applying $i$
  to the coefficients of the formal power series representing the $A$-module
  structure of $F$. 
\end{lem}

Note that every formal module law $F \in \FMLOver A R$ yields a functor
\begin{equation}\label{eq:fmnilpfunc}
  \Alg R \to \Mod A, \quad S \mapsto \Nil(S),
\end{equation}
where $\Nil(S)$, the set of nilpotent elements of $S$, is equipped with
addition and scalars given by 
\begin{equation*}
  s_1 + s_2 = F(s_1, s_2) \in \Nil(S), \quad a s = [a]_F(s) \in \Nil(S).
\end{equation*}
This construction yields a functor (with slight abuse of notation)
\begin{equation}\label{eq:formfunc}
  \FMLOver A R \to \Fun(\Alg R, \Mod A),
\end{equation}
where $\Fun$ denotes the functor category.

Passing from discrete $R$-algebras to admissible $R$-algebras (cf. Definition
\ref{def:admring}), this construction extends naturally to a functor 
\begin{equation*}
  \FMLOver AR \to \Fun (\Adm R, \Mod A), \quad F \mapsto \Spf R\llbr T \rrbr,
\end{equation*}
where we equip $\Spf R \llbr T \rrbr$ with the structure of an $A$-module object
using the endomorphisms coming from $F$. 
Following this line of thought leads naturally to the definition of
formal modules. 

\begin{defi}[Formal Group and Formal Module.]
  Let $X$ be an $A$-scheme. A formal $A$-module $\cF$ over $X$ 
  is an $A$-module object in $\FSchOver X$, the category of formal
  schemes over $X$, satisfying the following condition.
  That there is a Zariski-covering $(\spec(R_i))_{i \in I}$ of $X$ with $\cF
  \times_{X} U_i \cong \spf(R_i\llbr T \rrbr)$ for every $i\in I$ the induced
  $A$-module structure on $\spf(R_i\llbr T \rrbr)$ comes from a formal
  $A$-module law $F_i$ over $R_i$.
\end{defi}

\begin{rmk} 
  Formal schemes (over a base an $A$-scheme $X$, say) locally isomorphic to 
  $\spf \cO_X(U)\llbr T \rrbr$ are sometimes called (one-dimensional) Formal
  Lie Varieties \todo{reference}. Equivalently to the definition above, we could
  have defined formal $A$-modules as $A$-module objects (of relative dimension one
  over $X$) in the category of Formal Lie Varieties, such that the $A$-module structure
  on the tangent space at the identity agrees with the usual one.
\end{rmk}

\begin{defi}[Coordinate]
  Let $\cF$ be a formal $A$-module over $X$. The choice of a cover $\sqcup_{i
  \in I} \spec(R_i) \to X$ together with isomorphisms $\cF \times_X \spec(R_i)
  \cong \Spf(R_i\llbr T \rrbr)$ will be referred to as a coordinate of $\cF$. 
\end{defi}

Of course there is a functor 
\begin{equation*}
  \FMLOver AR \to \FMOver AR,
\end{equation*}
essentially forgetting the choice of module law. The observation of Lemma 
\ref{lem:FMLFunc} translates to formal modules, a morphism $p : R \to R'$ 
yields a functor 
\begin{equation*} 
  p_*: \FMOver AR \to \FMOver A{R'}, \quad \cF \mapsto \cF \otimes_R R'.
\end{equation*}

\begin{xpl}
  The additive group law $\GG_a$ extends to a formal $A$-module over an affine base
  $\spec R$ by setting 
  \begin{equation*}
    [a]_{\GG_a}(T) = aT
  \end{equation*}
  for $a \in A$. More generally, we obtain a formal $A$-module over an
  arbitrary base scheme $X$ over $A$.

  Over $\Z_p$, the formal group $\GG_m$ extends to a formal 
  $\Z_p$-moduel as follows. 
  As a functor, $\GG_m$ is isomorphic to the assignment
  \begin{equation*}
    \Adm {\Z_p} \to \Ab, \quad S \mapsto 1 + S^\cici \subset S^\times.
  \end{equation*}
  Here, we equipped $\Z_p$ with the $p$-adic topology.
  The subgroup $1 + S^\cici$ naturally carries the structure of a $\Z_p$-module.
  Indeed, for $k \in \N$, we have
  \begin{equation*}
    (1+s)^{p^k} = 1 + p^ks + \binom{p^k}2 s^2 + \dots + s^{p^k},
  \end{equation*}
  and given $s \in S^\cici$, this is of the form $1+ o(1)$ as $k$ gets large. 
  In particular, if $x = a_0 + a_1 p + a_2p^2 + \dots \in \Z_p$, expressions of
  the form
  \begin{equation*}
    (1+s)^x = \prod_{i = 1}^\infty (1+s)^{a_k p^k}
  \end{equation*}
  make sense by lemma \ref{lem:infiniteseriesandproducts}. This gives
  $\GG_{m,\Z_p}$ the structure of a formal $\Z_p$-module. 
  In the upcoming subsection, we discuss how this is the simplest example of a
  whole family of formal modules constructed by 
  Lubin and Tate. In section \ref{sec:Local Class Field Theory} we explain applications of these formal modules to local class field theory.
\end{xpl}
