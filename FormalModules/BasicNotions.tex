As promised in the introduction, we begin by defining {formal group
laws}.
For now, let $A$ be any ring.
\begin{defi}[Formal Group Laws of arbitrary dimension]
  A (commutative) formal group law  of dimension
  $n$ over $R$ is a tuple of
  power series $F = (F_1, \dots, F_n)$ with $$F_i(X_1, \dots, X_n,Y_1,
  \dots, Y_n) \in R\llbr X_1, \dots,
  X_n, Y_1, \dots, Y_n\rrbr, \quad 1 \leq i \leq n$$
  such that $F_i(\mathbf X, \mathbf Y) \equiv X_i + Y_i $ modulo degree $\geq 2$
  and the following equalities are satisfied:
  \begin{enumerate}
    \item $F(F (\mathbf X, \mathbf Y), \mathbf Z) = 
      F(\mathbf X, F(\mathbf Y, \mathbf Z))$.
    \item $F( \mathbf X, \mathbf 0) = \mathbf X$.
    \item $F( \mathbf X, \mathbf Y) = F(\mathbf Y, \mathbf X).$
  \end{enumerate}
  Here, and in the sequel, we abbreviate $\mathbf X =
  (X_1, \dots, X_n)$, et cetera.
  Given a formal group $F$ of dimension $n$ and a formal group law
  $G$ of dimension $m$,
  a morphism $F \to G$ is a $m$-tuple $f = (f_1, \dots, f_m)$ 
  of power series $f_i \in R\llbr X_1, \dots, X_n \rrbr$ such that $\bff(0)=0$ and
  \begin{equation*}
    G(f( \mathbf X), f( \mathbf Y) ) = f(F( \mathbf X, \mathbf Y)).
  \end{equation*}
  For any $n$-dimensional formal module $F$, the identity is given by 
  the morphism $\id_{F}$ with components $\id_{F,i}( \mathbf X) = X_i$. 
  Composition of morphisms is given by composition of tuples of power-series.
  This yields the category of formal modules of arbitrary dimension over $R$,
  which we denote by $\FGLArbOver R$. We will mostly be concerned with the full
  subcategory of one-dimensional formal groups, which we denote by $\FGLOver R$. 
\end{defi}

\begin{lem}\label{lem:FGLAdditive}
  \begin{enumerate}
    \item The set $\Hom_{\FGLArbOver R} (F,G)$
      is an abelian group with addition $f+g = G(f,g)$. 
      In particular, $\FGLArbOver R$ is pre-additive (cf.
      \cite[\href{https://stacks.math.columbia.edu/tag/00ZY}{Tag
      00ZY}]{stacks-project}).     
    \item Furthermore, $\FGLArbOver R$ admits finite products. Thereby it is 
      an additive category (cf. \cite[\href{https://stacks.math.columbia.edu/tag/0104}{Tag
      0104}]{stacks-project}).
      The unique final and initial object of $\FGLArbOver R$ is the unique
      $0$-dimensional formal $A$-module law.
    \item In particular $\End_{\FGLArbOver R}(F)$ is a (possibly non-commutative)
      ring.
  \end{enumerate}
\end{lem}

\begin{xpl} Let us introduce the following two formal group laws.
  \begin{itemize}
    \item \textit{The additive formal group law}. Write 
      $\GG_a$ for the formal group law with addition given by 
      $\GG_a(X,Y) = X + Y$. 
    \item We write $\GG_m$ for the formal group law associated with the 
      with $\GG_m(X,Y) = X + Y + XY$. Note that $\GG_m(X,Y) = (X+1)(Y+1) - 1$
  \end{itemize}
\end{xpl}

Next up is the definition of formal $A$-module laws. Naively, we would like to 
define formal $A$-module laws as formal group laws $F$
with $A$-module structure, i.e. a morphism of rings $[\cdot]_F: A \to
\End_\FGLArbOver R(F)$. But there is a subtlety, which becomes evident after defining
the Lie-algebra of a formal group law. 
\begin{defi}[Lie-algebra of formal group law]
Let $\Lie: \FGLArbOver R \to \Ab$ be the functor taking an $n$-dimensional 
formal group law $\mathbf F$ to the $R$-module
\begin{equation*}
  \Lie(\bF) = \Hom_{\Mod R} \left( \frac{(X_1, \dots, X_n)}{(X_1, \dots,
  X_n)^2}, R \right) 
\end{equation*}
Given an $m$-dimensional group law $\bG$ and a morphism $\mathbf f: \bF \to
\bG$, $\Lie(\bff)$ is the induced morphism
\begin{equation*}
  \Lie(\bF) \to \Lie(\bG), \quad \psi \mapsto \left( S_j \mapsto \psi
  (\overline{f_j})\right) \in 
  \Hom_\Mod R\left(\frac{(X_1, \dots, X_n)}{(X_1, \dots, X_n)^2}, R \right),
\end{equation*}
where $\overline{ f_j}$ is the reduction of $f_j$ mod $(\bX)^2$. 
\end{defi}
We have a canonical basis on both sides, and writing $\Lie(\bF) = R^n$, 
$\Lie(\bG) \cong R^m$, the induced map $\Lie(\bff) : R^n \to R^m$ is
given by multiplication with the matrix 
\begin{equation*}
  \left( \frac {\partial f_j}{\partial X_i} (0) \right)_{i,j}.
\end{equation*}

Given $F \in \FGLOver R$, the condition that 
$F(X,Y) \equiv X+Y$ modulo degree $2$ enforces that the induced map
$\End(F) \xto{\Lie} \End(R)$ is a morphism of rings. 
If we are given $[-]_F: A \to \End_{\FGLOver R}(F)$, this $A$-module structure on $F$ 
yields an $A$-module structure on $R$, given by the composition
\begin{equation*}
  A \xto{[\cdot]_F} \End(F) \xto{\Lie} \End(R), \quad a \mapsto \Lie([a]_F)
\end{equation*}
This is a morphism of rings, and we obtain an $A$-algebra structure on $R$. 
This motivates the following definition.
\begin{defi}[Formal $A$-Modules of arbitrary dimension]\label{def:formalmodulelaw}
  Let $R$ be an $A$-algebra with structure morphism $j: A \to R$. A formal
  $A$-module over $R$ of dimension $n$ is
  given by the data of 
  a formal $n$-dimensional group law $F$ over $R$ and a morphism of rings
  \begin{equation*}
    A \to \End_{\FGLArbOver R} (F), \quad a \mapsto ([a]_{F,i}
    )_{1 \leq i \leq n} \in (R \llbr X_1, \dots, X_n \rrbr )^n
  \end{equation*}
  such that $[a]_{F,i}(\mathbf X) \equiv j(a) X_i$ modulo terms of degree 
  $\geq 2$. Morphisms between formal $A$-modules of arbitrary dimension are 
  morphisms of formal groups respecting the $A$-module structure. 
  The resulting category is denoted $\FMLArbOver A R$. As before, the full
  subcategory of one-dimensional formal $A$ modules over $R$ is denoted 
  $\FMLOver A R$.
\end{defi}

Note that $\FGLOver R \cong \FMLOver \Z R$.
Slightly abusing notation, we usually do not explicitely mention the $A$-structure
when referring to formal module laws, simply writing $F \in \FMLOver AR$, for
example. 

The following lemma explains a the functoriality of the assignment
$R \mapsto \FMLArbOver A R$.

\begin{lem}\label{lem:FMLFunc}
  The assignment $R \mapsto \FMLArbOver A R$ is functorial in the following sense.
  If $i: R \to R'$ is a morphism of $A$-algebras, we obtain a functor
  \begin{equation*}
    \FMLArbOver A R \to \FMLArbOver A {R'}, \quad F \mapsto F \otimes_R R',
  \end{equation*}
  where $F \otimes_R R'$ is the formal $A$-module law obtained by applying $i$
  to the coefficients of the formal power series representing the $A$-module
  structure of $F$. 
\end{lem}

Note that every $n$-dimensional formal module law $F \in \FMLArbOver A R$
yields a functor
\begin{equation}\label{eq:fmnilpfunc}
  \Alg R \to \Mod A, \quad S \mapsto \Nil(S)^n,
\end{equation}
where $\Nil(S)^n$, the set of $n$-tuples of nilpotent elements of $S$, is equipped with
addition and scalars given by 
\begin{equation*}
  s_1 + s_2 = F(s_1, s_2) \in \Nil(S)^n, \quad a s = [a]_F(s) \in \Nil(S)^n.
\end{equation*}
This construction yields a functor (with slight abuse of notation)
\begin{equation}\label{eq:formfunc}
  \FMLOver A R \to \Fun(\Alg R, \Mod A),
\end{equation}
where $\Fun$ denotes the functor category.

Passing from discrete $R$-algebras to admissible $R$-algebras (cf. Definition
\ref{def:admring}), this construction extends naturally to a functor 
\begin{equation*}
  \FMLOver AR \to \Fun (\Adm R, \Mod A), \quad F \mapsto \Spf R\llbr \bT \rrbr,
\end{equation*}
where we equip $\Spf R \llbr \bT \rrbr$ with the structure of an $A$-module object
using the endomorphisms coming from $F$. 
Following this line of thought leads naturally to the definition of
formal modules. 

\begin{defi}[Formal Groups and Formal Modules.]
  Given an $A$-scheme $X$, we define the category
  $\FMArbOver A X$ as follows. 
  Objects are $A$-module objects $\cF$ in the category
  of formal schemes over $X$, having the property that 
  there is a cover of $X$ by Zariski-open affine subsets $U_i = \spec (R_i)$
  such that $\cF \times_X U_i$ is isomorphic to $\Spf R_i\llbr X_1, \dots,
  X_n\rrbr$ and the induced $A$-module structure on $\spf R_i\llbr X_1, \dots, X_n\rrbr$
  yields a formal $A$-module law on $R_i$. Given $\cF, \cG \in \FMLArbOver AX$,
  a morphism $\phi: \cF \to \cG$ in the category of formal schemes over $X$ is
  a morphism of formal $A$-modules if $\phi$ is a morphism of $A$-module objects.
  Again, we denote the category of one-dimensional formal $A$-modules over $X$
  by $\FMOver AX$. 
\end{defi}

\begin{rmk} 
  Formal schemes (over a base an $A$-scheme $X$, say) locally isomorphic to 
  $\spf \cO_X(U)\llbr \bT \rrbr$ are sometimes called Formal
  Lie Varieties \todo{reference}. Equivalently to the definition above, we could
  have defined formal $A$-modules as $A$-module objects in the category of
  Formal Lie Varieties, such that the $A$-module structure
  on the tangent space at the identity agrees with the usual one.
\end{rmk}

\begin{defi}[Coordinate]
  Let $\cF$ be a formal $A$-module over $X$. The choice of a cover $\sqcup_{i
  \in I} \spec(R_i) \to X$ together with isomorphisms $\cF \times_X \spec(R_i)
  \cong \Spf(R_i\llbr \bT \rrbr)$ will be referred to as a coordinate of $\cF$. 
\end{defi}

Of course there is a functor 
\begin{equation*}
  \FGG: \FMLOver AR \to \FMOver AR,
\end{equation*}
essentially forgetting the choice of module law. The observation of Lemma 
\ref{lem:FMLFunc} translates to formal modules, a morphism $j : R \to R'$ 
yields a functor 
\begin{equation*} 
  \FMOver AR \to \FMOver A{R'}, \quad \cF \mapsto \cF \otimes_R R'.
\end{equation*}

\begin{defi}[Lie functor]
  The functor $\Lie$ descents to a functor 
  \begin{equation*}
    \Lie: \FMArbOver A X \to \QCoh {\cO_X}, 
  \end{equation*}
  given by locally describing a formal $A$-module $\cF$ via formal group laws
  and glueing the local data. Alternatively, it arises from sending 
  sending a formal $A$-module $\cF$ to $(\cI / \cI^2)^\vee$, where $\cI$ is the
  ideal associated to the closed immersion $[0]_\cF: X \to \cF$. 
\end{defi}

\begin{lem}\label{lem:IsosCheckOnLie}
  A map $\phi: \cF \to \cG$ of formal $A$-modules (of arbitrary dimension) over
  $X$ is an isomorphism if and only if the induced
  morphism of Lie-algebras $\Lie(\phi): \Lie(\cF) \to \Lie(\cG)$ is an isomorphism.
\begin{proof}
  This is easily verified in the one-dimensional situation after choosing coordinates.
  The general case adds no complication. 
\end{proof}
\end{lem}

\begin{defi}[Formal Module associated to $R$-module]
  \label{def:additiveformalmoduleassociatedtomodule}
  Suppose that $M$ is a projective $R$-module. Then we write
  $\GG_a \otimes M$ for the additive formal $A$-module associated to $M$ over $R$.
  As a formal scheme, this formal module is given by
  \begin{equation*}
    \GG_a \otimes_A M \cong \spf R \llbr M^\vee \rrbr,
  \end{equation*}
  where $R \llbr M^\vee \rrbr$ denotes the completion of $\Sym_R(M^\vee)$ with respect
  to the ideal generated by $M^\vee$. The (formal) $A$-module structure is the
  canonical additive one. 
  Note that $\Lie(\GG_a \otimes M) = M$ by design. 
  More generally, if $X$ is a quasi-compact and quasi-separated $A$-scheme
  and $\cM$ is a finite locally free quasi-coherent 
  $\cO_X$-module, this construction yields a formal $A$-module
  $\GG_a \otimes \cM$ over $X$.
\end{defi}

\begin{xpl}
  The additive group law $\GG_a$ extends to a formal $A$-module over an affine base
  $\spec R$ by setting 
  \begin{equation*}
    [a]_{\GG_a}(T) = aT
  \end{equation*}
  for $a \in A$. More generally, we obtain a formal $A$-module over an
  arbitrary base scheme $X$ over $A$.

  Over $\Z_p$, the formal group $\GG_m$ extends to a formal 
  $\Z_p$-moduel as follows. 
  As a functor, $\GG_m$ is isomorphic to the assignment
  \begin{equation*}
    \Adm {\Z_p} \to \Ab, \quad S \mapsto 1 + S^\cici \subset S^\times.
  \end{equation*}
  Here, we equipped $\Z_p$ with the $p$-adic topology.
  The subgroup $1 + S^\cici$ naturally carries the structure of a $\Z_p$-module.
  Indeed, for $k \in \N$, we have
  \begin{equation*}
    (1+s)^{p^k} = 1 + p^ks + \binom{p^k}2 s^2 + \dots + s^{p^k},
  \end{equation*}
  and given $s \in S^\cici$, this is of the form $1+ o(1)$ as $k$ gets large. 
  In particular, if $x = a_0 + a_1 p + a_2p^2 + \dots \in \Z_p$, expressions of
  the form
  \begin{equation*}
    (1+s)^x = \prod_{i = 1}^\infty (1+s)^{a_k p^k}
  \end{equation*}
  make sense by lemma \ref{lem:infiniteseriesandproducts}. This gives
  $\GG_{m,\Z_p}$ the structure of a formal $\Z_p$-module. 
  In the upcoming subsection, we discuss how this is the simplest example of a
  whole family of formal modules constructed by 
  Lubin and Tate. In section \ref{sec:Local Class Field Theory} we explain applications of these formal modules to local class field theory.
\end{xpl}
