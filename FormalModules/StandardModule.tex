If, $A$ is an integral domain and $R$ is a flat $A$-module, the structure of a formal
$A$-module $F$ over $R$ is uniquely determined by its logarithm $\log_H
\in R \otimes_A K \llbr T \rrbr$. Indeed, we find
\begin{equation*}
  F(X,Y) = \exp_H(X+Y), \quad [a]_F(X) = \exp_H(a X).
\end{equation*}
It is therefore natural to wonder about conditions on power series $f \in
(R\otimes_A K) \llbr T \rrbr$ ensuring that $f$ is the logarithm of some
formal group law. Hazewinkel found such a condition in his functional equation
lemma.

\begin{prop}[Hazewinkel's Functional Equation Lemma] 
  Let $p$ be a prime and $q = p^e$. Given an inclusion of rings $B \subseteq
  L$, an ideal $\fa \subseteq B$ containing $p$, an endomorphism of rings
  $\sigma: L \to L$ and elements $s_1, s_2, \dots \in L$ subject to the conditions
  that 
  \begin{equation*}
    \sigma(b) \equiv b^q \pmod \fa \text{ for all } b \in B \quad \text{and} \quad 
    \sigma^r(s_i) \fa \subset B \text{ for all } r,s \geq 1.
  \end{equation*}
  Suppose now that $f \in L\llbr T \rrbr$ has $f'(0) \in L^\times$ and
  satisfies the functional equation condition
  \begin{equation*}
    f(X) - \sum_{i=1}^\infty s_i (\sigma^i_* f)(X^{q^i}) \in B\llbr X \rrbr.
  \end{equation*}
  Then we have 
  \begin{equation*}
    F(X,Y) = f^{-1}(f(X) + f(Y)) \in B \llbr X,Y \rrbr,
  \end{equation*}
  where $f^{-1}$ is the inverse power series as in Lemma \ref{lem:FGLeasyfacts}.
  Also, if $g(Z) \in L\llbr Z \rrbr$ is another power series satisfying the 
  same condition
  \begin{equation*}
    g(Z) - \sum_{i=1}^\infty s_i (\sigma^i_* f)(Z^{q^i}) \in B\llbr Z \rrbr,
  \end{equation*}
  then $f^{-1}(g(Z)) \in B\llbr Z \rrbr$. 
  Furthermore, if $\alpha(T) \in B\llbr T \rrbr$ and $\beta(T) \in B \llbr T \rrbr$, then
  \begin{equation} \label{eq:funceqlemcongruence}
    \alpha(T) \equiv \beta(T) \pmod {\fa^r} \iff f(\alpha(T)) \equiv f(\beta(T))
    \pmod {\fa^r}
  \end{equation}

  \begin{proof}
    A more general statement can be found in \cite[Section
    2]{hazewinkel1979funceqexp}. Proofs can be found in \cite[Sections 2 and
    10]{hazewinkel1978formal}.
  \end{proof}
\end{prop}
Note that by construction, $F(X,Y)$ as defined above yields a (commutative)
formal group law over $B$. 
Let $B^\sigma$ denote the subring of elements in $B$ fixed by $\sigma$. Then 
the second part of the Functional Equation Lemma implies that we even obtain
formal $B^\sigma$-modules with $[b]_F(T) = f^{-1}(b f(T))$, as $bf(T)$
satisfies the same functional equation if $b \in B^\sigma$. 

We use the Functional Equation Lemma to construct Lubin--Tate Formal Group Laws. 
Hence we now enter the situation where $K$ is a local field with ring of integers
$\cO_K$ and uniformizer $\pi$. A special role will play the power series
\begin{equation*}
  f(T) = \sum_{i=1}^\infty \frac{T^{q^{in}}}{\pi^i} \in K\llbr T \rrbr.
\end{equation*}
It satisfies the functional equation
\begin{equation*}
  f(T) = T + \frac 1\pi f(T^{q^n}),
\end{equation*}
which is a functional equation of the form above, with 
$B = \cO_K$, $\fa = (\pi)$, $L = K$, $s_1 = \pi^{-1}$, $s_2 = s_3 = \dots = 0$,
$\sigma = \id_L$. 
Hence $f$ arises as the logarithm of a formal $\cO_K$-module law $H$ over $\cO_K$.
The fact that $f^{-1}(X) = X - \frac 1\pi X^{q^n} + \dots$ reveals
$[\pi]_H(T) \equiv \pi T$ mod $(T^2)$. Additionally, note that 
\begin{equation*}
  f([\pi]_H(T)) = \pi f(T) = \pi T + f(T^{q^n}) \equiv f(T^{q^n}) \pmod \pi.
\end{equation*}
Hence, the equivalence in \eqref{eq:funceqlemcongruence} implies that 
$[\pi]_H(T) \equiv T^{q^n}$ mod $\pi$. So $H$ is a Lubin--Tate formal $\cO_K$-module
law of height $n$, we call it the standard Lubin--Tate formal module law of
height $n$. 
\begin{rmk} 
  The formal $\cO_K$-module $H$ is a member of the set of so called $A$-typical
  formal modules - formal $A$-modules $F$ with logarithm of the 
  form
  \begin{equation*}
    \log_F(T) = \sum_{i=0}^\infty b_i X^{q^i}
  \end{equation*}
  for elements $b_0, b_1, \dots \in R \otimes_A K$. If $R$ is flat over $A$,
  every formal $A$-module over $R$ is isomorphic to an $A$-typical one
  (cf. \cite[21.5.6]{hazewinkel1978formal}). The following discussion remains
  valid for $\cO_K$-typical formal modules.
\end{rmk}

It will be convenient to make the terms in the exact sequence of Proposition 
\ref{prop:ExplicitInterestingES} explicit for $F = H$. As $F$ is of height $n>0$, 
there is no non-trivial map $F \to \GG_a$ and the sequence becomes
\begin{equation*}
\begin{tikzcd}[ampersand replacement=\&]
	0 \& {\omega(H)} \& {\RigExt(H,\GG_a)} \& {\Ext(H,\GG_a)} \& 0 \\
	0 \& \begin{array}{c} \left\{\begin{gathered} {g \in TK\llbr T \rrbr : \delta g = 0} \\          \text{ and $g'(T) \in \cO_K\llbr T \rrbr$}\end{gathered} \right\} \end{array} \& {\QLog(H)} \& {\frac{\SymCoc^2(H, \GG_a^n)}{\{\delta g \mid g \in T \cO_K\llbr T \rrbr\}}} \& 0.
	\arrow[from=1-1, to=1-2]
	\arrow[from=1-2, to=1-3]
	\arrow[from=1-2, to=2-2]
	\arrow[from=1-3, to=1-4]
	\arrow[from=1-3, to=2-3]
	\arrow["0", from=1-4, to=1-5]
	\arrow[from=1-4, to=2-4]
	\arrow["", from=2-1, to=2-2]
	\arrow[from=2-2, to=2-3]
	\arrow["\delta", from=2-3, to=2-4]
	\arrow["0", from=2-4, to=2-5]
\end{tikzcd}   
\end{equation*}
We now have 
\begin{prop}
  The $R$-module $\omega(H)$ is free of rank $1$, generated by 
  $f(T) = \log_H(T)$. $\QLog(H)$ is free of rank $n$, generated by the classes of
  $(f(T), \frac 1\pi f(T^q), \dots, \frac 1\pi f(T^{q^{n-1}}))$. Consequently,
  the short exact sequence above is given by 
  \begin{equation*}
    0 \to \left \langle f(T) \right \rangle \to \left \langle f(T), \frac 1\pi
      f(T^q) , \dots,
    \frac 1\pi f(T^{q^{n-1}}) \right \rangle \xto \delta 
    \left \langle \delta \left(\frac 1\pi f(T^q)\right),\dots, \delta
    \left(\frac 1\pi f(T^{q^{n-1}})\right ) \right \rangle \to 0.
  \end{equation*}
\begin{proof}
  It is easily checked that $\frac 1\pi f(T^{q^k})$ is a quasi-logarithm for 
  $1 \leq k \leq n-1$. As $\delta f = 0$, we have $f(T) \in \QLog(F)$ as well. 
  The claim is \cite[Proposition 13.8]{hopkins1994equivariant} which is a 
  special case of [ibid., Proposition 9.8].
\end{proof}
\end{prop}
% subsubsection Deformations of Formal Modules and the Standard Formal Module (end)


