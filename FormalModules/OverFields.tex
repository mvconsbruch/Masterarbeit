As above, let $A$ be a discrete valuation ring with uniformizer $\varpi$ and finite 
residue field $k$; write $q$ for the cardinality of $k$. Let $K$ denote the
field of fractions of $A$.

We introduce the concept of height, which is an integer attached to
morphisms of formal group laws over fields. The height of a formal $A$-module
$\cF$ over $R$ will be defined as the height of it's endomorphism $[\varpi]_\cF$. 

We have seen in the previous section that if $R$ is a field extension of $K$,
then any morphism of formal group laws $f: F \to G$ over $R$ is either $0$, in
which case we say it has height $\infty$, or an isomorphism, in which case we
say it has height $0$. The height becomes interesting in positive
characteristic. 

We define the height over field extensions of the residue field. 
\begin{defi}[Height of morphisms of group laws]
  Assume that $R$ is a field extension of $k$ and $f: F \to G$ is a morphism of 
  formal groups laws over $R$, given by a formal series $f(T) \in R\llbr T \rrbr$. 
  If $f = 0$, we say that $f$ has infinite height. 
  If $f \neq 0$, the height of $f$ is defined as the largest integer $h$ such that 
  $f = g(T^{q^h})$ for some power series $g(T) = c_1 T + c_2 T^2 + \dots \in
  R\llbr T \rrbr$ with $c_1 \neq 0$. 
\end{defi}
One readily checks that if $f: \cF \to \cG$ is a morphism of formal groups over
a field extension $R$ of $k$, the height of $f$ does not depend on the choices
of group laws on $\cF$ and $\cG$. This allows us to define the height function 
attached to $f$. 
\begin{defi}[Height function]
  Let $f: \cF \to \cG$ be a morphism of formal groups over a scheme $X$.
  For a (scheme-theoretic) point $x \in \abs X$, let $f_x$ denote the 
  base-change of $f$ to the residue field of $x$. 
  The height function attached to $f$ is the upper-semicontinuous function
  \begin{equation} \label{eq:defheight}
    \height(f) : \abs X \to \Z_{\geq 0} \cup \{\infty\}, \quad x \mapsto 
    \height(f_x).
  \end{equation}
\end{defi}
It is not hard to see that the height function is additive, that is, we have
\begin{equation*}
  \height(f \circ g) = \height(f) + \height(g).
\end{equation*}


\begin{defi}[Isogeny]
  A morphism $f: \cF \to \cG$ of formal groups over a field $k$ is called an isogeny if
  $\ker(f)$ is a represented by a finite free $k$-scheme. More generally, a
  morphism of formal $A$-modules over a base scheme $X$ is an isogeny if and
  only if $\ker(f)$ is finite and locally free over $X$. 
\end{defi}

Isogenies can be described using the height function.

\begin{lem}
  A morphism $f: \cF \to \cG$ is a isogeny if and only if the height 
  function $\height(f)$ is locally constant with values in $\Z_{\geq 0}$. 
\end{lem}

\begin{defi}[$\varpi$-divisible $A$-module]
  We say that a formal $A$-module $H$ over $X$ is $\varpi$-divisible if 
  $[\varpi]_H$ is an isogeny. If $X$ is connected, the height of $H$ is the
  (constant) height of the endomorphism $[\varpi]_H: H \to H$. 
\end{defi}

We close this subsection with a discussion about the structure of 
formal $\cO_K$-modules over separably closed field extensions $k'$ of
$k$. 
\begin{lem}\label{lem:classificationofmodulelawsbyheight}
  Over $k'$, any two formal $\cO_K$-module laws of the same height are isomorphic.
\begin{proof}
  \cite[Proposition 1.7]{drinfel1974elliptic}.
\end{proof}
\end{lem}
In particular, any formal $\cO_K$-module of height $h$ is isomorphic to 
the formal $\cO_K$-module $F_\mathrm{norm}$ with $[\varpi]_{F_\mathrm{norm}}(T) =
  T^{q^h}$. We call this the normalized formal $\cO_K$-module.

\begin{lem}\label{lem:EndomorphismsOfFormalModulesOverksep}
  Suppose that $F \in \FMLOver {\cO_K}{k'}$. Then $\End_{\FMOver A{k'}}(F)$ is 
  isomorphic to the maximal order
  of the central division algebra $D$ over $K$ of rank $h^2$ and invariant
  $\frac 1h$.
\begin{proof}
  Also \cite[Proposition 1.7]{drinfel1974elliptic}. 
\end{proof}
\end{lem}

\begin{lem}\label{lem:inverseqisog}
  Let $f: F \to G$ be an isogeny of $\varpi$-divisible formal $\cO_K$-module laws
  over $k'$. Then there is an 
  integer $n \geq 0$ and an isogeny $g: G \to F$ with 
  \begin{equation*}
    f \circ g = [\varpi^n]_\cG \quad \text{and} \quad g \circ f = [\varpi^n]_\cF.
  \end{equation*}
  \begin{proof}[Proof]
    As the height is additive, we necessarily have 
    $\height(F) = \height(G)$, thus by Lemma
    \ref{lem:classificationofmodulelawsbyheight}, we may assume that $F$ and
    $G$ are given by the normalized formal $\cO_K$-module
    $F_\mathrm{norm}$.
    Write $f(T) = g(T^{q^n})$ for some power series $h(T) = c_1 T + c_2T^2 +
    \dots$, where $c_1 \neq 0$ is a unit in $k'$, and let $g(T) = h^{-1}(T)$ be the 
    formal inverse of $h$. Now $g$ is a morphism of formal $\cO_K$-module laws
    satisfying $f \circ g(T) = g\circ f(T) = T^{q^n}$. The claim follows.
  \end{proof}
\end{lem}

% subsubsection Formal DVR-Modules over Fields of Characteristic 0 (end)
