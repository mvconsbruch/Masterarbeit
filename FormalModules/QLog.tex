We keep the assumptions on $A$, $R$ and $S$ from the previous subsection. That is,
$A$ is a local ring with finite residue field and uniformizer $\pi$, 
$R$ is a local $A$-algebra with maximal ideal $I$ complete with respect to
the $I$-adic topology and algebraically closed residue field $k_R$, and 
$S$ denotes an admissible $R$-algebra (where $R \to S$ is continuous with
the $I$-adic topology on $R$) with ideal of definition $J \subseteq S$ containing
the image of $I$. 

The aim of this subsection is to define, attached to any $\pi$-divisible formal
$A$-module $H$ over $R$, a quasi-logarithm map
\begin{equation*}
  \qlog_H: \tilde H(S) \to (M(H_0) \otimes \GG_a)(S)
\end{equation*}
and give an explicit description of this map if $H$ is the standard $\cO_K$-module
over $\cO_{\breve K}$. 
% subsubsection The Quasilogarithm map (end)

\subsection{Determinants of Formal Modules} % (fold)
\label{sub:Determinants of Formal Modules}
\begin{itemize}
  \item "Functorial" description of the determinant. Either as in
    \cite{BoyarchenkoWeinstein2011MaxVar}, or as in \cite{weinstein2016semistable}.
  \item Construction.
  \item Approximations.
\end{itemize}

% subsubsection Determinants of Formal Modules (end)
% subsection Application: The Local Class Field Theory (end)

