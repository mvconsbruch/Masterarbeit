\subsection{Formal Modules of Arbitrary Dimension} % (fold)
\label{sub:Formal Modules of Arbitrary Dimension}
  
One readily checks that $\End_{\FGLArbOver R}(\mathbf F)$ is a (possibly 
non-commutative) algebra. Note that there is 
a functor 

Given a ring $A$ and a morphism of rings
$A \to \End_{\FGLArbOver R}(\mathbf F)$, we obtain an $A$-module structure on 
$\Lie(\bF) \cong R^n$. by composing with $\Lie(-): \End_{\FMLArbOver A
R}(\mathbf F) \to \End_{\Mod R}(\Lie(\bF))$. If $R$ is an $A$-algebra, we want the 
$A$-module structures on $\Lie(\bF)$ to be the same. This motivates the following 
definition.

Note that an $n$-dimensional formal module law $(\bF, [-]_\bF)$ equips the 
formal scheme $\FGG(\cF) \coloneqq \spf(R \llbr T_1, \dots, T_n\rrbr)$ with the
structure of an $A$-module object in the category $\FSchOver R$ of formal
schemes over $\spec R$. This leads naturally to the more geometric notion of 
formal $A$-modules.






% subsubsection Formal Modules of Arbitrary Dimension (end)

