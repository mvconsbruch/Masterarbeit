\subsubsection{Rigidified Extensions and Quasi-Logarithms} % (fold)
The aim of this subsection is to give a description of $\RigExt(\cF, \cF')$ 
similarly to the one given for $\Ext(\cF, \cF')$ in the previous 
subsection, at least in the case where $\cF' = \GG_a$ is the (one-dimensional)
additive formal group law and $\cF$ comes from a one-dimensional group law $F$. 
This description will be in terms of quasi-logarithms, which we shall now
define.


By Theorem \ref{thm:ExtClassCoc} and Theorem \ref{thm:RigExtStr}, we may equip the 
sets $\Ext(F, \GG_a)$ and $\RigExt(F, \GG_a)$ with the structure of an $R$-module. 
There is a natural map of $R$-modules $\RigExt(F, \GG_a) \to \Ext(F, \GG_a)$
sending $(E, \omega_E)$ to $E$. This map is surjective. The kernel is given by
pairs $(\GG_a \oplus F, \omega) \in \RigExt(F, \GG_a)$, where $\omega$ is an
invariant differential on
$\GG_a \oplus F$ pulling back to $\dc X$ on $\GG_a$. As $\omega(\GG_a \oplus F)
\cong \omega(\GG_a) \times \omega(F)$, this datum is equivalent to an invariant
differential $\nu \in \omega(F)$. 


% subsubsection Rigidified Extensions (end)

