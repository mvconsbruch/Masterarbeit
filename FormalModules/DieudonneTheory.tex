We explain construct a functor $\M^\vee: \FMOver AR \to \Mod R$, satisfying the 
following condition.
\begin{enumerate}
  \item bla
\end{enumerate}

\subsubsection{Higher-Dimensional Formal Modules and the Universal Additive Extension} % (fold)
\label{ssub:Higher-Dimensional Formal Modules and the Universal Additive Extension}
For now, let $A$ be any ring and let $R$ be an $A$-algebra.
We introduce the notion of higher-dimensional formal modules, which will play a role 
only in this chapter. 
\begin{defi}[Formal Group Laws of arbitrary dimension]
  A (commutative) formal group law $\mathbf F$ of dimension $n$ over $R$ is a tuple of
  power series $$F_i(X_1, \dots, X_n,Y_1, \dots, Y_n) \in R\llbr X_1, \dots,
  X_n, Y_1, \dots, Y_n\rrbr, \quad 1 \leq i \leq n$$
  such that $F_i(\mathbf X, \mathbf Y) \equiv X_i + Y_i $ modulo degree $\geq 2$.
  and the following equalities are satisfied:
  \begin{enumerate}
    \item $\mathbf F(\mathbf F (\mathbf X, \mathbf Y), \mathbf Z) = 
      \mathbf F(\mathbf X, \mathbf F(\mathbf Y, \mathbf Z))$.
    \item $\mathbf F( \mathbf X, \mathbf 0) = \mathbf X$.
    \item $\mathbf F( \mathbf X, \mathbf Y) = \mathbf F(\mathbf Y, \mathbf X).$
  \end{enumerate}
  Here, and in the sequel, we have $\mathbf F = (F_1, \dots, F_n)$, $\mathbf X =
  (X_1, \dots, X_n)$, et cetera.
  Given a formal group $\mathbf F$ of dimension $n$ and a formal group law
  $\mathbf G$ of dimension $m$,
  a morphism $\mathbf F \to \mathbf G$ is a $m$-tuple $\mathbf f = (f_1, \dots, f_m)$ 
  of power series $f_i \in R\llbr X_1, \dots, X_n \rrbr$ such that 
  \begin{equation*}
    \mathbf G(\mathbf f( \mathbf X), \mathbf f( \mathbf Y) ) = \mathbf
    f(\mathbf F( \mathbf X, \mathbf Y)).
  \end{equation*}
  For any $n$-dimensional formal module $\mathbf F$, the identity is given by 
  the morphism $\id_{\mathbf F}$ with components $\id_{\mathbf F,i}( \mathbf X) = X_i$. 
  Composition of morphisms is given by composition of tuples of power-series.
  This yields the category of formal modules of arbitrary dimension over $R$,
  which we denote by $\FGLArbOver R$.
\end{defi}
One readily checks that $\End_{\FGLArbOver R}(\mathbf F)$ is a (possibly 
non-commutative) $R$-algebra. Note that there is 
a functor $\Lie: \FGLArbOver R \to \Ab$, taking an $n$-dimensional 
formal group law $\mathbf F$ to the additive group $R^n$, and morphisms
$\mathbf f$ to multiplication with the matrix 
\begin{equation*}
  \Lie(\mathbf f) = \left(\frac{\partial f_j}{\partial X_i} (0) \right)_{\substack{1 \leq j \leq m\\1 \leq i \leq n}} \in \End_{\Mod R}(R^n).
\end{equation*}
This generalizes the one-dimensional situation.
Given a ring $A$ and a morphism of rings
$A \to \End_{\FGLArbOver R}(\mathbf F)$, we obtain an $A$-module structure on $R^n$
by composing with $\Lie(-): \End_{\FMLArbOver A R}(\mathbf F) \to \End_{\Mod R}(R^n)$.
The definition for formal modules generalizes as follows. 
\begin{defi}[Formal $A$-Modules of arbitrary dimension]
  Let $R$ be an $A$-algebra. A formal $A$-module over $R$ of dimension $n$ is
  given by the data of 
  a formal $n$-dimensional group law $\mathbf F$ over $R$ and a morphism of rings
  \begin{equation*}
    A \to \End_{\FGLArbOver R} (\mathbf F), \quad a \mapsto ([a]_{\mathbf F,i}
    )_{1 \leq i \leq n} \in (R \llbr X_1, \dots, X_n \rrbr )^n
  \end{equation*}
  such that $[a]_{\mathbf F,i}(\mathbf X) \equiv a X_i$ modulo terms of degree 
  $\geq 2$. Morphisms between formal $A$-modules of arbitrary dimension are 
  morphisms of formal groups respecting the $A$-module structure. 
  The resulting category is denoted $\FMLArbOver A R$.
\end{defi}
\color{blue}The following two definitions are only included for completeness.
\begin{defi}[Formal $A$-modules]
  Similarly to before, given an $A$-scheme $X$, we define the category
  $\FMArbOver A X$ as follows. Objects are group objects $\cF$ in the category
  of formal schemes over $X$, locally isomorphic to $U_i = \Spf R_i\llbr X_1,
  \dots, X_n\rrbr$, such that the induced $A$-module structure on $U_i$
  agrees with one coming from a formal $A$-module law on $R_i$. 
\end{defi}

\begin{defi}[Lie functor]
  The functor $\Lie$ descents to a functor 
  \begin{equation*}
    \Lie: \FMArbOver A X \to \QCoh {\cO_X}, 
  \end{equation*}
  given by locally describing a formal $A$-module $\cF$ via formal group laws
  and glueing the local data. Alternatively, it arises from sending 
  sending a formal $A$-module $\cF$ to $\cI / \cI^2$, where $\cI$ is the ideal
  associated to the closed immersion $[0]_\cF: X \to \cF$. 
\end{defi}
\color{black}

\subsubsection{Quasi-Logarithms and Rigidified Extensions} % (fold)
\label{ssub:Quasi-Logarithms and Rigidified Extensions}
We define short exact sequences.
\begin{defi}[Short exact sequence]
  A pair of composable morphisms $\bF' \to \bF \to \bF''$ in 
  $\FMLArbOver A R$ is called 
  a short exact sequence if the induced sequence 
  \begin{equation*}
    0 \to \Lie(\bF') \to \Lie(\bF) \to \Lie(\bF'') \to 0
  \end{equation*}
  is a short exact sequence of $R$-modules.
\end{defi}
\red{\textbf{Question}. Is $\FMArbOver A X$ an exact category (in the sense of Quillen)?}

\begin{defi}[Rigidified Extension]
  An extension of $\bF$ by $\bF'$ is a short exact sequence 
  \begin{equation*}
    0 \to \bF' \to \bE \to \bF \to 0.
  \end{equation*}
  We say that this extension is equivalent to another extension 
  \begin{equation*}
    0 \to \bF' \to \bE' \to \bF \to 0
  \end{equation*}
  if and only if there is a morphism $\bE \to \bE'$ making the diagram 
  \begin{equation*}
    \begin{tikzcd}[ampersand replacement=\&]
    	0 \& {\bF'} \& \bE \& \bF \& 0 \\
    	0 \& {\bF'} \& {\bE'} \& \bF \& 0
    	\arrow[from=1-1, to=1-2]
    	\arrow[from=1-2, to=1-3]
    	\arrow[Rightarrow, no head, from=1-2, to=2-2]
    	\arrow[from=1-3, to=1-4]
    	\arrow[from=1-3, to=2-3]
    	\arrow[from=1-4, to=1-5]
    	\arrow[Rightarrow, no head, from=1-4, to=2-4]
    	\arrow[from=2-1, to=2-2]
    	\arrow[from=2-2, to=2-3]
    	\arrow[from=2-3, to=2-4]
    	\arrow[from=2-4, to=2-5]
    \end{tikzcd}
  \end{equation*}
  commute. We denote the set of extensions of $\bF$ by $\bF'$ by 
  $\Ext(\bF, \bF')$. A rigidified extension is given by an extension as above
  together with a splitting $s: \Lie(\bF) \to \Lie(\bE)$ of the induced morphism
  $\Lie(\bE) \to \Lie(\bF)$. We denote the set of rigidified extensions of
  $\bF$ by $\bF'$ as $\RigExt(\bF, \bF')$.
\end{defi}
\begin{rmk} 
  Note that if $X = \spec R$ is affine,  such a splitting always exists because 
  $\Lie(\cF)$ is locally free, thereby projective. Giving a splitting 
  is the same as to give a differential form $\omega_E \in \omega(E)$ that pulls
  back to $\dc X' \in \GG_a$\todo{why?}.
  \todo{Also, what about $R$-module structure on $\RigExt$?}
\end{rmk}

Given an $n$-dimensional $A$-module law $\bF$, an $m$-dimensional $A$-module law 
$\bF'$, and an $m$-tuple of power series $\bff = (f_1, \dots, f_m)$ with $f_i \in
(R \otimes_A K)\llbr X_1, \dots, X_n \rrbr$ with no constant term, we define
\begin{gather*}
  \delta_{a,F,F''}(\bff) = [a]_{F'}(\bff) - \bff([a]_F) \in (R \otimes_A K)\llbr
  X_1, \dots, X_n \rrbr, \\
  \Delta_{F,F'}(\bff) = \bF'(\bff(\bX), \bff(\bY)) - \bff(\bF(\bX,\bY)) \in 
  (R \otimes_A K) \llbr X_1, \dots, X_n, Y_1, \dots, Y_n\rrbr.
\end{gather*}
Following \cite{hopkins1994equivariant}.

\begin{itemize}
  \item Definition using quasi-logarithms
  \item Definition with rigidified extensions as in \cite{hopkins1994equivariant} (?)
\end{itemize}
% subsubsection Quasi-Logarithms and Rigidified Extensions (end)
% subsubsection The Dieudonn`e functor (end)
