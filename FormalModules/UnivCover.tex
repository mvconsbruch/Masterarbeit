\subsubsection{Useful Calculations} % (fold)
\label{ssub:Useful Calculations}
Let $p$ be a prime.
Let $R$ be a Noetherian local ring with maximal ideal $I$ such that 
$p \in I$, $R$ is complete with respect to the $I$-adic topology and $k_R
\coloneqq R/I$ is an algebraically closed field (necessarily of characteristic
$p$). 
If $q$ is a power of $p$, we write $\cF_{R,q}$ for the set of power series $f
\in R\llbr T \rrbr$ satisfying 
\begin{equation} \label{eq:condonpowerseries}
  f(T) \equiv g(T^q) \pmod I
\end{equation}
for some power series $g(T) = c_1 T + c_2 T^2 + \dots \in R\llbr T \rrbr$ with 
$c_1 \in R^\times$. 
If $q'>q$ is another power of $p$, we have injections $\cF_{R,q} \inj \cF_{R,q'}$
given by sending $f(T)$ to its $(q'/q)$-fold self-composite $f^{q'/q}(T)$. 
Making use of these transition maps, we define
\begin{equation*}
  \cF_R \coloneqq \colim_{n \in \N} \cF_{R, p^n},
\end{equation*}
identifying any power series $f \in \cF_{R,q}$ with its image in $\cF_{R,q'}$ for 
higher $p$-powers $q'$. 
For any $f \in \cF_{R,q}$, we define the functor
\begin{equation*}
  U_f: \Adm R \to \Set, \quad S \mapsto \left\{(x_0,x_1, \dots) \in \prod_\N S^\cici 
                                          \mid f(x_{i+1}) = x_i\right\}.
\end{equation*}
This functor does, up to canonical isomorphism, only depend on the equivalence
class of $f$ in $\cF_{R}$. 
We write $U_{0,f}$ for the base change of $U_f$ to $k_R$, that is
\begin{equation*}
  U_{0,f}: \Adm {k_R} \to \Set, \quad S \mapsto \left\{(x_0,x_1, \dots) \in
                              \prod_\N S^\cici \mid \bar f(x_{i+1}) = x_i\right\}.
\end{equation*}
Here, $\bar f$ is the image of $f$ under the reduction map $R\llbr T \rrbr \to
k_R\llbr T \rrbr$. 

In the sequel, we denote $R$-algebras by $S$ and write $J$ for an ideal
of definition containing the image of $I$ (provided, for example, by \ref{lem:iodimage}). 
Given an element $f\in \cF_R$, we do not distinguish between $f$ and a choice of a 
representative $\tilde f \in \cF_{R,q}$ for some sufficiently large $p$-power.

The following observation lays the groundwork for many of the upcoming results.
\begin{lem}\label{lem:cryscalc}
  Let $f$ be any power series in $\cF_R$. For any two elements $s_1,s_2 \in S$ 
  with $s_1 \equiv s_2 \mod J$ such that $f(s_1)$ and $f(s_2)$ exist (for
  example if $f$ is a polynomial or $s_1, s_2 \in S^\cici$), we have 
  \begin{equation*}
    f^k(s_1) \equiv f^k(s_2) \pmod {J^{k+1}}.
  \end{equation*}
  Here, $f^k$ denotes $k$-fold composition of $f$.
\begin{proof}
  We will show that if $s_1 \equiv s_2$ mod $J^k$, then $f(s_1) \equiv f(s_2)$ mod 
  $J^{k+1}$, which suffices to prove the claim. 
  We may write $s_2 = s_1 + r$ for some $r\in J^k$. By the assumptions on $f$
  there exist power series 
  $g,h \in R\llbr T \rrbr$ such that $h$ only
  has coefficients in $I$ and $f(T) = g(T^q) + h(T)$. As $I$ is finitely generated,
  say by elements $(r_1, \dots, r_l)$, we obtain a representation 
  \begin{equation*}
    f(s_1) - f(s_2) \in g(s_1^{q})-g(s_2^{q}) + \sum_{i=1}^l r_i \left(h_i(s_1) -
    h_i(s_2)\right).
  \end{equation*}
  As $r$ divides $\left(h_i(s_1) - h_i(s_2)\right)$, we find
  $r_i(h_i(s_1) - h_i(s_2)) \in (r_i r) \subseteq J^{k+1}$. Also note that 
  for any $s\in S$ and $n \in \N$, 
  $$(s+r)^{nq} = s^{nq} + nqrs^{nq-1}r + \dots + r^{nq},$$
  so after cancellation, all monomials of $g(s_1^q) - g(s_2^q)$ lie in
  $(qr)$ or $(r^2)$. This implies
  \begin{equation*}
    g(s_1^q) - g\left((s_1+r)^q\right) \in (qr) + (r^2) \subseteq J^{k+1},
  \end{equation*}
  and we are done.
\end{proof}
\end{lem}

\begin{lem}\label{lem:reductioniso}
  The natural reduction map 
  \begin{equation*}
    U_f(S) \to U_{f}(S/J) = U_{0,f}(S/J)
  \end{equation*}
  is bijective.
\begin{proof}
  We first show surjectivity. Given a sequence $(x_0, x_1, \dots) \in U_{f}(S/J)$, 
  we can choose a sequence of arbitrary lifts $(y_0, y_1, \dots ) \in \prod_\N
  S^\cici$ and set 
  \begin{equation*}
    z_i = \lim_{r \to \infty} f^r(y_{i+r}).
  \end{equation*}
  The limit exists, because if $s\geq r$ are two non-negative integers, we calculate
  \begin{equation*}
    f^{s-r}(y_{i+s}) \equiv \bar f^{s-r}(x_{i+s}) = x_{i+r} \equiv y_{i+r}
    \pmod J,
  \end{equation*}
  implying by Lemma \ref{lem:cryscalc} that 
  \begin{equation*}
    f^{s}(y_{i+s}) \equiv f^r(y_{i+r}) \pmod{J^r}.
  \end{equation*}
  This shows that $(f^{r}(y_{i+r}))_{r \in \N}$ is a Cauchy-sequence for the 
  $J$-adic topology on $S$, thereby convergent (cf. Lemma
  \ref{lem:AdmAdicComp}). The sequence $(z_0, z_1, \dots)$ now lies in $U_f(S)$
  and lifts $(x_0, x_1, \dots)$. It remains to show that the lift is unique.
  Suppose that $(z'_0, z'_1, \dots)$ is another lift. Then, for any $i,k \in\N$
  we have $z_{i+k} \equiv z'_{i+k}$ mod $J$, and another application of Lemma
  \ref{lem:cryscalc} shows that 
  \begin{equation*}
    z_i = f^k(z_{i+k}) \equiv f^k(z'_{i+k}) = z'_i \pmod {J^k}.
  \end{equation*}
  Thereby $(z_i - z'_i) \in \bigcap_{k \in \N} J^k = \{0\}$. Hence,
  the lift is unique.
\end{proof}
\end{lem}

We write $\Nilp^\flat$ for the functor $U_{T^q}$. That is, 
$\Nilp^\flat(S) = \lim_{x \mapsto x^q}S^\cici$ is 
the set of $q$-power compatible sequences with values in $S^\cici$. 

\begin{lem}\label{lem:nilp0iso}
  For any $f \in \cF_R$, there is a canonical \todo{Use different $S$}
  bijection $U_{0,f}(S/J) \to \Nilp^\flat(S/J)$. This bijection is functorial in 
  $S$.
\begin{proof}
  By assumption on $f$ we have $f(T) = g(T^{q}) \in k_R\llbr T \rrbr$ for some 
  $g(T) = c_1T + c_2T^2 + \dots$ with $c_1 \neq 0$. For each coefficient $c_i$, let
  $d_i \in k_R$ be the unique element such that $d_i^{q} = c_i$. Let
  $h(T) \in k_R\llbr T \rrbr$ be the power series given by $d_1 T + d_2
  T^2 + \dots$. Now $(h(T))^{q}=f(T)$, and we find that 
  \begin{equation*}
      U_f(S/J) \to \Nilp^\flat(S/J): \quad
      (x_1, x_2, x_3, \dots) \mapsto (x_1, h(x_2), h(h(x_3)), \dots)
  \end{equation*}
  is a well-defined function, and (trivially) functorial in $S$. For the
  inverse, let $h^{-1}(T) \in k_R\llbr T \rrbr$ be the the unique power
  series with $h^{-1}(h(T))= h(h^{-1}(T)) = T$, see Lemma
  \ref{lem:FGLeasyfacts}. The map
  \begin{equation*}
      \Nilp^\flat(S/J) \to U_f(S/J), \quad 
      (x_1, x_2, \dots ) \mapsto (x_1, h^{-1}(x_2), h^{-1}(h^{-1}(x_3)), \dots)
  \end{equation*}
  is well-defined as
  \begin{equation*}
      f(h^{-1}(T)) = g((h^{-1}(T))^{q}) = (h(h^{-1}(T)))^{q} =
      T^{q}
  \end{equation*}
  and it is readily seen to be inverse to the map constructed above.
\end{proof}
\end{lem}

We collect results.
\begin{prop}\label{prop:pHTcalc}
  Given $f,g \in \cF_R$, we have bijections, functorial in $S$,
  \begin{equation} 
    U_f(S) \to U_f(S/J) \to \Nilp^\flat(S/J) \to U_g(S/J) \to U_g(S).
  \end{equation}
  Explicitely, the bijection $U_f(S) \to U_g(S)$ can be described as follows.
  Suppose that $f,g \in \cF_{R,q}$ for some sufficiently large $q$. 
  Let $h_f(T)$ and $h_g(T)$ be power series with coefficients in $A$ such that 
  $$h_f(T)^q \equiv f(T) \pmod I\quad\text{and}\quad h_g(T)^q \equiv g(T) \pmod
  I.$$
  Write $h_g^{-1}(T)$ for the (formal) inverse power series of $h_g$. 
  Now the isomorphism is given by the mapping
  \begin{equation*}
    (x_0, x_1, \dots) \mapsto (y_0, y_1, \dots), \quad \text{where} \quad y_i =
    \lim_{r \to \infty} g^r(h_g^{-(r+i)}(h_f^{r+i} (x_{i+r}))).
  \end{equation*}
  Here, the exponents are to be interpreted as iterated composition.
\begin{proof}
  The first part follows directly from repeated application of the previous
  two Lemmas. The second part follows by tracing through the previous lemmas.  
\end{proof}
\end{prop}
% subsubsection Useful Calculations (end)
\subsubsection{The Universal Cover} % (fold)
\label{ssub:The Universal Cover}
Let $A$ be an integral domain and $R$ be an $A$-algebra. Given $H \in \FMOver
AR$ and $a \in A$,
we define the functor 
\begin{equation*}
  \Tilde H_a : \Adm R \to \Mod {A}, \quad
  S \mapsto \left\{(x_1, x_2, \dots) \in \prod_{\N} H(S) \mid [a]_H(x_{i+1}) =
  x_i \right\}.
\end{equation*}
Here, the $A$-module structure is given by $b.(x_1, x_2,\dots) = ([b]_H(x_1), [b]_(x_2),
\dots)$. Note that multiplication by $a$ on $\tilde H_a(S)$ is an automorphism
(it sends $(x_1, x_2, \dots)$ to $([a]_H x_1, x_1, x_2, \dots)$, which has inverse given
by shifting to the left)  so that $\tilde H_a(S)$ is naturally an
$A[\frac1a]$-module.

From now on assume that $A$ is a discrete valuation ring with uniformizer
$\pi$, finite residue field $k$ and field of fractions $K$. Let $R$ be 
a local $A$-algebra with maximal ideal $I$ and algebraically closed residue field 
$k_R = R/I$. Let $H$ be a formal $A$-module over $R$. 
\begin{defi}[The Universal Cover and Tate Module]
  We write $\tilde H = \tilde H_\pi$. This functor
  takes values in the category of $K$-vector spaces.
  Up to natural isomorphism, $\tilde H$ does not depend on the choice of 
  $\pi$. We call this functor the universal cover of $H$. 

  The Tate-Module $T_\pi H$ is the subfunctor of $\Tilde H$ cut out out
  by the condition that $[\pi]_H(x_1) = 0$. Note that $T_\pi H$ does no longer 
  carry the structure of a $K$-vector space, it is an $A$-module. The Rational
  Tate Module $V_\pi H$ is the subfunctor of $\Tilde H$ cut out by the
  condition that $x_1$ has $[\pi]_H$-torsion. Equivalently, we have 
  \begin{equation*}
    V_\pi H (S) = T_\pi H(S) \otimes_A K.
  \end{equation*}
\end{defi}

\begin{lem}
  Let $H$ be a $\pi$-divisible formal $A$-module over $R$ and write $H_0 = H
  \otimes_R k_R$. Now the choice of a coordinate on $H_0$ gives rise to a
  an isomorphism 
  \begin{equation*}
    \tilde H_0 \cong \Nilp^\flat_{k_R}
  \end{equation*}
  of functors $\Adm {k_R} \to \Set$
  \begin{proof}
    Note that given any coordinate on $H$, we have $[\pi]_H(T) \in \cF_R$. Hence,
    the statement is an application of Lemma \ref{lem:nilp0iso}.
  \end{proof}
\end{lem}

\begin{lem}
  Suppose that $S$ is an admissible $R$-algebra admitting an ideal of definition
  $J$ such that $\pi \in J$. Then the natural reduction map
  \begin{equation*}
    \tilde H(S) \to \tilde H(S/J)
  \end{equation*}
  is an isomorphism.
  \begin{proof}[Proof]
    After choosing a coordinate on $H$, we have $[\pi]_H \in \cF_R$ and 
    $\tilde H(S) \cong U_{[\pi]_H}$, and the statement is given by 
    Lemma \ref{lem:reductioniso}.
  \end{proof}
\end{lem}

The following is analogous to Proposition \ref{prop:pHTcalc}.
\begin{prop}
  Let $S$ be an admissible $R$-algebra with ideal of definition $J$ such that 
  $\phi(I) \subseteq J$. Then there are canonical isomorphisms (of sets)
      \begin{equation*}
        \tilde H(S) \cong \tilde H(S/J) = \tilde H_0(S/J) \cong \Nilp^\flat(S/J) \cong
        \Nilp^\flat(S).
      \end{equation*}
  In particular, $\tilde H(S)$ is, as a functor to $\Set$, representable by
  $\spf(R \llbr T^{q^{-\infty}} \rrbr)$.
\end{prop}

\begin{rmk} 
  In case where $H$ comes from a Lubin--Tate $\cO_K$-module law, the bijections
  \begin{equation*}
    \tilde H(S) \rightleftarrows \Nilp^\flat(S), \quad (x_0, x_1, \dots) \mapsspamto (y_0, y_1, \dots)
  \end{equation*}
  are, in either direction, given by the equations
  \begin{equation*}
    y_i = \lim_{r \to \infty} x_{i+r}^{q^r} \quad \text{and} \quad 
    x_i = \lim_{s \to \infty} [\pi^s]_H(y_{i+s}).
  \end{equation*}
  This follows directly from the explicit description of the isomorphism in
  Proposition \ref{prop:pHTcalc}, as we may choose $h_{[\pi]_H}(T) = h_{T^q}(T)
  = T$.
\end{rmk}
% subsubsection The Universal Cover (end)

% subsection Tate Modules and the Universal Cover (end)


