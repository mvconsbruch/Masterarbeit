%! TeX root: ../main.tex
\documentclass[../main.tex]{subfiles}

\begin{document}

\section{Deligne--Lusztig Theory for Depth Zero Representations} % (fold)
\label{sub:Deligne-Lusztig Theory for Depth Zero Representations}
The aim of this section is to outline the construction of a correspondence
between certain characters of $\FF_{q^n}^\times$ (with values in $\C^\times$) and 
cuspidal representations of $\GL_n(\FF_q)$. The correspondence we construct here
is an instance of a more general theory developed by Deligne--Lusztig. 
In \cite{delignelusztig1976}, they construct
for any connected reductive algebraic group $G = G_0 \times_{\FF_q} \bar \FF_q$ and any
Frobenius-stable maximal torus $T \subseteq G$ a correspondence associating to
certain characters $\theta$ of $T^\Frob$ a virtual representation $R_{T,\theta}$ of 
$G^\Frob$. These virtual representations arise from the $\ell$-adic cohomology (with
$\ell \neq p$) of a certain variety $\DL_{G,T}$ admitting commuting actions by
$G^\Frob$ and $T^\Frob$. 
In this section, we give explicit descriptions of the occuring spaces 
in the situation where $G = \GL_{V_0 \otimes \bar \FF_q}$ for some
$n$-dimensional $\FF_q$-vector space $V_0$ and $T \subset G$ is a maximal 
Frobenius-stable torus with $T(\bar \FF_q) = \FF_{q^n}^\times$.
The main theorems of the theory are stated as facts, proofs are omitted.

\subsection{Deligne--Lusztig Varieties for the General Linear Group} % (fold)
\label{ssub:Deligne--Lusztig Varieties}
We begin by introducing (full) flags and their classifying objects, flag
varieties. Let $k$ be a field and let $V$ be a finite dimensional $k$-vector space
of dimension $n$. We write $\tilde V$ for the corresponding quasi-coherent sheaf
on $\spec k$, and $\GL_V$ for the general linear group scheme of $\tilde V$.

\begin{defi}[Flag Variety]\label{def:FlagVariety}
  Let $X: \SchOver k^\op \to \Set$ be the functor assigning to each $k$-scheme 
  $f:S \to \spec k$
  the set 
  \begin{equation*}
    X(S) = \\ \left\{\cF_1 \subset \cF_2 \subset \dots \subset \cF_{n-1} \subset
      f^* \tilde V \ \bigg \vert  \begin{array}c
      \text{$\cF_i$ is, for all $i$, a locally direct summand} \\
      \text{of $f^*\tilde V$, locally free of rank $i$}
  \end{array} \right\}.
  \end{equation*}
  Recall that a subsheaf $\cF_i \subset f^* \tilde V$ is
  locally a direct summand
  if it is quasi-coherent, and for each $s \in S$ there is some neighbourhood 
  $U$ of $s$ such that $\cF_i|_U$ is a direct summand of $f^* \tilde V|_U$. 
  The $S$-valued points of $X$ are called families of flags over $S$.
\end{defi}
Elements of $X(k)$ are called (full) flags. They are given by an increasing
$n-1$-tuple of vector spaces 
$$\Flag = (F_1 \subsetneq F_2 \subsetneq \dots F_{n-1} \subsetneq V) \in X(k).$$
  %By \cite[Prposition 8.10]{gortz2020algebraic} and the subsequent discussion,
  %a subsheaf $\cF \subset \cO_S^n$ being locally a direct summand is equivalent
  %to the quotient sheaf $\cO_S^n/\cF$ being locally free.
  There are natural morphisms
  \begin{equation*}
    \nu_i: X \to \Grass_{V,i}, \quad (\cF_1 \subset \dots \subset \cF_{n-1}
    \subset f^* \tilde V) \mapsto (f^* \tilde V^\vee \surj \cF_i^\vee),
  \end{equation*}
  where $\Grass_{V,i}$ denotes the Grassmannian parametrizing surjections of
  $f^*\tilde V^\vee$ to locally free coherent modules of rank $i$, as defined
  in \cite{Ricolfi2022}.

\begin{prop}\label{prop:FlagVarisProjective}
  The induced morphism of functors
  \begin{equation*}
    X \to \Grass_{V,1} \times_{\spec k} \dots \times_{\spec k}
    \Grass_{V,n-1}
  \end{equation*}
  is representable by a closed embedding. In particular, as $\Grass_{V,d}$ is
  representable by a projective scheme for integers $1 \leq d \leq n-1$, 
  the functor $X$ is representable by a projective scheme. 
\begin{proof}
  Upon picking a basis of $V$, the claim can be checked directly on the
  standard affine cover of the Grassmannians, where the condition that $\cF_i$ is
  contained in $\cF_{i+1}$ is cut out by a polynomial equation.
  For representability of $\Grass_{V,d}$, cf. \cite[Theorem 5.1.4]{Ricolfi2022}.
\end{proof}
\end{prop}

There is a natural $\GL_V$-action on $X$, induced by the natural action of 
$\GL_V(S)$ on $f^* \tilde V$. Given a flag $\Flag \in
X(k)$, we write $B_{\Flag} \subset \GL_V$ for the isotropy subgroup of $\Flag$
under this action. In \cite{delignelusztig1976}, the authors work with schemes
arising as quotients $G/B$ where $B$ is some Borel subgroup of a connected, reductive
algebraic group $G$. 
The following proposition shows that $X$ is isomorphic to the quotient of 
$\GL_V/B_{\Flag}$.

\begin{prop}\label{prop:FlagVarietyAsQuotient}
  The morphism of schemes $\mu_{\Flag}: \GL_V \to X$, $g \mapsto g.\Flag$
  yields an isomorphism $\GL_V/B_{\Flag} \to X$. 
\begin{proof}
  We show that Zariski-locally, $\mu_{\Flag}$ induces an isomorphism 
  $\GL_V(S)/B_{\Flag}(S) \to X(S)$.
  Let $(v_1, v_2, \dots, v_n)$ be a basis of $V$ such that each $F_i$ is generated
  by the first $i$ basis vectors. 
  Given any $k$-scheme $S$ and a family of flags $\Flag' \in X(S)$, 
  there is a Zariski-cover $\phi: S' \to S$ (with structure map to $\spec k$ denoted
  by $f'$) trivializing all of the quotients $F_i'/F_{i-1}'$ for $i = 1, \dots,
  n$. Hence we may choose generators $w_i \in \Gamma(S', \phi^*(F_i'/F_{i-1}'))$,
  and lift them to elements $\tilde w_i \in \Gamma(S', f'^* \tilde V)$. The 
  global sections $w_i$ generate $f'^* \tilde V$, and the $\cO_{S'}$-linear
  map $f'^* v_i \mapsto w_i$ yields an element in $\GL_V(S')$, unique up to 
  an element in $B_{\Flag}(S')$. Thereby $X(S') \cong \GL_V(S')/B_{\Flag}(S')$.
\end{proof}
\end{prop}

\textbf{Remark. }The proof shows that the quotient sheaf $\GL_V/B$, 

\begin{cor}\label{cor:XisSmoothAndOfDimension}
  The scheme $X$ is smooth over $k$, of dimension $\frac{n(n-1)}2$. 
\begin{proof}
  This follows as quotients of smooth algebraic groups by algebraic subgroups are 
  smooth (cf. \cite[Corollary 5.26]{milne2017algebraic}), and the fact that 
  $$n^2 = \dim \GL_V = \dim B + \dim X = \frac{n(n+1)}2 + \dim X,$$
  cf. \cite[Proposition 5.23]{milne2017algebraic}. 
\end{proof}
\end{cor}

Write $\TT_X$ for the sheaf assigning to an $X$-scheme $S \to X$ (corresponding
to a  family of flags $(\cF_i)_i \in X(S)$) the group 
$$\TT_X(S) = \Aut_{\cO_S}(\cF_1/\cF_0) \times \dots \times \Aut_{\cO_S}(\cF_n/\cF_{n-1}) 
= \GG_{m,X}^n(S).$$

\begin{defi}[Classifying Space of Marked Flags]\label{def:MarkedFlagVar}
  Let $Y$ be the functor $\SchOver X^\op \to \Set$ given by sending a morphism
  $S \to X$, corresponding to a family of flags $(\cF_i)_i \in X(S)$, to the set
  \begin{equation*}
    Y(S) = \left\{(e_1, \dots, e_n) \mid e_i: \cO_S \xto \sim 
    \cF_i/\cF_{i-1}\text{ for } i = 1, \dots, n-1\right \}.
  \end{equation*}
  Here, $\cF_0$ is the zero-sheaf.
\end{defi}
Just like $X$, the functor $Y$ comes with a natural action by $\GL_V$
and the natural morphism
\begin{equation}\label{eq:naturalmapYtoX}
  Y(S) \to X(S), \quad (\cF_i, e_i)_i \mapsto (\cF_i)_i
\end{equation}
is equivariant for this action. One readily checks that $Y$ is a sheaf on
$\SchOver X_{\mathrm{Zar}}$, the big Zariski site of schemes over $X$. 
By design, it is a $\TT_X$-torsor and thereby admits a Zariski-cover of open
subfunctors isomorphic to $\TT_X$. Hence it is representable (cf. \cite[Theorem
8.9]{gortz2020algebraic}), and the morphism $Y \to X$ is smooth and affine.

Furthermore, the scheme $Y$ is also isomorphic to certain quotients of
algebraic groups.
Let $(\Flag,e_\bullet) \in Y(k)$ be a marked flag, and write 
$U_{\Flag, e_\bullet} \subset \GL_V$ for the (unipotent) isotropy subgroup of
$(\Flag, e_\bullet)$ under the action of $\GL_V$. 
\begin{lem}\label{lem:MarkedFlagVarietyAsQuotient}
  In this situation, $Y \cong \GL_V/U_{\Flag, e_\bullet}$.
\begin{proof}
  This can be shown using the same arguments as in the proof of
  Proposition \ref{prop:FlagVarietyAsQuotient}.
\end{proof}
\end{lem}

If $(v_1, \dots, v_n)$ is a basis for $V$, we write $\vecFlag(v_1, \dots, v_n)$ for
the (marked) flag spanned by the vectors $(v_1, \dots, v_n)$. 
More generally, if $S$ is a $k$-scheme $(v_1, \dots, v_n)$ is a tuple of elements 
in $\Gamma(S, f^*\tilde V)$ such that the induced map $(v_1, \dots, v_n): \cO_S^n \to 
f^* \tilde V$ is an isomorphism (in which case we call $(v_i)_i$ a basis), we
write $\vecFlag(v_1, \dots, v_n)$ for the corresponding family of (marked) flags.

Recall the Bruhat decomposition for $\GL_V$.
Fixing a basis $(e_1, \dots, e_n)$ of $V$, we obtain an injection
$\Sigma_n \inj \GL_V$ (assigning to each $w \in \Sigma_n$ the corresponding
permutation matrix), and a (marked) flag $\Flag^\std = \vecFlag(e_1, \dots, e_n) \in
X(k)$.  For any such choice of a basis, we define $O_w$ as the $\GL_V$-orbit
of the pair of flags $(\Flag^\std, w.\Flag^\std) \in (X \times X)(k)$.
Note that this does not depend on the choice of basis.
The Bruhat decomposition states that all $\GL_V$-orbits inside $X \times X$ are of 
this form.

\begin{prop}[Bruhat Decomposition]\label{prop:BruhatDecomp}
  There is a decomposition of 
  $X \times X$ into $\GL_V$-stable locally closed subschemes
  \begin{equation*}
    X \times X = \bigcup_{w \in \Sigma_n}^{\cdot} O_w.
  \end{equation*}
  Each $O_w$ is smooth of dimension $\dim(X) + l(w)$, where $l(w)$ denotes the
  Coxeter-length of $w$.
\begin{proof}
   For each $w$, the scheme $O_w$ is locally closed as orbits are locally
   closed by \cite[Proposition 1.65 b)]{milne2017algebraic} and smooth,
   as it is isomorphic to a quotient of $\GL_V$. The remaining claims
   boil down to classical theory (in particular, the classical Bruhat decomposition),
   cf. \cite[Chapter 21]{milne2017algebraic}, and elementary considerations about 
   the dimensions of the isotropy subgroups of pairs $(\Flag, \Flag') \in X_w(k)$.
\end{proof}
\end{prop}

Let $(\Flag, \Flag') \in (X \times X)(S)$ be a pair of flags over a $k$-scheme
$S$. We say that $(\Flag,\Flag')$ is in relative position 
$w \in \Sigma_n$ if it lies inside the subset $O_w(S) \subset (X \times X)(S)$. 

Similarly, we can characterize the $\GL_V$-orbits in $Y \times_{\spec k} Y$.
For any choice of elements $w \in \Sigma_n \subset \GL_V(k)$ and $t \in
\TT_X(k)$, we define $\tilde O_{w,t}$ as the $\GL_V$-orbit of the element
$$((\Flag^\std, e_\bullet^\std), (w.\Flag^\std, w.te_\bullet^\std)) \in (Y
\times Y)(k).$$ 

A pair of marked flags over a $k$-scheme $S$ is said to be in relative position
$(w,t) \in \Sigma_n \times (k^\times)^n$ if it lies inside $\tilde O_{w,t}$.
The following proposition gives a convenient characterization of relative position.

\begin{lem}\label{lem:RelativePositionVecCrit}
  \begin{enumerate}
    \item A pair of families of flags $(\Flag, \Flag') \in (X \times X)(S)$ is 
      in relative position $w \in \Sigma_n$ if and only if there exists a
      Zariski-cover $ \phi: S' \to S$ (with structure map to $k$ denoted by
      $f'$) and a basis $(v_1, \dots, v_n) \in \Gamma(S', f'^*\tilde V)$ such
      that 
      \begin{equation*}
        \phi^* F_i  = \langle v_1, \dots, v_i \rangle
        \quad \text{ and } \quad 
        \phi^* F'_i  = \langle v_{w(1)}, \dots, v_{w(i)} \rangle 
        \quad \text{for all } i = 1, \dots, n-1.
      \end{equation*}
    \item A pair of families of marked flags $((\Flag, e_\bullet), (\Flag', e_\bullet'))
      \in (Y \times Y)(S)$
      is in relative position $(w,t)$ if and only if $(\Flag, \Flag')$ is in relative
      position $w$ and there is a basis as above furthermore satisfying
      \begin{equation*}
        \phi^* e_i \equiv v_i \mod \phi^*F_{i-1} \quad \text{and} \quad
        \phi^*e'_i \equiv t_{w(i)}v_{w(i)} \mod \phi^*F'_{w(i)-1} \quad
        \text{for all } i = 1, \dots, n.
      \end{equation*}
      Here, $\phi^*$ denotes the natural pullback of sections
      $\Gamma(S, f^*\tilde V) \to \Gamma(S', f'^* \tilde V)$.
  \end{enumerate}
\begin{proof}
  This is a mere reformulation of what it means to be in the corresponding
  $\GL_V$-orbits. Given any choice 
  of 'standard' basis $(e_1, \dots, e_n)$ of $V$ and a section $(\Flag, \Flag')$
  in the orbit of $(\Flag^\std, w.\Flag^\std)$, we
  may choose $S'$ such that there exists a $g \in \GL_V(S')$ 
  satisfying $g.(\Flag^\std|_{S'}, w.\Flag^{\std}|_{S'}) = (\Flag, \Flag')$.
  Now it is easily seen that the global sections $v_i = g(e_i) \in \Gamma(S',
  f^*\tilde V)$ satisfy the desired conditions. Conversely, any such basis
  yields an element in $\GL_V(S')$. The same ideas lead to the second statement.
  Note that here we only need to lift the sections $e_i$ to sections 
  in $\phi^* \tilde V$, which is possible once $S'$ is affine. 
\end{proof}
\end{lem}

We now specialize to the case where $k = \bar \FF_q$ is an algebraic closure of 
the finite field with $q$ elements, and $V = V_0 \otimes_{\FF_q} k$ for 
some $\FF_q$-vector space $V_0$. 
This equips $V$ with a $\Gal(k/\FF_q)$-action, and in particular the Frobenius
automorphism of $k$ (given on $k$ by $x\mapsto x^q$) yields a $k$-semilinear
automorphism $\Frob: V \to V$. As this automorphism sends subspaces to
subspaces, we obtain automorphisms 
\begin{equation*}
  \Frob: X \to X \quad \text{ and } \quad \Frob: Y \to Y.
\end{equation*}
Note that $X$ and $Y$ are defined over $\FF_q$, and these automorphisms are 
the same as the respective (relative) frobenii of $X$ and $Y$ over $k$. We write
$\gamma_\Frob$ for the corresponding graph-morphisms $X \to X \times_{\spec k} X$ and 
$Y \to Y\times_{\spec k} Y$.

For $w \in \Sigma_n$ and $t \in \TT_X$, we define the spaces
\begin{equation}\label{eq:defXYsigma}
  X_w \coloneqq O_w \times_{X \times_{\spec k}X, \gamma_\Frob} X
  \quad \text{and} \quad
  Y_{w,t} \coloneqq \tilde O_{w,t} \times_{Y \times_{\spec k}Y,
  \gamma_\Frob} Y
\end{equation}
As $\gamma_\Frob$ admits a section, $X_w$ is naturally a subscheme of $X$,
parametrizing those families of flags that are pointwise in relative position
$w$ to their Frobenius twist. Similarly, $Y_{w, t}$ is naturally a subscheme of $Y$,
parametrizing families of marked flags in relative position $(w,t)$ with their
Frobenius twist. 
We have natural maps $Y_{w, t} \to X_w$. 
As a pair of marked flags $((\Flag, e_\bullet), (\Flag', e_\bullet'))$ over $S$
lies in the same $\GL_V$-orbit as $((\Flag, t_\bullet e_\bullet), (\Flag', t'_\bullet
e_\bullet'))$ for $t_\bullet, t'_\bullet \in \GG_{m,X}^n(S)$ if and only if 
$t_\bullet = t'_\bullet$, we find that $Y_{w,t}$ is a
Zariski-torsor over $X_w$ for an affine group scheme $\TT_w^\Frob \times_k
X_w$. Here, $\TT_w = \Res_{\FF_{q^n}/\FF_{q}}(\GG_m) \times_{\FF_q} k$ is the
Weil restriction of the multiplicative group from $\FF_{q^n}$ to $\FF_q$. 
Hence, the $S$-valued points of $\TT_w^\Frob$ are given by
\begin{equation*}
  \TT_{w}^\Frob(S) = \{(t_1, \dots, t_n) \in \GG_{m}^n(S) \mid 
    t_i^q = t_{w(i)}\}.
\end{equation*}

Furthermore, an element $g \in \GL_V(S)$ stabilizes $Y_{w,t}(S) \subset Y(S)$ if
$g \in \GL_V^\Frob(S)$, so we obtain a $\GL_V^\Frob$ action on $Y_{w,t}$ and 
$X_w$. The morphism $Y_{w,t} \to X_w$ is equivariant for the $\GL_V^\Frob$-action.

One can show that the scheme $X_w$ is smooth (of pure dimension $l(w)$, 
as the intersection in \eqref{eq:defXYsigma} is transverse, cf.
\cite{delignelusztig1976}), so $Y_{w, t}$ is smooth and affine over $X_w$. 
To this end, we have constructed the spaces in the commutative diagram
\begin{equation} \label{eq:DLSquare}
\begin{tikzcd}[ampersand replacement=\&]
	{Y_{w,t}} \& {Y \cong \GL_V/U} \\
	{X_w} \& {X \cong \GL_V/B.}
	\arrow[hook, from=1-1, to=1-2]
	\arrow["{\TT_w^{\Frob}\text{-torsor}}"', from=1-1, to=2-1]
	\arrow["{\TT_X\text{-torsor}}", from=1-2, to=2-2]
	\arrow[hook, from=2-1, to=2-2]
\end{tikzcd}
\end{equation}
The interesting space is $Y_{w,t}$. It comes with commuting (left-)actions of 
$\GL_V^\Frob(k) = \GL_V(\FF_q)$ and $\TT_w^\Frob(k) = \FF_{q^n}^\times$. 
% subsection Deligne--Lusztig Varieties (end)

\subsection{An Explicit Example} % (fold)
\label{sub:An Explicit Example}
We keep the notation from the previous subsection. That is, $k = \bar \FF_q$, 
$V = V_0 \otimes_{\FF_q} k$, $X$ is the flag variety of $V$, and $Y$ is the 
variety of marked flags. In this subsection, we fix
$w = (1 \ 2 \ \dots \ n) \in \Sigma_n$ and $t = (1, \dots, 1) \in \GG_{X,m}^n$,
and give explicit descriptions of the resulting varieties appearing in the
square \eqref{eq:DLSquare}. To clearify notation, we write $\DL_V = Y_{w, t}$ in this
situation.

First, note that $\TT_w^\Frob(S) = \GG_m(S)^{\Frob^n}$, implying that
$\DL_V$ admits commuting actions 
by $\GL(V_0)$ and $\TT_w^\Frob(k) = \FF_{q^n}^\times$.

\begin{lem}\label{lem:FlagsInRelPosWareez}
  A pair of flags $(\Flag, \Flag') \in (X \times X)(S)$ is in relative position
  $(1 \ 2 \ \dots \ n)$ 
  if and only if for all $i = 1, \dots, n-1$ the condition $F_i + F_i' =
  F_{i+1}$ is satisfied. Here, the sum denotes the Zariski-sheafification of the
  corresponding presheaf. 
\begin{proof}
  It is easily seen that the criterion may be checked Zariski-locally.
  Here it follows quickly from Lemma \ref{lem:RelativePositionVecCrit} and 
  the Bruhat decomposition, Proposition \ref{prop:BruhatDecomp}.
  \end{proof}
\end{lem}

For a linear form $\mu \in V^\vee$, we write 
$D^+(\mu)$ for the affine open subscheme of $\PP(V)$ parametrizing
lines in $V$ that do not lie in the hyperplane defined by the equation $\mu(v) = 0$.

\begin{prop}\label{prop:XwExplicitely}
  If $w = (1 \ 2 \ \dots \ n)$, the morphism of functors defined on 
  $k$-schemes $S$ by
  \begin{equation*}
    \Phi(S): X_w(S) \to \PP(V)(S) \quad \Flag \mapsto F_1
  \end{equation*}
  yields an isomorphism 
  $$X_w \cong \bigcap_{\mu \in V_0^\vee} D^+(\mu) \subset \PP(V).$$
  That is, $X_w$ parametrizes lines in $V$ that do not lie inside any
  $\FF_q$-rational hyperplane. In particular $X_w$ is equal to a finite
  intersection of affine subschemes, hence an affine scheme itself.
\begin{proof}
  We first show that the image of any family of flags $\Flag \in X_w(S)$ lies inside
  $\cap_{\mu} D^+(\mu)(S)$. As the latter is an open subscheme, it suffices to
  show that any $s \in |S|$ maps into
  $\cap_{\mu} D^+(\mu)$, which is the case if any only if $F_1(s) = F_{1,s}
  \otimes_{\cO_{S,s}} \kappa(s)$  does not lie inside
  any $\FF_q$-rational hyperplane in $\kappa(s) \otimes_k V$. By Lemma
  \ref{lem:FlagsInRelPosWareez}, we find
  $$F_1(s) \oplus \Frob(F_1(s)) \oplus \dots \oplus \Frob^{n-1}(F_1(s)) = V
  \otimes_k \kappa(s),$$
  so $F_1(s)$ cannot lie inside any non-trivial Frobenius-stable linear subspace
  of $V \otimes_k \kappa(s)$. The claim follows. 

  To see bijectivity of $\Phi$, note that the inverse is, if well-defined,
  given by the morphism of functors $\Psi$ given on components by 
  \begin{equation*}
    \Psi(S): \bigcap_{\mu \in V_0^\vee}D^+(\mu)(S) \to X_w(S), \quad L \mapsto 
    \big(L \oplus \Frob(L) \oplus \dots \oplus \Frob^{i-1} L\big)_{i = 1, \dots, n}.
  \end{equation*}
  To see that this is indeed well-defined, choose a basis $(e_1, \dots, e_n)$
  of $V_0$ and take any section $\cL \in \cap_{\mu \in
  V_0^\vee}D^+(\mu)(S)$, interpreted as a locally direct summand of $f^* \tilde V$. 
  It suffices to work locally on $S$, and we may pick for any $s \in \abs S$
  some open affine $\spec R \subset S$ trivializing $\cL$. Write $L$ for the
  corresponding free rank $1$ direct summand of $R^n$. Then $L = \langle v
  \rangle$ for some $v = (v_1, \dots, v_n) \in R^n$. Now 
  $L$ constitutes a flag if and only if $(v, \Frob(v), \dots, \Frob^{n-1}(v))$
  is a basis for $R^n$ if and only if $\det(\Frob^{j-1}(v_i))_{i,j} \in R^\times$.
  The last condition is satisfied. Indeed, if not, we may 
  choose a maximal ideal $\fm \in \spec R$ containing
  $\det(\Frob^{j-1}(v_i))_{i,j}$. Let $\bar v$ denote the residue of $v$ in
  $(R/\fm)^n$. Now, the subspace
  $$\langle \bar v, \Frob(\bar v), \dots, \Frob^{n-1}(\bar v)\rangle \subset (R/\fm)^n$$ 
  is non-trivial and Frobenius-stable, and in particular
  contained in some $\FF_q$-rational hyperplane. This contradicts $\cL \in
  \cap_{\mu \in V_0^\vee}D^+(\mu)(S)$. 
\end{proof}
\end{prop}

We write $\Delta: \Sym(\bigwedge V^\vee) \to \Sym(V^\vee)$ for the morphism corresponding
to the $k$-linear morphism 
\begin{equation*}
  \bigwedge V^\vee \to \Sym(V^\vee), \quad \mu \mapsto \big[v \mapsto 
  \mu(v \wedge \Frob(v) \wedge \dots \Frob^{n-1}(v))\big].
\end{equation*}

\begin{prop}\label{prop:prop_YwExplicitely}
  The map of functors $\DL_V \to \spec \Sym(V^\vee)$ given by 
  $(\Flag, e_\bullet) \mapsto e_1$ 
  yields an isomorphism of $\DL_V$ and the subfunctor of $\spec \Sym(V^\vee)$ given on
  affine schemes by
  \begin{equation*}
    \spec R \mapsto \left\{v \in R \otimes_k V \, \bigg \vert
      \begin{array}c 
        (v, \Frob\, v, \dots, \Frob^{n-1} v) \text{ is a basis and } \\ 
        v \wedge \dots \wedge \Frob^{n-1} v = (-1)^{n-1} \Frob( v \wedge \dots
        \wedge \Frob^{n-1}v) 
      \end{array}\right\}.
  \end{equation*}
  Writing $S_1$ for the degree-$1$ part of $\Sym(\bigwedge V^\vee)$,
  this functor is readily seen to representable by the $k$-scheme
  \begin{equation*}
    \DL_V \coloneq \spec \left( \frac{\Sym(V^\vee)[\Delta(S_1 \setminus\{0\})^{-1}]}
    {(\Frob (\Delta(\lambda)) - (-1)^{n-1} \Delta(\lambda) \mid \lambda \in S_1)} \right).
  \end{equation*}
  Upon choosing a basis of $V_0 \cong \FF_q^n$, this takes on the form
  \begin{equation*}
    \DL_n \coloneq \spec \left(\frac{k[x_1, \dots, x_n]}{\left(
        \det D(\underline x)^{q-1} - (-1)^{n-1}\right)} \right), \quad
        \text{where} \quad D(\underline x) = 
          \begin{pmatrix}
            x_1 & \dots & x_1^{q^{n-1}}\\
            \vdots & \ddots  & \vdots \\
            x_n & \dots & x_n^{q^{n-1}}
          \end{pmatrix}.
  \end{equation*}
\begin{proof}
  Let $S = \spec R$ be an affine $k$-scheme and let $(\Flag, e_\bullet) \in \DL_V(S)$. 
  By Lemma \ref{lem:RelativePositionVecCrit}, there is a basis 
  $(v_1, \dots, v_n)$ of $R \otimes_k V$ such that 
  \begin{equation}\label{eq:conditionsonvecs}
    \begin{aligned}
      v_i &\equiv e_i \mod F_{i-1} \quad &\text{and} \quad v_{i+1} &\equiv
      \Frob(e_i) \mod \Frob(F_{i-1}) \quad \text{for } 1 \leq i \leq n-1,\\ 
      v_n &\equiv e_n \mod F_{n-1} \quad &\text{and}\hspace{0.8em}\quad  v_1 &\equiv
    \Frob(e_n) \mod \Frob(F_{n-1}).
  \end{aligned}
\end{equation}
  From here we quickly find
  \begin{equation*}
    \Frob(v_1 \wedge v_2 \wedge \dots \wedge v_n) = 
    \Frob\, v_1 \wedge \dots \wedge \Frob\, v_n = v_2 \wedge v_3
    \wedge \dots \wedge v_{n} \wedge v_1.
  \end{equation*}
  The equivalences in \eqref{eq:conditionsonvecs} also imply that for integers
  $2 \leq m \leq n$, we have $\Frob^{m-1} \, v_1
  \equiv v_m \mod \Frob(F_{m-2})$. Also, we find $v_1 \equiv \Frob^n \, v_1 \mod \Frob
  (F_{n-1})$. Altogether, writing $v = v_1 = e_1$, this yields 
  \begin{equation}\label{eq:WedgeConditionMarkedFlagInDL}
    \Frob (v \wedge \Frob(v) \wedge \dots \wedge \Frob^{n-1} v) = 
    (-1)^{n-1} (v \wedge \dots \wedge \Frob^{n-1}v).
  \end{equation}
  This shows that the map given in the statement of the proposition is
  well-defined. To see that it is bijective, note that it has an inverse. Indeed,
  given any $v \in R \otimes_k V$ such that $(v, \Frob\, v, \dots, \Frob^{n-1} v)$
  is a basis and $v$ satisfies the equation \eqref{eq:WedgeConditionMarkedFlagInDL},
  Gaussian elimination shows that the corresponding marked flag is in relative
  position $(w,1)$ to its Frobenius-twist.

  If we are given a basis of $V_0$, we may write $v = (x_1, \dots, x_n)$
  and identify $v \wedge \Frob\, v \wedge \dots \Frob^{n-1} v$ with
  $\det\left[ (x_i^{q^{j-1}})_{1 \leq i,j \leq n} \right]$.
  Thereby $v$ gives a marked flag in $\DL_n$ if and only if
  \begin{equation*}
    \det((x_i^{q^{j-1}})_{i,j})^{q-1} = (-1)^{n-1}.
  \end{equation*}
  This gives the representability statement of $\DL_n$.
\end{proof}
\end{prop}

Note that $\DL_n$ has $q-1$ disjoint irreducible components, parametrized by the set
of solutions $b \in k$ to the equation $z^{q-1}=(-1)^{n-1}$. For any such $b
\in k$ we write (in accordance with notation in \cite{mieda2016geometric})
\begin{equation}\label{eq:DefYb}
  Y_b \coloneqq \spec \left(\frac{k[x_1, \dots, x_n]}{(\det D(\underline x) - b)}\right),
\end{equation}
and obtain $\DL_n = \sqcup_{b^{q-1} = (-1)^{n-1}} Y_b$.

The right action of $\GL_n(\FF_q) \times \FF_{q^n}^\times$ on $\DL_n$ has the following
explicit description.

\begin{lem}\label{lem:ActionsOnDLn}
  Let $g \in \GL_n(\FF_q)$ be an element with matrix entries $(a_{ij})_{1 \leq
  i,j \leq n}$. Then $g$ acts on the global sections of $\DL_n$ from the left via
  \begin{equation*}
    x_i  \mapsto  \sum_{j = 1}^n a_{ji}x_j \quad \text{for} \quad i = 1, \dots, n. 
  \end{equation*}
  Similarly, an element $d \in \FF_{q^n}^\times$ acts via
  \begin{equation*}
    x_i \mapsto d^{-1} x_i \quad \text{for} \quad i = 1, \dots, n.
  \end{equation*}
\end{lem}

Through the $q$-th power Frobenius automorphism on $\bar \FF_q$, we may also
define an action of $\Z$ on $\DL_n$, sending $1$ to the automorphism given by
$x_i \mapsto x_i$, $a \mapsto a^{-q}$. Note that this action is defined over 
$\FF_q$, not over $\bar \FF_q$. 
By construction, the action of $\Z$ commutes with the action of $\GL_n(\FF_q)$, 
but in order to make it commute with the action of $\FF_{q^n}^\times$, we 
have to restrict to $n\Z \subset \Z$. 

Note that $g \in \GL_n(\FF_q)$ induces a morphism of schemes
$Y_b \to Y_{b \det(g)}$. Similarly, $\zeta \in \FF_{q^\times}$ 
restricts to $Y_b \to Y_{b \Norm(\zeta)^{-1}}$, where $\Norm = \Norm_{\FF_{q^n}/\FF_q}$
denotes the norm map of the extension $\FF_{q^n}/\FF_q$. The action of 
$1 \in \Z$ sends $Y_b$ to $Y_{(-1)^{n-1} b}$, therefore the action of 
$n\Z$ stabilizes each component $Y_b$. In particular, 
the subgroup 
\begin{equation*}
  (\GL_n(\FF_q) \times \FF_{q^n}^\times)^1 \times n\Z 
  \coloneq \{(g,d,n) \in \GL_n(\FF_q) \times \FF_{q^n}^\times \times n\Z 
  \mid \det(g) \Norm(d) = 1\}
\end{equation*}
of $\GL_n \times \FF_{q^n}^\times \times \Z$
stabilizes $Y_b$ for every choice of $b$.

We write $\hH_\DL$ for the $\bar \Q_l$-vector space
\begin{equation*}
  \hH_\DL = \hH_c^{n-1}(\DL_n, \bar \Q_l).
\end{equation*}
By the above, this is a representation of $\GL_n(\FF_q) \times \FF_{q^n}^\times
\times n\Z$. The following subsection is concerned with the study of this
representation.

% subsection An Explicit Example (end)

\subsection{An Example of the Deligne--Lusztig Correspondence} % (fold)
\label{sub:The Deligne--Lusztig Correspondence for the Explicit Example}

In this subsection, we are concerned with the \'etale cohomology of the variety
$\DL_n$ introduced in the previous subsection. As before, $k$ denotes an algebraic
closure of $\FF_q$.

We give a short review of \cite{delignelusztig1976}.
For a connected, reductive algebraic group $G$ defined over $\FF_q$ and a maximal
Frobenius-stable torus $T \subset G$ contained in a Borel subgroup $B
\subset G$, Deligne and Lusztig construct varieties with right $G^\Frob$-actions
$X_{T\subset B}$ and $\tilde X_{T \subset B}$, constituting a $G^\Frob$-equivariant
Galois covering
\begin{equation*}
  \tilde X_{T \subset B} \to X_{T \subset B}
\end{equation*}
with Galois group $T^\Frob$. See \eqref{eq:ExplicitDLVarieties} for explicit
descriptions of these spaces. The space $\tilde X_{T \subset B}$ comes with
commuting actions of $G^\Frob$ and $T^\Frob$, and for characters $\theta$ of $T^\Frob(k)$,
the main concern of \cite{delignelusztig1976} is the study of the resulting
virtual representations
\begin{equation} \label{eq:DLRepresentation}
  R_{T}^\theta = \sum_{i} (-1)^i \hH_c^i(\tilde X_{T \subset B}, \bar
  \Q_l)_\theta \in \cR(T^\Frob(k) \times G^\Frob(k)).
\end{equation}
Here, the subscript $\theta$ denotes the direct summand of $\hH^i_c(\tilde X_{T
\subset B}, \bar \Q_l)$ where $T^\Frob(k)$ acts by $\theta$.
These representations do only depend on the choice of the torus, not on the
choice of the Borel sugroup
(\cite[Corollary 4.3]{delignelusztig1976}). To formulate the results of the theory 
precisely, we need the notion of regular position.

\begin{defi}[Regular Position]\label{def:DLGeomConjRegularPos}
  Let $T$ be a Frobenius-stable maximal  torus of $G$ and let $\theta$ be a character
  of $T^\Frob(k)$. We say that $\theta$ is in general position if it is not fixed by any
  non-trivial element of $\Weyl_G(T)^\Frob$. Here $\Weyl_G(T)$ denotes the Weyl
  group $(\Norm_G(T)/T)$, which acts on $T$ by conjugation.
\end{defi}

Recall that for a finite group $H$, the Grothendieck group $\cR(H)$ of finite 
dimensional $H$-representations over $\bar \Q_l$ comes with a natural inner
product. Indeed, any representation takes values in the maximal cyclotomic
subfield $\cup_{r} \Q(\zeta_r) \subset \bar \Q_l$, which has the unique 
"complex conjugation" automorphism given by $\zeta_r \mapsto \zeta_r^{-1}$ for 
$r \in \N$. 
The inner product is now defined on finite-dimensional representations $\rho,
\rho' \in \Rep H$ as
\begin{equation*}
  \langle \rho, \rho' \rangle = \frac 1 {\# H} \sum_{h \in H} \Tr(\rho(h)) \bar {\Tr (\rho'(h))} \in \bar \Q_l.
\end{equation*}
This definition linearly extends to $\cR(H)$, and the irreducible
finite-dimensional representations of $H$ give an orthogonal basis for $\cR(H)
\otimes \Q$. If $\rho$ is an irreducible representation of $H$ and 
$R$ is an element in $\cR(H) \otimes \Q$, we say that $\rho$ occurs in $R$ if 
$\langle \rho, R\rangle \neq 0$.

The following results of Deligne-Lusztig theory are important for our purposes.

\begin{thm}[Some Results of Deligne-Lusztig Theory]\label{thm:DLTheoryGeneralResults}
  \leavevmode
  \begin{enumerate}
      \item \textnormal{\cite[Corollary 1.22]{delignelusztig1976}} Let $Z \subset G$
        denote the center of $G$. Then $Z^\Frob$ acts on $\hH^{n-1}_c(\tilde
        X_{T \subset B}, \bar \Q_l)_{\theta}$ through $\theta|_{Z^\Frob}$.
      \item \textnormal{\cite[Corollary 7.3]{delignelusztig1976}} If 
        $(T, \theta)$ is in general position, one of $\pm R_T^\theta$ is an 
        irreducible representation.
      \item \textnormal{\cite[Corollary 8.3]{delignelusztig1976}} If furthermore
        $T$ is not contained in any proper Frobenius-stable parabolic subgroup,
        one of $\pm R^\theta_T$ is a cuspidal representation of $G^\Frob(k)$.
      \item \textnormal{\cite[Corollary 9.9]{delignelusztig1976}} If furthermore
        $\tilde X_{T \subset G}$ (or equivalently, $X_{T \subset G}$) is affine, 
        and we denote by $w \in \Weyl_G(T)$ the relative position of $B$ and
        $\Frob(B)$ (given by the Bruhat decomposition for $G$),
        we have $$\hH_c^i(\tilde X_{T \subset G}, \bar \Q_l)_\theta = 0 \quad \text{if}
        \quad i \neq l(w).$$
        Here, $l(w)$ denotes the Coxeter-length of $w$.
      \item \textnormal{\cite[Theorem 9.8]{delignelusztig1976}} For a character 
        $\theta$ in general position, the natural map
        \begin{equation*}
          \hH^i_c(Y_{T \subset B},\bar \Q_l)_\theta \to \hH^i(Y_{T \subset
          B}, \bar \Q_l)_\theta
        \end{equation*}
        is an isomorphism.
    \end{enumerate}
\end{thm}

We next explain how these results apply to the compactly supported \'etale cohomology
of $\DL_n$. We set $G = \GL_n$, and we denote by $B^\std$ the standard Borel
subgroup of upper triangular matrices, by $T^\std$ the standard torus of
diagonal matrices
and by $U^\std$ the standard unipotent subgroup of upper triangular matrices with
diagonal entries equal to $1$. We have $T^\std U^\std = B^\std$. 

Choose a flag $\Flag \in X_w(k)$, where $w$ is, as usual, $(1\ 2\ \dots \ n)
\in \Sigma_n \cong \Weyl_G(T^\std)$. Let $B \subset \GL_n$ denote the isotropy
subgroup of $\Flag$, and let 
$T \subset B$ be a Frobenius-stable maximal torus in $B$.
Write $U$ for the unipotent radical of $B$.
We have the explicit descriptions (cf. \cite[Definition 1.17]{delignelusztig1976})
\begin{equation} \label{eq:ExplicitDLVarieties}
  \begin{array}c
  X_{T \subset B} = \{g \in \GL_n \mid g^{-1} \Frob(g) \in \Frob(U)\}/(T^\Frob(U \cap \Frob(U)))\\
  \text{and} \\
  \tilde X_{T \subset B} = \{g \in \GL_n \mid g^{-1} \Frob(g) \in \Frob(U)\}(U
  \cap \Frob(U)).
  \end{array}
\end{equation}
Given any marked flag $(\Flag', e'_\bullet) \in X(S)$, we write $g(\Flag', e'_\bullet) \in
\GL_n/U^\std(S)$ for the corresponding
section under the isomorphism of Proposition \ref{lem:MarkedFlagVarietyAsQuotient}
(i.e., $g.(\Flag^\std,e_\bullet^\std) = (\Flag',e'_\bullet)$). Also, by 
\cite[Proposition 17.13]{milne2017algebraic}, we may
choose $h \in \GL_n(k)$ such that $h(T^\std, B^\std)h^{-1} = (T,B)$. 
Furthermore, for any $k$-scheme $S$, we may identify $\TT_w^\Frob$ with the subgroup 
\begin{equation*}
  \{t \in T^\std(S) \mid \ad w^{-1} t = \Frob(t)\} \subset T^\std(S).
\end{equation*}

Using these identifications, \cite[Proposition 1.19]{delignelusztig1976}
implies that the map
\begin{equation*}
\DL_n \to \tilde X_{T \subset B}, \quad (\Flag', e'_\bullet) \mapsto g(\Flag',
e'_\bullet)h^{-1}
\end{equation*}

gives an isomorphism of $\GL_n^\Frob$-equivariant torsors

\begin{equation*}
\begin{tikzcd}[ampersand replacement=\&]
	{\DL_n} \& {\tilde X_{T\subset B}} \\
	{X_w} \& {X_{T \subset B.}}
  \arrow[""{name=0, anchor=center, inner sep=0},
  "{\TT_w^\Frob\text{-torsor}}"', from=1-1, to=2-1]
  \arrow[""{name=1, anchor=center, inner sep=0}, "{T^\Frob\text{-torsor}}",
  from=1-2, to=2-2] \arrow["\sim", shorten <=15pt, shorten >=15pt, from=0, to=1]
\end{tikzcd}
\end{equation*}

Given any representation $\theta$ of $\TT^\Frob_w(k)$, we write
\begin{equation*}
  R_\theta = \sum_i (-1)^i \hH_c^i(\DL_n, \bar \Q_l)_\theta.
\end{equation*}

By definition, we have $R_{\theta \circ \ad h} \cong R^\theta_{T\subset B}$,
where $h$ is chosen as above. If $\theta$ is a
character of $\TT_w^\Frob(k) \cong \FF_{q^n}^\times$, the pair $(T, w \circ \ad
h)$ is in regular position if and only if $\theta$ is regular, in the following
sense.

\begin{defi}[Regular Character on $\FF_{q^n}^\times$]
  We say that a character $\theta: \FF_{q^n}^\times \to \C^\times$ is 
  regular if it does not factor through the norm morphism
  $$\Norm_{\FF_{q^n}/\FF_{q^m}}:\FF_{q^n}^\times \to \FF_{q^m}^\times$$ 
  for any $m < n$.
\end{defi}


The statements of Theorem \ref{thm:DLTheoryGeneralResults} reduce to
the following.

\begin{thm}[Deligne--Lusztig Correspondence]\label{thm:DLCorrespondenceForUs}
  Let $\theta$ be a regular character of $\FF_{q^n}^\times$, and let
  $\hH_{\DL,\theta} = \hH^{n-1}_c(\DL_n, \bar \Q_l)_\theta$ denote the direct
  summand of $\HDL$ on which $\FF_{q^n}^\times$
  acts through $\theta$. 
  \begin{enumerate}
    \item We have
      \begin{equation*}
        \hH_{\DL,\theta} = R_\theta \boxtimes
        \theta
      \end{equation*}
      as representations of $\GL_n(\FF_q) \times \FF_{q^n}^\times$.
      The representation $R_\theta$ is irreducible and cuspidal, and its central
      character is given by $\theta|_{\FF_q^\times}$ under the identification
      $\FF_q^\times \cong Z_{\GL_n}$.
    \item The natural map 
      \begin{equation*}
        \hH^{n-1}_c(\DL_n, \bar \Q_l)_\theta \to
        \hH^{n-1}(\DL_n, \bar \Q_l)_\theta
      \end{equation*}
      is an isomorphism.
  \end{enumerate}
\end{thm}

This finishes the discussion about the $\GL_n(\FF_q) \times \FF_{q^n}^\times$-action
on $\hH_\DL$. 

It remains to make the action of $n\Z$ explicit. The Frobenius automorphism on
$\bar \FF_q$ yields an action of $n\Z$ on $\hH_\DL$. This action admits the
following partial description.

\begin{prop}
  Let $\theta$ be a regular character. The subgroup $n\Z \subset \Z$ acts on
  $\hH_{\DL,\theta}$ through the character $\gamma:
  n\Z \to \Q^\times$, given by
  $$\gamma(nm) =  (-1)^{(n-1)m} q^{m \frac{n(n-1)}2}.$$
  \begin{proof}
    Note that, as the absolute Frobenius morphism $\DL_n \to \DL_n$
    induces the identity on \'etale cohomology (cf.
    \cite[\href{https://stacks.math.columbia.edu/tag/03SN}{Tag
    03SN}]{stacks-project}), the claim is equivalent to showing that pullback 
    along the relative Frobenius 
    \begin{equation*}
      \Frob_{q^n}: \DL_n \to \DL_n^{(q^n)} = \DL_n, \quad
      x_i \mapsto x_i^q,\quad a \mapsto a \quad (a \in \bar \FF_q)
    \end{equation*}
    induces multiplication of $(-1)^{n-1}q^{\frac{n(n-1)}2}$
    on $\hH^{n-1}_c(\DL_n, \bar \Q_\ell)_\theta$. 
    This result is essentially due to Digne--Michel, cf. \cite[Remarque
    3.14]{digne1985fonctions}, but their proof contains a sign error. This mistake
    was identified and corrected by Wang in \cite[Théorème 3.1.12]{wang2014espace}.
  \end{proof}
\end{prop}

Concludingly, we obtain the following structural result about the 
$\GL_n(\FF_q) \times \FF_{q^n}^\times \times n\Z$-representation 
$\hH_\DL$.

\begin{thm}[Structure of $\hH_\DL$]\label{thm:MainThmHDlStructure}
  Let $\theta: \FF_{q^n}^\times \to \bar \Q_l^\times$ be a regular character.  
  The $\GL_n(\FF_q) \times \FF_{q^n}^\times \times n\Z$-representation
  $\hH_{\DL, \theta}$ is given by 
  \begin{equation*}
    {\HDL}_{,\theta} = R_\theta \boxtimes \theta \boxtimes \gamma.
  \end{equation*}
  Here, $R_\theta$ is the irreducible cuspidal representation from
  Deligne--Lusztig theory (cf. Theorem \ref{thm:DLCorrespondenceForUs}), and
  $\gamma$ is the character defined in the previous proposition.
\end{thm}

% subsection The Deligne--Lusztig Correspondence for the Explicit Example (end)
% section Deligne-Lusztig Theory for Depth Zero Representations (end)

\end{document}


