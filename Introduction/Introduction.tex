%! TeX root: ../main.tex
\documentclass[../main.tex]{subfiles}

\begin{document}
\section*{Introduction} % (fold)

When \textsc{Jonathan Lubin} and \textsc{John T. Tate} published their
paper \textit{Formal Complex Multiplication in Local Fields} in 1965 (cf.
\cite{LubinTateFormalMult}), they achieved one of the last major breakthroughs
in Class Field Theory. Perhaps inpsired by the theory of complex multiplication
of elliptic curves, they show that formal group laws, which can be seen as local
analogues of abelian varieties, may be used to construct abelian extensions of
arbitrary non-Archimedian local fields $E$. This approach, colloqually
reguarded to as Lubin--Tate Theory, provides a strikingly simple solution to a
complicated problem, that of finding generators for the maximal abelian
extension of $E$ and describing how the Galois group acts on them. 

Two years later, in 1967, \textsc{Robert Langlands} wrote his famous letter 
to \textsc{André Weil}, introducing the web of conjectures that 
is now known as the Langlands Program. The ‘‘local part’’ of these conjectures 
predicts, in vague terms, a correspondence between smooth representations
of $\GL_n(E)$ and representations of the absolute Galois group of $E$, thus
conjecturing an analogue of local class field theory for non-Abelian extensions. 
Inspired by results of \textsc{Pierre Deligne} on the cohomology of certain
modular curves with bad reduction at $p$ (where $p$ denotes the residue
characteristic of $E$), \textsc{Henri Carayol} predicted in his 1990 paper
\textit{Non-Abelian Lubin--Tate Theory} that this (at the time conjectural)
correspondence appears, along with the Jacquet--Langlands Correspondence, in
the $\ell$-adic cohomology of the Lubin--Tate tower (cf. \cite{carayol1990non}),
which is a certain projective system of rigid analytic spaces related to
deformation spaces of formal $\cO_E$-modules.
% The latter is a certain projective system of rigid analytic
% spaces arising from moduli of deformations of formal $\cO_E$-modules with
% Drinfeld level $\varpi^m$
% structure $\{\cM_m\}_{m \in \N}$. The latter are formal schemes admitting a
% natural right action by the group $\GL_n(E) \times D^\times \times \Weil_E$,
% where $D$ is a division algebra over $E$ with invariant $1/n$ and $\Weil_E$ is
% the Weil group attached to $E$. 
The cohomology group in question, denoted $\HLT$, carries the structure of a
representation of $\GL_n(E) \times D^\times \times \Weil_E$, where $D$ is a
division algebra over $E$ with invariant $1/n$ and $\Weil_E$ is the Weil group
attached to $E$. Carayol's conjecture predicts that for supercuspidal
representations $\pi$ of $\GL_n(E)$, the $\pi$-isotypic component of $\HLT$
decomposes (up to duals and twists) as
\begin{equation*}
  \hH_{\LT,\pi} = \pi \boxtimes \JL(\pi) \boxtimes \rec_E(\pi),
\end{equation*}
where $\JL(\pi)$ and $\rec_E(\pi)$ denote the representations 
of $D^\times$ and $\Weil_E$ corresponding to $\pi$ under the Jacquet--Langlands and 
local Langlands correspondences. The study of the representation
$\HLT$ is colloqually regarded to as Non-Abelian Lubin--Tate Theory.

Roughly a decade later, this approach was used by \textsc{Michael Harris} and
\textsc{Richard Taylor} to define the local Langlands correspondence for
$\GL_n$ over $p$-adic fields, verifying (a part of)
Carayol's conjectures along the way (cf. \cite{HTShimura}). 
Around the same time, \textsc{Pierre Boyer} 
proved Carayol's conjectures for local function fields (cf.
\cite{boyer1999mauvaise}). 

Whilst these results are milestone accomplishments, they leave a few questions
unanswered, two of which we aim to address in the present text. 
First, the results of Harris--Taylor offer no insight into the question of
providing an explicit description of the correspondence.
Secondly, and this point is of rather conceptual nature, their proof makes
essential use of global arguments, ultimately requiring to realize $E$ as the
completion of some global field at one of its places.

In order to have any hope of attacking the latter issue, it appears essential
to gain a deeper understanding of the geometry of the Lubin--Tate tower. 
A step in this direction was taken by \textsc{Teruyoshi Yoshida} in his
2004 paper \textit{On non-abelian Lubin--Tate theory via vanishing cycles}
(cf. \cite{yoshida2010non}).
Yoshida uses a series of blow-ups to construct a
semistable model of the generic fiber of the first layer of the Lubin--Tate space,
whose reduction contains a Deligne--Lusztig variety for $\GL_n(\FF_q)$ (where
$\FF_q$ is the residue field of $E$) as an affine open subscheme. This allowed
him to prove, purely locally, that a part of non-Abelian Lubin--Tate theory
boils down to Deligne--Lusztig theory. 
Building on this work, \textsc{Jared Weinstein} initiated a program
to obtain similar results in 2010. 

Around the same time, \textsc{Peter Scholze} developed the theory of perfectoid spaces,
providing a framework that allows to analyze the geometry of the projective limit 
of the spaces in the Lubin--Tate tower. 
While no longer a locally finite-type analytic space, the resulting
Lubin--Tate perfectoid space $M_\infty$ turns out to be easier to work with than 
the individual layers of the Lubin--Tate tower. For example, it admits a
coordinate that is not available at finite level.
Recent results along the lines of Weinstein's program
(cf., for example, \cite{weinstein2016semistable},
\cite{BoyarchenkoWeinstein2011MaxVar}, \cite{imaitsushima2020affinoids} and 
\cite{tokimoto2020affinoids}) make use of this developement to constrcut
certain formal models of rational subsets of the Lubin--Tate perfectoid space
$M_{\infty}$ and observe parts of the local Langlands correspondence
inside the $\ell$-adic cohomology of their reduction. This approach relies on
previously known explicit descriptions of the local Langlands
correspondence.

Work of \textsc{Yoichi Mieda} (cf. \cite{mieda2016geometric}) reverses this
approach. Given a formal model $\cX$ of a rational subset $U \subset M_\infty$,
his results show that parts of the cohomology of the
special fiber of $\cX$ appear inside $\HLT$. Using the results of non-Abelian
Lubin--Tate theory, this allows, together with the zoo of formal models
constructed in the works referenced above, for an explicit description of
certain parts of the local Langlands correspondence.

The aim of this thesis is to give a (mostly) self-contained proof of
the following results, which are due to Mieda (cf. \cite[Section 5]{mieda2016geometric}).
\begin{thm*}[Main Results of this Thesis]
  \leavevmode \begin{enumerate}
    \item There is an affinoid $U \subset M_\infty$ inside the Lubin--Tate perfectoid
      space admitting a formal model $\cX$ whose reduction is isomorphic to the
      perfection of a Deligne--Lusztig variety for $\GL_n(\FF_q)$. Furthermore,
      this isomorphism is compatible for certain group actions on both sides.
    \item For certain (so-called depth zero supercuspidal) representations
      $\pi$ of $\GL_n(E)$, one can calculate $\rec_E(\pi)$ and $\JL(\pi)$
      explicitly, using only the geometry of the Lubin--Tate tower, and the
      fact that these representations appear inside its cohomology.
  \end{enumerate}
\end{thm*}
The first part is an infinite level analogue of the construction by Yoshida.
With this result at hand, the second part is an application of Mieda's ideas.
For precise statements, see \cref{prop:SpecialFiberOfAffinoidIsLusztig}
and \cref{thm:MainRes1} below. 

The structure of the thesis is as follows.

\textbf{\cref{sec:FormalModules}} reviews the theory of formal $\cO_E$-modules, which
will play a large role in the analysis of the Lubin--Tate tower. It encompasses
all results we need in later parts of the text.

In \textbf{Sections \ref{sec:Local Class Field Theory}  and \ref{sec:Non-Abelian Lubin-Tate Theory: An Overview}}, we review classical Lubin--Tate theory as in
\cite{LubinTateFormalMult}, and the theorems of non-Abelian Lubin--Tate theory
due to Boyer and Harris--Taylor. In particular,
we introduce the Lubin--Tate tower $\{\cM_m\}_{m \in \N}$ and describe the
relevant group actions.

\textbf{\cref{sec:The Lubin--Tate Space at Infinite Level}} introduces the
Lubin--Tate space at infinite level. 
We follow ideas of Weinstein to define a certain coordinate on $\cM_\infty = 
\lim_{m \in \N} \cM_m$. The main result of this section is a description of
the group actions on $\cM_\infty$ in terms of this coordinate. 

\textbf{\cref{sec:Mieda's Approach to the Explicit Local Langlands
Correspondence}} reviews the results of Mieda's paper
\cite{mieda2016geometric}. We do not go into details, but hope to provide some
insight into his methods. 

\textbf{\cref{sec:Deligne--Lusztig Theory}} deals with Deligne--Lusztig theory
for $\GL_n(\FF_q)$. 
We develop the theory exclusively for $\GL_n$, describing in detail the 
geometry of the moduli of flags and the related spaces that occur as 
special cases of the varieties considered by 
\textsc{Pierre Deligne} and \textsc{George Lusztig} in
\cite{delignelusztig1976}. The cohomological results of their paper are 
stated as facts and applied to our situation.

Finally, in \textbf{\cref{sec:Explicit Non-Abelian Lubin-Tate Theory for Depth
Zero Supercuspidal Representations}}, we collect the results of the 
previous sections to prove the results announced above. 

\subsection*{Notation and Conventions} % (fold)
\addcontentsline{toc}{subsection}{Notation and Conventions}

\textbf{Fields.} Throughout the text, the letter $E$ usually denotes
a non-Archimedian local field, for which we make a choice of uniformizer
$\varpi \in E$. We then denote by $\FF_q$ the residue
field of $E$, whose characteristic we denote by $p \in \N$, and we furthermore fix
an algebraic closure $\bar E$ of $E$. The $\varpi$-adic absolute value $\abs
-_\varpi$ on $E$ (normalized with $\abs \varpi_\varpi = 1/q$) extends uniquely
to $\bar E$, and we denote the completion of $\bar E$ with respect to $\abs
-_\varpi$ by $\Cp$. If $E \subseteq E' \subseteq \Cp$ is any
field extension, we denote by $\cO_{E'} = \{x \in E' : \abs x_\varpi \leq 1\}$
its ring of integers and by $\fm_{E'} = \{x \in E' : \abs x_{\varpi} < 1\}$ its
maximal ideal. Inside $\Cp$, we have for $m\in \N$ the degree $m$ unramified
extension $E_m$ of $E$ with residue field $\FF_{q^m}$, and the maximal unramified extension
$E^\nr$ of $E$, whose residue field we denote by $\Fqbar$ and whose closure
inside $\Cp$ we denote by $\br E$. We have the maximal abelian extension $E^\ab$ of 
$E$ with closure $\hat E^\ab \subset \Cp$. In \cref{sec:Local Class Field Theory} we 
introduce the field extensions $E_{\varpi, m}$ and $E^\LT_m$ of $E$.

\textbf{Categories.} 
We denote the category of sets by $\Set$ and the category of (unital,
commututative) rings by $\Ring$. If $A$ is a ring, we write $\Alg A$ for the
category of $A$-algebras, and $\Mod A$ for the category of $A$-modules.
In \cref{def:admring} we introduce the categories $\Adm A$ and $\Adic A$ of 
admissible and adic $A$-algebras, and in \cref{sub:Basic Notions} we introduce 
for $R \in \Alg A$ the categories $\FMLOver AR$ of formal $A$-modules over $R$
(with various alterations). We follow this style of notation for almost all categories
that occur, and their meaning should be clear from the context, with one exception:
inspired by \cite{drinfel1974elliptic}, we define $\cC$ as the category of
local, Noetherian, complete $\cO_\br E$-algebras with residue field $\Fqbar$,
and continuous homomorphisms.

\textbf{Cohomology.} 
Throughout the text, we also fix a prime number $\ell \neq p$.
Assume that $X$ is either 
\begin{itemize}
  \item a scheme; separated and of finite type over $\spec(\Fqbar)$, or
  \item an analytic adic space; locally of finite type, separated and taut over
    $\spa(\Cp, \cO_\Cp)$. 
\end{itemize}
Then we denote the \'etale site of $X$ (cf. \cite{LeiFuEtale} for schemes and
\cite{huber2013etale} for analytic adic spaces) by $X_\et$. For an
$\ell$-torsion ring $\Lambda = \Z/\ell^m \Z$ (for some integer $m>0$) we denote
by $\underline \Lambda_X$ the associated constant sheaf on $X_\et$. This gives
rise to the Grothendieck Abelian category $\Mod {\underline \Lambda_X}$ of
$\underline \Lambda_X$-module objects in $\Sh(X_\et)$, whose derived category
we denote by $\derD(X_\et, \underline \Lambda_X)$. 
Denoting the structure morphism of $X$ by $f$, we write 
$\derR \Gamma = \derR f_*$ and $\derR \Gammac = \derR f_!$ (interpreted as functors
$\derD^+(X_\et, \underline \Lambda_X) \to \derD^+\Mod \Lambda$).
For any $\cF \in \derD^+(X_\et, \underline \Lambda_X)$, we denote 
by $\derR^i\Gamma\cF = \hH^i(X, \cF) \in \Mod \Lambda$ (resp. $\derR^i \Gammac \cF = 
\hHc^i(X, \cF) \in \Mod \Lambda$) its $i$-th cohomology group (resp, its
$i$-th cohomology with compact support).
We also need the $\ell$-adic cohomology groups $\hH^i(X, \Qlbar)$ 
and $\hHc^i(X, \Qlbar)$. The classical $\ell$-adic formalism (cf. \cite{LeiFuEtale},
\cite{jannsen1988continuous} for schemes and \cite{huber1998comparison} for
analytic adic spaces) defines these groups via a limiting procedure in the 
\'etale topos. Recent developements by Scholze and Bhatt--Scholze (cf.
\cite{scholze2017etale} and \cite{bhattscholze2013pro}) allow a definition
as honest sheaf cohomology groups over the pro-\'etale site.

\textbf{Formal Composition.}
For a ring $A$ and a formal power series $f(T) = c_1 T + c_2 T^2 + \dots \in
A\llbr T \rrbr$, we write $f^k(T)$ for the $k$-fold self composite of $f$, that is
\begin{equation*}
  f^k(T) = \underbrace{f(f( \cdots (f}_{k-\text{fold}}(T)) \cdots) \quad \text{not to be confused with}
  \quad f(T)^k = \underbrace{f(T) f(T) \cdots f(T)}_{k-\text{fold}}.
\end{equation*}
We call $f^k(T)$ the $k$-fold self-composite of $f$. If $f^{-1}(T) \in A\llbr T \rrbr$
is a power series such that $f(f^{-1}(T)) = f^{-1}(f(T)) = T$, we call
$f^{-1}(T)$ a formal inverse to $f$.


\textbf{Other.}\vspace{-1em}
\begin{itemize}
  \item If $R$ is a local ring, we write $\fm_R$ for its maximal ideal and 
    $k_R = R/\fm_R$ for its residue field. 
  \item If $S$ is a topological ring, we write $S^\circ$ for the set of power-bounded
    elements and $S^\cici$ for the set of topologically nilpotent elements.
  \item If $\cF$ is a set (or a presheaf), and $\sigma\colon \cF \to \cF$ is an
    automorphism, we denote by $\cF^\sigma$ the set (or presheaf) of 
    $\sigma$-invariants.
  \item Starting in \cref{sec:Local Class Field Theory}, we fix a positive
    integer $n \in \N$, along with a division algebra $D$ over $E$ of invariant $1/n$. 
\end{itemize}

% subsection Notation (end)

\subsection*{Acknowledgements} % (fold)
\addcontentsline{toc}{subsection}{Acknowledgements}

% subsection Acknowledgements (end)
% section Introduction (end)
\end{document}
