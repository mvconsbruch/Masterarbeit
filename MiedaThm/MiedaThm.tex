%! TeX root: ../main.tex
\documentclass[../main.tex]{subfiles}

\begin{document}
\section{Mieda's Approach to the Explicit Local Langlands Correspondence} % (fold)
\label{sec:Mieda's Approach to the Explicit Local Langlands Correspondence}

We give a brief summary of the content of results in \cite{mieda2016geometric}.

\subsection{The Specialization Map} % (fold)
\label{sub:The Injectivity FinLevel}
Let $R$ be a complete discrete valuation ring with residue field $\Fqbar$ and
write $K$ for the field of fractions of $R$.
Let $\cY$ be a flat and topologically of finite type formal scheme over $\Spf R$.
Utilizing a classical construction of Raynauld \cite{raynaud1974geometrie}, we 
can consider the corresponding rigid generic fiber $d(\cY)$, which we may
consider as an analytic adic space over $\Spa(K,R)$ by \cite[Section
1.9]{huber2013etale}. We denote by $\cY_s = \cY \times_{\Spf(R)} \spec
\Fqbar$ the special fiber of $\cY$, which is of finite type over $\spec
\Fqbar$. Recall that we may identify the \'etale sites $(\cY_\reduc)_\et$ and
$\cY_{s,\et}$ via the isomorphism induced by the closed immersion of formal schemes
$\cY_s \inj \cY_\reduc$.

By \cite[Lemma 3.5.1]{huber2013etale} we have a morphism of sites
\begin{equation*}
  \lambda_\cY \colon d(\cY)_\et \to (\cY_\reduc)_\et = \cY_{s, \et}.
\end{equation*}
For a positive integer $m$, let us fix the torsion ring $\Lambda = \Z/\ell^m
\Z$ and write $\underline \Lambda_{d(\cY)}$ (respectively $\underline
\Lambda_{\cY_{s}}$)
for the corresponding constant sheaf on $d(\cY)_\et$ (respectively $\cY_{s,\et}$). Then 
pushforward along $\lambda_\cY$ induces a left-exact functor of Grothendieck
Abelian categories
\begin{equation*}
  \lambda_{\cY,*}\colon \Mod {\underline \Lambda_{d(\cY)}} \to \Mod{\underline \Lambda_{\cY_{s}}},
\end{equation*}
which is right-adjoint to the exact pullback functor $\lambda_\cY^*$.
This allows for the following definitions.

\begin{defi}[Formal Nearby Cycle Functor and Specialization
  Map]\label{def:FormalNearbyCycleFunctor}
  We denote by
  \begin{equation*}
    \derR \Psi_\cY\colon \derD (d(\cY)_\et, {\underline \Lambda_{d(\cY)}}) \to
    \derD (\cY_{s, \et},{\underline \Lambda_{\cY_{s}}})
  \end{equation*}
  the right derived functor of $\lambda_{\cY,*}$, which we call the formal nearby
  cycle functor. The unit of the adjunction $\lambda_{\cY}^* \ladj 
  \derR \Psi_\cY$ induces, evaluated at $\underline \Lambda_{\cY_{s}}$, a morphism
  \begin{equation*}
    \specmap^* \colon \underline \Lambda_{\cY_{s}} \to 
    \derR \Psi_\cY( \lambda_{\cY}^* \underline \Lambda_{\cY_{s}}) = 
    \derR \Psi_\cY(\underline \Lambda_{d(\cY)}).
  \end{equation*}
  which we shall call the specialization map.
\end{defi}

Our goal is to compare the compact supported cohomologies of $\cY_s$ and $d(\cY)$.
From classical results (cf. for example \cite[Corollary 0.7.9]{huber2013etale}, which
essentially deals with the case where $\cY$ is algebraizable),
we might hope that under mild assumptions, there is an isomorphism
\begin{equation}\label{eq:NearbyCycleIsomorphism}
  \derR\Gamma_c(\cY_s, \derR\Psi_\cY \underline \Lambda_{d(\cY)}) \xto\sim \derR\Gamma_c
  (d(\cY), \underline \Lambda_{d(\cY)})
\end{equation}
inducing the morphism (which we also call specialization map)
\begin{equation*}
  \specmap^*\colon \derR\Gamma_c(\cY_s, \underline \Lambda_{\cY_s}) \to \derR
  \Gamma_c(d(\cY), \underline \Lambda_{d(\cY)}).
\end{equation*}
And indeed, in \cite[Corollary 4.29]{mieda2014variants}, Mieda constructs an
isomorphism as in \eqref{eq:NearbyCycleIsomorphism}
if $\cY$ is pseudo-compactifiable over $\Spf A$ in the sense of \cite[Definition
4.24]{mieda2014variants}. We do not explain this notion here, but we remark that
all affine formal schemes that are topologically of finite type are
pseudo-compactifiable (cf. \cite[Example 4.25]{mieda2014variants}), which is
the only important case for us.

Let us assume that $d(\cY)$ is smooth over $\Spa(\Cp, \cO_\Cp)$. Then, by
\cite[Theorem 7.3.4]{huber2013etale}, we have the trace map
\begin{equation*}
  \Tr_{d(\cY)} \colon \derR  \Gamma_c( d(\cY), \underline \Lambda_{d(\cY)} ) \to 
  \Lambda (-d)[-2d].
\end{equation*}
Let us write $f \colon \cY_s \to \spec\Fqbar$ for the structure map. Then by the 
adjunction $\derR f_! \ladj f^!$ and the isomorphism in \eqref{eq:NearbyCycleIsomorphism},
we obtain a map
\begin{equation*}
  \cospecmap^* \colon \derR\Psi_\cY \underline\Lambda_{d(\cY)} \to f^!
  \underline\Lambda_{\Spec(\Fqbar)}(-d)[-2d].
\end{equation*}
Together with the specialization map, we obtain a morphism
\begin{equation*}
  \underline \Lambda_{\cY_s} \xto {\specmap^*} \derR\Psi_\cY \underline\Lambda_{d(\cY)} 
  \xto{\cospecmap^*} f^! \underline\Lambda_{\Spec(\Fqbar)}(-d)[-2d].
\end{equation*}

There is another natural map with the same source and target, namely the 
Gysin map, which we define as the map obtained by adjunction from the trace map
\begin{equation*}
  \Tr_f\colon \derR f_! \underline \Lambda_{\cY_s} \to \underline \Lambda_{\spec \Fqbar}(-d)[-2d].
\end{equation*}
As $f$ is not necessarily smooth, this is not
simply the trace map from Poincar\'e duality. The trace map exists
nonetheless (cf. \cite[XVIII, Théorème 2.9]{SGA4}), and we shortly 
recall how to construct it from the usual trace map. We may assume that $\cY_s$
is reduced. Then, by \cite[\href{https://stacks.math.columbia.edu/tag/056V}{Tag
056V}]{stacks-project} 
there exists a dense open subscheme $j \colon U \inj \cY_s$ that is smooth over 
$\spec(\Fqbar)$, and we denote by $i$ the closed complement 
$i \colon Z \inj \cY_s$. Any component of $Z$ has dimension $< d$, implying that
$\hH^{2d}_c(Z, \underline \Lambda_U) = 0$ by \cite[Theorem 7.4.5]{LeiFuEtale}.
From the excision exact triangle (cf. \cite[Theorem 7.4.4 (iii)]{LeiFuEtale}),
we obtain a natural isomorphism
\begin{equation*}
  \derR^{2d} (fj)_! \underline \Lambda_U \xto \sim \derR^{2d} f_! \underline
  \Lambda_{\cY_s}.
\end{equation*}
As $U$ is smooth over $\Spec(\Fqbar)$, there is the usual trace morphism from
Poincar\'e duality
\begin{equation*}
  \Tr_{fj}\colon \derR^{2d}(fj)_! \underline \Lambda_U \to \underline
  \Lambda_{\spec(\Fqbar)}(-d),
\end{equation*}
constructed, for example, in \cite[Section 8.2]{LeiFuEtale}. By the isomorphism
above, this gives a map $\derR^{2d} f_! \underline \Lambda_{\cY_s} \to \underline
\Lambda_{\spec(\Fqbar)}(-d)$, and as $\derR f_! \underline \Lambda_{\cY_s}$ is
(up to quasi-isomorphism) concentrated in degree $\leq 2d$, this suffices to
construct the desired map. As the trace map commutes with basechange, this construction
does not depend on the choice of $U$.

With the trace map defined, we can formulate Mieda's main result about the
specializazion map.
\begin{prop}[Mieda's theorem about the Specialization Map]\label{prop:MiedaFirstStepSpecMap}
  The composite 
  \begin{equation*}
    \underline{\Lambda}_{\cY_s} \xto {\specmap^*} \derR \Psi_\cY \underline
    \Lambda_{d(\cY)} \xto{\cospecmap^*} f^! \underline
    \Lambda_{\spec(\Fqbar)}(-d)[-2d]
  \end{equation*}
  is equal to the Gysin map with respect to $f$, that is, the map
  \begin{equation*}
    \Gys_f \colon \underline \Lambda_{\cY_s} \to f^! \underline
    \Lambda_{\spec(\Fqbar)}(-d)[-2d]
  \end{equation*}
  obtained by the trace map from the adjunction $\derR f_! \ladj f^!$.
  \begin{proof}
    This is \cite[Theorem 2.1]{mieda2016geometric}.
  \end{proof}
\end{prop}

Now, let $X$ be a purely $d$-dimensional separated smooth scheme of finite type
over $\spec(\Fqbar)$, with structure map denoted by $g$. Assume that there is a
finite surjective morphism
\begin{equation*}
  \pi: \cY_s \to X.
\end{equation*}
As a corollary to the theorem above, Mieda constructs a commutative diagram
\begin{equation}\label{eq:MiedasCommutingDiagram}
\begin{tikzcd}[ampersand replacement=\&]
  {\hH_c^i(X, \underline \Lambda_X)} \& {\hH^i_c(\cY_s, \underline
  \Lambda_{\cY_s})} \& {\hH^i_c(\cY_s, \derR \Psi_\cY \underline
    \Lambda_{d(\cY)})} \\
  {\hH^i_c(X, \underline \Lambda_{X})} \& {\hH^i_c(\cY_s, f^! \underline
  \Lambda_{\spec(\Fqbar)}(-d)[-2d])} \& {\hH^i_c(\cY_s, \derR \Psi_\cY
  \underline \Lambda_{d(\cY)})} \\
  {\hH^i(X, \underline \Lambda_{X})} \& {\hH^i(\cY_s, f^! \underline
  \Lambda_{\spec(\Fqbar)}(-d)[-2d])} \& {\hH^i(\cY_s, \derR \Psi_\cY \underline
  \Lambda_{d(\cY)}),}
	\arrow["{\pi^*}", from=1-1, to=1-2]
	\arrow["{\times \deg\pi}"', from=1-1, to=2-1]
	\arrow["{\specmap^*}", from=1-2, to=1-3]
	\arrow["{\Gys_f}"', from=1-2, to=2-2]
	\arrow[Rightarrow, no head, from=1-3, to=2-3]
	\arrow[from=2-1, to=3-1]
	\arrow["{\pi_*}"', from=2-2, to=2-1]
	\arrow[from=2-2, to=3-2]
	\arrow["{\cospecmap^*}"', from=2-3, to=2-2]
	\arrow[from=2-3, to=3-3]
	\arrow["{\pi_*}"', from=3-2, to=3-1]
	\arrow["{\cospecmap^*}"', from=3-3, to=3-2]
\end{tikzcd}
\end{equation}
cf. \cite[Corollary 2.7]{mieda2016geometric}.
In this diagram, the maps $\pi_*$ and $\times \deg \pi$ are given as follows.
\begin{itemize}
  \item For the map $\pi_*$, note that, as $g$ is smooth, the Gysin map 
    \begin{equation*}
    \Gys_g \colon \underline \Lambda_{X} \to g^! \underline
    \Lambda_{\spec(\Fqbar)}(-d)[-2d] 
    \end{equation*}
    is an isomorphism. 
    Furthermore, as $\pi$ is finite, we have $\derR \pi_* = \pi_* =
    \derR \pi_!$. This allows us to define $\pi_*$ as the map induced by the
    composition
    \begin{equation*}
      \pi_* f^! \underline \Lambda_{\spec(\Fqbar)}(-d)[-2d]
      \cong \derR\pi_! \pi^! g^! \underline \Lambda_{\spec(\Fqbar)}(-d)[-2d] 
      \xto \sim \derR \pi_! \pi^! \underline \Lambda_{X}
      \to \underline \Lambda_{X},
    \end{equation*}
    where the last map is the counit of the adjunction $\derR \pi_! \ladj \pi^!$
    evaluated at $\underline \Lambda_{X}$.
  \item For the map $\times \deg \pi$, decompose $X$ into connected components
    $X = \coprod_j X_j$ and write $\delta_j \in \Z$ for the generic degree of 
    $\pi$ over $X_j$. Then $\times \deg\pi$ is the map obtained from the morphism
    of \'etale sheaves $\underline \Lambda_X \to \underline \Lambda_X$ admitting
    the local description
    \begin{equation*}
      \underline \Lambda_{X}(X_j) \xto{\times \delta_j} \underline \Lambda_{X}(X_j).
    \end{equation*}
\end{itemize}
Commutativity of the square in the upper right hand side is the content of
Theorem \ref{prop:MiedaFirstStepSpecMap}, and commutativity of the square in the
upper left is essentially a consequence of the naturality of the trace map.
Furthermore, Mieda shows that the square
\begin{equation*}
\begin{tikzcd}[ampersand replacement=\&]
	{\hH_c^i(\cY_s, \derR \Psi_\cY \underline \Lambda_{d(\cY)})} \& 
  {\hH_c^i(d(\cY), \underline \Lambda_{d(\cY)})} \\
	{\hH^i(\cY_s, \derR \Psi_\cY \underline \Lambda_{d(\cY)})} \& 
  {\hH^i(d(\cY), \underline \Lambda_{d(\cY)})} 
	\arrow["\sim", from=1-1, to=1-2]
	\arrow[from=1-1, to=2-1]
	\arrow[from=1-2, to=2-2]
	\arrow[Rightarrow, no head, from=2-1, to=2-2]
\end{tikzcd}
\end{equation*}
is commutative. The following statement is a direct consequence.

\begin{prop}\label{thm:SubspaceInjectionFiniteLevel}
  Let $X$ and $\cY$ be as above, i.e., let $X$ be a purely $d$-dimensional
  separated smooth scheme of finite type over $\Fqbar$, and let 
  $\cY$ be a flat and topologically of finite type formal scheme over $R$ with 
  generic fiber $d(\cY)$ smooth over $\spa(K,R)$. Furthermore, assume that 
  $Z$ is an adic space that is locally of finite type, separated and taut
  over $\spa(K,R)$, and suppose that there is an open immersion
  $d(\cY) \inj Z$.

  Then, the map 
  \begin{equation*}
    \hH_c^i(X, \underline \Lambda_X) \xto {\times \deg \pi} 
    \hH_c^i(X, \underline \Lambda_X) \to 
    \hH^i(X,   \underline \Lambda_X)
  \end{equation*}
  factors through the composition
  \begin{equation*}
    \hH_c^i(X,      \underline \Lambda_X) \xto {\pi^*}
    \hH_c^i(\cY_s,  \underline \Lambda_{\cY_s}) \xto {\specmap^*}
    \hH_c^i(d(\cY), \underline \Lambda_{d(\cY)}) \to 
    \hH_c^i(Z,      \underline \Lambda_Z).
  \end{equation*}
\begin{proof}
  By construction of the lower shriek for taut morphisms (cf. \cite[Section 0.4
  D)]{huber2013etale}), formation of the lower shriek functor is covariant with
  respect open embeddings, and we have the commutative square
  \begin{equation*}
    \begin{tikzcd}[ampersand replacement=\&]
      {\hH^i_c(d(\cY), \underline \Lambda_{d(\cY)})} \& 
      {\hH^i_c(Z, \underline \Lambda_{Z})} \\
    	{\hH^i(d(\cY), \underline \Lambda_{d(\cY)})} \& 
      {\hH^i(Z, \underline \Lambda_{Z})}.
    	\arrow[from=1-1, to=1-2]
    	\arrow[from=1-1, to=2-1]
    	\arrow[from=1-2, to=2-2]
    	\arrow[from=2-2, to=2-1]
    \end{tikzcd}
  \end{equation*}
  The claim now essentially follows from gluing together all the commutative
  diagrams above. 
\end{proof}
\end{prop}

Passing from torsion cohomology to $\ell$-adic cohomology, we quickly arrive
the following result.

\begin{thm}[Mieda's Injectivity Result at Finite
  Level]\label{thm:MiedaInjectivityAtFiniteLevel}
  Let $X$ and $Z$ be as in Proposition \ref{thm:SubspaceInjectionFiniteLevel}.
  If $V \subseteq \hH^i_c(X, \Qlbar)$ is a subspace such that the 
  composition $V \inj \hH^i_c(X, \Qlbar) \to \hH^i(X, \Qlbar)$ is injective,
  the map 
  \begin{equation*}
    V \inj \hH^i_c(X, \Qlbar) \xto {\specmap^* \circ \pi^*}
    \hH^i_c(d(\cY), \Qlbar) \to \hH^i_c(Z, \Qlbar)
  \end{equation*}
  is injective as well.
\begin{proof}
  Up to concerns about the formation of $\ell$-adic cohomology, this is a
  direct consequence of Proposition \ref{thm:SubspaceInjectionFiniteLevel}
  above. We refer to \cite[Theorem 2.8]{mieda2016geometric} for details.
\end{proof}
\end{thm}




% subsection The Injectivity Result (end)

\subsection{Application to the Lubin--Tate Tower} % (fold)
\label{sub:Application to the Lubin--Tate Tower}

We now explain how the results of the previous subsection apply to the 
cohomology of the Lubin--Tate tower.

Suppose we are given a rational subset $U$ of the Lubin--Tate perfectoid space
$M_{\infty, \cO_\Cp}^{(0)}$ and a $\varpi$-adically complete, flat $\cO_\Cp$-algebra
$A$ such that $\cX = \spf(A)$ is a formal model of $U$. 

For our purposes the following results, taken from \cite[Corollary
4.6]{mieda2016geometric}, suffice.
\begin{thm}[Mieda's Result for the Lubin--Tate Tower]\label{thm:MiedaAppliedToLTT}
  Let $J$ be a subgroup of $G^1$ whose action on $M_{\infty, \Cp}^{(0)}$ 
  stabilizes $U$ and extends to an action on $\cY$. Assume that there exists an
  affine scheme $Y$ of finite type over $\Fqbar$ equipped with a right action of 
  $J$ such that there is an isomorphism
  \begin{equation*}
    \cY_s = \spec(A \otimes_{R} \Fqbar) \xto \sim Y^\perf
  \end{equation*}
  of schemes over $\spec \Fqbar$, equivariant for the action of $J$ on both sides.
  \begin{enumerate}
    \item There is a $J$-equivariant homomorphism
      \begin{equation*}
        \specmap^*\colon \hH_c^{n-1}(Y, \bar \Q_\ell) \to \colim_K \hH_c^{n-1}
        (M_{K, E}^{(0)}, \bar \Q_\ell) \eqcolon \HLT'.
      \end{equation*}
    \item If $Y$ is pure-dimensional and smooth over $\spec(\Fqbar)$
      and $V$ is a subspace of $\hH^{n-1}_c(Y, \bar \Q_\ell)$ such that the
      composite 
      $$V \inj H^{n-1}_c(Y, \bar \Q_\ell) \inj \hH^{n-1}_c(Y, \bar \Q_\ell)$$
      is injective, the composite
      \begin{equation*}
        V \inj \hH^{n-1}_c(Y, \bar \Q_\ell) \xto {\specmap^*} \HLT'
      \end{equation*}
      is injective as well.
  \end{enumerate}
\end{thm}

% subsubsection Injectivity of the Specialization Map at Infinite Level (end)

% section Mieda's Approach to the Explicit Local Langlands Correspondence (end)
\end{document}
