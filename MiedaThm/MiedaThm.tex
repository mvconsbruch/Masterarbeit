%! TeX root: ../main.tex
\documentclass[../main.tex]{subfiles}

\begin{document}
\section{Mieda's Approach to the Explicit Local Langlands Correspondence} % (fold)
\label{sec:Mieda's Approach to the Explicit Local Langlands Correspondence}

We give a brief summary of the content of results in \cite{mieda2016geometric}.

\subsection{The Specialization Map} % (fold)
\label{sub:The Injectivity FinLevel}
Let $\cY$ be an affine, flat and topologically of finite type formal scheme over $\Spf(\cO_\Cp)$.
By \cite{raynaud1974geometrie} and \cite[§1.9]{huber2013etale},
we
have the corresponding rigid generic fiber $d(\cY)$, which we may
consider as an analytic adic space over $\Spa(\Cp,\cO_\Cp)$. We denote by $\cY_s = \cY \times_{\Spf(\cO_\Cp)} \spec
(\Fqbar)$ the special fiber of $\cY$, which is a finite-type scheme over $\spec(
\Fqbar)$. By the discussion at the beginning of \cite[§3.5]{huber2013etale}, we have a morphism of sites
\begin{equation*}
  \lambda_\cY \colon d(\cY)_\et \to (\cY_\reduc)_\et \cong \cY_{s, \et}.
\end{equation*}
For a positive integer $m$, let us fix the torsion ring $\Lambda = \Z/\ell^m
\Z$ and write $\underline \Lambda_{d(\cY)}$ (respectively, $\underline
\Lambda_{\cY_{s}}$)
for the corresponding constant sheaf on $d(\cY)_\et$ (respectively, $\cY_{s,\et}$). Then 
push-forward along $\lambda_\cY$ induces a left exact functor of Grothendieck
Abelian categories
\begin{equation*}
  \lambda_{\cY,*}\colon \Sh(d(\cY)_\et, {\underline \Lambda_{d(\cY)}}) \to
  \Sh(\cY_{s,\et}, {\underline \Lambda_{\cY_{s}}}),
\end{equation*}
which is right-adjoint to the exact pullback functor $\lambda_\cY^*$.
This allows for the following definitions.

\begin{defi}[Formal Nearby Cycle Functor and Specialization
  Map]\label{def:FormalNearbyCycleFunctor}
  We denote by
  \begin{equation*}
    \derR \Psi_\cY\colon \derD^+ (d(\cY)_\et, {\underline \Lambda_{d(\cY)}}) \to
    \derD ^+ (\cY_{s, \et},{\underline \Lambda_{\cY_{s}}})
  \end{equation*}
  the right derived functor of $\lambda_{\cY,*}$, which we call the formal \emph{nearby
  cycle functor}. The unit of the adjunction $\lambda_{\cY}^* \ladj 
  \derR \Psi_\cY$ induces, evaluated at $\underline \Lambda_{\cY_{s}}$, a morphism
  \begin{equation*}
    \specmap^* \colon \underline \Lambda_{\cY_{s}} \to 
    \derR \Psi_\cY( \lambda_{\cY}^* \underline \Lambda_{\cY_{s}}) = 
    \derR \Psi_\cY(\underline \Lambda_{d(\cY)}).
  \end{equation*}
  which we shall call \emph{specialization map}.
\end{defi}

By \cite[Theorem 5.7.6]{huber2013etale}, there is a canonical isomorphism
\begin{equation}\label{eq:NearbyCycleIsomorphism}
  \derR\Gamma_c(\cY_s, \derR\Psi_\cY \underline \Lambda_{d(\cY)}) \xto\sim \derR\Gamma_c
  (d(\cY), \underline \Lambda_{d(\cY)}).
\end{equation}
Upon choosing a compactification $j \colon \cY \inj \bar \cY$, 
this morphism is essentially constructed by adjunction from the map
\begin{equation*}
    \lambda^*_{\sbar \cY} d(j)_! \derR \Psi_\cY
    \to j_! \lambda_\cY^* \derR \Psi_\cY
    \to j_!.
\end{equation*}

Thus, the formal nearby cycle functor provides sheaves on $\cY_s$ that allow us to calculate the cohomology of torsion sheaves on $d(\cY)$. We obtain the morphism (which we also call specialization map)
\begin{equation*}
  \specmap^*\colon \derR\Gamma_c(\cY_s, \underline \Lambda_{\cY_s}) \to \derR
  \Gamma_c(d(\cY), \underline \Lambda_{d(\cY)}).
\end{equation*}

Let us assume that $d(\cY)$ is smooth over $\Spa(\Cp, \cO_\Cp)$. Then, by
\cite[Theorem 7.3.4]{huber2013etale}, we have the trace map
\begin{equation*}
  \Tr_{d(\cY)} \colon \derR  \Gamma_c( d(\cY), \underline \Lambda_{d(\cY)} ) \to 
  \Lambda (-d)[-2d].
\end{equation*}
Let us write $f \colon \cY_s \to \spec(\Fqbar)$ for the structure morphism. Then by the
adjunction $\derR f_! \ladj f^!$ and the isomorphism in \eqref{eq:NearbyCycleIsomorphism},
we obtain a map
\begin{equation*}
  \cospecmap^* \colon \derR\Psi_\cY \underline\Lambda_{d(\cY)} \to f^!
  \underline\Lambda_{\Spec(\Fqbar)}(-d)[-2d].
\end{equation*}
Together with the specialization map, we obtain a morphism
\begin{equation*}
  \underline \Lambda_{\cY_s} \xto {\specmap^*} \derR\Psi_\cY \underline\Lambda_{d(\cY)} 
  \xto{\cospecmap^*} f^! \underline\Lambda_{\Spec(\Fqbar)}(-d)[-2d].
\end{equation*}

There is another natural map with the same source and target, namely the 
\emph{Gysin map}, which we define as the map obtained by adjunction from the trace map
\begin{equation*}
  \Tr_f\colon \derR f_! \underline \Lambda_{\cY_s} \to \underline
  \Lambda_{\spec(\Fqbar)}(-d)[-2d].
\end{equation*}
As $f$ is not necessarily smooth, this is not
simply the trace map of Poincar'e duality. However, the trace map exists (cf. \cite[XVIII, Théorème 2.9]{SGA4}), and we briefly
recall how to construct it from the usual trace map. We may assume that $\cY_s$
is reduced. Then, by \cite[\href{https://stacks.math.columbia.edu/tag/056V}{Tag
056V}]{stacks-project} 
there exists a dense open subscheme $j \colon U \inj \cY_s$ that is smooth over 
$\spec(\Fqbar)$, and we denote by $i$ the closed complement 
$i \colon Z \inj \cY_s$. Any component of $Z$ has dimension $< d$, which implies that
$\hH^{2d}_c(Z, \underline \Lambda_U) = 0$ by \cite[Theorem 7.4.5]{LeiFuEtale}.
From the excision exact triangle (cf. \cite[Theorem 7.4.4 (iii)]{LeiFuEtale}),
we obtain a natural isomorphism
\begin{equation*}
  \derR^{2d} (fj)_! \underline \Lambda_U \xto \sim \derR^{2d} f_! \underline
  \Lambda_{\cY_s}.
\end{equation*}
As $U$ is smooth over $\Spec(\Fqbar)$, there is the usual trace morphism from
Poincar\'e duality
\begin{equation*}
  \Tr_{fj}\colon \derR^{2d}(fj)_! \underline \Lambda_U \to \underline
  \Lambda_{\spec(\Fqbar)}(-d),
\end{equation*}
constructed, for example, in \cite[§8.2]{LeiFuEtale}. By the isomorphism
above, this gives a map $\derR^{2d} f_! \underline \Lambda_{\cY_s} \to \underline
\Lambda_{\spec(\Fqbar)}(-d)$, and as $\derR f_! \underline \Lambda_{\cY_s}$ is
(up to quasi-isomorphism) concentrated in degree $\leq 2d$, this suffices to
construct the desired map. As the trace map commutes with base change, this construction
does not depend on the choice of $U$.

With the trace map defined, we can formulate Mieda's main result about the
specialization map.
\begin{prop}[Mieda's Result about the Specialization Map]\label{prop:MiedaFirstStepSpecMap}
  The composite 
  \begin{equation*}
    \underline{\Lambda}_{\cY_s} \xto {\specmap^*} \derR \Psi_\cY \underline
    \Lambda_{d(\cY)} \xto{\cospecmap^*} f^! \underline
    \Lambda_{\spec(\Fqbar)}(-d)[-2d]
  \end{equation*}
  is equal to the Gysin map with respect to $f$, that is, the map
  \begin{equation*}
    \Gys_f \colon \underline \Lambda_{\cY_s} \to f^! \underline
    \Lambda_{\spec(\Fqbar)}(-d)[-2d]
  \end{equation*}
  obtained by the trace map from the adjunction $\derR f_! \ladj f^!$.
  \begin{proof}
    This is \cite[Theorem 2.1]{mieda2016geometric}.
  \end{proof}
\end{prop}

In practice, $\cY$ will serve as a “finite level version”
of certain affinoids constructed inside the Lubin--Tate
perfectoid space, which in our application admit surjections
to certain smooth varieties over $\Fqbar$. Let $X$ be a purely $d$-dimensional separated smooth scheme of finite type
over $\spec(\Fqbar)$, with structure morphism denoted by $g$. Assume that there is a finite surjective morphism
\begin{equation*}
  \pi\colon \cY_s \to X.
\end{equation*}
As a corollary to the theorem above, Mieda constructs a commutative diagram
\begin{equation}\label{eq:MiedasCommutingDiagram}
\begin{tikzcd}[ampersand replacement=\&]
  {\hHc^i(X, \underline \Lambda_X)} \& {\hHc^i(\cY_s, \underline
  \Lambda_{\cY_s})} \& {\hHc^i(\cY_s, \derR \Psi_\cY \underline
    \Lambda_{d(\cY)})} \\
  {\hHc^i(X, \underline \Lambda_{X})} \& {\hHc^i(\cY_s, f^! \underline
  \Lambda_{\spec(\Fqbar)}(-d)[-2d])} \& {\hHc^i(\cY_s, \derR \Psi_\cY
  \underline \Lambda_{d(\cY)})} \\
  {\hH^i(X, \underline \Lambda_{X})} \& {\hH^i(\cY_s, f^! \underline
  \Lambda_{\spec(\Fqbar)}(-d)[-2d])} \& {\hH^i(\cY_s, \derR \Psi_\cY \underline
  \Lambda_{d(\cY)}),}
	\arrow["{\pi^*}", from=1-1, to=1-2]
	\arrow["{\times \deg\pi}"', from=1-1, to=2-1]
	\arrow["{\specmap^*}", from=1-2, to=1-3]
	\arrow["{\Gys_f}"', from=1-2, to=2-2]
	\arrow[Rightarrow, no head, from=1-3, to=2-3]
	\arrow[from=2-1, to=3-1]
	\arrow["{\pi_*}"', from=2-2, to=2-1]
	\arrow[from=2-2, to=3-2]
	\arrow["{\cospecmap^*}"', from=2-3, to=2-2]
	\arrow[from=2-3, to=3-3]
	\arrow["{\pi_*}"', from=3-2, to=3-1]
	\arrow["{\cospecmap^*}"', from=3-3, to=3-2]
\end{tikzcd}
\end{equation}
cf. \cite[Corollary 2.7]{mieda2016geometric}.
In this diagram, the maps $\pi_*$ and $\times \deg \pi$ are given as follows.
\begin{itemize}
  \item For the map $\pi_*$, note that, as the structure map $g\colon X \to \spec(\Fqbar)$ is smooth, the Gysin map 
    \begin{equation*}
    \Gys_g \colon \underline \Lambda_{X} \to g^! \underline
    \Lambda_{\spec(\Fqbar)}(-d)[-2d] 
    \end{equation*}
    is an isomorphism. 
    Furthermore, as $\pi$ is finite, we have $\derR \pi_* = \pi_* =
    \derR \pi_!$. This allows us to define $\pi_*$ as the map induced by the
    composition
    \begin{equation*}
      \pi_* f^! \underline \Lambda_{\spec(\Fqbar)}(-d)[-2d]
      \cong \derR\pi_! \pi^! g^! \underline \Lambda_{\spec(\Fqbar)}(-d)[-2d] 
      \xto \sim \derR \pi_! \pi^! \underline \Lambda_{X}
      \to \underline \Lambda_{X},
    \end{equation*}
    where the last map is the counit of the adjunction $\derR \pi_! \ladj \pi^!$
    evaluated at $\underline \Lambda_{X}$.
  \item For the map $\times \deg \pi$, decompose $X$ into connected components
    $X = \coprod_j X_j$ and write $\delta_j \in \Z$ for the generic degree of 
    $\pi$ over $X_j$. Then $\times \deg\pi$ is the map obtained from the morphism
    of \'etale sheaves $\underline \Lambda_X \to \underline \Lambda_X$ admitting
    the local description
    \begin{equation*}
      \underline \Lambda_{X}(X_j) \xto{\times \delta_j} \underline \Lambda_{X}(X_j).
    \end{equation*}
\end{itemize}
Commutativity of the square in the upper right hand side is the content of
\cref{prop:MiedaFirstStepSpecMap}, and commutativity of the square in the
upper left is essentially a consequence of the naturality of the trace map. The
lower two squares commute by definition. Furthermore, the square
\begin{equation*}
\begin{tikzcd}[ampersand replacement=\&]
	{\hHc^i(\cY_s, \derR \Psi_\cY \underline \Lambda_{d(\cY)})} \& 
  {\hHc^i(d(\cY), \underline \Lambda_{d(\cY)})} \\
	{\hH^i(\cY_s, \derR \Psi_\cY \underline \Lambda_{d(\cY)})} \& 
  {\hH^i(d(\cY), \underline \Lambda_{d(\cY)})} 
	\arrow["\sim", from=1-1, to=1-2]
	\arrow[from=1-1, to=2-1]
	\arrow[from=1-2, to=2-2]
	\arrow[Rightarrow, no head, from=2-1, to=2-2]
\end{tikzcd}
\end{equation*}
is commutative. The following statement is a direct consequence.

\begin{prop}\label{thm:SubspaceInjectionFiniteLevel}
  Let $X$ and $\cY$ be as above, that is, let $X$ be a purely $d$-dimensional
  separated smooth scheme of finite type over $\Fqbar$, and let $\cY$ be a flat and topologically finite-type formal scheme over $\Spf(\cO_\Cp)$ with generic fiber $d(\cY)$ smooth over $\spa(\Cp,\cO_\Cp)$.
  Let $\pi\colon \cY_s \to X$ be a finite surjective morphism.
  Furthermore, assume that there is an open immersion
  $d(\cY) \inj Z$, where
  $Z$ is an adic space that is locally of finite type, separated, and taut over $\spa(\Cp, \cO_\Cp)$.
  Then, the map 
  \begin{equation*}
    \hHc^i(X, \underline \Lambda_X) \xto {\times \deg \pi} 
    \hHc^i(X, \underline \Lambda_X) \to 
    \hH^i(X,   \underline \Lambda_X)
  \end{equation*}
  factors through the composition
  \begin{equation*}
    \hHc^i(X,      \underline \Lambda_X) \xto {\pi^*}
    \hHc^i(\cY_s,  \underline \Lambda_{\cY_s}) \xto {\specmap^*}
    \hHc^i(d(\cY), \underline \Lambda_{d(\cY)}) \to 
    \hHc^i(Z,      \underline \Lambda_Z).
  \end{equation*}
\begin{proof}[Sketch of Proof]
  By construction of the lower shriek for taut morphisms (cf. \cite[§0.4
  D)]{huber2013etale}), formation of the lower shriek functor is covariant with
  respect open embeddings, and we have the commutative square
  \begin{equation*}
    \begin{tikzcd}[ampersand replacement=\&]
      {\hHc^i(d(\cY), \underline \Lambda_{d(\cY)})} \& 
      {\hHc^i(Z, \underline \Lambda_{Z})} \\
    	{\hH^i(d(\cY), \underline \Lambda_{d(\cY)})} \& 
      {\hH^i(Z, \underline \Lambda_{Z})}.
    	\arrow[from=1-1, to=1-2]
    	\arrow[from=1-1, to=2-1]
    	\arrow[from=1-2, to=2-2]
    	\arrow[from=2-2, to=2-1]
    \end{tikzcd}
  \end{equation*}
  The claim now essentially follows from gluing together all the commutative diagrams above. 
\end{proof}
\end{prop}

Passing from torsion cohomology to $\ell$-adic cohomology, we quickly arrive at the following result.

\begin{thm}[Mieda's Injectivity Result at Finite
  Level]\label{thm:MiedaInjectivityAtFiniteLevel}
  Let $X$ and $Z$ be as in Proposition \ref{thm:SubspaceInjectionFiniteLevel}.
  If $V \subseteq \hHc^i(X, \Qlbar)$ is a subspace such that the 
  composition $V \inj \hHc^i(X, \Qlbar) \to \hH^i(X, \Qlbar)$ is injective,
  the map 
  \begin{equation*}
    V \inj \hHc^i(X, \Qlbar) \xto {\specmap^* \circ \pi^*}
    \hHc^i(d(\cY), \Qlbar) \to \hHc^i(Z, \Qlbar)
  \end{equation*}
  is also injective.
  \begin{proof}[Sketch of Proof] 
  We have 
  \begin{equation*}
    \hHc^i(X, \Z_\ell) = \lim_{m \in \N} \hHc^i(X, \underline{\Z/\ell^m\Z}) 
    \quad \text{and} \quad \hHc^i(d(\cY), \Z_\ell) = \lim_{m \in \N}
    \hHc^i(d(\cY), \underline{\Z/\ell^m\Z})
  \end{equation*}
  by \cite[Corollary 9.3.3 and Proposition 10.1.15]{LeiFuEtale} and 
  \cite[Theorem 3.1]{huber1998comparison}. The corresponding statement for
$Z$ is in general wrong. In the limit over $m \in \N$,
  \cref{thm:SubspaceInjectionFiniteLevel} thus yields that the map
  \begin{equation*}
    \hHc^i(X, \Qlbar) \xto{\times \deg \pi} \hHc^i(X, \Qlbar) \to \hH^i(X, \Qlbar)
  \end{equation*}
  factors over the composition
  \begin{equation*}
    \hHc^i(X,      \Qlbar) \xto {\pi^*}
    \hHc^i(\cY_s,  \Qlbar) \xto {\specmap^*}
    \hHc^i(d(\cY), \Qlbar) \to 
    \left(\lim_{m \in \N} \hHc^i(Z, \underline{\Z/\ell^m\Z}_Z) \right) \otimes_{\Z_\ell}
    \Qlbar.
  \end{equation*}
  The right-most map above factors over the natural homomorphism
  $$\hHc^i(Z, \Qlbar) \to \left(\lim_{m \in \N} \hHc^i(Z,
  \underline{\Z/\ell^m\Z}_Z) \right) \otimes_{\Z_\ell} \Qlbar.$$ 
  The claim follows.
\end{proof}
\end{thm}

% subsection The Injectivity Result (end)

\subsection{Application to the Lubin--Tate Tower} % (fold)
\label{sub:Application to the Lubin--Tate Tower}

We now explain how the results of the previous subsection apply to the 
cohomology of the Lubin--Tate tower. In order to set the stage, recall the 
definition of the Lubin--Tate perfectoid space $M'_{\infty, \Cp}$ and its
right action by the group $G^1$, as defined in the previous two sections.
Assume that we are in the following situation.

\begin{sit}\label{sit:MiedaLTTSit}
  We are given
\begin{itemize}
  \item a rational subset $U\subset M'_{\infty, \Cp}$ and an open
    subgroup $K_U \subset \SL_n(\cO_E)$ stabilizing $U$,
  \item a $\varpi$-adically complete, flat $\cO_\Cp$-algebra
    $Q$ such that $\spf(Q) \eqqcolon \cX$ is a formal model of $U$ and the action of 
    $K_U$ extends to $\cX$ (which is to say, the induced action of $K_U$ on $Q[\tfrac 1\varpi]$ stabilizes $Q$),
  \item an affine scheme $Y=\spec(R)$, pure-dimensional, smooth and 
    of finite type over $\Fqbar$,
  \item an isomorphism
    \begin{equation}\label{eq:isoofreductionwithperfection}
      R^\perf \coloneq \colim_{x \mapsto x^p} R \cong Q \otimes \Fqbar,
    \end{equation}
  \item a subgroup $J^1$ of $G^1$ whose action on $M'_{\infty, \Cp}$ stabilizes
    $U$ and extends to $\cX \coloneq \spf(Q)$, 
  \item and a left action of $J^1$ on $R$ such that the isomorphism
    \eqref{eq:isoofreductionwithperfection} is compatible with respect to the
    induced $J^1$-actions on both sides.
\end{itemize}
\end{sit}

We want to use the ideas of \cref{sub:The Injectivity FinLevel} to compare the
$J^1$-representations
$\hHc^{n-1}(Y, \Qlbar)$ and $\colim_K \hHc^{n-1}(M'_{K,\Cp}, \Qlbar) = \hH'_\LT$. 
We will need to construct finite level versions of the rational subset $U$. 
For $m \in \N$, recall the definition of the $m$-th congruence subgroup $K_m$ from 
\eqref{eq:defCongruenceSubgroup} and denote $K'_m \coloneq K_m \cap \SL_n(\cO_E)$. 
The following is part of \cite[Proposition 4.5]{mieda2016geometric}.

\begin{prop}\label{prop:FiniteLevelU}
  There exists an integer $m \geq 0$ such that $K'_m \subset K_U$ and an open affinoid
  $U_m$ in $M'_{m, \Cp}$ such that the inverse image of $U_m$ under the projection
  $M'_{\infty, \Cp} \to M'_{m, \Cp}$ is equal to $U$. 
\end{prop}
With $m$ as in the proposition above, we write $K_0 = K'_m$ and $U_{K_0} = 
U_m$. For an open normal subgroup $K \subseteq K_0$, we write $U_K = \spa(B_K,
B^\circ_K) \subset M'_{K, \Cp}$ for the preimage of $U_{K_0}$ under 
$M_{K,\Cp}' \to M'_{K_0, \Cp}$. For any such $K$, we find
that $B_K$ is a complete Huber $\Cp$-algebra, 
topologically of finite type and endowed with an action of $K_0/K$. Furthermore,
for $K' \subseteq K$ we have $K_0$-equivariant transition maps $B_{K} \to B_{K'}$,
and $\{B_K\}_{K \subseteq K_0}$ becomes an inductive system (with $K$ running over 
open normal subgroups). For each such $K$, we have $K_0$-equivariant morphisms
of Huber pairs $(B_K, B_K^\circ) \to (Q[\tfrac 1\varpi], \tilde Q)$.

It can be shown that the morphism $B_K \to Q[\tfrac 1\varpi]$ is injective
(cf. \cite[Lemma 3.4]{mieda2016geometric}), and we define $B' =
\bigcup_{K\subseteq K_0} B_K \subseteq Q[\tfrac 1\varpi]$ as the union
of the images of $\{B_K\}_K$ inside $Q[\tfrac 1\varpi]$. Now $B'$ 
is a dense, $K_0$-stable subgalgebra of $Q[\tfrac 1\varpi]$, cf.
\cite[Proposition 4.5]{mieda2016geometric}. We furthermore define
$Q' = Q \cap B'$ and $Q_K = (Q')^K \subset Q$ for any open normal subgroup $K
\subseteq K_0$. 

\begin{lem}\label{lem:BKisIntegralOverQK}
  Inside $Q[\tfrac 1\varpi]$, the algebra $Q_K$ is a subset of $B_K^\circ$,
  and $B_K^\circ$ is integral over $Q_K$. 
\begin{proof}
  This follows from Definition 3.5, Lemma 3.7 and Lemma 3.8 of \cite{mieda2016geometric}. 
\end{proof}
\end{lem}

By \cite[Lemma 3.16]{mieda2016geometric}, it is possible to find a normal
subgroup $K \subseteq K_0$,
an $\cO_\Cp$-subalgebra $Q_0$ which is a ring of definition for $B_K$ and topologically finitely generated (that is,
admitting a surjection from $\cO_\Cp \langle T_1, \dots, T_r \rangle$ for some $r
\in \N$), and a finite surjective morphism $\pi\colon \spec(Q_0 \otimes \Fqbar)
\to \spec(R) = Y$ that fits into a commutative square
\begin{equation}\label{eq:PiFitsDiag}
\begin{tikzcd}[ampersand replacement=\&]
  {\spec(Q \otimes_{\cO_\Cp} \Fqbar)} \& {\spec(Q_0 \otimes_{\cO_\Cp}\Fqbar)} \\
	{Y^\perf} \& {Y.}
	\arrow[from=1-1, to=1-2]
	\arrow["\sim", from=1-1, to=2-1]
	\arrow["\pi", from=1-2, to=2-2]
	\arrow[from=2-1, to=2-2]
\end{tikzcd}
\end{equation}
We set $\cY \coloneqq \spf(Q_0)$ and note that $d(\cY) = U_K$. Using the
specialization map introduced in \cref{sub:The Injectivity FinLevel}, we define
the specialization map at infinite level as the composition
\begin{equation} \label{eq:DefSpecMapAtInfLevel}
  \specmap^*\colon \hHc^i(Y, \Qlbar) \xto{\pi^*} \hHc^i(\cY_s, \Qlbar) 
  \xto{\specmap^*} \hHc^i(d(\cY), \Qlbar) = \hHc^i(U_K, \Qlbar)
  \to \colim_K \hHc^i(U_K, \Qlbar).
\end{equation}
A priori, it is not clear that this definition is independent of the choices made 
above. In order to get a hold of these choices, we introduce the set
\begin{equation*}
  \cD \coloneqq \left\{ (K, Q_0, \pi) \, \middle \vert \begin{array}c
      K \subseteq K_0 \text{ is a normal open subgroup} \\
      Q_0 \subset Q \text{ is a topologically finitely generated ring of definition 
      for $B_K$} \\
      \pi\colon \spec(Q_0 \otimes_{\cO_\Cp} \Fqbar) \to Y \text{ is finite
        surjective and fits into \eqref{eq:PiFitsDiag}}
  \end{array}\right\}.
\end{equation*}
This set comes equipped with a natural partial order, we say that 
$(K, Q_0, \pi) \leq (K', Q_0', \pi')$ if $K' \subseteq K$ and $Q_0 \subseteq Q_0'$,
and moreover $\pi'$ arises as the composition of $\pi$ with the induced morphism 
$\spec(Q_0' \otimes_{\cO_\Cp} \Fqbar) \to \spec(Q_0 \otimes_{\cO_\Cp} \Fqbar)$. 

\begin{lem}\label{lem:MiedaCategoryOfChoicesIsFiltered}
  This partial order turns $\cD$ into a filtered category. Moreover, for any
  triple $(K, Q_0, \pi) \in \cD$ and any open normal subgroup $K' \subset K_0$ 
  contained in $K$, there is a triple $(K', Q_0', \pi') \in \cD$ with $(K, Q_0,
  \pi) \leq (K', Q_0', \pi')$. 
\begin{proof}
  This is a combination of Lemma 3.11, Corollaries 3.12 and 3.13 and 
  Lemma 3.14 in \cite{mieda2016geometric}.
\end{proof}
\end{lem}

\begin{prop}\label{lem:specmapisindependentofchoice}
  The specialization map as defined in \eqref{eq:DefSpecMapAtInfLevel} is
  independent of the choice of triple $(K, Q_0, \pi) \in \cD$. 
  \begin{proof}
    This is Lemma 3.18 in \cite{mieda2016geometric}.
  \end{proof}
\end{prop}

The condition that $Q_0$ must be topologically finitely generated is imposed 
because it allows to show that the specialization map is equivariant for the 
$J^1$-action on both sides. 

\begin{defi}[Morphism of Finite Level]\label{def:MorphismOfFinLevel}
  An automorphism $\sigma\colon Q \to Q$ is said to be of finite level if it 
  preserves $Q' = (\bigcup_K B_K)\cap Q$. 
\end{defi}

\begin{lem}\label{lem:FinLevelAutoYIeldAutOfColim}
  Any automorphism of finite level
  induces a natural automorphism of the vector space $\colim_K \hHc^i(U_K, \Qlbar)$.
  \begin{proof}[Sketch of Proof] 
    This is \cite[Lemma 3.20]{mieda2016geometric}.  The crucial point 
    is that for each open normal subgroup $K \subset K_0$, and any $(K, Q_0, \pi) \in \cD$, there is an open subgroup $K' \subset K$ fixing the 
    finitely many topological generators of $Q_0$. Now, it can be shown that 
    $\sigma(B_K) \subset B_{K'}$, and that the resulting morphism
    $\sigma\colon B_K \to B_{K'}$ is finite. This induces finite morphisms
    $U_{K'} \to U_K$, and we obtain the desired automorphism after taking compactly supported cohomology and colimit over $K$. 
\end{proof}
\end{lem}
By \cite[Proposition 3.21]{mieda2016geometric}, the specialization map respects
automorphisms of finite level.

\begin{prop}\label{prop:SpecMapIsEquivariantWithFinLev}
  Let $\sigma$ be an automorphism of finite level of $Q$ and let 
  $\bar \sigma$ be an automorphism of $R$. Assume that the isomorphism
  \eqref{eq:isoofreductionwithperfection} is compatible with the automorphisms
  induced by $\bar \sigma$ and $\sigma$. Then, the specialization map
  \begin{equation*}
    \specmap^* \colon \hHc^i(Y, \Qlbar) \to \colim_K \hHc^i(U_K, \Qlbar)
  \end{equation*}
  is compatible with the automorphisms induced by $\bar \sigma$ and 
  $\sigma$. 
\end{prop}

\begin{prop}\label{prop:ActinOfJ1IsFinLevel}
  The group $J^1$ acts on $Q$ through automorphisms of finite level. 
  By \cref{lem:FinLevelAutoYIeldAutOfColim}, this endows $\colim_K \hHc^i(U_K, \Qlbar)$
  with the structure of a $J^1$-representation. The natural morphism $\colim_K
  \hHc^i(U_K, \Qlbar) \to \colim_K \hHc^i(M'_{K,\Cp}, \Qlbar)$ is equivariant
  for the $J^1$-action on both sides.
\begin{proof}
  This is \cite[Propsition 4.5]{mieda2016geometric}.
\end{proof}
\end{prop}

We are now in a position to apply \cref{thm:MiedaInjectivityAtFiniteLevel}.

\begin{thm}[Mieda's Injectivity Theorem for the Lubin--Tate
  Tower]\label{thm:MiedaAppliedToLTT}
  Assume that we are in \cref{sit:MiedaLTTSit}.
  Let $V \subset \hHc^i(Y, \Qlbar)$ be a subspace such that the composition 
  $V \inj \hHc^i(Y, \Qlbar) \to \hH^i(Y, \Qlbar)$ is injective. Then, the composition
  \begin{equation}\label{eq:MiedaInfLevelMap}
  V \inj \hHc^i(Y, \Qlbar) \xto{\specmap^*} \colim_K \hHc^{i}(U_K, \Qlbar)
  \to \colim_K \hHc^i(M'_{K,\Cp}, \Qlbar) = \HLT'
  \end{equation}
  is injective as well. If $V$ is $J^1$-stable, this injection is equivariant 
  for the action of $J^1$ on both sides.
\begin{proof}
  Fix a triple $(K, Q_0, \pi) \in \cD$, let $K'$ be an open normal subgroup 
  of $K_0$ contained in $K$. Then there is a triple $(K, Q_0, \pi) \leq (K', Q_0', \pi') \in \cD$. We write $\cY' = \Spf(Q_0')$,
  which is flat and topologically of finite type over $\spf(\cO_\Cp)$. 
  Furthermore, it has generic fiber $U_{K'}$, which is smooth over 
  $\spa(\Cp, \cO_\Cp)$, and $\pi'\colon Y \to \cY_s$ is a finite surjective
  morphism.
  Thereby, \cref{thm:MiedaInjectivityAtFiniteLevel} yields an injection
  \begin{equation*}
    V \inj \hHc^i(Y, \Qlbar) \xto{\specmap^* \circ \pi'^*} 
    \hHc^i(d(\cY'), \Qlbar) = \hHc^i(U_{K'}, \Qlbar) \to \hHc^i(M'_{K', \Cp}, \Qlbar).
  \end{equation*}
  Taking colimit over $K'$, we obtain a map
  \begin{equation*}
    V \inj \hHc^i(Y, \Qlbar) \xto{\specmap^*} \colim_K \hHc^{i}(U_K, \Qlbar)
    \to \colim_K \hHc^i(M'_{K,\Cp}, \Qlbar) = \HLT'
  \end{equation*}
  which is injective since filtered colimits are exact.
  Moreover, this map is the same as the one defined in
  \eqref{eq:MiedaInfLevelMap}.

  The map $\colim_K \hHc^{i}(U_K, \Qlbar) \to \colim_K \hHc^i(M'_{K,\Cp}, \Qlbar)$
  is $J^1$-equivariant by \cref{prop:ActinOfJ1IsFinLevel}, and together with 
  \cref{prop:SpecMapIsEquivariantWithFinLev} it follows that
  \eqref{eq:MiedaInfLevelMap} is $J^1$-equivariant if $V$ is $J^1$-stable. This finishes the proof.
\end{proof}
\end{thm}

% subsubsection Injectivity of the Specialization Map at Infinite Level (end)

% section Mieda's Approach to the Explicit Local Langlands Correspondence (end)
\end{document}
