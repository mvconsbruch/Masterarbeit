%! TeX root: ../main.tex
\documentclass[../main.tex]{subfiles}

\begin{document}
\section{The Lubin--Tate Space at Infinite Level}
\label{sec:The Lubin--Tate Space at Infinite Level}

% Intro (fold)
In this section, we introduce, attached to a formal $\cO_E$-module
$\XX \in \FMOver {\cO_E}{\Fqbar}$ of height $n \in \N$, the inverse limit
\begin{equation*}
  \cM_{\infty}^{(0)} = \lim_{m \in \N} \left(\cM_{m}^{(0)}\right).
\end{equation*}
The first question is in what category this objects lives. It is
certainly no longer interesting as a covariant functor on $\cC$: let $(y_1,
y_2, \dots)$ be a system of $[\varpi]_F$-torsion points,
for an arbitrary $\varpi$-divisible formal $\cO_E$-module law $F$ over
$R \in \cC$. As $R$ is Noetherian, the ideal generated by the sequence $(y_1,
y_2, \dots)$ is finitely generated, hence it has to be trivial.

However, we can make sense of the limit as a presheaf on the category of formal
schemes over $\cO_\br E$. We have seen in \cref{sec:Non-Abelian Lubin-Tate
Theory: An Overview} that $\cM_{m}^{(0)}$ is 
representable by the formal spectrum of a local ring $A_m$ which
is finite, flat and generically \'etale
over $\spf(A_0) = \spf(\cO_{\br E}\llbr u_1, \dots, u_{n-1}\rrbr)$. In particular,
$\cM_{\infty}^{(0)}$ is the formal spectrum of the completed colimit 
\begin{equation*}
  A_\infty = (\colim_m A_m)^\wedge_{\fm}.
\end{equation*}
Here, $\fm$ denotes the ideal generated by the image of the maximal ideal of
$A_0$ (or of any $A_m$, it doesn't matter). 
Note that the completion along an arbitrary ideal $I \subset A_\infty$ is, in
general, not $I$-adically complete (see
\cite[\href{https://stacks.math.columbia.edu/tag/05JA}{Tag
05JA}]{stacks-project} for an example). However, the maximal ideal $\fm$ is
finitely generated, so this pathology does not occur and $\cM_\infty^{(0)}
\coloneqq \spf A_\infty$ makes sense as a formal scheme. A priori, it is not
clear if there is a good moduli description of $\cM_\infty^{(0)}$ as a functor
$\FSchOver{\cO_\br E}^{\opp} \to \Set$.

This section is concerned with the study of the ring $A_\infty$. 
The main results are summarized as follows. 
First, we review some of the constructions in Chapter 2 of
\cite{weinstein2016semistable}. As a first step, making use of the determinants
of formal $\cO_E$-modules constructed in \cite{hedayatzadeh2015det}, we obtain
a natural homomorphism
\begin{equation*}
   \cO_{\widehat{E^\ab}} \to  A_\infty.
\end{equation*}
We then introduce the notion of universal covers of formal $\cO_E$-modules; the 
passage from $H$ to its universal cover $\tilde H$ can be interpreted as a sort
of tilting procedure. Still following \cite{weinstein2016semistable}, this
notion and its relation with the moduli problem $\cM_{\infty,
\cO_{\C_p}}^{(0)}$ makes it possible to construct an isomorphism
\begin{equation}\label{eq:ExplicitAinftyBCtoOC}
  A_{\infty, \cO_{\C_p}} \coloneq A_\infty \cotimes_{\cO_{\widehat{E^\ab}}}
  \cO_{\C_p} 
  \cong \cO_{\C_p} \llbr X_1^{q^{-\infty}}, \dots, X_n^{q^{-\infty}} \rrbr /
  (\Delta(X_1^{q^{-\infty}}, \dots, X_n^{q^{-\infty}})^{q^{-m}} - \tau^{q^{-m}} \mid 
  m \in \N)^- ,
\end{equation}
where $\Delta \in \cO_{\br E} \llbr X_1^{q^{-\infty}}, \dots,
X_n^{q^{-\infty}}\rrbr$ denotes the power series constructed in 
\cref{sub:Determinants}, and a certain element $\tau \in \Nilp^\flat(\cO_\Cp)$
(suspect to some choice). The superscript minus denotes the completion of the
ideal.

The main effort is taken in an description of the various group actions on
$\cM_\infty \times_{\Spf(\cO_\br E)} \Spf(\cO_\Cp)$ in terms of this
isomorphism, following \cite[ 1.2]{imaitsushima2020affinoids}.
Together with the approximation of the determinant map in 
\cref{lem:DeltaApprox}, this gives the necessary information to observe the
perfection of a Deligne--Lusztig variety as the special fiber of some affinoid
inside the Lubin--Tate perfectoid space, in a certain way compatible with the
respective group actions. 

%Let $G$ denote the product $\GL_n(E) \times D^\times \times \Weil_E$.
%We have seen in  \cref{sec:Non-Abelian Lubin-Tate Theory: An Overview} 
%that $G$ acts on the tower $\{\cM_{m} \times_{\spf(\cO_\br E)} \cO_{\C_p}\}_m$,
%inducing an action on the infinite-level space 
%$\cM_\infty \times_{\Spf(\cO_\br E)} \Spf(\cO_\Cp)$,
%which we will see to admit a decomposition
%\begin{equation}\label{eq:DecompositionOfMinfty}
%  \cM_\infty \times_{\Spf(\cO_\br E)} \Spf(\cO_\Cp) = \coprod_{\alpha \in E^\times}
%  \cM_{\infty,\cO_\Cp}^\alpha = \coprod_{\alpha \in E^\times} \Spf
%  (A_{\infty,\cO_\Cp}^\alpha).
%\end{equation}


% Intro (end)

\subsection{The Case of Height One} % (fold)
\label{sub:The Case of Height One}
Let $\XX$ be a formal $\cO_E$-module law over $\Fqbar$ of height one, 
let $E^\LT_m = \br E E_{\varpi, m}$ denote the degree $m$ Lubin--Tate extension of $E$
and let $(\cF, \iota) \in \cM_{m}^{(0)}(\cO_{E^\LT_m})$ be the unique deformation
of $\XX$.
We have seen in \cref{ssub:The Case of Height One} that any choice of Drinfeld
level $\varpi^m$ structure of $\cF$ yields an isomorphism 
$\cM_M^{(0)} \cong \cO_{E^\LT_m}$. By the construction of the maximal Abelian extension
of $E$ in \cref{sec:Local Class Field Theory}, we have we have 
$(\colim_{m} \cO_{E^\LT_m})^{\wedge}_\varpi = \cO_{\hat E^\ab}$. The following
is a direct consequence.

\begin{prop}\label{prop:InfiniteLevelOfHeightOneIsRepbleByAbExt}
  Assume that $\XX$ is of height one and let $(\cF, \iota) \in \cM_0(\cO_\br E)$ 
  be a deformation of $\XX$. 
  Then any choice of compatible system of level structures on $\cF(\cO_{\hat
  E^\ab})$ yields an isomorphism 
  \begin{equation*}
    \cM_{\infty}^{(0)} \xto \sim \cO_{\hat E^\ab},
  \end{equation*}
  where $\hat E^\ab$ is the closure of the maximal Abelian extension of $E$ inside
  $\Cp$.
\end{prop}

% subsection The Case of Height One (end)

\subsection{The Deformation Space in terms of the Universal Cover} % (fold)
\label{ssub:Relation to the Deformation Space at Infinite Level}
Let $(e_1, \dots, e_n)$ denote the standard basis of $\cO_E^n$. By 
Theorem \ref{thm:RepresentabilityOfDefSpaceWithLevel}, there is for 
every positive integer $m$ a universal triple 
$$(\cF^\univ, \iota^{\univ}, \phi^\univ_m) \in \cM_{m}^{(0)}(A_m)$$
and the Drinfeld level $\varpi^m$-structure 
\begin{equation*}
  \phi_m^\univ\colon {(\varpi^{-m}\cO_{E}/\cO_E)}^n \to \cF^\univ(A_m)
\end{equation*}
gives rise to elements $x_i^{(m)} = \phi_m(e_i) \in
\cF^{\univ}(A_\infty)$ for $i = 1, \dots, n$. 
This gives rise to an $n$-tuple 
of compatible systems
\begin{equation*}
  (x_1, \dots, x_n) \in \tilde \cF^\univ (A_\infty). 
\end{equation*}
Now let $\cF$ be an arbitrary deformation of $\XX$ to $\cO_{\br E}$ and let $I \subset A_\infty$ be an ideal of definition
containing $\varpi$. By 
Proposition \ref{prop:UnivCoverReductionIso}, we have isomorphisms
\begin{equation*}
  \tilde \cF^{\univ,n}(A_\infty) \cong \tilde \cF_0^{\univ, n}(A_\infty/I)
  \xto{\iota} \tilde \cF_0^n(A_\infty/I)
  \cong \tilde \cF^n(A_\infty).
\end{equation*}
This constructs a morphism of formal schemes over $\spf \cO_\br E$
\begin{equation*} 
  \cM_\infty^{(0)} \to \tilde \cF^n.
\end{equation*}
Similarly, we obtain natural maps $\cM_\infty^{(j)} \to \tilde \cF^n$, yielding in the colimit over $j$ a morphism
\begin{equation}\label{eq:MapFromMtoTildeF}
\cM_\infty \to \tilde \cF^n.
\end{equation}

Suppose we are given a system $S_0 \to S_1 \to S_2 \to \dots$ of objects in 
$\cC$, together with a system of compatible triples 
$(\cG, \iota, \phi_m) \in \cM_m(S_m)$. Furthermore, assume that there
is an $\cO_\br E$-algebra 
$S_\infty$ with compatible morphisms $S_m \to S_\infty$. This data determines a 
morphism $S_\infty \to \cM_\infty$, and the morphism above yields an $n$-tuple 
$(X_1, \dots, X_n) \in \tilde \cF^n(S_\infty)$, which has the following description. Let $J \subset S_\infty$
be an ideal of definition with $\varpi \in J$. Then $X_i$ is exactly the image
of the $i$-th basis vector in the composition
\begin{equation*}
    {\cO_E^n} \xto{(\phi_m)_{m \in \N}} T_\varpi \cG(S_\infty) \to \tilde \cG_0(S_\infty/J) \xto{\iota^{-1}} \tilde \XX(S_\infty/J)
    \xto{\sim} \tilde \cF(S_\infty).
\end{equation*}



By naturality of the determinant map, this yields, for
$j \in \Z$, a commutative square 
\begin{equation*}
    \begin{tikzcd}[ampersand replacement=\&]
      {\cM_\infty^{(j)}} \& {\cM_{\wedge, \infty}^{(j)}} \\
    	{\widetilde \cF^n} \& {\widetilde{\wedge^n \cF}.}
    	\arrow["\det", from=1-1, to=1-2]
    	\arrow[from=1-1, to=2-1]
    	\arrow[from=1-2, to=2-2]
    	\arrow["\delta", from=2-1, to=2-2]
    \end{tikzcd}
\end{equation*}

All of the formal schemes in this square are affine, and with careful analysis
of the underlying adic algebras, Weinstein obtained the following key
result.

\begin{thm}[Structure of $\cM_\infty$]\label{thm:WeinsteinsCartesianSquare}
  The square above is cartesian.
  \begin{proof}
  This is \cite[Theorem 2.17]{weinstein2016semistable}.
\end{proof}
\end{thm}

By \cref{prop:UnivCoverReductionIso}, any coordinate $\cF \cong \FGG(F)$ yields
isomorphisms $\tilde \cF \cong \spf(\cO_\br E \llbr X^{q^{-\infty}}\rrbr)$ and
$\widetilde{\wedge^n \cF} \cong \spf(\cO_\br E \llbr X^{q^{-\infty}}\rrbr)$.
Let $t = (t_1, t_2, \dots) \in T_\varpi F(\cO_{\hat E^\ab})$ be a primitive element,
that is, an element such that $t_1 \neq 0$ and $[\varpi]_F(t_1) = 0$. 
This is equivalent to a system of compatible level structure $(\phi_m)_{m \in
\N}$ with codomain $\wedge^n F(\cO_{\hat E ^\ab})$. 
The element $t$ yields an isomorphism $\cM_{\wedge,
\infty}^{(j)} \cong  \cO_{\hat E^\ab}.$ We obtain the following description of
$A_{\infty, \cO_\Cp}$. 

\begin{cor}\label{cor:StructureOfAinfty}
  Let $(\tau^{1/q^m})_{m \in \N} \in \Nilp^\flat(\cO_{\hat E^\ab})$ be the
  system of $q$-th power roots corresponding to $t$ under the inclusion
  $T_\varpi F \inj \tilde F$. Then we have 
  \begin{equation*}
    A_\infty \cong \cO_{\hat E^\ab}\llbr X_1^{q^{-\infty}}, \dots,
    X_n^{q^{-\infty}}\rrbr/(\Delta^{q^{-m}} - \tau^{q^{-m}} \mid m \in \N)^-.
  \end{equation*}
\end{cor}

We note that any other primitive $t' \in T_\varpi(F)(\cO_{\hat E^\ab})$ satisfies
$t' = [\alpha]_F(t)$ for some unique $\alpha \in \cO_E^\times$. 
By definition of the Artin map, the element $\tau' \in \Nilp^\flat(\cO_{\hat
E^\ab})$ corresponding to $t' \in \tilde F(\cO_{\hat E^\ab})$ is given by
$\Art_E(\alpha)(\tau)$.

Let $M'_{\infty, \Cp}$ be the generic fiber of $\cM'_{\infty,\cO_\Cp}$, it 
is an adic space over $\spa(\Cp, \cO_\Cp)$. We call $M'_{\infty, \Cp}$ the 
Lubin--Tate perfectoid space. This name is justified by the following result.
\begin{thm}[The Lubin--Tate Perfectoid Space]\label{thm:LTPerfSpace}
  The adic space $M'_{\infty, \Cp}$ can be covered by perfectoid affinoids,
  hence it is a perfectoid space.
  \begin{proof}[Sketch of Proof] 
    We sketch the proof in \cite[Lemma 2.32]{weinstein2016semistable}. 
    Let $I$ denote a finitely generated ideal of definition for $A_{\cO_\Cp}
    \coloneqq A_\infty \cotimes_{\cO_{\hat E^\ab}} \cO_\Cp$ and let $(f_1,
    \dots, f_k)$ be a tuple of generators for $I$. 

    Let $x \in M'_{\infty, \Cp}$ be an arbitrary point. We have
    $\abs{\varpi(x)} \neq 0$ by definition of the generic fiber. The element
    $f_i$ is topologically nilpotent in $A_{\infty,\cO_\Cp}$,
    and it follows that there is some $r \in \N$ with $\abs{f_i(x)}^r <
    \abs{\varpi(x)}$ for all $i = 1, \dots, k$. We may now define the subspace 
    $U_r = \spa(R_r, R_r^+)$, where $R_r$ is defined as
    $$R_r \coloneqq A_{\cO_\Cp}\left\langle \tfrac{f_1^r}\varpi, \dots, \tfrac
    {f_k^r}\varpi \right\rangle [\tfrac 1\varpi],$$
    and $R_r^+$ is defined as the integral closure of $A_{\cO_\Cp}\left\langle
    \tfrac{f_1^r}\varpi, \dots, \tfrac {f_k^r}\varpi \right\rangle$ inside
    $R_r$. Now $R_r^+ = R_r^\circ$. As the Frobenius map is surjective on
    $A_{\cO_\Cp}/\varpi$, it is surjective on $R_r^+/\varpi$, and we find
    that $R_r$ is perfectoid. The valuation
    belonging to $x$ extends uniquely to $R_r$ and satisfies $\abs{R_r^+} \leq 1$,
    so it is point in $\spa(R_r, R_r^+)$. The claim follows.
  \end{proof}
\end{thm}

% subsubsection Relation to the Deformation Space at Infinite Level (end)

\subsection{Reviewing the Group Actions} % (fold)
\label{ssub:Group Actions on the Universal Cover}
Write
\begin{equation*}
  G \coloneqq \GL_n(E) \times D^\times \times \Weil_E,
\end{equation*}
and define a homomorphism
\begin{equation*}
  \theta \colon G \to E^\times, \quad (g,d,\sigma) \mapsto \det(g)^{-1} \Nrd(d)
  \Art^{-1}_E(\sigma).
\end{equation*}

Let $G^{\mathrm u}$ be the preimage of $\cO_E^\times$ under $\theta$, so that 
$G^{\mathrm u}$ acts on $\cM_{\infty}^{(0)} \times_{\Spf \br E} \Spf(\cO_\Cp)$. 
Given a $\varpi$-divisible formal $\cO_E$-module $\cF \in \FMOver
{\cO_E}{\cO_{\br E}}$ of height $n$, we write $\cF_{\cO_\Cp} = \cF \otimes_{\cO_\br E}
\cO_\Cp$ and describe a natural right action on $\tilde \cF_{\cO_\Cp}^n$ by $G$ 
such that the map $\cM_{\infty}^{(0)} \times_{\spf \br E} \Spf(\cO_\Cp) \to
\tilde \cF_{\cO_\Cp}^n$,
induced by the map constructed above is equivariant for the respective
$G^{\mathrm u}$-actions.

The action of $G$ on $\tilde \cF^n$ is easy to describe. For the action of
$\GL_n(E)$, note that $\tilde \cF$ carries the structure of a $E$-vector space
object. Hence $\tilde \cF^n$ obtains a natural right action by $\GL_n(E)$: an
object $g \in \GL_n(E)$ with entries $g = (a_{ij})_{i,j}$ acts by matrix
multiplication from the right, as in 
\begin{equation}\label{eq:UnivCoverGLnAction}
  (x_1, \dots, x_n).g = (y_1, \dots, y_n), \quad
  \text{where} \quad y_j = \sum_{i=1}^n a_{ij} x_i. 
\end{equation}

For the action of $D^\times$, note that by Proposition
\ref{prop:UnivCoverReductionIso}, we have 
a natural left action of $D^\times$ on $\tilde \cF$. 
Indeed, given $0 \neq d_0 \in \End_{\FMOver A{\Fqbar}}(F_0) = \cO_D$, an
integer $r \in \Z$ and any $S \in \Adm {\cO_\br E}$ with ideal of definition
containing $\varpi$, 
we let the element $d = \varpi^r d_0 \in D^\times$ act on $\tilde \cF$ via the
automorphism
\begin{equation*}
  \tilde \cF(S) \to \tilde \cF_0(S/J) \xto{\varpi^{r}} \cF_0(S/J) \xto{d_0}
  \tilde \cF_0(S/J) \to \tilde \cF(S), \quad x \mapsto dx.
\end{equation*}
Note that multiplication by $\varpi$ and $d_0$ commute as $\varpi$ lies in the 
center of the multiplicative monoid $\cO_D$, so this yields a well-defined
left action. We define the right action of $D^\times$ on $\tilde \cF^n$ via
\begin{equation*}
  (x_1, \dots, x_n).d = (d^{-1} x_1, \dots, d^{-1} x_n).
\end{equation*}


The map $\Pi: \Tilde \cF \to \Phi^{-1,*}\Tilde \cF$ from Definition 
\ref{def:FrobOnUnivCov} equips $\cF$ with the Weil descent datum
\begin{equation*}
  (\Phi^*{\Pi})^{-1}: \cF \to \Phi^*\cF,
\end{equation*}
and in particular yields an action of $\Weil_E$ on $\tilde \cF_{\cO_\Cp}^n$. 

It is easy to see that the actions of $\GL_n(E)$ and $D^\times$ commute,
and that both these actions commute with the Weil descent datum. Hence we
obtain a right action by $G$ on $\tilde \cF^n_{\cO_\Cp}$.

The maps constructed in \eqref{eq:MapFromMtoTildeF} induce a 
map $\cM_\infty \to \tilde \cF^n$.

\begin{lem}\label{lem:EquivarianceForD}
    The map $\cM_\infty \to \tilde \cF^n$ is equivariant 
    for the action of $D^\times$ on both sides. More precisely,
    given any $d \in D^\times$ and any $j \in \Z$, the diagram
    \begin{equation*}
        \begin{tikzcd}[ampersand replacement=\&]
        	{\cM_\infty^{(j)}} \& {\tilde \cF^n} \\
        	{\cM_\infty^{(j')}} \& {\tilde \cF^n},
        	\arrow[from=1-1, to=1-2]
        	\arrow["{\cdot d}", from=1-1, to=2-1]
        	\arrow["{\cdot d}", from=1-2, to=2-2]
        	\arrow[from=2-1, to=2-2]
        \end{tikzcd}
    \end{equation*}
    with $j' = j + \val_\varpi(\Nrd d)$, commutes.
    \begin{proof}
        This follows directly from the description of the $D^\times$-action on $\cM_\infty$ and $\tilde \cF^n$.
    \end{proof}
\end{lem}

The proof of the corresponding statement for the action of $\GL_n(E)$ is more
tedious. 

\begin{lem}\label{lem:EquivarianceForGLn}
    The map $\cM_\infty \to \tilde \cF^n$ is equivariant 
    for the action of $\GL_n(E)$ on both sides. More precisely,
    given any $g \in \GL_n(E)$ and any $j \in \Z$, the diagram
    \begin{equation*}
        \begin{tikzcd}[ampersand replacement=\&]
        	{\cM_\infty^{(j)}} \& {\tilde \cF^n} \\
        	{\cM_\infty^{(j')}} \& {\tilde \cF^n},
        	\arrow[from=1-1, to=1-2]
        	\arrow["{\cdot g}", from=1-1, to=2-1]
        	\arrow["{\cdot g}", from=1-2, to=2-2]
        	\arrow[from=2-1, to=2-2]
        \end{tikzcd}
    \end{equation*}
    with $j' = j - \val_\varpi(\det g)$, commutes.
\begin{proof}
    We denote by $(X_1, \dots, X_n) \in \tilde \cF^n(A_\infty^{(j)})$
    the elements corresponding to the morphism
    $\cM_\infty^{(j)} \to \tilde \cF^n$ and denote by 
    $(Z_1, \dots, Z_n)$ the elements corresponding to the composition
    $\cM_\infty^{(j)} \xto{\cdot g} \cM_\infty^{(j')} \to \tilde \cF^n$.
    We need to show that $(X_1, \dots, X_n).g = (Z_1, \dots, Z_n)$.

    This essentially boils down to the description of the group action
    on $\cM_\infty$. As the elements $(\varpi, \varpi, 1)$ act trivially 
    on both sides, we may assume that $g \in \GL_n(E)$ satisfies
    $g^{-1} \in \Mat_{n \times n}(\cO_E)$. Let $k \in \N$
    be an integer such that $\varpi^k g \in \Mat_{n\times n}
    (\cO_E)$. Let $m>k$ be another integer and
    take an arbitrary triple $(\cF, \iota, \phi) \in \cM_m^{(j)}(R)$
    for any $R \in \cC.$ Let 
    $S \in \Adm R$ be any admissible $R$-algebra and let $J
    \subset S$ be an ideal of definition with $\varpi \in J$. Write 
    $$(\cF', \iota', \phi') = (\cF, \iota, \phi). g
    \in \cM_m^{(j')}(R).$$
    By construction of the $\GL_n(E)$-action on 
    $\cM_m$, we have a commutative diagram
    \begin{equation*}
\begin{tikzcd}[ampersand replacement=\&]
	\& {{(\varpi^{-m}\cO_E^n/\cO_E^n)}} \& {\cF(S)} \& {\cF_0(S/J)} \& {\XX(S/J)} \\
	{{(\varpi^{m-k}\cO_E^n/\cO_E^n)}} \& {{(\varpi^{-m}\cO_E^n/g\cO_E^n)}} \& {\cF'(S)} \& {\cF_0'(S/J)} \& {\XX(S/J).}
	\arrow["{\phi_m}", from=1-2, to=1-3]
	\arrow[from=1-2, to=2-2]
	\arrow[from=1-3, to=1-4]
	\arrow[from=1-3, to=2-3]
	\arrow["{\iota^{-1}}", dashed, from=1-4, to=1-5]
	\arrow[from=1-4, to=2-4]
	\arrow[Rightarrow, no head, from=1-5, to=2-5]
	\arrow["g", from=2-1, to=2-2]
	\arrow["{\phi'_{m-k}}"', curve={height=18pt}, from=2-1, to=2-3]
	\arrow["{\bar \phi_m}", from=2-2, to=2-3]
	\arrow[from=2-3, to=2-4]
	\arrow["{\iota'^{-1}}", dashed, from=2-4, to=2-5]
\end{tikzcd}
    \end{equation*}
    Let us now consider the diagrams arising with $R = A_m^{(j)}$, $S = A_\infty^{(j)}$ with any ideal of definition $I \subset A_\infty^{(j)}$ with $\varpi \in I$ and the universal 
    triple $(\cF^{\univ}, \iota^\univ, \phi_m^{\univ}) \in \cM_m^{(j)}(A_m^{(j)})$. In the limit over $m$, these
    diagrams yield a commutative square of Abelian groups
    \begin{equation}\label{eq:diagramimportantforequiv}
\begin{tikzcd}[ampersand replacement=\&]
	{\cO_E^n} \& {T_\varpi \cF^\univ(A_\infty^{(j)})} \& {\tilde \cF^\univ_0(A_\infty^{(j)}/I)} \& {\tilde \XX(A_\infty^{(j)}/I)} \\
	{\cO_E^n} \& {T_\varpi \cF'(A_\infty^{(j)})} \& {\tilde \cF'_0(A_\infty^{(j)}/I)} \& {\tilde \XX(A_\infty^{(j)}/I)}.
	\arrow["{{(\phi_m)_{m \in \N}}}", from=1-1, to=1-2]
	\arrow["{g^{-1}}"', from=1-1, to=2-1]
	\arrow[from=1-2, to=1-3]
	\arrow[from=1-2, to=2-2]
	\arrow["{\iota^{-1}}", from=1-3, to=1-4]
	\arrow[from=1-3, to=2-3]
	\arrow[Rightarrow, no head, from=1-4, to=2-4]
	\arrow["{(\phi'_m)_{m \in \N}}", from=2-1, to=2-2]
	\arrow[from=2-2, to=2-3]
	\arrow["{\iota'^{-1}}", from=2-3, to=2-4]
\end{tikzcd}
    \end{equation}
    Under the natural bijection $\tilde \XX(A_\infty^{(j)}/I) \xto \sim \tilde \cF(A_\infty)$, the images 
    of the basis vectors $(e_1, \dots, e_n)$ of the top row are
    mapped to the the elements $(X_1, \dots, X_n) \in \tilde \cF^n(A_\infty^{(j)})$, while the basis vectors of the bottom row map to $(Z_1, \dots, Z_n) \in \tilde \cF^n(A_\infty^{(j)})$. Commutativity of the diagram now implies 
    $$(X_1, \dots, X_n).g = (Z_1, \dots, Z_n),$$
    and this is what we had to show. If $g$ is general, the same
    argument works, we only have to replace $\cO_E^n$ with 
    $E^n$ and the tate module with the rational tate module.
\end{proof}
\end{lem}

\begin{lem}\label{lem:EquivarianceForWeilDD}
    The map $\cM_\infty \to \tilde \cF^n$ is equivariant
    for the Weil descent datum on both sides. More precisely,
    for any $j \in \Z,$ the square
    \begin{equation*}
        \begin{tikzcd}[ampersand replacement=\&]
        	{\cM_\infty^{(j)}} \& {\tilde \cF^n} \\
        	{\Phi^*\cM_\infty^{(j-1)}} \& {\Phi^*\tilde \cF^n}
        	\arrow[from=1-1, to=1-2]
        	\arrow[from=1-1, to=2-1]
        	\arrow[from=1-2, to=2-2]
        	\arrow[from=2-1, to=2-2]
        \end{tikzcd}
    \end{equation*}
    with vertical maps given by the respective Weil descent data,
    commutes.
\end{lem}
\begin{proof}
    As in the proof of \cref{lem:EquivarianceForGLn}, we denote by $(X_1, \dots, X_n) \in \tilde \cF^n(A_\infty^{(j)})$
    the elements corresponding to the morphism
    $\cM_\infty^{(j)} \to \tilde \cF^n$ and denote by 
    $(Z_1, \dots, Z_n)$ the elements corresponding to the composition
    $\cM_\infty^{(j)} \to \Phi^*\cM_\infty^{(j)} \to \Phi^*\tilde \cF^n$.
    We need to show that $(\Phi^* \Pi)^{-1}(X_1, \dots, X_n) = (Z_1, \dots, Z_n)$.
    Let $J \subset A_\infty^{(j)}$ be an ideal of definition with 
    $\varpi \in J$, and denote by $x_i \in \tilde \cF_0(A_\infty/J) = \XX_0(A_\infty/J)$ the reduction 
    of $X_i$, and similarly denote by $z_i \in \tilde \XX^{(-q)}_0(A_\infty/J)$ the reduction 
    of $Z_i$. Now, the statement $Z_i = (\Phi^* \Pi)^{-1}(X_i)$ is equivalent to $\Frob_q(z_i) = x_i$.

     As above, let $(\cF^\univ, \iota, \phi_m) \in \cM_m^{(j)}(A_m^{(j)})$ denote a universal
     triple and write 
     $(\Phi^*\cF^\univ, \iota', \Phi^* \phi) \in \Phi^* \cM_m^{(j-1)}(A_m^{(j)})$ for the 
     triple obtained by the Weil descent datum on $\cM_m$. By construction, we have the following commutative diagram.
\begin{equation*}
\begin{tikzcd}[ampersand replacement=\&]
	{\varpi^{-m}\cO_E^n/\cO_E^n} \& {\cF^\univ(A_\infty)} \& { \cF^\univ_0(A_\infty/J)} \& {\XX(A_\infty/J)} \\
	{\varpi^{-m}\cO_E^n/\cO_E^n} \& {\Phi^*\cF^\univ(A_\infty)} \& { \cF^{\univ,{(-q)}}_0(A_\infty/J)} \& { \XX^{(-q)}(A_\infty/J)}
	\arrow["{\phi_m}", from=1-1, to=1-2]
	\arrow[Rightarrow, no head, from=1-1, to=2-1]
	\arrow[from=1-2, to=1-3]
	\arrow["{\iota^{-1}}", dashed, from=1-3, to=1-4]
	\arrow["{\Phi^*\phi_m}", from=2-1, to=2-2]
	\arrow[from=2-2, to=2-3]
	\arrow["{\Frob_q}"', from=2-3, to=1-3]
	\arrow["{\iota'^{-1}}"{description}, dashed, from=2-3, to=1-4]
	\arrow["{(\iota^{(-q)})^{-1}}", dashed, from=2-3, to=2-4]
	\arrow["{\Frob_q}"', from=2-4, to=1-4]
\end{tikzcd}
\end{equation*}
    In the limit over $m$, we obtain the diagram
    \begin{equation*}
\begin{tikzcd}[ampersand replacement=\&]
	{\cO_E^n} \& {T_\varpi\cF^\univ(A_\infty)} \& {\tilde \cF^\univ_0(A_\infty/J)} \& {\tilde \XX(A_\infty/J)} \\
	{\cO_E^n} \& {T_\varpi[\Phi^*\cF^\univ](A_\infty)} \& {\tilde \cF^{\univ,{(-q)}}_0(A_\infty/J)} \& {\tilde \XX^{(-q)}(A_\infty/J)}
	\arrow["{(\phi_m)_m}", from=1-1, to=1-2]
	\arrow[Rightarrow, no head, from=1-1, to=2-1]
	\arrow[from=1-2, to=1-3]
	\arrow["{\iota^{-1}}", from=1-3, to=1-4]
	\arrow["{(\Phi^*\phi_m)_m}", from=2-1, to=2-2]
	\arrow[from=2-2, to=2-3]
	\arrow["{\Frob_q}"', from=2-3, to=1-3]
	\arrow["{\iota'^{-1}}"{description}, from=2-3, to=1-4]
	\arrow["{(\iota^{(-q)})^{-1}}", from=2-3, to=2-4]
	\arrow["{\Frob_q}"', from=2-4, to=1-4]
\end{tikzcd}
    \end{equation*}
    and the claim follows immediately.
\end{proof}

As a direct corollary from the three previous lemmas, we obtain the following result.
\begin{prop}\label{lem:MapToUnivCovIsEquivariant}
  The morphism $\cM_{\infty}^{(0)} \times_{\spf \cO_\br E} \spf(\cO_\Cp)
  \to \tilde \cF_{\cO_\Cp}^n$
  in \eqref{eq:MapFromMtoTildeF} is equivariant for the action of $G^{\mathrm u}$ on both
  sides.
\end{prop}

% subsubsection Group Actions on the Universal Cover (end)

\subsection{Description of the Group Actions in Coordinates} % (fold)
\label{ssub:Making the Group Actions Explicit}

We now choose coordinates for $\cM_{\infty}^{(0)}$ and make the group action
of $G^{\mathrm u}$ explicit in terms of these coordinates. Let $H$ be the standard formal
$\cO_E$-module over $\br E$ of height $n$. By the monomorphism constructed in
\eqref{eq:MapFromMtoTildeF} and Proposition \ref{lem:MapToUnivCovIsEquivariant},
the action of $G^{\mathrm u}$ on $\cM_{\infty}^{(0)}$ is determined by the action of 
$G^{\mathrm u}$ on $\tilde H^n$. As canonically $\tilde H^n \cong \Nilp^{\flat, n}$, the
right action
of $G$ on $\tilde H^n$ is equivalent to a left action on the algebra
$$\Xi_n \coloneqq \cO_\Cp \llbr X_1^{q^{-\infty}}, \dots, X_n^{q^{-\infty}}\rrbr.$$ 
Our first aim is to make this action explicit.

We begin with an explicit description of the isomorphisms $\mu$ and $\lambda$
of Proposition \ref{prop:UnivCoverReductionIso}.

\begin{lem}\label{lem:ExplicitDescrOfMuAndLambda}
  The bijections
  \begin{equation*}
    \lambda \colon \tilde H(S) \rightleftarrows \Nilp^\flat(S) \colon
    \mu, \quad (x_0, x_1, \dots) \mapsspamto (y, y^{q^{-1}}, y^{q^{-2}},
    \dots)
  \end{equation*}
  are, in either direction, given by the equations
  \begin{equation*}
    y^{1/q^{ni}} = \lim_{r \to \infty} x_{r+i}^{q^{nr}} \quad \text{and} \quad 
    x_i = \lim_{s \to \infty} [\varpi^s]_H(y^{q^{-n(i+s)}}).
  \end{equation*}
\begin{proof}
  This follows directly from the fact that $[\varpi]_H(T) \equiv T^{q^n}$ modulo
  $\varpi$ and the explicit description of the isomorphism in Proposition
  \ref{prop:pHTcalc}.
\end{proof}
\end{lem}

This lemma allows us to make the $\cO_E$-module structure on $\Nilp^\flat$
explicit. Let $S$ be an adic $\cO_\br E$-algebra admitting an ideal of definition $J$ 
containing $\varpi$. 

\begin{lem}\label{lem:GroupStructureOnNilp}
  The $E$-vector space structure on $\Nilp^\flat(S)$ takes on the following form.
  \begin{itemize}
    \item Given two $q$-th power compatible systems $y_1, y_2
      \in \Nilp^\flat(S)$ corresponding to compatible systems
      $\mu(y_1) = x_1$, $\mu(y_2) = x_2 \in \tilde H(S)$, the sum $x_1 + x_2 \in
      \tilde H(S)$ corresponds to the element $\lambda(x_1 + x_2) = y_1 +_H y_2 \in
      \Nilp^\flat(S)$, where 
      \begin{equation*}
        (y_1 +_H y_2)^{1/q^j} = \lim_{r \to \infty} H( y_1^{q^{-r}},
        y_2^{q^{-r}})^{q^{r-j}}.
      \end{equation*}
      If $G \in \FMLOver {\cO_E}{\cO_\br E}$ with $G \otimes \Fqbar = H \otimes \Fqbar$,
      the systems of $q$-th power roots $(y_1 +_H y_2)$ and $(y_1 +_G y_2)$ agree. 
    \item Similarly, given $a \in \cO_E$ and $y \in \Nilp^\flat(S)$ with
      $\mu(y) = x \in \Tilde H(S)$, we have 
      \begin{equation*}
        (a_H y)^{1/q^j} = \lambda([a]_H(x)) = \lim_{r \to \infty}
        [a]_H(y^{q^{-r}})^{q^{r-j}}.
      \end{equation*}
      For $G$ as above, we have $[a]_H(x) = [a]_G(x)$. 
  \end{itemize}
\begin{proof}
  The first statement follows directly from the lemma above, after tracing
  through the commutative diagram
  \begin{equation*}
    \begin{tikzcd}[ampersand replacement=\&]
    	\& {\Tilde H(S)^2} \& {\Tilde H(S)} \\
    	{\Nilp^\flat(S)^2} \& {\Tilde H_0(S/J)^2} \& {\Tilde H_0(S/J)} \& {\Nilp^\flat(S).}
    	\arrow["{H(-,-)}", from=1-2, to=1-3]
    	\arrow[ from=1-2, to=2-2]
    	\arrow[ from=1-3, to=2-3]
    	\arrow["\lambda", from=1-3, to=2-4]
    	\arrow["\mu", from=2-1, to=1-2]
    	\arrow[ from=2-1, to=2-2]
    	\arrow["{H_0(-,-)}", from=2-2, to=2-3]
    	\arrow[from=2-3, to=2-4]
    \end{tikzcd}
  \end{equation*}
  Similarly one proves the third statement.
\end{proof}
\end{lem}

We obtain the following description of the $E^\times \times D^\times$-action.

\begin{cor}\label{cor:ExplicitDescriptionOfActionOnUnivCov}
  Let $a \in E$ and $d \in D^\times$ be elements so that
  \begin{itemize}
    \item the element $a$ is, for some $l \in \Z$, of the form
      \begin{equation*}
        a = \sum_{i = l}^\infty a_i \varpi^i \quad \text{with} \quad
        a_i \in \mu_{q-1}(E) \cup \{0\}.
      \end{equation*}
    \item the element $d$ is, for some $l' \in \Z$ and 
      $\vartheta \in \End_{\FMLOver {\cO_E}\Fqbar} (H_0)$ the endomorphism given
      by $\vartheta(T) = T^q$ (cf. the description of $D^\times$ at the 
      end of \cref{sub:Formal DVR-Modules over Fields}) of the form
      \begin{equation*}
        d = \sum_{i = l'}^\infty d_i \vartheta^i \quad \text{with} \quad
        d_i \in \mu_{q^n-1}(E_n) \cup \{0\}.
      \end{equation*}
      Here $E_n$ denotes the unramified extension of $E$ with residue field 
      $\FF_{q^n}$. 
  \end{itemize}
  Then, given any $x = (x_1, x_2, \dots)
  \in \tilde H(S)$ with $\lambda(x)^{1/q^i} = y^{1/q^i} \in \Nilp^\flat(S)$, we
  have the explicit descriptions
  \begin{equation*}
    a_H(y)^{1/q^j}  = \sum_{i=l}^\infty\sumH a_i y^{q^{ni-j}}
  \quad \text{and} \quad
  d_H(y)^{1/q^j} \coloneqq \lambda(dx)^{1/q^j} = \sum_{i=l'}^\infty \sumH d_i y^{q^{i-j}}
  \end{equation*}
  Here the symbol $\sum \sumH$ \hspace{-10pt} denotes addition of $q$-th power
  root systems with
  respect to the addition in $H$, as defined in the previous lemma.
  \begin{proof}
    This is an immediate corollary of the lemma above, 
    \cref{lem:MultByROUForStandardModule} and the fact that 
    $[\varpi]_H(T) \equiv T^{q^n} \pmod \varpi$. 
  \end{proof}
\end{cor}

It is now not hard to deduce an explicit description of the $\GL_n(E)$-action
on $\tilde H^n(S)$, we write it down in terms of $\Xi_n$ below. Finally, we
describe the action of $\Weil_E$ on $(\tilde H \otimes \cO_\Cp)$ in terms
of $\Nilp^\flat \times_{\spf \cO_\br E} \Spf(\cO_\Cp) \eqcolon \Nilp^\flat_{\cO_\Cp}$.

\begin{lem}\label{lem:ExplicitWeilGroupActionOnTildeH}
  Let $\sigma \in \Weil_E$ and let $m \in \Z$ be an integer such that
  $\sigma|_\br E = \Phi^m$. Let $(y, y^{1/q}, \dots)$ be an element in
  $\Nilp_{\cO_\Cp}^\flat(S)$.
  Then
  \begin{equation*}
    \sigma.(y_1, y_2, \dots) = (\sigma(y)^{q^{-m}}, \sigma(y)^{q^{-(m+1)}}, \dots).
  \end{equation*}
\begin{proof}
  This follows from the definition of the Weil descent datum and the fact that 
  the square
  \begin{equation*}
    \begin{tikzcd}[ampersand replacement=\&]
    	{\tilde H(S)} \& {\Nilp^\flat(S)} \\
    	{\tilde H(S)} \& {\Nilp^\flat(S)}
    	\arrow["\sim", from=1-1, to=1-2]
    	\arrow["\Pi"', from=1-1, to=2-1]
    	\arrow["{y \mapsto y^q}", from=1-2, to=2-2]
    	\arrow["\sim", from=2-1, to=2-2]
    \end{tikzcd}
  \end{equation*}
  is commutative.
\end{proof}
\end{lem}
\begin{rmk} 
  Note that  as $\Xi_n$ is defined over $\cO_E$, it comes with a natural action
  by $\Weil_E$. This action differs from the one defined above and is not
  compatible with the Weil group action on $\cM_{\infty, \cO_\Cp}^{(0)}$.
\end{rmk}

We obtain the following description of the left action of $G$ on $\Xi_n = 
\cO_\Cp\llbr X_1^{q^{-\infty}}, \dots, X_n^{q^{-\infty}}\rrbr$.

\begin{lem}\label{lem:ExplicitActionOnQthPowerRootSystems}
  Let $g = (a_{ij})_{i,j} \in \GL_n(E)$, $d \in D^\times$ and $\sigma \in
  \Weil_E$. Let $m \in \Z$ be such that $\sigma|_{\br E} = \Phi^m$. Then the
  morphism $\Xi_n \to \Xi_n$ induced by the element $(g, d, \sigma) \in G$ is given by
  the composition of the morphsims
  \begin{itemize}
    \item $g^*\colon \Xi_n \to \Xi_n$, the morphism of $\cO_\Cp$ algebras given
      by $X_i \mapsto \sum_{j=1}^n \sumH {a_{ji}}_H(X_j)$. Writing
      $a_{ij} = \sum_{k=l}^\infty a_{ij}^{(k)} \varpi^k$ for  $a_{ij}^{(k)}$
      either vanishing or a $(q-1)$th root of unity and a sufficiently chosen integer
      $l$, we obtain, by Corollary
      \ref{cor:ExplicitDescriptionOfActionOnUnivCov}, the description
      \begin{equation*}
        X_i \mapsto \gamma_{g,i}(X_1, \dots, X_n) \coloneqq 
        \sum_{k=l}^\infty {\hspace{-3pt}}_H\hspace{3pt}
        \sum_{j=1}^n{\hspace{-3pt}}_H\hspace{3pt} a_{ji}^{(k)} X_j^{q^{nk}}.
      \end{equation*}
      Note that both the left and the right hand side are systems of $q$-th power roots.

    \item $d^{-1,*} \colon \Xi_n \to \Xi_n$, the isomorphism of 
      $\cO_\Cp$-algebras given by $X_i \mapsto d_H(X_i)$. 
      Writing $d^{-1} = \sum_{k=l}^\infty d_i \vartheta^k$ for
       a sufficiently small integer $l$, and $\vartheta$ and $d_i$ as in
       Corollary \ref{cor:ExplicitDescriptionOfActionOnUnivCov}, we obtain the
       description
      \begin{equation*}
        X_i \mapsto \delta_{d, i}(X_i) \coloneqq \sum_{k=l}^\infty
        {\hspace{-3pt}}_H\hspace{3pt} d_k X_i^{q^{k}}.
      \end{equation*}
      Note that both the left and the right hand side are systems of $q$-th power roots.

    \item $\sigma^*\colon \Xi_n \to \Xi_n$, the isomorphism of 
      $\cO_E$-algebras given by $X_i \mapsto X_i^{q^{-m}}$ and 
      $a \mapsto \sigma(a)$ for $a \in \cO_\Cp$. 
  \end{itemize}
\end{lem}

Let us now turn our attention to the space 
$\cM_{\infty,\cO_\Cp}' = \spf(A_\infty \cotimes_{\cO_{\hat E^\ab}} \cO_\Cp)$. 
We choose a primitive element $t = (t_1, t_2, \dots) \in T_\varpi H(\cO_{\hat
E^\ab})$. Denote by $\tau$ the corresponding element of $\Nilp^\flat(\cO_{\hat E^\ab})$,
so that
$$\tau^{1/q^m} = \lim_{r \to \infty} (-1)^{m(n-1)}t_{r+m}^{q^{m}}  \in
\Nilp^\flat(\cO_{\hat E^\ab})$$
by \cref{prop:UnivCoverReductionIso} and the description of the determinant at the end
of \cref{sub:Determinants}. 
By \cref{cor:StructureOfAinfty}, this datum yields an isomorphism
\begin{equation*}
  A_\infty \cotimes_{\cO_{\hat E^\ab}} \cO_\Cp \cong 
  \frac{\cO_\Cp\llbr X_1^{q^{-\infty}}, \dots, X_n^{q^{-\infty}} \rrbr }{
  (\Delta(X_1, \dots, X_n)^{q^{-m}} - \tau^{q^{-m}} \mid m \in \N)^-} \eqcolon
  A_{\infty, \cO_\Cp}^{1}.
\end{equation*}
For $\alpha \in \cO_{E}^\times$, we define
\begin{equation*}
  A_{\infty, \cO_\Cp}^{\alpha} \coloneqq
  \frac{\cO_\Cp\llbr X_1^{q^{-\infty}}, \dots, X_n^{q^{-\infty}} \rrbr }{
  (\Delta(X_1, \dots, X_n)^{q^{-m}} - \Art_E(\alpha)(\tau)^{q^{-m}} \mid m \in \N)^-}.
\end{equation*}
Note that for any admissible $\cO_\Cp$-algebra $R$, we have
\begin{equation*}
  \Hom(A_\infty \cotimes_{\cO_\br E} \cO_\Cp, R) 
  \cong \coprod_{\alpha \in \cO_E^\times} \Hom(A_{\infty,\cO_\Cp}^{\alpha}, R)
\end{equation*}
as sets. We obtain the following description of the $G^{\mathrm u}$ action on
$A_{\infty}\cotimes_{\cO_\br E} \cO_\Cp$.
\begin{prop}\label{prop:ExplicitDescriptionOfActionOnAinfty}
  The group $G^{\mathrm u}$ is generated by elements of the form
  \begin{itemize}
    \item $(a, a, 1) \in G$ for $a \in E^\times$.
    \item $(g, d, 1) \in G$ such that $\det(g) \Nrd(d)^{-1} \in \cO_E^\times$.
    \item $(1, \vartheta^{-m}, \sigma)$ for $\sigma \in \Weil_E$ with
      $\sigma|_{\br E} = \Phi^m$. 
  \end{itemize}
  These elements act on $A_{\infty} \otimes_{\cO_\br E} \cO_\Cp$ as follows.
  \begin{itemize}
    \item $(a,a,1)$ acts trivially.
    \item $(g,d,1)$ acts by the morphism of $\cO_\Cp$-algebras
      $$A_{\infty, \cO_\Cp}^\alpha \to A_{\infty, \cO_\Cp}^{\det(g)^{-1} \Nrd(d) \alpha},
    \quad X_i \mapsto (g,d^{-1})^*(X_i) \ \text{for} \ i = 1, \dots,n.$$
    \item $(1, \vartheta^{-m}, \sigma)$ acts by the morphism of $\cO_E$-algebras 
      $A_{\infty, \cO_\Cp}^\alpha \to A_{\infty, \cO_\Cp}^{\Art_E^{-1}(\sigma)\alpha}$
      given by $X_i \mapsto X_i$ for $i = 1, \dots, n$ and $a \mapsto
      \sigma(a)$ for $a \in \cO_\Cp$.
  \end{itemize}
\end{prop}
\begin{proof}
  This follows from the description of the action on $\Xi_n$ and the fact
  that $\delta((g,d)^*\allowbreak (X_1, \dots, X_n)) = \tau$ is equivalent to 
  $\delta(X_1, \dots, X_n) = [\det(g)^{-1} \Nrd(d)]_{\wedge H}(t)$ by the fact
  that $\delta$ is multilinear and alternating, and
  \cref{lem:WeinsteinDeterminantAndNorm}. 
  By the construction of the Artin map, the corresponding statement on systems
  of $q$-th power roots is $\Delta(X_1, \dots, X_n) =
  \Art_E(\det(g)^{-1}\Nrd(d))(\tau)$. The remaining statements are immediate.
\end{proof}

In particular, this gives an explicit description of the action of 
the group $G^1 = \{(g,d,\sigma) \in G \mid \theta(g,d,\sigma) = 1\}$ on 
$A^1_{\infty, \cO_\Cp}$. 

% subsubsection Making the Group Actions Explicit (end)

% subsection Tate Modules and the Universal Cover (end)


\end{document}
