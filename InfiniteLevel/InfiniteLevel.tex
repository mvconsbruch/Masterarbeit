%! TeX root: ../main.tex
\documentclass[../main.tex]{subfiles}

\begin{document}
\section{The Lubin--Tate Space at Infinite Level}
In this section, we introduce, attached to a formal group law
$H_0 \in \FMLOver {\cO_E}{\Fqbar}$ of height $n \in \N$, the infinite level
deformation moduli problem
\begin{equation*}
  \cM_{\infty}^{(0)} = \lim_{m \in \N} \left(\cM_{m}^{(0)}\right).
\end{equation*}
We have seen in the previous section that $\cM_{m}^{(0)}$ is 
representable by the formal spectrum of a local ring $A_m$, finite and \'etale
over $\spf(A_0) = \spf(\cO_{\br E}\llbr u_1, \dots, u_{n-1}\rrbr)$. In particular,
$\cM_{\infty}^{(0)}$ is represented by 
the formal spectrum of the completed colimit 
\begin{equation*}
  A_\infty = (\colim_m A_m)^\wedge_{\fm}.
\end{equation*}
Here, $\fm$ denotes the image of the maximal ideal of $A_0$ (or of any $A_m$, it doesn't
matter). Note that $\colim_m A_m$ is not Noetherian, so the completion 
along an arbitrary ideal $I \subset A_\infty$ is, in general, not $I$-adically
complete (see \cite[\href{https://stacks.math.columbia.edu/tag/05JA}{Tag
05JA}]{stacks-project} for an example). However, we have seen that $\fm$ is
finitely generated, so this pathology does not occur and $\cM_\infty^{(0)}
\coloneqq \spf A_\infty$ makes senes as a formal scheme.

Let $G$ denote the product $\GL_n(E) \times D^\times \times \Weil_E$
and write $G^0 \subset G$ for the subgroup given by the kernel of the homomorphism
\begin{equation*}
  G \to \Z, \quad (g, d, \sigma) \mapsto \val( \det(g)^{-1} \Nrd_{D/E}(d)
  \Art_E^{-1}(\sigma)). 
\end{equation*}
The action of $G$ on the tower $\{\cM_{m, \cO_{\C_p}}\}_m$ constructed in
Section \ref{ssub:The
Tower of Deformation Spaces} restricts to an action of $G^0$ on the tower
$\{\cM_{m, \cO_{\C_p}}^{(0)}\}_m$. This yields a right action of $G^0$ on the formal 
scheme $\cM_{\infty}\times_{\spf(\cO_\br E)} \spf(\cO_{\C_p})$, equivalent to
a left action on the ring $A_\infty \cotimes_{\cO_\br E} \cO_{\C_p}$.

This section is concerned with the study of the ring $A_\infty$. 
The main results are summarized as follows. 
First, we review some of the constructions in Chapter 2 of \cite{weinstein2016semistable}.
As a first step, making use of the determinants of formal $\cO_E$-modules constructed in 
\cite{hedayatzadeh2015det}, we obtain a natural homomorphism
\begin{equation*}
   \cO_{\widehat{E^\ab}} \to  A_\infty.
\end{equation*}
We then introduce the notion of universal covers of formal $\cO_E$-modules; the 
passage from $H$ to its universal cover $\tilde H$ can be interpreted as a sort
of tilting procedure. Still following \cite{weinstein2016semistable}, this
notion and its relation with the moduli problem $\cM_{\infty,
\cO_{\C_p}}^{(0)}$ makes it possible to construct an isomorphism
\begin{equation*}
  A_{\infty, \cO_{\C_p}} \coloneq A_\infty \cotimes_{\cO_{\widehat{E^\ab}}} \cO_{\C_p} 
  \cong \cO_{\C_p} \llbr X_1^{q^{-\infty}}, \dots, X_n^{q^{-\infty}} \rrbr /
  (\delta(X_1^{q^{-\infty}}, \dots, X_n^{q^{-\infty}})^{q^{-m}} - \tau^{q^{-m}} \mid 
  m \in \N)^- ,
\end{equation*}
where the superscript minus denotes the completion of the ideal. 

The main effort is taken in an description of the action by $G^0$ 
on $A_{\infty,\cO_{\C_p}}$ in terms of this isomorphism following
\cite[Section 1.2]{imaitsushima2020affinoids}, 
and an approximation of $\delta$ following \cite[Section
2.10]{BoyarchenkoWeinstein2011MaxVar}.
These results give the necessary information to observe the perfection of a
Deligne--Lusztig variety as the special fiber of some affinoid inside the
Lubin--Tate perfectoid space. This space is constructed at the end of this section.

\subsection{Determinants of Formal modules} % (fold)
\label{sub:Determinants of Formal modules}
bla bla
% subsubsection Determinants of Formal modules (end)

\subsection{The Universal Cover} % (fold)
\label{sub:Tate Modules and the Universal Cover}
\subsubsection{Useful Calculations} % (fold)
\label{ssub:Useful Calculations}
Let $p$ be a prime. Let $R$ be a Noetherian local ring with maximal ideal $I$
such that $p \in I$, $R$ is complete with respect to the $I$-adic topology and
$k_R \coloneqq R/I$ is an algebraically closed field (necessarily of
characteristic $p$). If $q$ is a power of $p$, we write $\cF_{R,q}$ for the set
of power series $f \in R\llbr T \rrbr$ satisfying 
\begin{equation} \label{eq:condonpowerseries}
  f(T) \equiv g(T^q) \pmod I
\end{equation}
for some power series $g(T) = c_1 T + c_2 T^2 + \dots \in R\llbr T \rrbr$ with 
$c_1 \in R^\times$. 
If $q'>q$ is another power of $p$, we have injections $\cF_{R,q} \inj \cF_{R,q'}$
given by sending $f(T)$ to its $(q'/q)$-fold self-composite $f^{q'/q}(T)$. 
Making use of these transition maps, we define
\begin{equation*}
  \cF_R \coloneqq \colim_{n \in \N} \cF_{R, p^n},
\end{equation*}
identifying any power series $f \in \cF_{R,q}$ with its image in $\cF_{R,q'}$ for 
higher $p$-powers $q'$. 
For any $f \in \cF_{R,q}$, we define the functor
\begin{equation*}
  U_f: \Adm R \to \Set, \quad S \mapsto \left\{(x_0,x_1, \dots) \in \prod_\N S^\cici 
                                          \mid f(x_{i+1}) = x_i\right\}.
\end{equation*}
This functor does, up to canonical isomorphism, only depend on the equivalence
class of $f$ in $\cF_{R}$. 
We write $U_{0,f}$ for the base change of $U_f$ to $k_R$, that is
\begin{equation*}
  U_{0,f}: \Adm {k_R} \to \Set, \quad S \mapsto \left\{(x_0,x_1, \dots) \in
                              \prod_\N S^\cici \mid \bar f(x_{i+1}) = x_i\right\}.
\end{equation*}
Here, $\bar f$ is the image of $f$ under the reduction map $R\llbr T \rrbr \to
k_R\llbr T \rrbr$. 

In the sequel, we denote $R$-algebras by $S$ and write $J$ for an ideal
of definition containing the image of $I$ (provided, for example, by \ref{lem:iodimage}).
Given an element $f\in \cF_R$, we do not distinguish between $f$ and a choice of a 
representative $\tilde f \in \cF_{R,q}$ for some sufficiently large $p$-power.

The following observation lays the groundwork for many of the upcoming results.
\begin{lem}\label{lem:cryscalc}
  Let $f$ be any power series in $\cF_R$. For any two elements $s_1,s_2 \in S$ 
  with $s_1 \equiv s_2 \mod J$ such that $f(s_1)$ and $f(s_2)$ exist (for
  example if $f$ is a polynomial or $s_1, s_2 \in S^\cici$), we have 
  \begin{equation*}
    f^k(s_1) \equiv f^k(s_2) \pmod {J^{k+1}}.
  \end{equation*}
  Here, $f^k$ denotes $k$-fold composition of $f$.
\begin{proof}
  We will show that if $s_1 \equiv s_2$ mod $J^k$, then $f(s_1) \equiv f(s_2)$ mod 
  $J^{k+1}$, which suffices to prove the claim. 
  We may write $s_2 = s_1 + r$ for some $r\in J^k$. By the assumptions on $f$
  there exist power series 
  $g,h \in R\llbr T \rrbr$ such that $h$ only
  has coefficients in $I$ and $f(T) = g(T^q) + h(T)$. As $I$ is finitely generated,
  say by elements $(r_1, \dots, r_l)$, we obtain a representation 
  \begin{equation*}
    f(s_1) - f(s_2) \in g(s_1^{q})-g(s_2^{q}) + \sum_{i=1}^l r_i \left(h_i(s_1) -
    h_i(s_2)\right).
  \end{equation*}
  As $r$ divides $\left(h_i(s_1) - h_i(s_2)\right)$, we find
  $r_i(h_i(s_1) - h_i(s_2)) \in (r_i r) \subseteq J^{k+1}$. Also note that 
  for any $s\in S$ and $n \in \N$, 
  $$(s+r)^{nq} = s^{nq} + nqrs^{nq-1}r + \dots + r^{nq},$$
  so after cancellation, all monomials of $g(s_1^q) - g(s_2^q)$ lie in
  $(qr)$ or $(r^2)$. This implies
  \begin{equation*}
    g(s_1^q) - g\left((s_1+r)^q\right) \in (qr) + (r^2) \subseteq J^{k+1},
  \end{equation*}
  and we are done.
\end{proof}
\end{lem}

\begin{lem}\label{lem:reductioniso}
  The natural reduction map 
  \begin{equation*}
    U_f(S) \to U_{f}(S/J) = U_{0,f}(S/J)
  \end{equation*}
  is bijective.
\begin{proof}
  We first show surjectivity. Given a sequence $(x_0, x_1, \dots) \in U_{f}(S/J)$, 
  we can choose a sequence of arbitrary lifts $(y_0, y_1, \dots ) \in \prod_\N
  S^\cici$ and set 
  \begin{equation*}
    z_i = \lim_{r \to \infty} f^r(y_{i+r}).
  \end{equation*}
  The limit exists, because if $s\geq r$ are two non-negative integers, we calculate
  \begin{equation*}
    f^{s-r}(y_{i+s}) \equiv \bar f^{s-r}(x_{i+s}) = x_{i+r} \equiv y_{i+r}
    \pmod J,
  \end{equation*}
  implying by Lemma \ref{lem:cryscalc} that 
  \begin{equation*}
    f^{s}(y_{i+s}) \equiv f^r(y_{i+r}) \pmod{J^r}.
  \end{equation*}
  This shows that $(f^{r}(y_{i+r}))_{r \in \N}$ is a Cauchy-sequence for the 
  $J$-adic topology on $S$, thereby convergent (cf. Lemma
  \ref{lem:AdmAdicComp}). The sequence $(z_0, z_1, \dots)$ now lies in $U_f(S)$
  and lifts $(x_0, x_1, \dots)$. It remains to show that the lift is unique.
  Suppose that $(z'_0, z'_1, \dots)$ is another lift. Then, for any $i,k \in\N$
  we have $z_{i+k} \equiv z'_{i+k}$ mod $J$, and another application of Lemma
  \ref{lem:cryscalc} shows that 
  \begin{equation*}
    z_i = f^k(z_{i+k}) \equiv f^k(z'_{i+k}) = z'_i \pmod {J^k}.
  \end{equation*}
  Thereby $(z_i - z'_i) \in \bigcap_{k \in \N} J^k = \{0\}$. Hence,
  the lift is unique.
\end{proof}
\end{lem}

We write $\Nilp^\flat$ for the functor $U_{T^q}$. That is, 
$\Nilp^\flat(S) = \lim_{x \mapsto x^q}S^\cici$ is 
the set of $q$-power compatible sequences with values in $S^\cici$. 

\begin{lem}\label{lem:nilp0iso}
  For any $f \in \cF_R$, there is a canonical \todo{Use different $S$}
  bijection $U_{0,f}(S/J) \to \Nilp^\flat(S/J)$. This bijection is functorial in 
  $S$.
\begin{proof}
  By assumption on $f$ we have $f(T) = g(T^{q}) \in k_R\llbr T \rrbr$ for some 
  $g(T) = c_1T + c_2T^2 + \dots$ with $c_1 \neq 0$. For each coefficient $c_i$, let
  $d_i \in k_R$ be the unique element such that $d_i^{q} = c_i$. Let
  $h(T) \in k_R\llbr T \rrbr$ be the power series given by $d_1 T + d_2
  T^2 + \dots$. Now $(h(T))^{q}=f(T)$, and we find that 
  \begin{equation*}
      U_f(S/J) \to \Nilp^\flat(S/J): \quad
      (x_1, x_2, x_3, \dots) \mapsto (x_1, h(x_2), h(h(x_3)), \dots)
  \end{equation*}
  is a well-defined function, and functorial in $S$. For the
  inverse, let $h^{-1}(T) \in k_R\llbr T \rrbr$ be the unique power
  series with $h^{-1}(h(T))= h(h^{-1}(T)) = T$, see Lemma
  \ref{lem:IsosCheckOnLie}. The map
  \begin{equation*}
      \Nilp^\flat(S/J) \to U_f(S/J), \quad 
      (x_1, x_2, \dots ) \mapsto (x_1, h^{-1}(x_2), h^{-1}(h^{-1}(x_3)), \dots)
  \end{equation*}
  is well-defined as
  \begin{equation*}
      f(h^{-1}(T)) = g((h^{-1}(T))^{q}) = (h(h^{-1}(T)))^{q} =
      T^{q},
  \end{equation*}
  and it is readily seen to be inverse to the map constructed above.
\end{proof}
\end{lem}

We collect results.
\begin{prop}\label{prop:pHTcalc}
  Given $f,g \in \cF_R$, we have bijections, functorial in $S$,
  \begin{equation} 
    U_f(S) \to U_f(S/J) \to \Nilp^\flat(S/J) \to U_g(S/J) \to U_g(S).
  \end{equation}
  Explicitly, the bijection $U_f(S) \to U_g(S)$ can be described as follows.
  Suppose that $f,g \in \cF_{R,q}$ for some sufficiently large $q$. 
  Let $h_f(T)$ and $h_g(T)$ be power series with coefficients in $A$ such that 
  $$h_f(T)^q \equiv f(T) \pmod I\quad\text{and}\quad h_g(T)^q \equiv g(T) \pmod
  I.$$
  Write $h_g^{-1}(T)$ for the (formal) inverse power series of $h_g$. 
  Now the isomorphism is given by the mapping
  \begin{equation*}
    (x_0, x_1, \dots) \mapsto (y_0, y_1, \dots), \quad \text{where} \quad y_i =
    \lim_{r \to \infty} g^r(h_g^{-(r+i)}(h_f^{r+i} (x_{i+r}))).
  \end{equation*}
  Here, the exponents are to be interpreted as iterated composition.
\begin{proof}
  The first part follows directly from repeated application of the previous
  two Lemmas. The second part follows by tracing through the previous lemmas.  
\end{proof}
\end{prop}
% subsubsection Useful Calculations (end)

\subsubsection{The Universal Cover} % (fold)
\label{ssub:The Universal Cover}
Let $A$ be an integral domain and $R$ be an $A$-algebra. Given $H \in \FMOver
AR$ and $a \in A$,
we define the functor 
\begin{equation*}
  \Tilde H_a : \Adm R \to \Mod {A}, \quad
  S \mapsto \left\{(x_1, x_2, \dots) \in \prod_{\N} H(S) \mid [a]_H(x_{i+1}) =
  x_i \right\}.
\end{equation*}
Here, the $A$-module structure is given by $b.(x_1, x_2,\dots) = ([b]_H(x_1), [b]_(x_2),
\dots)$. Note that multiplication by $a$ on $\tilde H_a(S)$ is an automorphism
(it sends $(x_1, x_2, \dots)$ to $([a]_H x_1, x_1, x_2, \dots)$, which has inverse given
by shifting to the left)  so that $\tilde H_a(S)$ is naturally an
$A[\frac1a]$-module.

From now on assume that $A$ is a discrete valuation ring with uniformizer
$\varpi$, finite residue field $k$ and field of fractions $K$. 
Write $q = \# k$. Let $R$ be a local $A$-algebra with maximal ideal $I$ and
algebraically closed residue field $k_R = R/I$. Let $H$ be a formal
$\varpi$-divisible $A$-module over $R$ of height $n$. 
\begin{defi}[The Universal Cover and Tate Module]
  We write $\tilde H = \tilde H_\varpi$. This functor
  takes values in the category of $K$-vector spaces.
  Up to natural isomorphism, $\tilde H$ does not depend on the choice of 
  $\varpi$. We call this functor the universal cover of $H$. 

  The Tate-Module $T_\varpi H$ is the subfunctor of $\Tilde H$ cut out out
  by the condition that $[\varpi]_H(x_1) = 0$. Note that $T_\varpi H$ does no longer 
  carry the structure of a $K$-vector space, it is an $A$-module. The Rational
  Tate Module $V_\varpi H$ is the subfunctor of $\Tilde H$ cut out by the
  condition that $x_1$ has $[\varpi]_H$-torsion. Equivalently, we have 
  \begin{equation*}
    V_\varpi H (S) = T_\varpi H(S) \otimes_A K.
  \end{equation*}
\end{defi}

\begin{lem}
  Let $H$ be a $\varpi$-divisible formal $A$-module over $R$ and write $H_0 = H
  \otimes_R k_R$. Now the choice of a coordinate on $H_0$ gives rise to
  an isomorphism 
  \begin{equation*}
    \tilde H_0 \cong \Nilp^\flat_{k_R}
  \end{equation*}
  of functors $\Adm {k_R} \to \Set$
  \begin{proof}
    Note that given any coordinate on $H$, we have $[\varpi]_H(T) \in \cF_R$. Hence,
    the statement is an application of Lemma \ref{lem:nilp0iso}.
  \end{proof}
\end{lem}

\begin{lem}
  Suppose that $S$ is an admissible $R$-algebra admitting an ideal of definition
  $J$ such that $\varpi \in J$. Then the natural reduction map
  \begin{equation*}
    \tilde H(S) \to \tilde H(S/J)
  \end{equation*}
  is an isomorphism.
  \begin{proof}[Proof]
    After choosing a coordinate on $H$, we have $[\varpi]_H \in \cF_R$ and 
    $\tilde H(S) \cong U_{[\varpi]_H}$, and the statement is given by 
    Lemma \ref{lem:reductioniso}.
  \end{proof}
\end{lem}

The following is analogous to Proposition \ref{prop:pHTcalc}.
\begin{prop}\label{prop:UnivCoverReductionIso}
  Let $S$ be an admissible $R$-algebra with ideal of definition $J$ such that 
  $\phi(I) \subseteq J$. Then there are canonical isomorphisms (of sets)
  \begin{equation*}
    \tilde H(S) \cong \tilde H(S/J) = \tilde H_0(S/J) \cong \Nilp^\flat(S/J) \cong
    \Nilp^\flat(S).
  \end{equation*}
  In particular, $\tilde H(S)$ is, as a functor to $\Set$, representable by
  $\spf(R \llbr T^{q^{-\infty}} \rrbr)$.
\end{prop}
We write $\lambda$ for the isomorphism $\Tilde H \to \Nilp^\flat$, and
$\lambda_i: \Tilde H \to (-)^\cici$ for projection on the $i$-th component.
Similarly, we write $\mu: \Nilp^\flat \to \Tilde H$ for the inverse of 
$\lambda$ and $\mu_i$ for the $i$-th component of $\mu$. 

By Proposition \ref{prop:UnivCoverReductionIso}, we obtain an action of 
$\End(H \otimes_R k_R)$ on $\Tilde H$.

\begin{defi}[Frobenius on $\Tilde H$] \label{def:FrobOnUnivCov}
  Write $\Pi: \Tilde H \to \Tilde H$ for the automorphism of $\Tilde H$ corresponding
  to the Frobenius quasi-isogeny (which sends $T$ to $T^q$) of $H_0$.
\end{defi}

Note that $\lambda_i(\Pi x) = \lambda_i(x)^q$ for $x \in \Tilde H(S)$ and 
$i=0,1,\dots$. 

\begin{rmk} 
  In case where $\cF$ comes from a $\cO_K$-module law $F$ over 
  $\cO_K$ with $[\varpi]_F(T) \equiv T^{q^n}$ mod $(\varpi)$,
    This follows directly from the explicit description of the isomorphism in
  Proposition \ref{prop:pHTcalc}, as we may choose $h_{[\varpi]_H}(T) = h_{T^{q^n}}(T)
  = T$.
\end{rmk}

By the isomorphism above and the fact that $\tilde H$ is a $K$-vector space
object, we obtain a left action of $K$ on $\Nilp^\flat$, which corresponds
to a right action on $R\llbr T^{q^{-\infty}} \rrbr$. We can make this action
explicit if $H$ comes from a Lubin--Tate formal module law.

\begin{lem}\label{lem:GroupStructureOnNilp}
  Suppose that $H$ is a formal module law over $R$ such that $[\varpi]_H(T) \equiv
  T^{q^n}$ mod $I$. Now, the following assertions hold.
  \begin{itemize}
    \item The bijections
        \begin{equation*}
          \lambda \colon \tilde H(S) \rightleftarrows \Nilp^\flat(S) \colon
          \mu, \quad (x_0, x_1, \dots) \mapsspamto (y, y^{q^{-n}}, y^{q^{-2n}},
          \dots)
        \end{equation*}
        are, in either direction, given by the equations
        \begin{equation*}
          y^{1/q^{ni}} = \lim_{r \to \infty} x_{r+i}^{q^{nr}} \quad \text{and} \quad 
          x_i = \lim_{s \to \infty} [\varpi^s]_H(y^{q^{-n(i+s)}}).
        \end{equation*}
    \item Given two $q$-th power compatible systems $y_1, y_2
      \in \Nilp^\flat(S)$ corresponding to compatible systems
      $\mu(y_1) = x_1, \mu(y_2) = x_2 \in \tilde H(S)$, the sum $x_1 + x_2 \in
      \tilde H(S)$ corresponds to the element $\lambda(x_1 + x_2) = z \in
      \Nilp^\flat(S)$, where 
      \begin{equation*}
        z^{1/q^j} = \lim_{r \to \infty} H( y_1^{q^{-r}}, y_2^{q^{-r}})^{q^{r-j}}.
      \end{equation*}
    \item Similarly, given $a \in \cO_E$ and $y \in \Nilp^\flat(S)$ with
      $\mu(y) = x \in \Tilde H(S)$, we have 
      \begin{equation*}
        \lambda([a]_H(x)) = \lim_{r \to \infty} [a]_H(y^{q^{-r}})^{q^{r-j}}.
      \end{equation*}
  \end{itemize}
\begin{proof}
  The first statement is an immediate corollary of the second part of
  Proposition \ref{prop:pHTcalc} applied with $f(T) = [\varpi]_H(T)$ and $g(T)
  = T^{q^n}$, as we may choose $h_f(T) = h_g(T) = T$. 
  The second assertion follows directly after tracing through the commutative diagram
  \begin{equation*}
    \begin{tikzcd}[ampersand replacement=\&]
    	\& {\Tilde H(S)^2} \& {\Tilde H(S)} \\
    	{\Nilp^\flat(S)^2} \& {\Tilde H_0(S/J)^2} \& {\Tilde H_0(S/J)} \& {\Nilp^\flat(S).}
    	\arrow["{H(-,-)}", from=1-2, to=1-3]
    	\arrow[ from=1-2, to=2-2]
    	\arrow[ from=1-3, to=2-3]
    	\arrow["\lambda", from=1-3, to=2-4]
    	\arrow["\mu", from=2-1, to=1-2]
    	\arrow[ from=2-1, to=2-2]
    	\arrow["{H_0(-,-)}", from=2-2, to=2-3]
    	\arrow[from=2-3, to=2-4]
    \end{tikzcd}
  \end{equation*}
  Similarly one proves the third statement.
\end{proof}
\end{lem}

We add calculations regarding the interplay of $\lambda$ and $\log_H$ which will prove
useful later.
\begin{lem}\label{lem:LogInTermsOfNil}
  Let $H$ be the standard formal $\cO_K$-module of height $n$ over $R =
  \cO_{\breve K}$. We have a commutative diagram (cf. \cite[Lemma
  2.6.1]{BoyarchenkoWeinstein2011MaxVar})
  \begin{equation*}
  \begin{tikzcd}[ampersand replacement=\&]
  	{(x_0, x_1, \dots)} \& {\Tilde H(S)} \&\& {\Nil^\flat(S)} \& {(y,y^{1/q}, \dots)} \\
  	{\sum_{i=0}^\infty \frac{x_0^{q^{ni}}}{\varpi^i}} \&\& {S\otimes_{\cO_K}K} \&\& {\sum_{i = -\infty}^\infty\frac{y^{q^{ni}}}{\varpi^i}}
  	\arrow["\in"{description}, draw=none, from=1-1, to=1-2]
  	\arrow[maps to, from=1-1, to=2-1]
  	\arrow["\lambda", from=1-2, to=1-4]
  	\arrow["{\log_H}"', from=1-2, to=2-3]
  	\arrow[from=1-4, to=2-3]
  	\arrow["\ni"{description}, draw=none, from=1-5, to=1-4]
  	\arrow[maps to, from=1-5, to=2-5]
  \end{tikzcd}
  \end{equation*}
  With this terminology, we have $\log_H((\Pi^j x)_0) = \sum_{i = -\infty}^\infty
  \frac{ y^{ni + j}}{\varpi^i}$. 
\begin{proof}
  This follows directly from the remark above. Let $x \in \Tilde H(S)$ and write
  $(y, y^{1/q}, \dots)$ for $\lambda(x)$. We have $x_0 = \lim_{s \to \infty}
  [\varpi^s]_H(y^{-ns})$, hence
  \begin{equation*}
    \log_H(x_0) = \lim_{s \to \infty}\left( \varpi^s \log_H(y^{1/q^{ns}})\right) = 
    \lim_{s \to \infty}\left( \sum_{i=0}^\infty  \frac{y^{q^{n(i-s)}}}{\varpi^{i-s}}\right)
    = \sum_{i=-\infty}^\infty \frac{y^{q^{ni}}}{\varpi^i}.
  \end{equation*}
  The second part is an immediate consequence.
\end{proof}
\end{lem}
% subsubsection The Universal Cover (end)

\subsubsection{Group Actions on the Universal Cover} % (fold)
\label{ssub:Group Actions on the Universal Cover}
We quickly note that $\tilde H \otimes_{\cO_\br E} \cO_{\C_p}$ comes with commuting actions
by the groups 
$D^\times$ and $W_E$. 
Both these actions come from the isomorphism of functors $\tilde H \cong \tilde H_0$. 

By this isomorphism, we find that any $d \in \cO_D = \End_{\FMLOver
{\cO_E}{\Fqbar}}(H_0)$ yields a morphism $\tilde H \to \tilde H$. As the morphism
corresponding by $\varpi$ is an isomorphism, we obtain a (right-) action of 
$D^\times \subset \cO_D[\varpi^{-1}]$ on $\tilde H$. 

There is also an action of $\GL_n(E)$ on $\tilde H^n$. Given 
$g \in \GL_n(E)$ with entries $a_{ij} \in E$, a complete, local $\cO_{\br E}$-algebra
$R \in \cC$ and a section $(x_1, \dots, x_n) \in \tilde H^n(R)$, we set
\begin{equation*}
  g.(x_1, \dots, x_n) = (y_1, \dots, y_n), \quad \text{where}\quad
  y_i = \sum_{j=1}^n{}^{(H)} a_{ij}x_j.
\end{equation*}
% subsubsection Group Actions on the Universal Cover (end)

\subsubsection{Relation to the Deformation Space at Infinite Level} % (fold)
\label{ssub:Relation to the Deformation Space at Infinite Level}
As usual, let $H$ denote the standard formal $\cO_E$-module law over 
$\cO_\br E$ of height $n$, and let $H_0$ be its reduction to 
$\Fqbar$. 

Let $(e_1, \dots, e_n)$ denote the standard basis of $\cO_E^n$. Given $R \in
\cC$ and a sequence of elements
\begin{equation*}
  (F, \iota, \phi_m)_m \in \cM_\infty(R),
\end{equation*}
the corresponding sequence of Drinfeld basis vectors $x_i^{(m)} =
\phi_m(\varpi^{-m}e_i) \in F(R)$ yield an element
\begin{equation*}
  \mathbf{x}_i \coloneqq \left(x_i^{(0)}, x_i^{(1)}, x_i^{(2)}, \dots \right)
  \in \tilde F(R). 
\end{equation*}

By the isomorphism $\iota: F \otimes \Fqbar \to H_0$ and Proposition
\ref{prop:UnivCoverReductionIso}, there is a canonical
isomorphism $\tilde F \cong \tilde H$, so we may consider
$\mathbf x_i$ elements of $\tilde H(R)$. We obtain a map
\begin{equation*}
  \cM_\infty(R) \to \tilde H(R), \quad (F, \iota, \phi_m)_m \mapsto 
  (\mathbf x_1, \dots, \mathbf x_n) \in \tilde H(R),
\end{equation*}
functorial in $R$.
\begin{lem}\label{lem:EquivarianceOfMap}
  The functor constructed above is equivariant for the action of 
  $\GL_n(E) \times D^\times \times \Weil_E$. 
\begin{proof}
  \red{todo.}
\end{proof}
\end{lem}

% subsubsection Relation to the Deformation Space at Infinite Level (end)

% subsection Tate Modules and the Universal Cover (end)

\subsection{The Quasilogarithm Map} % (fold)
\label{sub:The Quasilogarithm map}
We keep the assumptions on $A$, $R$ and $S$ from the previous subsection. That is,
$A$ is a local ring with finite residue field and uniformizer $\varpi$, 
$R$ is a local $A$-algebra with maximal ideal $I$ complete with respect to
the $I$-adic topology and algebraically closed residue field $k_R$, and 
$S$ denotes an admissible $R$-algebra (where $R \to S$ is continuous with
the $I$-adic topology on $R$) with ideal of definition $J \subseteq S$ containing
the image of $I$. 

The aim of this subsection is to define, attached to any $\varpi$-divisible formal
$A$-module $\cF$ over $R$, a map
\begin{equation*}
\qlog_\cF: \tilde \cF(S) \to \Dio(\cF) \otimes_R (S \otimes_A K),
\end{equation*}
called the quasi-logarithm map.
We give an explicit description of this map if $\cF = \FGG(H)$ is the standard
$\cO_K$-module over $\cO_{\breve K}$. 

The construction of $\qlog_\cF$ is as follows.  
Let $0 \to \cV \xto\psi \cE \xto\phi \cF \to 0$ be the universal additive
extension of $\cF$. For any sequence $(x_1, x_2, \dots) \in \tilde \cF(S)$, choose an arbitrary sequence $(y_1, y_2, \dots) \in
\tilde \cE(S)$ such that $y_i$ is a lift of $x_i$ under the map $\cE(S) \to \cF(S)$. 
Let $y$ be the limit $y = \lim_{i \to \infty} [\varpi]_{\cE}^i(y_i)$ and put 
$$\qlog_\cF((x_1, x_2, \dots)) = \log_\cE(y) \in \Dio(\cF) \otimes_R(S \otimes_A K).$$ 

\begin{prop}
  This construction yields a well-defined map. 
\begin{proof}
  We may assume that $\cF$ and $\cV$ come from formal module laws $F$ and 
  $V$, and we may furthermore assume that $\cE = \FGG(E)$ 
  for an $\cO_K$-module law $E$ obtained by Lemma \ref{lem:SESStandardForm}. Now
  $(x_1, x_2, \dots)$ is a sequence in $S^\cici$ and $(y_1, y_2, \dots)$ is a
  sequence of elements in $(S^\cici)^n$.

  It suffices to show that $y = \lim_{i\to \infty} [\varpi]^i_E(y_i)$ exists and 
  that it is independent of the choice of lifts $(y_1, y_2, \dots)$. 
  Both claims follow from the additivity of $\cV$, implying that 
  $[\varpi]_{V}(T) = \varpi T$. The sequence 
  $([\varpi^i]_{E}(y_i))$ converges, as for positive integers $i\leq j$, we have 
  \begin{equation*}
    [\varpi^i](y_i) - [\varpi^j](y_j) = [\varpi^i]([\varpi^{i-j}]y_j - y_i)
    \in \psi(\varpi^i (S^\cici)^{n-1}) \subseteq J^i (S^\cici)^{n}. 
  \end{equation*}

  If $(y_1', y'_2, \dots)$ is another sequence of lifts, put
  $y' = \lim_{i \to \infty} [\varpi^i]_E (y_i') \in S^\cici$. Now there exists some
  $z \in \cV(S)$ such that 
  $y - y' = \psi(z)$. But by construction
  $z \in \bigcap_{i \in \N} \varpi^i(S^{\cici})^{n-1} = 0$.
\end{proof}
\end{prop}

Let us now consider the case where $\cF = \FGG(H)$ comes from the standard
formal $\cO_K$-module of height $n$ over $\cO_{\breve K}$. Then 
from Proposition \ref{prop:InterestingSequenceStdMod} we have the distinguished
basis elements of $\Ext(H,\Ghat_a)$ corresponding to the symmetric $2$-cocycles
$\delta f_i$, $1 \leq i \leq n-1$ where $f_i(T) = \frac 1 \varpi
\log_H(T^{q^i})$. Also recall that, setting $f_0(T) = \log_H(T)$, the elements 
$(f_0, f_1, \dots, f_{n-1})$ freely generate $\QLog(H)$. The universal
additive extension now corresponds to the symmetric $2$-cocycle $(\delta f_1,
\dots, \delta f_{n-1}) \in \SymCoc^2(H,V)$.
We can make the quasi-logarithm map explicit.

\begin{prop}\label{prop:qlogmapExplicit}
  Let $x = (x_0, x_1, \dots) \in \Tilde H(S)$. With respect to the basis
  $(\log_H(T),\allowbreak \log_H(T^q), \allowbreak \dots, \allowbreak \log_H(T^{q^{n-1}}))$ of 
  $\QLog(H) \otimes_{\cO_K} K$, the quasi-logarithm map is given by
  \begin{equation*}
    \qlog_H(x) = \left(\log_H(x_0), \log_H((\Pi x)_0), \dots, \log_H((\Pi^{n-1}
    x)_0)\right) \in (S\otimes K)^n.
  \end{equation*}
  Here, $\Pi x = ((\Pi x)_0, (\Pi x)_1, \dots)$ is the image of $x$ under
  $\Pi$, the automorphism of $\Tilde
  H(S)$ induced by the (relative) Frobenius quasi-isogeny on $H_0$, cf. Definition
  \ref{def:FrobOnUnivCov}.
\end{prop}
We postpone the proof to state the following auxiliary result.
\begin{lem}\label{lem:FrobOnTildeHExpl}
  Let $x = (x_0 ,x_1, \dots) \in \Tilde H(S)$. For positive integers $i$ and
  $j$ we have
  \begin{equation*}
    \log_H((\Pi^j x)_i) = \lim_{r \to \infty} \varpi^r \log_H(x_{r+i}^{q^j}).
  \end{equation*}
\begin{proof}
  Tracing through the commutative square (with $\lambda$ and $\mu$ the
  isomorphisms from the previous subsection)
    \begin{equation*}
    \begin{tikzcd}[ampersand replacement=\&]
    	{\Tilde H(S)} \& {\Nilp^\flat(S)} \\
    	{\Tilde H(S)} \& {\Nilp^\flat(S),}
    	\arrow["\lambda", from=1-1, to=1-2]
    	\arrow["\Pi", from=1-1, to=2-1]
    	\arrow["{(y_i)_i \mapsto (y_i^{q})_i}", from=1-2, to=2-2]
    	\arrow["\mu"', from=2-2, to=2-1]
    \end{tikzcd}
    \end{equation*}
    we find 
    \begin{equation} \label{eq:CompsOfPiExplicit}
      (\Pi^j x)_i = \lim_{s \to \infty} \lim_{r\to\infty} \left([\varpi]_H^s
      (x_{r+s+i}^{q^{nr + j}})\right).
    \end{equation}
    The claim follows after applying $\log_H$ and making repeated use of the
    functional equation $\log_H(T^{q^n}) = \varpi \log_H(T) + \varpi T$.
\end{proof}
\end{lem}
\begin{proof}[Proof of Proposition \ref{prop:qlogmapExplicit}]
  Using the coordinates provided by $(\delta f_1, \dots, \delta f_{n-1})$, the
  universal additive extension of $H$ is isomorphic to 
  \begin{equation*}
    0 \to \Ghat_a^{n-1} \to E \to H \to 0,
  \end{equation*}
  where $E$ is a module law with 
  \begin{equation*}
    [\varpi]_E(\bX, T) = \big(\varpi X_1 + (\delta_{\varpi}f_1)(T), \dots, \varpi X_{n-1} + 
    (\delta_{\varpi}f_{n-1})(T), [\varpi]_H(T)\big).
  \end{equation*}
  Beginning with $x = (x_0, x_1, \dots) \in \Tilde H (S)$, lifting to $(y_0,
  y_1, \dots) \in E(S)^\N$ 
  and writing $y = \lim_{i\to\infty} [\varpi]_E^i (y_i)$, we find
  \begin{equation*}
    y = \left(\lim_{r\to\infty} (\delta_{\varpi^r} f_1)(x_r), \dots, \lim_{r\to \infty}
    (\delta_{\varpi^r}f_{n-1})(x_r), x_0\right) \in E(S).
  \end{equation*}
  Now, Lemma \ref{lem:FrobOnTildeHExpl} provides the equality
  $$\lim_{r\to\infty}\delta_{\varpi^r} f_i(x_r) = \frac 1\varpi \lim_{r \to \infty}
  \varpi^r \log_H(x_r^{q^{nr+i}}) - \frac 1\varpi \log_H\left(x_0^{q^i}\right) = 
  \frac 1\varpi \left(\log_H((\Pi^i x)_0)- \log_H(x_0^{q^i})\right).$$
  We need to calculate $\log_E(y)$, which calls for an explicit description of 
  $\log_E : E \otimes (R \otimes_AK) \to (\Ghat_a \otimes(R \otimes_A K))^n$. 
  Tracing through the procedure provided in Subsection \ref{sub:Logarithms},
  we find
  \begin{equation*}
    \log_E(\bX, T) = \left(X_1 + \tfrac 1\varpi \log_H(T^q), \dots, 
    X_{n-1} + \tfrac 1\varpi \log_H(T^{q^{n-1}}), \log_H(T)\right).
  \end{equation*}
  This representation is with respect to the basis $(f_1, \dots, f_{n-1}, f_0)$. 
  The claim follows.
\end{proof}



% subsubsection The Quasilogarithm map (end)


\subsection{Determinants} % (fold)
\label{sub:Determinants}
Let $H$ be the standard formal $\cO_K$-module over $\cO_{\breve K}$ of height
$n$. Write $\wedge H$ for the formal $\cO_K$-module over $\cO_{\breve K}$ with
logarithm
\begin{equation*}
  \log_{\wedge H}(T) = \sum_{i = 0}^\infty (-1)^{(n-1)i} \frac{T^{qi}}{\varpi^i}.
\end{equation*}
By Hazewinkel's integrality Lemma (cf. Theorem \ref{thm:HazewinkelIntegrality}), 
such a module law exists. We have $\Dio(\wedge H) = \wedge^n \Dio(H)$. 
We follow \cite[Theorem 2.10.3]{BoyarchenkoWeinstein2011MaxVar} to describe a map $\delta:
\Tilde H^n \to \Tilde {\wedge H}$ making the square
\begin{equation}\label{diag:qlogsquare}
\begin{tikzcd}[ampersand replacement=\&]
	{\Tilde H^n(S)} \& {\Tilde{\wedge H}(S)} \\
	{\Dio(H)^n \otimes (S\otimes_{\cO_K} K)} \& {\Dio(\wedge H) \otimes(S\otimes_{\cO_K}K)}
	\arrow["\delta", from=1-1, to=1-2]
	\arrow["\qlog_H \times \dots \times \qlog_H"', from=1-1, to=2-1]
  \arrow["\qlog_{\wedge H}", from=1-2, to=2-2]
	\arrow["\det", from=2-1, to=2-2]
\end{tikzcd}
\end{equation}
commute. 

Let $(s_1, \dots, s_n) \in \Tilde H(S)^n$, and write $x_i = \lambda(s_i) \in
\Nilp^\flat(S)$, which are elements in $S^\cici$ with distinguished $q$-power
roots. Here $\lambda: \Tilde H \to \Nilp^\flat$ is the isomorphism from Section
\ref{sub:Tate Modules and the Universal Cover} with inverse $\mu = (\mu_0,
\mu_1, \dots)$. 
We set
\begin{equation*}
  \delta_0(s_1, \dots, s_n) = \sum_{(a_1, \dots, a_n)} \varepsilon(a_1, \dots
  a_n) \mu_0(x_1^{q^{a_1}} \cdots x_n^{q^{a_n}}) \in \wedge H(S),
\end{equation*}
where 
\begin{itemize}
  \item The sum takes place in ${\wedge H}(S)$.
  \item The sum ranges over $n$-tuples $(a_1, \dots, a_n)$ of (possibly negative) integers 
    satisfying $a_1 + \dots + a_n = n (n-1)/2$, subject to the
    condition that each $a_i$ occupies a distinct residue class modulo $n$.
  \item The expression $\varepsilon(a_1, \dots, a_n)$ denotes the sign of the 
    permutation $i \mapsto a_{i+1}$ (mod $n$) of $(0, \dots, n-1)$.
\end{itemize}

\begin{prop}
  The map $\delta_0$ makes the diagram 
  \begin{equation*}
    \begin{tikzcd}[ampersand replacement=\&]
    	{\tilde H^n(S)} \& {\wedge H(S)} \\
    	{\Dio(H)^n \otimes (S \otimes K)} \& {\Dio(\wedge H) \otimes(S \otimes K)}
    	\arrow["{\delta_0}", from=1-1, to=1-2]
    	\arrow["{\qlog_H^n}"', from=1-1, to=2-1]
    	\arrow["{\log_{\wedge H}}", from=1-2, to=2-2]
    	\arrow["\det", from=2-1, to=2-2]
    \end{tikzcd}
  \end{equation*}
  commute. It is $\cO_K$-multilinear and alternating.
\begin{proof}
  This is part of the proof of \cite[Theorem
  2.10.3]{BoyarchenkoWeinstein2011MaxVar}.
  Commutativity follows from 
  \begin{multline*}
    \log_{\wedge H}(\delta_0(s_1, \dots, s_n)) = \sum_{(a_1, \dots, a_n)} 
    \varepsilon(\mathbf a) \log_{\wedge H} \mu_0( x_1^{q^{a_1}}\cdots x_n^{q^{a_n}})
     \\ = \sum_{(a_1, \dots, a_n)} \varepsilon(\mathbf a) \sum_{m \in \Z} (-1)^{(n-1)m}
    \frac{x_1^{q^{a_1+m}} \cdots x_n^{q^{a_n+m}}}{\varpi^m} 
    = \det \left( \sum_{m \in \Z} \frac{x_i^{q^{mn + j-1}}}{\varpi^m} \right)_{\substack{
      1 \leq i \leq n,\\
      1 \leq j \leq n}},
  \end{multline*}
  which is equal to $\det(\qlog_H^n(s_1, \dots, s_n))$ by Proposition 
  \ref{prop:qlogmapExplicit} and Lemma \ref{lem:LogInTermsOfNil}. 
  The fact that $\delta_0$ is multilinear and alternating ultimately follows
  from the corresponding properties of $\det$, the fact that $\ker(\log_H) =
  \wedge H[\varpi^\infty]$ (cf. Lemma \ref{lem:KernelOfLog}) and topological considerations
  in the induced diagram in the category of adic spaces over $(\breve K, \cO_{\breve K})$.
\end{proof}
\end{prop}

This allows us to define the sought for morphism of functors
$\delta: \Tilde H^n \to \Tilde {\wedge H}$. 
\begin{defi}\label{def:DeltaMap}
  Put $\delta_i(s_1, \dots, s_n) = \delta_0(\varpi^{-i} s_1, \dots, s_n)$. Then 
  $\delta = (\delta_0, \delta_1, \dots)$ yields a map 
  $\Tilde H^n \to \Tilde{\wedge H}$. It is $K$-multilinear and alternating.
\end{defi}

Using the canonical identifications $\Tilde {H}^n \cong (\Nilp^\flat)^n$
and $\Tilde{\wedge H} \cong \Nilp^\flat$, the morphism $\delta$ yields a 
map $(\Nilp^\flat)^n \to \Nilp^\flat$, which in turn is the same as a power series 
\begin{equation*}
  \Delta(X_1, \dots, X_n) \in \cO_{\breve K}\llbr X_1^{q^{-\infty}}, \dots,
  X_n^{q^{-\infty}}\rrbr
\end{equation*}
together with distinguished $q$-th power roots. 
We have the following approximation of $\Delta$. 
\begin{lem}\label{lem:DeltaApprox}
  We have $\Delta(X_1, \dots, X_n) \equiv \det(X_i^{q^{j-1}})_{\substack{1 \leq i \leq n,\\ 1 \leq j \leq n}}$ modulo $(X_1, \dots, X_n)^{q^n}$.
\begin{proof}
  \red{TODO. This is \cite[Lemma 2.10.4]{BoyarchenkoWeinstein2011MaxVar}, but they
  don't explain the proof.}
\end{proof}
\end{lem}

\begin{thm}[]\label{thm:}
  There is a cartesian square
\begin{equation*}
\begin{tikzcd}[ampersand replacement=\&]
	{\cM_{H_0, \infty}} \& {\cM_{\wedge^n H_0, \infty}} \\
	{\tilde H^n} \& {\widetilde{\wedge H}}
	\arrow[from=1-1, to=1-2]
	\arrow[from=1-1, to=2-1]
	\arrow[from=1-2, to=2-2]
	\arrow["\delta", from=2-1, to=2-2]
\end{tikzcd}
\end{equation*}
Here, the vertical arrows are given by the morphism constructed in 
Section \ref{ssub:Relation to the Deformation Space at Infinite Level}. 
\begin{proof}
\end{proof}
\end{thm}

% subsubsection Determinants of Formal Modules (end)


% subsection Determinants of Formal Modules (end)

\end{document}
