%! TeX root: ../main.tex
\documentclass[../main.tex]{subfiles}

\begin{document}
\section{The Lubin--Tate Space at Infinite Level}
In this section, we introduce, attached to a formal group law
$H_0 \in \FMLOver {\cO_E}{\Fqbar}$ of height $n \in \N$, the infinite level
deformation moduli problem
\begin{equation*}
  \cM_{\infty}^{(0)} = \lim_{m \in \N} \left(\cM_{m}^{(0)}\right).
\end{equation*}
We have seen in the previous section that $\cM_{m}^{(0)}$ is 
representable by the formal spectrum of a local ring $A_m$, finite and \'etale
over $\spf(A_0) = \spf(\cO_{\br E}\llbr u_1, \dots, u_{n-1}\rrbr)$. In particular,
$\cM_{\infty}^{(0)}$ is represented by 
the formal spectrum of the completed colimit 
\begin{equation*}
  A_\infty = (\colim_m A_m)^\wedge_{\fm}.
\end{equation*}
Here, $\fm$ denotes the image of the maximal ideal of $A_0$ (or of any $A_m$, it doesn't
matter). Note that $\colim_m A_m$ is not Noetherian, so the completion 
along an arbitrary ideal $I \subset A_\infty$ is, in general, not $I$-adically
complete (see \cite[\href{https://stacks.math.columbia.edu/tag/05JA}{Tag
05JA}]{stacks-project} for an example). However, we have seen that $\fm$ is
finitely generated, so this pathology does not occur and $\cM_\infty^{(0)}
\coloneqq \spf A_\infty$ makes senes as a formal scheme.

Let $G$ denote the product $\GL_n(E) \times D^\times \times \Weil_E$
and write $G^0 \subset G$ for the subgroup given by the kernel of the homomorphism
\begin{equation*}
  G \to \Z, \quad (g, d, \sigma) \mapsto \val( \det(g)^{-1} \Nrd_{D/E}(d)
  \Art_E^{-1}(\sigma)). 
\end{equation*}
The action of $G$ on the tower $\{\cM_{m, \cO_{\C_p}}\}_m$ constructed in
Section \ref{ssub:The
Tower of Deformation Spaces} restricts to an action of $G^0$ on the tower
$\{\cM_{m, \cO_{\C_p}}^{(0)}\}_m$. This yields a right action of $G^0$ on the formal 
scheme $\cM_{\infty}\times_{\spf(\cO_\br E)} \spf(\cO_{\C_p})$, equivalent to
a left action on the ring $A_\infty \cotimes_{\cO_\br E} \cO_{\C_p}$.

This section is concerned with the study of the ring $A_\infty$. 
The main results are summarized as follows. 
First, we review some of the constructions in Chapter 2 of \cite{weinstein2016semistable}.
As a first step, making use of the determinants of formal $\cO_E$-modules constructed in 
\cite{hedayatzadeh2015det}, we obtain a natural homomorphism
\begin{equation*}
   \cO_{\widehat{E^\ab}} \to  A_\infty.
\end{equation*}
We then introduce the notion of universal covers of formal $\cO_E$-modules; the 
passage from $H$ to its universal cover $\tilde H$ can be interpreted as a sort
of tilting procedure. Still following \cite{weinstein2016semistable}, this
notion and its relation with the moduli problem $\cM_{\infty,
\cO_{\C_p}}^{(0)}$ makes it possible to construct an isomorphism
\begin{equation}\label{eq:ExplicitAinftyBCtoOC}
  A_{\infty, \cO_{\C_p}} \coloneq A_\infty \cotimes_{\cO_{\widehat{E^\ab}}}
  \cO_{\C_p} 
  \cong \cO_{\C_p} \llbr X_1^{q^{-\infty}}, \dots, X_n^{q^{-\infty}} \rrbr /
  (\delta(X_1^{q^{-\infty}}, \dots, X_n^{q^{-\infty}})^{q^{-m}} - \tau^{q^{-m}} \mid 
  m \in \N)^- ,
\end{equation}
where the superscript minus denotes the completion of the ideal. 

The main effort is taken in an description of the action by $G^0$ 
on $A_{\infty,\cO_{\C_p}}$ in terms of this isomorphism following
\cite[Section 1.2]{imaitsushima2020affinoids}, 
and an approximation of $\delta$ following \cite[Section
2.10]{BoyarchenkoWeinstein2011MaxVar}.
These results give the necessary information to observe the perfection of a
Deligne--Lusztig variety as the special fiber of some affinoid inside the
Lubin--Tate perfectoid space. This space is constructed at the end of this section.

\subsection{Determinants of Formal modules} % (fold)
\label{sub:Determinants of Formal modules}
In \cite{hedayatzadeh2015det}, Hedayatzadeh constructs determinants of 
$\varpi$-divisible formal $\cO_E$-modules over \red{properties} rings. 
We cite the result of main importance for us.
\begin{thm}[Determinants of Formal Modules]\label{thm:HedayatzadehsResult}
  ABCDE
\begin{proof}
\end{proof}
\end{thm}

Let $\cF_0 \in \FMOver {\cO_E}{\Fqbar}$ be a formal module of height $n$ and 
write $\wedge^n\cF_0$ for the associated determinant module, that is, the 
formal $\cO_E$ module with $\Dio(\wedge^n\cF_0) = \wedge^n\Dio(\cF_0)$. Write
$\cM_m$ for the Deformation space of $\cF_0$ with Drinfeld level
$\varpi^m$-structure, and write $\cM_{m, \wedge}$ for the deformation space of
$\wedge^n\cF$. 
Following \cite{weinstein2016semistable}, we sketch how this result can be
used to construct a functor $\cM_{m} \to \cM_{m, \wedge}$. 

For $R\in\cC$ and $(\cF, \iota) \in \cM_{0}(R)$, we write $\lambda_m$ for the
induced multilinear and alternating morphisms
\begin{equation*}
  \lambda_m: \cF[\varpi^m]^n \to \wedge^n\cF[\varpi^m].
\end{equation*}

We now have the following result.

\begin{lem}\label{lem:DeterminantOfDrinfeldStructure}
  Let $(x_1, \dots, x_n) \in \cF[\varpi^{m}]^n(R)$ be a Drinfeld level
  $\varpi^m$ structure. Then 
  \begin{equation*}
    \lambda_m(x_1, \dots, x_n) \in \wedge^n\cF[\varpi^m](R) 
  \end{equation*}
  is a Drinfeld level $\varpi^m$ structure.
\begin{proof}
  This is \cite[Proposition 2.11]{weinstein2016semistable}.
\end{proof}
\end{lem}

In particular, we obtain the desired map

\begin{equation*}
  \cM_m^{(0)}(R) \to \cM^{(0)}_{m, \wedge}(R), \quad, (\cF, \iota, \phi)
  \mapsto (\wedge^n \cF, \wedge^n \iota, \lambda^m \circ \phi).
\end{equation*}

We also need the following result.

\begin{lem}\label{lem:WeinsteinDeterminantAndNorm}
  Let $L/E$ be a separable extension of degree $n$ and suppose that there 
  is an action $\cO_L \inj \End(\cF)$ making $\cF$ into a formal $\cO_L$-module
  of height $1$. Then, for all $m \geq 1$, the identity
  \begin{equation*}
    \lambda_m(\alpha x_1, \dots, \alpha x_n) = \Norm_{L/E}(\alpha) \lambda_m
    (x_1, \dots, x_n)
  \end{equation*}
  holds. 
  \begin{proof}
    This is \cite[Lemma 2.12]{weinstein2016semistable}.
  \end{proof}
\end{lem}

We remark that the Lemma above in particular applies to the standard formal 
$\cO_E$-module $H$ over $\cO_\br E$. We turn our attention to the determinant of the standard
formal $\cO_E$-module in the following example.

\begin{xpl}
  The determinant $\wedge^n H$ is the formal $\cO_E$-module over 
  $\cO_\br E$ with logarithm given by the power series
  \begin{equation*}
    f_\wedge(T) = \sum_{i=0}^\infty (-1)^{(n-1)i} \frac{T^{q^i}}{\varpi^i}.
  \end{equation*}
  We do not prove this, but we note that this can be witnessed on the
  corresponding Dieuodonn\'e-modules: $\Dio(\wedge^n H)$ and 
  $\wedge^n \Dio(H)$ are naturally isomorphic.
  By Theorem \ref{thm:HazewinkelIntegrality} we have that 
  $[\varpi]_{(\wedge^n H)_0}(T) = (-1)^{n-1} T^q$. 
\end{xpl}

% subsubsection Determinants of Formal modules (end)

\subsection{The Universal Cover} % (fold)
\label{sub:Tate Modules and the Universal Cover}
\subsubsection{Useful Calculations} % (fold)
\label{ssub:Useful Calculations}
Let $p$ be a prime. Let $R$ be a Noetherian local ring with maximal ideal $I$
such that $p \in I$, $R$ is complete with respect to the $I$-adic topology and
$k_R \coloneqq R/I$ is an algebraically closed field (necessarily of
characteristic $p$). If $q$ is a power of $p$, we write $\cP_{R,q}$ for the set
of power series $f \in R\llbr T \rrbr$ satisfying 
\begin{equation} \label{eq:condonpowerseries}
  f(T) \equiv g(T^q) \pmod I
\end{equation}
for some power series $g(T) = c_1 T + c_2 T^2 + \dots \in R\llbr T \rrbr$ with 
$c_1 \in R^\times$. 
If $q'>q$ is another power of $p$, we have injections $\cP_{R,q} \inj \cP_{R,q'}$
given by sending $f(T)$ to its $(q'/q)$-fold self-composite $f^{q'/q}(T)$. 
Making use of these transition maps, we define
\begin{equation*}
  \cP_R \coloneqq \colim_{n \in \N} \cP_{R, p^n},
\end{equation*}
identifying any power series $f \in \cP_{R,q}$ with its image in $\cP_{R,q'}$ for 
higher $p$-powers $q'$. 
For any $f \in \cP_{R,q}$, we define the functor
\begin{equation*}
  U_f: \Adm R \to \Set, \quad S \mapsto \left\{(x_0,x_1, \dots) \in \prod_\N S^\cici 
                                          \mid f(x_{i+1}) = x_i\right\}.
\end{equation*}
This functor does, up to canonical isomorphism, only depend on the equivalence
class of $f$ in $\cP_R$. 
We write $U_{0,f}$ for the base change of $U_f$ to $k_R$, that is
\begin{equation*}
  U_{0,f}: \Adm {k_R} \to \Set, \quad S \mapsto \left\{(x_0,x_1, \dots) \in
                              \prod_\N S^\cici \mid \bar f(x_{i+1}) = x_i\right\}.
\end{equation*}
Here, $\bar f$ is the image of $f$ under the reduction map $R\llbr T \rrbr \to
k_R\llbr T \rrbr$. 

In the sequel, we denote $R$-algebras by $S$ and write $J$ for an ideal
of definition containing the image of $I$ (provided, for example, by \ref{lem:iodimage}).
Given an element $f\in \cP_R$, we do not distinguish between $f$ and a choice of a 
representative $\tilde f \in \cP_{R,q}$ for some sufficiently large $p$-power.

The following observation lays the groundwork for many of the upcoming results.
\begin{lem}\label{lem:cryscalc}
  Let $f$ be any power series in $\cP_R$. For any two elements $s_1,s_2 \in S$ 
  with $s_1 \equiv s_2 \mod J$ such that $f(s_1)$ and $f(s_2)$ exist (for
  example if $f$ is a polynomial or $s_1, s_2 \in S^\cici$), we have 
  \begin{equation*}
    f^k(s_1) \equiv f^k(s_2) \pmod {J^{k+1}}.
  \end{equation*}
  Here, $f^k$ denotes $k$-fold composition of $f$.
\begin{proof}
  We will show that if $s_1 \equiv s_2$ mod $J^k$, then $f(s_1) \equiv f(s_2)$ mod 
  $J^{k+1}$, which suffices to prove the claim. 
  We may write $s_2 = s_1 + r$ for some $r\in J^k$. By the assumptions on $f$
  there exist power series 
  $g,h \in R\llbr T \rrbr$ such that $h$ only
  has coefficients in $I$ and $f(T) = g(T^q) + h(T)$. As $I$ is finitely generated,
  say by elements $(r_1, \dots, r_l)$, we obtain a representation 
  \begin{equation*}
    f(s_1) - f(s_2) \in g(s_1^{q})-g(s_2^{q}) + \sum_{i=1}^l r_i \left(h_i(s_1) -
    h_i(s_2)\right).
  \end{equation*}
  As $r$ divides $\left(h_i(s_1) - h_i(s_2)\right)$, we find
  $r_i(h_i(s_1) - h_i(s_2)) \in (r_i r) \subseteq J^{k+1}$. Also note that 
  for any $s\in S$ and $n \in \N$, 
  $$(s+r)^{nq} = s^{nq} + nqrs^{nq-1}r + \dots + r^{nq},$$
  so after cancellation, all monomials of $g(s_1^q) - g(s_2^q)$ lie in
  $(qr)$ or $(r^2)$. This implies
  \begin{equation*}
    g(s_1^q) - g\left((s_1+r)^q\right) \in (qr) + (r^2) \subseteq J^{k+1},
  \end{equation*}
  and we are done.
\end{proof}
\end{lem}

\begin{lem}\label{lem:reductioniso}
  The natural reduction map 
  \begin{equation*}
    U_f(S) \to U_{f}(S/J) = U_{0,f}(S/J) 
  \end{equation*}
  is bijective.
\begin{proof}
  We first show surjectivity. Given a sequence $(x_0, x_1, \dots) \in U_{f}(S/J)$, 
  we can choose a sequence of arbitrary lifts $(y_0, y_1, \dots ) \in \prod_\N
  S^\cici$ and set 
  \begin{equation*}
    z_i = \lim_{r \to \infty} f^r(y_{i+r}).
  \end{equation*}
  The limit exists, because if $s\geq r$ are two non-negative integers, we calculate
  \begin{equation*}
    f^{s-r}(y_{i+s}) \equiv \bar f^{s-r}(x_{i+s}) = x_{i+r} \equiv y_{i+r}
    \pmod J,
  \end{equation*}
  implying by Lemma \ref{lem:cryscalc} that 
  \begin{equation*}
    f^{s}(y_{i+s}) \equiv f^r(y_{i+r}) \pmod{J^r}.
  \end{equation*}
  This shows that $(f^{r}(y_{i+r}))_{r \in \N}$ is a Cauchy-sequence for the 
  $J$-adic topology on $S$, thereby convergent (cf. Lemma
  \ref{lem:AdmAdicComp}). The sequence $(z_0, z_1, \dots)$ now lies in $U_f(S)$
  and lifts $(x_0, x_1, \dots)$. It remains to show that the lift is unique.
  Suppose that $(z'_0, z'_1, \dots)$ is another lift. Then, for any $i,k \in\N$
  we have $z_{i+k} \equiv z'_{i+k}$ mod $J$, and another application of Lemma
  \ref{lem:cryscalc} shows that 
  \begin{equation*}
    z_i = f^k(z_{i+k}) \equiv f^k(z'_{i+k}) = z'_i \pmod {J^k}.
  \end{equation*}
  Thereby $(z_i - z'_i) \in \bigcap_{k \in \N} J^k = \{0\}$. Hence,
  the lift is unique.
\end{proof}
\end{lem}

We write $\Nilp^\flat$ for the functor $U_{T^q}$. That is, 
$\Nilp^\flat(S) = \lim_{x \mapsto x^q}S^\cici$ is 
the set of $q$-power compatible sequences with values in $S^\cici$. 

\begin{lem}\label{lem:nilp0iso}
  For any $f \in \cP_R$, there is a canonical \todo{Use different $S$}
  bijection $U_{0,f}(S/J) \to \Nilp^\flat(S/J)$. This bijection is functorial in 
  $S$.
\begin{proof}
  By assumption on $f$ we have $f(T) = g(T^{q}) \in k_R\llbr T \rrbr$ for some 
  $g(T) = c_1T + c_2T^2 + \dots$ with $c_1 \neq 0$. For each coefficient $c_i$, let
  $d_i \in k_R$ be the unique element such that $d_i^{q} = c_i$. Let
  $h(T) \in k_R\llbr T \rrbr$ be the power series given by $d_1 T + d_2
  T^2 + \dots$. Now $(h(T))^{q}=f(T)$, and we find that 
  \begin{equation*}
      U_f(S/J) \to \Nilp^\flat(S/J): \quad
      (x_1, x_2, x_3, \dots) \mapsto (x_1, h(x_2), h(h(x_3)), \dots)
  \end{equation*}
  is a well-defined function, and functorial in $S$. For the
  inverse, let $h^{-1}(T) \in k_R\llbr T \rrbr$ be the unique power
  series with $h^{-1}(h(T))= h(h^{-1}(T)) = T$, see Lemma
  \ref{lem:IsosCheckOnLie}. The map
  \begin{equation*}
      \Nilp^\flat(S/J) \to U_f(S/J), \quad 
      (x_1, x_2, \dots ) \mapsto (x_1, h^{-1}(x_2), h^{-1}(h^{-1}(x_3)), \dots)
  \end{equation*}
  is well-defined as
  \begin{equation*}
      f(h^{-1}(T)) = g((h^{-1}(T))^{q}) = (h(h^{-1}(T)))^{q} =
      T^{q},
  \end{equation*}
  and it is readily seen to be inverse to the map constructed above.
\end{proof}
\end{lem}

We collect results.
\begin{prop}\label{prop:pHTcalc}
  Given $f,g \in \cP_R$, we have bijections, functorial in $S$,
  \begin{equation} 
    U_f(S) \to U_f(S/J) \to \Nilp^\flat(S/J) \to U_g(S/J) \to U_g(S).
  \end{equation}
  Explicitly, the bijection $U_f(S) \to U_g(S)$ can be described as follows.
  Suppose that $f,g \in \cP_{R,q}$ for some sufficiently large $q$. 
  Let $h_f(T)$ and $h_g(T)$ be power series with coefficients in $A$ such that 
  $$h_f(T)^q \equiv f(T) \pmod I\quad\text{and}\quad h_g(T)^q \equiv g(T) \pmod
  I.$$
  Write $h_g^{-1}(T)$ for the (formal) inverse power series of $h_g$. 
  Now the isomorphism is given by the mapping
  \begin{equation*}
    (x_0, x_1, \dots) \mapsto (y_0, y_1, \dots), \quad \text{where} \quad y_i =
    \lim_{r \to \infty} g^r(h_g^{-(r+i)}(h_f^{r+i} (x_{i+r}))).
  \end{equation*}
  Here, the exponents are to be interpreted as iterated composition.
\begin{proof}
  The first part follows directly from repeated application of the previous
  two Lemmas. The second part follows by tracing through the previous lemmas.  
\end{proof}
\end{prop}
% subsubsection Useful Calculations (end)

\subsubsection{The Universal Cover} % (fold)
\label{ssub:The Universal Cover}
Let $A$ be an integral domain and $R$ be an $A$-algebra. Given $\cF \in \FMOver
AR$ and $a \in A$,
we define the functor 
\begin{equation*}
  \Tilde \cF_a : \Adm R \to \Mod {A}, \quad
  S \mapsto \left\{(x_1, x_2, \dots) \in \prod_{\N} \cF(S) \mid [a]_\cF(x_{i+1}) =
  x_i \right\}.
\end{equation*}
Here, the $A$-module structure is given by $b.(x_1, x_2,\dots) = ([b]_\cF(x_1), [b]_(x_2),
\dots)$. Note that multiplication by $a$ on $\tilde \cF_a(S)$ is an automorphism
(it sends $(x_1, x_2, \dots)$ to $([a]_\cF x_1, x_1, x_2, \dots)$, which has inverse given
by shifting to the left)  so that $\tilde \cF_a(S)$ is naturally an
$A[\frac1a]$-module.

From now on assume that $A$ is a discrete valuation ring with uniformizer
$\varpi$, finite residue field $k$ and field of fractions $K$. 
Write $q = \# k$. Let $R$ be a local $A$-algebra with maximal ideal $I$ and
algebraically closed residue field $k_R = R/I$. Let $\cF$ be a formal
$\varpi$-divisible $A$-module over $R$ of height $n$. 
\begin{defi}[The Universal Cover and Tate Module]
  We write $\tilde \cF = \tilde \cF_\varpi$. This functor
  takes values in the category of $K$-vector spaces.
  Up to natural isomorphism, $\tilde \cF$ does not depend on the choice of 
  $\varpi$. We call this functor the universal cover of $\cF$. 

  The Tate-Module $T_\varpi \cF$ is the subfunctor of $\Tilde \cF$ cut out out
  by the condition that $[\varpi]_\cF(x_1) = 0$. Note that $T_\varpi \cF$ does no longer 
  carry the structure of a $K$-vector space, it is an $A$-module. The Rational
  Tate Module $V_\varpi \cF$ is the subfunctor of $\Tilde \cF$ cut out by the
  condition that $x_1$ has $[\varpi]_\cF$-torsion. Equivalently, we have 
  \begin{equation*}
    V_\varpi \cF (S) = T_\varpi \cF(S) \otimes_A K.
  \end{equation*}
\end{defi}

Fix a coordinate $\cF \cong \spf(R\llbr T \rrbr)$ so that $\cF = \FGG(F)$ for
some $A$-module law $F \in \FMLOver AR$. Then $[\varpi]_F(T) \in \cP_R$, and
we obtain an isomorphism $\tilde \cF \cong U_{[\varpi]_F} \eqqcolon \tilde F$.
Write $F_0 = F \otimes k_R$, and $\tilde F_0 = U_{0, [\varpi]_F}$. 

\begin{lem}
  We have an isomorphism 
  \begin{equation*}
    \tilde \cF_0 \cong \Nilp^\flat_{k_R}
  \end{equation*}
  of functors $\Adm {k_R} \to \Set$
  \begin{proof}
    Any lift of $[\varpi]_{F_0}(T) \in k_R\llbr T \rrbr$ lies inside $\cP_R$. Hence,
    the statement is an application of Lemma \ref{lem:nilp0iso}.
  \end{proof}
\end{lem}

\begin{lem}
  Suppose that $S$ is an admissible $R$-algebra admitting an ideal of definition
  $J$ such that $\varpi \in J$. Then the natural reduction map
  \begin{equation*}
    \tilde \cF(S) \to \tilde \cF(S/J) = \cF_0(S/J)
  \end{equation*}
  is an isomorphism.
  \begin{proof}[Proof]
    After choosing a coordinate $\cF = \FGG(F)$, we have $[\varpi]_F \in \cP_R$
    and hence $\tilde \cF(S) \cong U_{[\varpi]_F}$. Thereby the statement is
    given by Lemma \ref{lem:reductioniso}.
  \end{proof}
\end{lem}

The following is analogous to Proposition \ref{prop:pHTcalc}.
\begin{prop}\label{prop:UnivCoverReductionIso}
  Let $S$ be an admissible $R$-algebra with ideal of definition $J$ such that 
  $\phi(I) \subseteq J$. Then there are canonical isomorphisms (of sets)
  \begin{equation*}
    \tilde \cF(S) \cong \tilde \cF(S/J) = \tilde \cF_0(S/J) \cong \Nilp^\flat(S/J) \cong
    \Nilp^\flat(S).
  \end{equation*}
  In particular, $\tilde \cF(S)$ is, as a functor to $\Set$, representable by
  $\spf(R \llbr T^{q^{-\infty}} \rrbr)$.
\end{prop}
We write $\lambda$ for the isomorphism $\Tilde \cF \to \Nilp^\flat$, and
$\lambda_i: \Tilde \cF \to (-)^\cici$ for projection on the $i$-th component.
Similarly, we write $\mu: \Nilp^\flat \to \Tilde \cF$ for the inverse of 
$\lambda$ and $\mu_i$ for the $i$-th component of $\mu$. 

By the proposition above, quasi-isogenies on $\cF_0$ induce isomorphisms on 
$\tilde \cF$. In particular, there is a natural left action of 
$D^\times = \End_{\FMOver {\cO_E} \Fqbar}(\cF_0)[\varpi^{-1}]^\times$ on $\tilde \cF$. 
The relative Frobenius morphism lifts as well.

\begin{defi}[Relative Frobenius on $\Tilde \cF$] \label{def:FrobOnUnivCov}
  Write $\Pi: \Tilde \cF \to \Phi^{-1,*}\Tilde \cF$ for the isomorphism coming from
  the Frobenius quasi-isogeny 
  \begin{equation*}
    \Frob_q: \cF_0 \to \cF_0^{(q)} = \Phi^{-1,*} \cF_0.
  \end{equation*}
\end{defi}

By the explicit description of $\lambda$ in Lemma
\ref{lem:GroupStructureOnNilp}, we find that $\lambda_i(\Pi x) =
\lambda_i(x)^q$ for $x \in \Tilde \cF(S)$ and $i\in \N_0$. 

As $\tilde \cF$ is a $K$-vector space
object, the isomorphism $\lambda: \tilde \cF \to \Nilp^\flat$ equips
$\Nilp^\flat$ with the stucture of a $K$-vector space object. Upon choosing a 
suitable coordinate for $\cF$, this structure admits the following description.

\begin{lem}\label{lem:GroupStructureOnNilp}
  Suppose that $F$ is a formal module law over $R$ with $[\varpi]_F(T) \equiv
  T^{q^n}$ mod $I$. Then the following assertions hold.
  \begin{itemize}
    \item The bijections
        \begin{equation*}
          \lambda \colon \tilde F(S) \rightleftarrows \Nilp^\flat(S) \colon
          \mu, \quad (x_0, x_1, \dots) \mapsspamto (y, y^{q^{-1}}, y^{q^{-2}},
          \dots)
        \end{equation*}
        are, in either direction, given by the equations
        \begin{equation*}
          y^{1/q^{ni}} = \lim_{r \to \infty} x_{r+i}^{q^{nr}} \quad \text{and} \quad 
          x_i = \lim_{s \to \infty} [\varpi^s]_F(y^{q^{-n(i+s)}}).
        \end{equation*}
    \item Given two $q$-th power compatible systems $y_1, y_2
      \in \Nilp^\flat(S)$ corresponding to compatible systems
      $\mu(y_1) = x_1$, $\mu(y_2) = x_2 \in \tilde \cF(S)$, the sum $x_1 + x_2 \in
      \tilde \cF(S)$ corresponds to the element $\lambda(x_1 + x_2) = y_1 +_F y_2 \in
      \Nilp^\flat(S)$, where 
      \begin{equation*}
        (y_1 +_F y_2)^{1/q^j} = \lim_{r \to \infty} F( y_1^{q^{-r}},
        y_2^{q^{-r}})^{q^{r-j}}.
      \end{equation*}
      If $G \in \FMLOver AR$ with $G \otimes \Fqbar = F \otimes \Fqbar$,
      the systems of $q$-th power roots $(y_1 +_F y_2)$ and $(y_1 +_G y_2)$ agree. 
    \item Similarly, given $a \in \cO_E$ and $y \in \Nilp^\flat(S)$ with
      $\mu(y) = x \in \Tilde F(S)$, we have 
      \begin{equation*}
        a_F y = \lambda([a]_F(x)) = \lim_{r \to \infty} [a]_F(y^{q^{-r}})^{q^{r-j}}.
      \end{equation*}
      For $G$ as above, we have $[a]_F(x) = [a]_G(x)$. 
  \end{itemize}
\begin{proof}
  The first statement is an immediate corollary of the second part of
  Proposition \ref{prop:pHTcalc} applied with $f(T) = [\varpi]_F(T)$ and $g(T)
  = T^{q^n}$, as we may choose $h_f(T) = h_g(T) = T$. 
  The second assertion follows directly after tracing through the commutative diagram
  \begin{equation*}
    \begin{tikzcd}[ampersand replacement=\&]
    	\& {\Tilde F(S)^2} \& {\Tilde F(S)} \\
    	{\Nilp^\flat(S)^2} \& {\Tilde F_0(S/J)^2} \& {\Tilde F_0(S/J)} \& {\Nilp^\flat(S).}
    	\arrow["{F(-,-)}", from=1-2, to=1-3]
    	\arrow[ from=1-2, to=2-2]
    	\arrow[ from=1-3, to=2-3]
    	\arrow["\lambda", from=1-3, to=2-4]
    	\arrow["\mu", from=2-1, to=1-2]
    	\arrow[ from=2-1, to=2-2]
    	\arrow["{F_0(-,-)}", from=2-2, to=2-3]
    	\arrow[from=2-3, to=2-4]
    \end{tikzcd}
  \end{equation*}
  Similarly one proves the third statement.
\end{proof}
\end{lem}


In the case where $H$ is the standard formal $\cO_E$-module over $\cO_\br E$,
we have the following description of the actions of $K^\times$ and $D^\times$ on
$\Nilp^\flat$.

\begin{cor}\label{cor:ExplicitDescriptionOfActionOnUnivCov}
  Let $a \in E$ and $d \in D^\times$ be elements so that
  \begin{itemize}
    \item the element $a$ is, for some $l \in \Z$, of the form
      \begin{equation*}
        a = \sum_{i = l}^\infty a_i \varpi^i \quad \text{for} \quad
        a_i \in \mu_{q-1}(E) \cup \{0\}.
      \end{equation*}
    \item the element $d$ is, for some $l' \in \Z$ and 
      $\vartheta \in \End_{\FMLOver {\cO_E}\Fqbar} (H_0)$ the endomorphism given
      by $\vartheta(T) = T^q$, of the form
      \begin{equation*}
        d = \sum_{i = l'}^\infty d_i \vartheta^i \quad \text{for} \quad
        d_i \in \mu_{q^n-1}(E_n) \cup \{0\}.
      \end{equation*}
      Here $E_n$ denotes the unramified extension of $E$ with residue field 
      $\FF_{q^n}$. 
  \end{itemize}
  Then, given any $S \in \Adm R$ \red{Correct category?} and $x = (x_1, x_2, \dots)
  \in \tilde H(S)$ with $\lambda(x)^{1/q^i} = y^{1/q^i} \in \Nilp^\flat(S)$, we
  have the explicit descriptions
  \begin{equation*}
    a_H(y)^{1/q^j}  = (H) \sum_{i=l}^\infty a_i y^{q^{ni-j}}
  \quad \text{and} \quad
  d_H(y)^{1/q^j} \coloneqq \lambda(dx)^{1/q^j} = (H) \sum_{i=l'}^\infty d_i y^{q^{i-j}}
  \end{equation*}
  Here the prefix $(H)$ denotes addition of $q$-th power root systems with
  respect to the addition in $H$, as defined in the previous lemma.
  \begin{proof}
    This is an immediate corollary of the Lemma above, Lemma
    \ref{lem:MultByROUForStandardModule} and the fact that 
    $[\varpi]_H(T) \equiv T^{q^n} \pmod \varpi$. 
  \end{proof}
\end{cor}

Finally, we remark how to express the logarithm map in terms of the isomorphism
$\lambda$.


\begin{lem}\label{lem:LogInTermsOfNil}
  Let $H$ be the standard formal $\cO_K$-module of height $n$ over $R =
  \cO_{\breve K}$. We have a commutative diagram (cf. \cite[Lemma
  2.6.1]{BoyarchenkoWeinstein2011MaxVar})
  \begin{equation*}
  \begin{tikzcd}[ampersand replacement=\&]
  	{(x_0, x_1, \dots)} \& {\Tilde H(S)} \&\& {\Nil^\flat(S)} \& {(y,y^{1/q}, \dots)} \\
  	{\sum_{i=0}^\infty \frac{x_0^{q^{ni}}}{\varpi^i}} \&\& {S\otimes_{\cO_K}K} \&\& {\sum_{i = -\infty}^\infty\frac{y^{q^{ni}}}{\varpi^i}}
  	\arrow["\in"{description}, draw=none, from=1-1, to=1-2]
  	\arrow[maps to, from=1-1, to=2-1]
  	\arrow["\lambda", from=1-2, to=1-4]
  	\arrow["{\log_H}"', from=1-2, to=2-3]
  	\arrow[from=1-4, to=2-3]
  	\arrow["\ni"{description}, draw=none, from=1-5, to=1-4]
  	\arrow[maps to, from=1-5, to=2-5]
  \end{tikzcd}
  \end{equation*}
  With this terminology, we have $\log_H((\Pi^j x)_0) = \sum_{i = -\infty}^\infty
  \frac{ y^{ni + j}}{\varpi^i}$. 
\begin{proof}
  This follows directly from the remark above. Let $x \in \Tilde H(S)$ and write
  $\lambda(x) = (y, y^{1/q}, \dots)$. We have $x_0 = \lim_{s \to \infty}
  [\varpi^s]_H(y^{-ns})$, hence
  \begin{equation*}
    \log_H(x_0) = \lim_{s \to \infty}\left( \varpi^s \log_H(y^{1/q^{ns}})\right) = 
    \lim_{s \to \infty}\left( \sum_{i=0}^\infty  \frac{y^{q^{n(i-s)}}}{\varpi^{i-s}}\right)
    = \sum_{i=-\infty}^\infty \frac{y^{q^{ni}}}{\varpi^i}.
  \end{equation*}
  The second part is an immediate consequence.
\end{proof}
\end{lem}


% subsubsection The Universal Cover (end)

\subsubsection{Relation to the Deformation Space at Infinite Level} % (fold)
\label{ssub:Relation to the Deformation Space at Infinite Level}
Let $(e_1, \dots, e_n)$ denote the standard basis of $\cO_E^n$. By 
Theorem \ref{thm:RepresentabilityOfDefSpaceWithLevel}, there is for
every positive integer $m$ a universal triple 
$$(\cF^\univ_m, \iota^{\univ}_m, \phi^\univ_m) \in \cM_{m}^{(0)}(A_m),$$
\red{where the pair $(\cF_m^\univ, \iota_m^\univ) = (\cF^\univ, \iota^\univ)$
  can be chosen independently of $m$} and the Drinfeld level
  $\varpi^m$-structure 
\begin{equation*}
  \phi_m^\univ: \underline{(\varpi^{-m}\cO_{E}/\cO_E)}^n \to \cF^\univ
\end{equation*}
gives, evaluated at $A_\infty$, rise to elements $x_i^{(m)} = \phi_m(e_i) \in
\cF^{\univ}(A_\infty)$ for $i = 1, \dots, n$. This gives rise to an $n$-tuple 
of compatible systems
\begin{equation*}
  (x_1, \dots, x_n) \in \tilde \cF^\univ (A_\infty). 
\end{equation*}
Now let $\cF$ be an arbitrary deformation of $\cF_0$ to $\cO_{\br E}$. By 
Proposition \ref{prop:UnivCoverReductionIso}, we have isomorphisms
\begin{equation*}
  \tilde \cF^{\univ,n}(A_\infty) \cong \tilde \cF_0^n(A_\infty/I_\infty)
  \cong \tilde \cF^n(A_\infty).
\end{equation*}
This constructs a morphism of formal schemes over $\spf R$
\begin{equation} \label{eq:MapFromMtoTildeF}
  \cM_\infty^{(0)} \to \tilde \cF^n.
\end{equation}

\begin{thm}[Structure of $\cM_\infty$]\label{thm:WeinsteinsCartesianSquare}
  There is a cartesian square of formal schemes
  \begin{equation*}
    \begin{tikzcd}[ampersand replacement=\&]
      {\cM_\infty^{(0)}} \& {\cM_{\wedge, \infty}^{(0)}} \\
    	{\widetilde \cF^n} \& {\widetilde{\wedge^n \cF}.}
    	\arrow["\det", from=1-1, to=1-2]
    	\arrow[from=1-1, to=2-1]
    	\arrow[from=1-2, to=2-2]
    	\arrow["\delta", from=2-1, to=2-2]
    \end{tikzcd}
  \end{equation*}
  Here, the vertical maps are the ones constructed above, the horizontal ones
  come from the determinant construction of \cite{hedayatzadeh2015det}. 
\begin{proof}
  This is \cite[Theorem 2.17]{weinstein2016semistable}.
\end{proof}
\end{thm}

\begin{cor}\label{cor:StructureOfAinfty}
  Let $(\tau^{1/q^m})_{m \in \N} \in \cO_{\hat E^\ab}$ be a \red{primitive, make
  precise} system of $q$-th power roots. Then we have 
  \begin{equation*}
  A_\infty \cong \cO_{\hat E^\ab}\llbr X_1^{q^{-\infty}}, \dots,
  X_n^{q^{-\infty}}\rrbr/(\delta^{q^{-m}} - \tau^{q^{-m}} \mid m \in \N)^-.
  \end{equation*}
  In particular, we obtain the equality in \eqref{eq:ExplicitAinftyBCtoOC}.
\begin{proof}
\end{proof}
\end{cor}


% subsubsection Relation to the Deformation Space at Infinite Level (end)

\subsubsection{Group Actions on the Universal Cover} % (fold)
\label{ssub:Group Actions on the Universal Cover}
Given a $\varpi$-divisible formal $A$-module $\cF \in \FMOver AR$ of height
$n$, we write $\cF_{\cO_\Cp} = \cF \otimes \cO_\Cp$ and describe a natural right
action on $\tilde \cF_{\cO_\Cp}^n$ by the group 
\begin{equation*}
  G = \GL_n(E) \times D^\times \times W_E,
\end{equation*}
such that the map $\cM_{\infty, \cO_\Cp} \to \tilde \cF_{\cO_\Cp}^n$,
induced by the map constructed above is equivariant for this action. 

The action of $G$ on $\tilde \cF^n$ is easy to describe. For the action of
$\GL_n(E)$, note that $\tilde \cF$ carries the structure of a $E$-vector space
object. Hence $\tilde \cF^n$ obtains a natural right action by $\GL_n(E)$: an
object $g \in \GL_n(E)$ with entries $g = (a_{ij})_{i,j}$ acts by matrix
multiplication from the right, as in 
\begin{equation}\label{eq:UnivCoverGLnAction}
  (x_1, \dots, x_n).g = (y_1, \dots, y_n), \quad
  \text{where} \quad y_j = \sum_{i=1}^n a_{ij} x_i. 
\end{equation}

For the action of $D^\times$, note that by Proposition
\ref{prop:UnivCoverReductionIso}, we have 
a natural left action of $D^\times$ on $\tilde \cF$. 
Indeed, given $0 \neq d_0 \in \End_{\FMOver A{\Fqbar}}(F_0) = \cO_D$, an
integer $r \in \Z$ and any $S \in \Adm R$ \red{Correct Category?}, 
we let the element $d = \varpi^r d_0 \in D^\times$ act on $\tilde \cF$ via the
automorphism
\begin{equation*}
  \tilde \cF(S) \to \tilde \cF_0(S/J) \xto{\varpi^{r}} \cF_0(S/J) \xto{d_0}
  \tilde \cF_0(S/J) \to \tilde \cF(S), \quad x \mapsto dx.
\end{equation*}
Note that multiplication by $\varpi$ and $d_0$ commute (as $\varpi$ lies in the 
center of the multiplicative monoid $\cO_D$), so this yields a well-defined
left action. We define the right action of $D^\times$ on $\tilde \cF^n$ via
\begin{equation*}
  (x_1, \dots, x_n).d = (d^{-1} x_1, \dots, d^{-1} x_n).
\end{equation*}


The map $\Pi: \Tilde \cF \to \Phi^{-1,*}\Tilde \cF$ from Definition 
\ref{def:FrobOnUnivCov} equips $\cF$ with the Weil descent datum
\begin{equation*}
  (\Phi^*{\Pi})^{-1}: \cF \to \Phi^*\cF,
\end{equation*}
and in particular yields an action of $\Weil_E$ on $\tilde \cF_{\cO_\Cp}^n$. 

It is easy to see that the actions of $\GL_n(E)$ and $\Weil_E$ commute,
and that both these actions commute with the Weil descent datum. Hence we
obtain a right action by $G$ on $\tilde \cF^n_{\cO_\Cp}$.

\begin{lem}\label{lem:MapToUnivCovIsEquivariant}
  The morphism $\cM_{\infty, \cO_\Cp}^{(0)} \to \tilde \cF_{\cO_\Cp}^n$
  in \eqref{eq:MapFromMtoTildeF} is equivariant for the action of $G^1$ on both
  sides.
\begin{proof}
  It suffices to check $G$-equivariance of the induced map
  $$\cM_{\infty, \cO_\Cp}^{(0)}(A_\infty \cotimes_{R} \cO_\Cp) \to \tilde
  \cF^n_{\cO_\Cp}(A_\infty \cotimes_R \cO_\Cp).$$
  Here it suffices to show that the morphism $\cM_{\infty}^{(0)}(A_\infty) \to
  \tilde \cF^n(A_\infty)$ is equivariant for the action of 
  $\GL_n(E) \times D^\times$ and that it preserves the Weil descent datum in 
  the sense that the square
  \begin{equation*}
    \begin{tikzcd}[ampersand replacement=\&]
    	{\cM_\infty^{(0)}(A_\infty)} \& {\tilde \cF^n(A_\infty)} \\
    	{\Phi^*\cM_\infty(A_\infty)} \& {\Phi^*\tilde \cF^n(A_\infty)}
    	\arrow[from=1-1, to=1-2]
    	\arrow[from=1-1, to=2-1]
    	\arrow[from=1-2, to=2-2]
    	\arrow[from=2-1, to=2-2]
    \end{tikzcd}
  \end{equation*}
  commutes. 
  \red{Todo: more explicit description of group action on $A_\infty$.}
\end{proof}
\end{lem}

Suppose that $\cF = \FGG(H)$ comes from the standard formal $\cO_E$-module of
height $n$ over $\cO_\br E$. Under the isomorphism $\lambda: \tilde H \to
\Nilp^\flat$, the right action of $G$ on $\tilde H^n \otimes \cO_\Cp$
corresponds to a left action of $G$ on $B_n = \cO_\Cp \llbr X_1^{q^{- \infty}}, \dots,
X_n^{q^{- \infty}} \rrbr$. By tracing through the definitions, we obtain the 
following description.

\begin{lem}\label{lem:ExplicitActionOnQthPowerRootSystems}
  Let $g = (a_{ij})_{i,j} \in \GL_n(E)$, $d \in D^\times$ and $\sigma \in
  \Weil_E$. Let $m \in \Z$ be such that $\sigma|_{\br E} = \Phi^m$. Then the
  morphism $B_n \to B_n$ induced by the element $(g, d, \sigma) \in G$ is given by
  the composition of the morphsims
  \begin{itemize}
    \item $g^*\colon B_n \to B_n$ is the morphism of $\cO_\Cp$ algebras given
      by $X_i \mapsto \sum_{j=1}^n \sumH {a_{ij}}_H(X_j)$. Writing
      $a_{ij} = \sum_{k=l}^\infty a_{ij}^{(k)} \varpi^k$ for  $a_{ij}^{(k)}$
      either vanishing or a $(q-1)$th root of unity and a sufficiently small integer
      $k$, we obtain, by Corollary
      \ref{cor:ExplicitDescriptionOfActionOnUnivCov}, the description
      \begin{equation*}
        g^*(X_i) = \sum_{k=l}^\infty {\hspace{-3pt}}_H\hspace{3pt} \varpi^i_H
        \sum_{j=1}^n{\hspace{-3pt}}_H\hspace{3pt} a_{ij}^{(k)} X_i^{q^{nk}}
      \end{equation*}
      
    \item $d^{-1,*} \colon B_n \to B_n$ is the isomorphism of 
      $\cO_\Cp$-algebras given by $X_i \mapsto d_H(X_i)$. 
      Writing $d^{-1} = \sum_{k=l}^\infty d_i \vartheta^k$ for
       a sufficiently small integer $k$, and $\vartheta$ and $d_i$ as in
       Corollary \ref{cor:ExplicitDescriptionOfActionOnUnivCov}, we obtain the
       description
      \begin{equation*}
        d^{-1, *}(X_i) =  \sum_{k=l}^\infty {\hspace{-3pt}}_H\hspace{3pt}
        d_k X_i^{1/q^k}.
      \end{equation*}

    \item $\sigma^*: B_n \to B_n$ is the isomorphism of 
      $\cO_E$-algebras given by $X_i \mapsto X_i^{q^{-m}}$ and 
      $a \mapsto \sigma(a)$ for $a \in \cO_\Cp$. 
  \end{itemize}
\end{lem}

Finally, we express the action of $G^1$ on $\cM_{\infty, \cO_\Cp}^{(0)} =
A_\infty \cotimes_{\cO_\br E} \cO_\Cp$ in terms of the isomorphism
\eqref{eq:ExplicitAinftyBCtoOC}.
For $\alpha \in \Gal(\hat E^\ab / \br E) \cong \cO_E^\times$, we write
$A_{\infty, \cO_\Cp}^{\alpha}$ for the ring
$A_\infty \cotimes_{\hat E^\ab, \alpha} \cO_\Cp$.
Under the isomorphism $\Gal(\hat E^\ab / \br E) \cong \cO_E^\times$ from 
the local Kronecker--Weber theorem, this yields a decomposition
\begin{equation*}
  A_{\infty} \cotimes_{\cO_\br E} \cO_\Cp = \prod_{\alpha \in \cO_E^\times}
  A_{\infty, \cO_\Cp}^{\alpha}.
\end{equation*}
We have 
\begin{equation*}
  A_{\infty, \cO_\Cp}^{\alpha} = \cO_{\Cp}\llbr X_1^{q^{-\infty}}, \dots, 
  X_n^{q^{-\infty}} \rrbr / (\delta(X_1, \dots, X_n)^{1/q^m} - \Art_E(\alpha)(t)^{1/q^m} 
  \mid m \in \N)^-,
\end{equation*}
and by local class field theory \red{make precise}, the action of $\Weil_E$ on
$t$ induces a character $a_E: \Weil_E \to \cO_E^\times$. 


\begin{prop}\label{prop:ExplicitDescriptionOfActionOnAinfty}
  The group $G^1$ is generated by elements of the form
  \begin{itemize}
    \item $(a, a, 1) \in G$ for $a \in E^\times$.
    \item $(g, d, 1) \in G$ such that $\det(g) \Nrd(d)^{-1} \in \cO_E^\times$.
    \item $(1, \vartheta^{-m}, \sigma)$ for $\sigma \in \Weil_E$ with
      $\sigma|_{\br E} = \Phi^m$. 
  \end{itemize}
  These elements act on $A_{\infty, \cO_\Cp}$ as follows.
  \begin{itemize}
    \item $(a,a,1)$ acts trivially.
    \item $(g,d,1)$ acts by the morphism of $\cO_\Cp$-algebras
      $$A_{\infty, \cO_\Cp}^\alpha \to A_{\infty, \cO_\Cp}^{\det(g)^{-1} \Nrd(d) \alpha},
    \quad X_i \mapsto (g,d^{-1})^*(X_i) \ \text{for} \ i = 1, \dots,n.$$
    \item $(1, \vartheta^{-m}, \sigma)$ acts by the morphism of $\cO_E$-algebras 
      $A_{\infty, \cO_\Cp}^\alpha \to A_{\infty, \cO_\Cp}^{a_E(\sigma)\alpha}$
      given by $X_i \mapsto X_i$ for $i = 1, \dots, n$ and $a \mapsto
      \sigma(a)$ for $a \in \cO_\Cp$.
  \end{itemize}
\begin{proof}
\end{proof}
\end{prop}


% subsubsection Group Actions on the Universal Cover (end)

% subsection Tate Modules and the Universal Cover (end)


\subsection{The Quasilogarithm Map} % (fold)
\label{sub:The Quasilogarithm map}
We keep the assumptions on $A$, $R$ and $S$ from the previous subsection. That is,
$A$ is a local ring with finite residue field and uniformizer $\varpi$, 
$R$ is a local $A$-algebra with maximal ideal $I$ complete with respect to
the $I$-adic topology and algebraically closed residue field $k_R$, and 
$S$ denotes an admissible $R$-algebra (where $R \to S$ is continuous with
the $I$-adic topology on $R$) with ideal of definition $J \subseteq S$ containing
the image of $I$. 

The aim of this subsection is to define, attached to any $\varpi$-divisible formal
$A$-module $\cF$ over $R$, a map
\begin{equation*}
\qlog_\cF: \tilde \cF(S) \to \Dio(\cF) \otimes_R (S \otimes_A K),
\end{equation*}
called the quasi-logarithm map.
We give an explicit description of this map if $\cF = \FGG(H)$ is the standard
$\cO_K$-module over $\cO_{\breve K}$. 

The construction of $\qlog_\cF$ is as follows.  
Let $0 \to \cV \xto\psi \cE \xto\phi \cF \to 0$ be the universal additive
extension of $\cF$. For any sequence $(x_1, x_2, \dots) \in \tilde \cF(S)$, choose an arbitrary sequence $(y_1, y_2, \dots) \in
\tilde \cE(S)$ such that $y_i$ is a lift of $x_i$ under the map $\cE(S) \to \cF(S)$. 
Let $y$ be the limit $y = \lim_{i \to \infty} [\varpi]_{\cE}^i(y_i)$ and put 
$$\qlog_\cF((x_1, x_2, \dots)) = \log_\cE(y) \in \Dio(\cF) \otimes_R(S \otimes_A K).$$ 

\begin{prop}
  This construction yields a well-defined map. 
\begin{proof}
  We may assume that $\cF$ and $\cV$ come from formal module laws $F$ and 
  $V$, and we may furthermore assume that $\cE = \FGG(E)$ 
  for an $\cO_K$-module law $E$ obtained by Lemma \ref{lem:SESStandardForm}. Now
  $(x_1, x_2, \dots)$ is a sequence in $S^\cici$ and $(y_1, y_2, \dots)$ is a
  sequence of elements in $(S^\cici)^n$.

  It suffices to show that $y = \lim_{i\to \infty} [\varpi]^i_E(y_i)$ exists and 
  that it is independent of the choice of lifts $(y_1, y_2, \dots)$. 
  Both claims follow from the additivity of $\cV$, implying that 
  $[\varpi]_{V}(T) = \varpi T$. The sequence 
  $([\varpi^i]_{E}(y_i))$ converges, as for positive integers $i\leq j$, we have 
  \begin{equation*}
    [\varpi^i](y_i) - [\varpi^j](y_j) = [\varpi^i]([\varpi^{i-j}]y_j - y_i)
    \in \psi(\varpi^i (S^\cici)^{n-1}) \subseteq J^i (S^\cici)^{n}. 
  \end{equation*}

  If $(y_1', y'_2, \dots)$ is another sequence of lifts, put
  $y' = \lim_{i \to \infty} [\varpi^i]_E (y_i') \in S^\cici$. Now there exists some
  $z \in \cV(S)$ such that 
  $y - y' = \psi(z)$. But by construction
  $z \in \bigcap_{i \in \N} \varpi^i(S^{\cici})^{n-1} = 0$.
\end{proof}
\end{prop}

Let us now consider the case where $\cF = \FGG(H)$ comes from the standard
formal $\cO_K$-module of height $n$ over $\cO_{\breve K}$. Then 
from Proposition \ref{prop:InterestingSequenceStdMod} we have the distinguished
basis elements of $\Ext(H,\Ghat_a)$ corresponding to the symmetric $2$-cocycles
$\delta f_i$, $1 \leq i \leq n-1$ where $f_i(T) = \frac 1 \varpi
\log_H(T^{q^i})$. Also recall that, setting $f_0(T) = \log_H(T)$, the elements 
$(f_0, f_1, \dots, f_{n-1})$ freely generate $\QLog(H)$. The universal
additive extension now corresponds to the symmetric $2$-cocycle $(\delta f_1,
\dots, \delta f_{n-1}) \in \SymCoc^2(H,V)$.
We can make the quasi-logarithm map explicit.

\begin{prop}\label{prop:qlogmapExplicit}
  Let $x = (x_0, x_1, \dots) \in \Tilde H(S)$. With respect to the basis
  $(\log_H(T),\allowbreak \log_H(T^q), \allowbreak \dots, \allowbreak \log_H(T^{q^{n-1}}))$ of 
  $\QLog(H) \otimes_{\cO_K} K$, the quasi-logarithm map is given by
  \begin{equation*}
    \qlog_H(x) = \left(\log_H(x_0), \log_H((\Pi x)_0), \dots, \log_H((\Pi^{n-1}
    x)_0)\right) \in (S\otimes K)^n.
  \end{equation*}
  Here, $\Pi x = ((\Pi x)_0, (\Pi x)_1, \dots)$ is the image of $x$ under
  $\Pi$, the automorphism of $\Tilde
  H(S)$ induced by the (relative) Frobenius quasi-isogeny on $H_0$, cf. Definition
  \ref{def:FrobOnUnivCov}.
\end{prop}
We postpone the proof to state the following auxiliary result.
\begin{lem}\label{lem:FrobOnTildeHExpl}
  Let $x = (x_0 ,x_1, \dots) \in \Tilde H(S)$. For positive integers $i$ and
  $j$ we have
  \begin{equation*}
    \log_H((\Pi^j x)_i) = \lim_{r \to \infty} \varpi^r \log_H(x_{r+i}^{q^j}).
  \end{equation*}
\begin{proof}
  Tracing through the commutative square (with $\lambda$ and $\mu$ the
  isomorphisms from the previous subsection)
    \begin{equation*}
    \begin{tikzcd}[ampersand replacement=\&]
    	{\Tilde H(S)} \& {\Nilp^\flat(S)} \\
    	{\Tilde H(S)} \& {\Nilp^\flat(S),}
    	\arrow["\lambda", from=1-1, to=1-2]
    	\arrow["\Pi", from=1-1, to=2-1]
    	\arrow["{(y_i)_i \mapsto (y_i^{q})_i}", from=1-2, to=2-2]
    	\arrow["\mu"', from=2-2, to=2-1]
    \end{tikzcd}
    \end{equation*}
    we find 
    \begin{equation} \label{eq:CompsOfPiExplicit}
      (\Pi^j x)_i = \lim_{s \to \infty} \lim_{r\to\infty} \left([\varpi]_H^s
      (x_{r+s+i}^{q^{nr + j}})\right).
    \end{equation}
    The claim follows after applying $\log_H$ and making repeated use of the
    functional equation $\log_H(T^{q^n}) = \varpi \log_H(T) + \varpi T$.
\end{proof}
\end{lem}
\begin{proof}[Proof of Proposition \ref{prop:qlogmapExplicit}]
  Using the coordinates provided by $(\delta f_1, \dots, \delta f_{n-1})$, the
  universal additive extension of $H$ is isomorphic to 
  \begin{equation*}
    0 \to \Ghat_a^{n-1} \to E \to H \to 0,
  \end{equation*}
  where $E$ is a module law with 
  \begin{equation*}
    [\varpi]_E(\bX, T) = \big(\varpi X_1 + (\delta_{\varpi}f_1)(T), \dots, \varpi X_{n-1} + 
    (\delta_{\varpi}f_{n-1})(T), [\varpi]_H(T)\big).
  \end{equation*}
  Beginning with $x = (x_0, x_1, \dots) \in \Tilde H (S)$, lifting to $(y_0,
  y_1, \dots) \in E(S)^\N$ 
  and writing $y = \lim_{i\to\infty} [\varpi]_E^i (y_i)$, we find
  \begin{equation*}
    y = \left(\lim_{r\to\infty} (\delta_{\varpi^r} f_1)(x_r), \dots, \lim_{r\to \infty}
    (\delta_{\varpi^r}f_{n-1})(x_r), x_0\right) \in E(S).
  \end{equation*}
  Now, Lemma \ref{lem:FrobOnTildeHExpl} provides the equality
  $$\lim_{r\to\infty}\delta_{\varpi^r} f_i(x_r) = \frac 1\varpi \lim_{r \to \infty}
  \varpi^r \log_H(x_r^{q^{nr+i}}) - \frac 1\varpi \log_H\left(x_0^{q^i}\right) = 
  \frac 1\varpi \left(\log_H((\Pi^i x)_0)- \log_H(x_0^{q^i})\right).$$
  We need to calculate $\log_E(y)$, which calls for an explicit description of 
  $\log_E : E \otimes (R \otimes_AK) \to (\Ghat_a \otimes(R \otimes_A K))^n$. 
  Tracing through the procedure provided in Subsection \ref{sub:Logarithms},
  we find
  \begin{equation*}
    \log_E(\bX, T) = \left(X_1 + \tfrac 1\varpi \log_H(T^q), \dots, 
    X_{n-1} + \tfrac 1\varpi \log_H(T^{q^{n-1}}), \log_H(T)\right).
  \end{equation*}
  This representation is with respect to the basis $(f_1, \dots, f_{n-1}, f_0)$. 
  The claim follows.
\end{proof}



% subsubsection The Quasilogarithm map (end)


\subsection{Determinants} % (fold)
\label{sub:Determinants}
Let $H$ be the standard formal $\cO_K$-module over $\cO_{\breve K}$ of height
$n$. Write $\wedge H$ for the formal $\cO_K$-module over $\cO_{\breve K}$ with
logarithm
\begin{equation*}
  \log_{\wedge H}(T) = \sum_{i = 0}^\infty (-1)^{(n-1)i} \frac{T^{qi}}{\varpi^i}.
\end{equation*}
By Hazewinkel's integrality Lemma (cf. Theorem \ref{thm:HazewinkelIntegrality}), 
such a module law exists. We have $\Dio(\wedge H) = \wedge^n \Dio(H)$. 
We follow \cite[Theorem 2.10.3]{BoyarchenkoWeinstein2011MaxVar} to describe a map $\delta:
\Tilde H^n \to \Tilde {\wedge H}$ making the square
\begin{equation}\label{diag:qlogsquare}
\begin{tikzcd}[ampersand replacement=\&]
	{\Tilde H^n(S)} \& {\Tilde{\wedge H}(S)} \\
	{\Dio(H)^n \otimes (S\otimes_{\cO_K} K)} \& {\Dio(\wedge H) \otimes(S\otimes_{\cO_K}K)}
	\arrow["\delta", from=1-1, to=1-2]
	\arrow["\qlog_H \times \dots \times \qlog_H"', from=1-1, to=2-1]
  \arrow["\qlog_{\wedge H}", from=1-2, to=2-2]
	\arrow["\det", from=2-1, to=2-2]
\end{tikzcd}
\end{equation}
commute. 

Let $(s_1, \dots, s_n) \in \Tilde H(S)^n$, and write $x_i = \lambda(s_i) \in
\Nilp^\flat(S)$, which are elements in $S^\cici$ with distinguished $q$-power
roots. Here $\lambda: \Tilde H \to \Nilp^\flat$ is the isomorphism from Section
\ref{sub:Tate Modules and the Universal Cover} with inverse $\mu = (\mu_0,
\mu_1, \dots)$. 
We set
\begin{equation*}
  \delta_0(s_1, \dots, s_n) = \sum_{(a_1, \dots, a_n)} \varepsilon(a_1, \dots
  a_n) \mu_0(x_1^{q^{a_1}} \cdots x_n^{q^{a_n}}) \in \wedge H(S),
\end{equation*}
where 
\begin{itemize}
  \item The sum takes place in ${\wedge H}(S)$.
  \item The sum ranges over $n$-tuples $(a_1, \dots, a_n)$ of (possibly negative) integers 
    satisfying $a_1 + \dots + a_n = n (n-1)/2$, subject to the
    condition that each $a_i$ occupies a distinct residue class modulo $n$.
  \item The expression $\varepsilon(a_1, \dots, a_n)$ denotes the sign of the 
    permutation $i \mapsto a_{i+1}$ (mod $n$) of $(0, \dots, n-1)$.
\end{itemize}

\begin{prop}
  The map $\delta_0$ makes the diagram 
  \begin{equation*}
    \begin{tikzcd}[ampersand replacement=\&]
    	{\tilde H^n(S)} \& {\wedge H(S)} \\
    	{\Dio(H)^n \otimes (S \otimes K)} \& {\Dio(\wedge H) \otimes(S \otimes K)}
    	\arrow["{\delta_0}", from=1-1, to=1-2]
    	\arrow["{\qlog_H^n}"', from=1-1, to=2-1]
    	\arrow["{\log_{\wedge H}}", from=1-2, to=2-2]
    	\arrow["\det", from=2-1, to=2-2]
    \end{tikzcd}
  \end{equation*}
  commute. It is $\cO_K$-multilinear and alternating.
\begin{proof}
  This is part of the proof of \cite[Theorem
  2.10.3]{BoyarchenkoWeinstein2011MaxVar}.
  Commutativity follows from 
  \begin{multline*}
    \log_{\wedge H}(\delta_0(s_1, \dots, s_n)) = \sum_{(a_1, \dots, a_n)} 
    \varepsilon(\mathbf a) \log_{\wedge H} \mu_0( x_1^{q^{a_1}}\cdots x_n^{q^{a_n}})
     \\ = \sum_{(a_1, \dots, a_n)} \varepsilon(\mathbf a) \sum_{m \in \Z} (-1)^{(n-1)m}
    \frac{x_1^{q^{a_1+m}} \cdots x_n^{q^{a_n+m}}}{\varpi^m} 
    = \det \left( \sum_{m \in \Z} \frac{x_i^{q^{mn + j-1}}}{\varpi^m} \right)_{\substack{
      1 \leq i \leq n,\\
      1 \leq j \leq n}},
  \end{multline*}
  which is equal to $\det(\qlog_H^n(s_1, \dots, s_n))$ by Proposition 
  \ref{prop:qlogmapExplicit} and Lemma \ref{lem:LogInTermsOfNil}. 
  The fact that $\delta_0$ is multilinear and alternating ultimately follows
  from the corresponding properties of $\det$, the fact that $\ker(\log_H) =
  \wedge H[\varpi^\infty]$ (cf. Lemma \ref{lem:KernelOfLog}) and topological considerations
  in the induced diagram in the category of adic spaces over $(\breve K, \cO_{\breve K})$.
\end{proof}
\end{prop}

This allows us to define the sought for morphism of functors
$\delta: \Tilde H^n \to \Tilde {\wedge H}$. 
\begin{defi}\label{def:DeltaMap}
  Put $\delta_i(s_1, \dots, s_n) = \delta_0(\varpi^{-i} s_1, \dots, s_n)$. Then 
  $\delta = (\delta_0, \delta_1, \dots)$ yields a map 
  $\Tilde H^n \to \Tilde{\wedge H}$. It is $K$-multilinear and alternating.
\end{defi}

Using the canonical identifications $\Tilde {H}^n \cong (\Nilp^\flat)^n$
and $\Tilde{\wedge H} \cong \Nilp^\flat$, the morphism $\delta$ yields a 
map $(\Nilp^\flat)^n \to \Nilp^\flat$, which in turn is the same as a power series 
\begin{equation*}
  \Delta(X_1, \dots, X_n) \in \cO_{\breve K}\llbr X_1^{q^{-\infty}}, \dots,
  X_n^{q^{-\infty}}\rrbr
\end{equation*}
together with distinguished $q$-th power roots. 
We have the following approximation of $\Delta$. 
\begin{lem}\label{lem:DeltaApprox}
  We have $\Delta(X_1, \dots, X_n) \equiv \det(X_i^{q^{j-1}})_{\substack{1 \leq i \leq n,\\ 1 \leq j \leq n}}$ modulo $(X_1, \dots, X_n)^{q^n}$.
\begin{proof}
  \red{TODO. This is \cite[Lemma 2.10.4]{BoyarchenkoWeinstein2011MaxVar}, but they
  don't explain the proof.}
\end{proof}
\end{lem}

\begin{thm}[]\label{thm:}
  There is a cartesian square
\begin{equation*}
\begin{tikzcd}[ampersand replacement=\&]
	{\cM_{H_0, \infty}} \& {\cM_{\wedge^n H_0, \infty}} \\
	{\tilde H^n} \& {\widetilde{\wedge H}}
	\arrow[from=1-1, to=1-2]
	\arrow[from=1-1, to=2-1]
	\arrow[from=1-2, to=2-2]
	\arrow["\delta", from=2-1, to=2-2]
\end{tikzcd}
\end{equation*}
Here, the vertical arrows are given by the morphism constructed in 
Section \ref{ssub:Relation to the Deformation Space at Infinite Level}. 
\begin{proof}
\end{proof}
\end{thm}

% subsubsection Determinants of Formal Modules (end)


% subsection Determinants of Formal Modules (end)

\end{document}
